\documentclass[11pt]{extarticle}

\usepackage{color}
\usepackage{graphicx}
\usepackage{fullpage}
\usepackage{verbatim}
\usepackage{tikz}
\usepackage{listings}
\usepackage[yyyymmdd,hhmmss]{datetime}
\usepackage{rotating}
\usepackage{authblk}
\usepackage{amsfonts}
 \usepackage{todonotes}

\usepackage{titlesec}

\setcounter{secnumdepth}{3}

\titleformat{\paragraph}
{\normalfont\normalsize\bfseries}{\theparagraph}{1em}{}
\titlespacing*{\paragraph}
{0pt}{3.25ex plus 1ex minus .2ex}{1.5ex plus .2ex}

\begin{document}

\title{Proposal for a GraphBLAS C API\\ (\emph{\large Working document from the \emph{GraphBLAS} Signatures Subgroup})}
\author{Aydin Buluc, Timothy Mattson, Scott McMillan, Jos\'e Moreira, Carl Yang}
\affil{GraphBLAS Signatures Subgroup}
\date{Generated on \today\ at \currenttime\ EDT}

\renewcommand{\vector}[1]{{\bf #1}}
\renewcommand{\matrix}[1]{{\bf #1}}
\newcommand{\zip}{{\mbox{zip}}}
\newcommand{\zap}{{\mbox{zap}}}
\newcommand{\ewiseadd}{{\mbox{\bf ewiseadd}}}
\newcommand{\ewisemult}{{\mbox{\bf ewisemult}}}
\newcommand{\mxm}{{\mbox{\bf mxm}}}
\newcommand{\vxm}{{\mbox{\bf vxm}}}
\newcommand{\mxv}{{\mbox{\bf mxv}}}
\newcommand{\gpit}[1]{{\sf #1}}
\newcommand{\ie}{\emph{i.e.}}
\newcommand{\nan}{{\sf NaN}}
\newcommand{\nil}{{\bf nil}}
\newcommand{\ifif}{{\bf if}}
\newcommand{\ifthen}{{\bf then}}
\newcommand{\ifelse}{{\bf else}}
\newcommand{\ifendif}{{\bf endif}}
\newcommand{\zero}{{\bf 0}}
\newcommand{\one}{{\bf 1}}

\newcommand{\aydin}[1]{{{\color{orange}[Aydin: #1]}}}
\newcommand{\scott}[1]{{{\color{violet}[Scott: #1]}}}
\newcommand{\tim}[1]{{{\color{teal}[Tim: #1]}}}
\newcommand{\jose}[1]{{{\color{red}[Jose: #1]}}}
\newcommand{\carl}[1]{{{\color{blue}[Carl: #1]}}}
\newcommand{\ajy}[1]{{{\color{brown}[Yzelman: #1]}}}

\setlength{\parskip}{0.5\baselineskip}
\setlength{\parindent}{0ex}

\maketitle

\carl{testing}
\scott{testing}
\aydin{testing}
\tim{testing}
\jose{testing}
\ajy{testing}

\section{Introduction}

This is a proposal for the C programming language binding of the
GraphBLAS interface. We adopt C99 as the standard definition of the C
programming language.  Furthermore, we assume that the language has been
extended with the {\tt \_Generic} construct from C11.  After establishing
some basic concepts, we proceed by describing the objects in GraphBLAS:
spaces, vectors, matrices and descriptors. We then describe the various
methods that operate on those objects. The appendix includes examples
of GraphBLAS in C.

\ajy{Is the use of \_Generic an implementation detail or will it show up in 
the API?  If the former, could it be dropped from the spec?}

\section{Basic concepts}

\subsection{Domains}

GraphBLAS defines two kinds of collections: matrices and vectors.
For any given collection, the elements of the collection belong to
a \emph{domain}, which is the set of valid values for the element.
In GraphBLAS, domains correspond to the valid values for types from
the host language (in our case, the C programming language).  For any
variable or object $V$ in GraphBLAS we denote as $D(V)$ the domain of
$V$. That is, the set of possible values that elements of $V$ can take.
The predefined types, and corresponding domains, of GraphBLAS are shown
in Table~\ref{Tab:PredefinedTypes}.  The Boolean type is defined in
{\tt stdbool.h}, the integral types are defined in {\tt stdint.h}, and
the floating-point types are native to the language.  GraphBLAS also
supports user defined types. In that case, the domain is the set of
valid values for a variable of that type.

\begin{table}
\hrule
\begin{center}
\caption{Predefined types and corresponding domains for GraphBLAS in C. 
              \scott{Domains and types are not always equivalent although they may be in this
                         table and the third column of this table does not indicate that these items (are they enums?) 
                         really are indicating domains.  This is a naming nitpick, sorry.  I think we should
                         also go through the exercise of supporting a Domain that does not have a 1-to-1
                         correllation with any type e.g. \{0,1\}.  Also need to understand how a user
                         extends the set of domains supported.}
                         \aydin{There will be a way to introduce new GraphBLAS identifiers, similar to MPI\_Type\_Commit, these are just predefined stuff}}
\label{Tab:PredefinedTypes}
\begin{tabular}{l|l|l}
type	& domain & GraphBLAS identifier \\ \hline
{\tt bool}	& $\{ {\tt false}, {\tt true} \}$	& {\sf GrB\_BOOL} \\
{\tt int8\_t}	& $\mathbb{Z} \cap [-2^{7},2^{7})$ 	& {\sf GrB\_INT8} \\
{\tt uint8\_t}	& $\mathbb{Z} \cap [0,2{^8})$ 		& {\sf GrB\_UINT8} \\
{\tt int16\_t}	& $\mathbb{Z} \cap [-2^{15},2^{15})$ 	& {\sf GrB\_INT16} \\
{\tt uint16\_t}	& $\mathbb{Z} \cap [0,2^{16})$ 		& {\sf GrB\_UINT16} \\
{\tt int32\_t}	& $\mathbb{Z} \cap [-2^{31},2^{31})$ 	& {\sf GrB\_INT32} \\
{\tt uint32\_t}	& $\mathbb{Z} \cap [0,2^{32})$ 		& {\sf GrB\_UINT32} \\
{\tt int64\_t}	& $\mathbb{Z} \cap [-2^{63},2^{63})$ 	& {\sf GrB\_INT64} \\
{\tt uint64\_t}	& $\mathbb{Z} \cap [0,2^{64})$ 		& {\sf GrB\_UINT64} \\
{\tt float}	& IEEE 754 {\sf binary32} 		& {\sf GrB\_FLOAT} \\
{\tt double}	& IEEE 754 {\sf binary64} 		& {\sf GrB\_DOUBLE} \\
\end{tabular}
\end{center}
\hrule
\end{table}

\subsection{Operations}

In GraphBLAS, a \emph{binary operation} is a function that maps two input
values to one output value. A \emph{unary operation} is a function that 
maps one input value to one output value. The value of the output is uniquely
determined by the value of the input(s).
Binary functions are defined over two input domains and produce an output from
a (possibly different) third domain. Unary functions are specified
over one input domain and produce an output from a (possibly different)
second domain.
The predefined operations of GraphBLAS are listed in
Table~\ref{Tab:PredefinedOperations}.

\begin{table}
\hrule
\begin{center}
\caption{Predefined operations for GraphBLAS in C. (Just a sample.)
\scott{Rather than a sample we should strive to specify the complete set of predefined operations; add min and max. What C construct are the 
symbols in the second column.  They look like enums or constants that need to be completely specified by the API.}}
\label{Tab:PredefinedOperations}
\begin{tabular}{l|l|l|l}
kind		& operation 		& domains								& description \\ \hline
		& {\sf GrB\_NOP}	& 									& no operation \\
unary		& {\sf GrB\_LNOT}	& ${\tt bool} \rightarrow {\tt bool}$      				& logical inverse \\
binary		& {\sf GrB\_LAND}	& ${\tt bool} \times {\tt bool} \rightarrow {\tt bool}$      		& logical AND \\
binary		& {\sf GrB\_LOR}	& ${\tt bool} \times {\tt bool} \rightarrow {\tt bool}$      		& logical OR \\
binary		& {\sf GrB\_LXOR}	& ${\tt bool} \times {\tt bool} \rightarrow {\tt bool}$      		& logical XOR \\
binary		& {\sf GrB\_PLUS}	& ${\tt int64\_t} \times {\tt int64\_t} \rightarrow {\tt int64\_t}$ 	& signed integer addition \\
binary		& {\sf GrB\_PLUS}	& ${\tt uint64\_t} \times {\tt uint64\_t} \rightarrow {\tt uint64\_t}$ 	& unsigned integer addition \\
binary		& {\sf GrB\_PLUS}	& ${\tt double} \times {\tt double} \rightarrow {\tt double}$ 		& floating-point addition \\
binary		& {\sf GrB\_MINUS}	& ${\tt int64\_t} \times {\tt int64\_t} \rightarrow {\tt int64\_t}$ 	& signed integer subtraction \\
binary		& {\sf GrB\_MINUS}	& ${\tt uint64\_t} \times {\tt uint64\_t} \rightarrow {\tt uint64\_t}$ 	& unsigned integer subtraction \\
binary		& {\sf GrB\_MINUS}	& ${\tt double} \times {\tt double} \rightarrow {\tt double}$ 		& floating-point subtraction \\
binary		& {\sf GrB\_TIMES}	& ${\tt int64\_t} \times {\tt int64\_t} \rightarrow {\tt int64\_t}$ 	& signed integer multiplication \\
binary		& {\sf GrB\_TIMES}	& ${\tt uint64\_t} \times {\tt uint64\_t} \rightarrow {\tt uint64\_t}$ 	& unsigned integer multiplication \\
binary		& {\sf GrB\_TIMES}	& ${\tt double} \times {\tt double} \rightarrow {\tt double}$ 		& floating-point multiplication \\
binary		& {\sf GrB\_DIV}	& ${\tt int64\_t} \times {\tt int64\_t} \rightarrow {\tt int64\_t}$ 	& signed integer division \\
binary		& {\sf GrB\_DIV}	& ${\tt uint64\_t} \times {\tt uint64\_t} \rightarrow {\tt uint64\_t}$ 	& unsigned integer division \\
binary		& {\sf GrB\_DIV}	& ${\tt double} \times {\tt double} \rightarrow {\tt double}$ 		& floating-point division \\
binary		& {\sf GrB\_EQ}		& ${\tt int64\_t} \times {\tt int64\_t} \rightarrow {\tt bool}$		& signed integer  equal \\
binary		& {\sf GrB\_EQ}		& ${\tt uint64\_t} \times {\tt uint64\_t} \rightarrow {\tt bool}$ 	& unsigned integer  equal \\
binary		& {\sf GrB\_EQ}		& ${\tt double} \times {\tt double} \rightarrow {\tt bool}$ 		& floating-point  equal \\
binary		& {\sf GrB\_NE}		& ${\tt int64\_t} \times {\tt int64\_t} \rightarrow {\tt bool}$		& signed integer not equal \\
binary		& {\sf GrB\_NE}		& ${\tt uint64\_t} \times {\tt uint64\_t} \rightarrow {\tt bool}$ 	& unsigned integer not equal \\
binary		& {\sf GrB\_NE}		& ${\tt double} \times {\tt double} \rightarrow {\tt bool}$ 		& floating-point not equal \\
binary		& {\sf GrB\_GT}		& ${\tt int64\_t} \times {\tt int64\_t} \rightarrow {\tt bool}$		& signed integer greater than  \\
binary		& {\sf GrB\_GT}		& ${\tt uint64\_t} \times {\tt uint64\_t} \rightarrow {\tt bool}$ 	& unsigned integer greater than  \\
binary		& {\sf GrB\_GT}		& ${\tt double} \times {\tt double} \rightarrow {\tt bool}$ 		& floating-point greater than  \\
binary		& {\sf GrB\_LT}		& ${\tt int64\_t} \times {\tt int64\_t} \rightarrow {\tt bool}$		& signed integer less than  \\
binary		& {\sf GrB\_LT}		& ${\tt uint64\_t} \times {\tt uint64\_t} \rightarrow {\tt bool}$ 	& unsigned integer less than  \\
binary		& {\sf GrB\_LT}		& ${\tt double} \times {\tt double} \rightarrow {\tt bool}$ 		& floating-point less than  \\
binary		& {\sf GrB\_GE}		& ${\tt int64\_t} \times {\tt int64\_t} \rightarrow {\tt bool}$		& signed integer greater than or equal \\
binary		& {\sf GrB\_GE}		& ${\tt uint64\_t} \times {\tt uint64\_t} \rightarrow {\tt bool}$ 	& unsigned integer greater than or equal \\
binary		& {\sf GrB\_GE}		& ${\tt double} \times {\tt double} \rightarrow {\tt bool}$ 		& floating-point greater than or equal \\
binary		& {\sf GrB\_LE}		& ${\tt int64\_t} \times {\tt int64\_t} \rightarrow {\tt bool}$		& signed integer less than or equal \\
binary		& {\sf GrB\_LE}		& ${\tt uint64\_t} \times {\tt uint64\_t} \rightarrow {\tt bool}$ 	& unsigned integer less than or equal \\
binary		& {\sf GrB\_LE}		& ${\tt double} \times {\tt double} \rightarrow {\tt bool}$ 		& floating-point less than or equal \\
\end{tabular}
\end{center}
\hrule
\end{table}

\section{Objects}

\subsection{Spaces}

\scott{Space is too overloaded.  We should decide on a better name.  I propose calling the section "Algebraic Structures" or"Algebra" for short.}

A GraphBLAS \emph{space} $S=\langle D_1,D_2,D_3,\oplus,\otimes,\zero [,\one] \rangle$ is defined by three domains $D_1$, $D_2$ and $D_3$, an additive 
operation $\oplus : D_3 \times D_3 \rightarrow D_3$, with corresponding identity $\zero \in D_3$, and a multiplicative
operation $\otimes : D_1 \times D_2 \rightarrow D_3$, with optional corresponding identity $\one \in D_1 \cap D_2$. 
If $\one$ is specified, then $D_1 \subseteq D_3$ and $D_2 \subseteq D_3$ must be satisfied. For a given GraphBLAS
space $S=\langle D_1, D_2, D_3,\oplus,\otimes,\zero,\one \rangle$ we define $D_1(S) = D_1$, $D_2(S) = D_2$, $D_3(S) = D_3$, $\oplus(S) = \oplus$,
$\otimes(S) = \otimes$, $\zero(S) = \zero$ and $\one(S) = \one$. We note that, in the special case of $D_1 = D_2 = D_3$ and $\one$ defined,
a GraphBLAS space reduces to the conventional \emph{semiring} algebraic structure \scott{and $\zero$ must be $\oplus$ identity and $\otimes$ annihilator in order to be a semiring as well}.

\ajy{$D_i$, etc is overloaded or used recursively}

\ajy{There could be 2 $\one$'s: one in each of $D_1$ and $D_2$ and both could be optional.  This would relax the constraint that
 $D_1 \subseteq D_3$ and $D_2 \subseteq D_3$.}

\ajy{more far-fetched: a Space could consist of four domains: $\otimes: D_1 \times D_2 \rightarrow D_3$, and 
$\oplus: D_3 \times D_3 \rightarrow D_4$ where $D_3$ can be reduced to something in $D_4$.}

\scott{I think we should consider another object which is a monoid: a binary function and its identity.  This space or algebraic structure could
then be built from two monoids.  Further the monoids have three domains each: two for input and one for output.  There are some things to discuss:
This implies a more relaxation of the "space" to up to six domains if implicit or explicit casting allows it.}

\scott{I think the Monoid is what is most natural to provide to ewiseadd and ewisemult.}

\subsection{Vectors}

A vector $\vector{v} = \langle D, N, \{ (i,\vector{v}(i)) \} \rangle$ is defined by a domain $D$, a size $N>0$ and a set of tuples
$(i,\vector{v}(i))$ where $0 \leq i < N$ and $\vector{v}(i) \in D$. A
particular value of $i$ can only appear at most once in $\vector{v}$. We define $n(\vector{v}) = N$ and $L(\vector{v}) = \{ (i,\vector{v}(i)) \}$. We
also define the set $\vector{i(\vector{v})} = \{ i : (i,\vector{v}(i))
\in \vector{v} \}$, and $D(\vector{v}) = D$.

\ajy{Overloaded use of $\vector{v}$ in the definition.}

\ajy{Why not define in terms of $D^{N}$}

\scott{Going out on a limb here: matrices and vectors should be thought of more as storage containers rather than the pure mathematical concepts.  In this context I need a compelling reason to have vectors {\it and} matrices.  The only good reason I can think to to have vectors is that they are strictly
one dimensional; i.e., saves storage not having to store a second index for each stored value.  Implication: if you want to perform multiple BFS traversals,
it should be with a Matrix of wavefronts and not a 2D stack of 'vectors'.  Further if we have a strictly 1D structure, I believe we must give it an implicit orientation
for matrix operations (I prefer column vector) and discuss all the other vector operations proposed a while back.}

\subsection{Matrices}
\label{Sec:Matrices}

A matrix $\matrix{A} = \langle D, M, N,  \{ (i,j,\matrix{A}(i,j)) \} \rangle$ is defined by a domain $D$. its number of rows $M>0$, its number
of columns $N>0$ and a set of tuples
$(i,j,\matrix{A}(i,j))$ where $0 \leq i < M$,
$0 \leq j < N$, and $\matrix{A}(i,j)i \in D$. A particular
pair of values $i,j$ can only appear at most once in $\matrix{A}$. We define $n(\matrix{A}) = N$,  $m(\matrix{A}) = M$
and $L(\matrix{A}) = \{ (i,j,\matrix{A}(i,j)) \}$.
We also define the sets $\vector{i(\matrix{A})} = \{ i : \exists (i,j,\matrix{A}(i,j)) \in \matrix{A} \}$ and
$\vector{j(\matrix{A})} = \{ j : \exists (i,j,\matrix{A}(i,j)) \in \matrix{A} \}$.
(These are the sets of nonempty rows and columns of $\matrix{A}$, respectively.)
Finally, $D(\matrix{A}) = D$.

\ajy{Overloaded use of $\matrix{A}$ in the definition.}

\ajy{Why not define in terms of $D^{M \times N}$.}

If $\matrix{A}$ is a matrix and $0 \leq j < N$, then $\matrix{A}(:,j) = \langle D, M, \{(i,\matrix{A}(i,j)) : (i,j,\matrix{A}(i,j)) \in L(\matrix{A}) \} \rangle$
is a vector called the $j$-th \emph{column} of $\matrix{A}$. Correspondingly,
if $\matrix{A}$ is a matrix and $0 \leq i < M$, then $\matrix{A}(i,:) = \langle D, N, \{(j,\matrix{A}(i,j)) : (i,j,\matrix{A}(i,j)) \in L(\matrix{A}) \} \rangle$
is a vector called the $i$-th \emph{row} of $\matrix{A}$.

\subsection{Descriptors}

Descriptors are used as input parameters in various GraphBLAS methods to
provide more details of the operation to be performed by those methods.
In particular, descriptors specify how the other input parameters
should be processed before the main operation of a method is performed.
For example, a descriptor may specify that a particular input matrix
needs to be transposed or that a mask needs to be inverted before using
it in the operation. Some methods may also allow additional processing
of the result before generating the final output parameter.

\scott{'inverted' above is ambiguous, we need to define a better term like "structural complement".  Further we should specify behaviour if the mask
is a dense container not just when it is sparse.}

For the purpose of constructing descriptors, the parameters of a method
are identified by specific names. The output parameter (typically
the first parameter in a GraphBLAS method) is {\sf OUTP}.  The input
parameters are named {\sf ARG0}, {\sf ARG1}, {\sf ARG2} and so on from
the first input parameter to the last. The mask (typically the next to
last parameter in a method) is named {\sf MASK}. Finally, the descriptor
(typically the last parameter in a method) is not named, since GraphBLAS
does not support modifications of descriptors themselves.

\scott{Is negate/invert/structural complement only restricted to masks?}

\scott{We must specify the behaviour of the descriptor's transpose.  E.g. is it allowed to mutate the operand for the duration of
the operation, or is this strictly a flag that affects the operation only -- how it accesses the operand's values?} 

\section{Methods}

\subsection{Vector-matrix multiply ({\sf vxm})}

Multiplies a vector by a matrix within a space. The result is a vector.

\paragraph{C99 Syntax}

\begin{verbatim}
#include "GraphBLAS.h"
GrB_info GrB_vxm(GrB_Vector *u, const GrB_Space s, const GrB_Vector v, 
                 const GrB_Matrix A, const GrB_Vector m, const GrB_Descriptor d)
\end{verbatim}

\ajy{Should {\sf u} parameter have {\tt restrict} keyword?}

\paragraph{Input Parameters}

\begin{itemize}
	\item[{\sf s}] ({\sf ARG0}) Space used in the vector-matrix multiply.
	\item[{\sf v}] ({\sf ARG1}) Vector to be multiplied.
	\item[{\sf A}] ({\sf ARG2}) Matrix to be multiplied.
	\item[{\sf m}] ({\sf MASK}) Operation mask. The mask specifies which elements of the result vector are to be computed.
		       If no mask is necessary (i.e., compute all elements of result vector), {\sf GrB\_NULL} can be used.
	\item[{\sf d}] Operation descriptor. The descriptor is used to specify details of the operation, such as transpose the matrix or not,
		       invert the mask or not (see below). If a \emph{default} descriptor is desired, {\sf GrB\_NULL} can be used.
\end{itemize}

\ajy{Can {\sf v} and {\sf m} refer to the same container?}
\ajy{Should ({\sf u} and {\sf v}) or ({\sf u} and {\sf m} NOT refer to the same container...use restrict?}

\paragraph{Output Parameter}

\begin{itemize}
	\item[{\sf u}] ({\sf OUTP}) Address of result vector.
\end{itemize}

\paragraph{Return Value}

\scott{We should specify anything that we can about the behaviour/program state when any error condition is returned.  What gaurantees are we giving (think consistency models)?  Definitely related to mutability question earlier.}

\begin{tabular}{rl} 
{\sf GrB\_SUCCESS} 	& operation completed successfully \\
{\sf GrB\_PANIC}	& unknown internal error \\
{\sf GrB\_OUTOFMEM}	& not enough memory available for operation \\
{\sf GrB\_MISMATCH}	& mismatch among vectors, matrix and/or space
\end{tabular}

\scott{More return values possible: domain/type mismatches, dimension mismatches}

\paragraph{Description}

Vectors $\vector{v}, \vector{m}$ and matrix $\matrix{A}$ are computed
from input parameters {\sf v}, {\sf m} and {\sf A}, respectively, as
specified by descriptor {\sf d}. (See below for the properties of a
descriptor. In the simplest form, these are just copies, but 
additional preprocessing, including casting, can be specified.)
$D(\vector{v}) \equiv D_1({\sf s})$ and $D(\matrix{A}) \equiv D_2({\sf s})$.
If {\sf m} is {\sf GrB\_NULL} then
$\vector{m}$ is a Boolean vector of size $n(\vector{A})$ and with
all elements set to {\sf true}.  

A consistency check is performed
to verify that $n(\vector{v}) = m(\matrix{A})$ and $n(\vector{m}) =
n(\matrix{A})$. If a consistency check fails, the operation is aborted
and the method returns {\sf GrB\_MISMATCH}.

A new vector $\vector{u} = \langle D_3({\sf s}), n(\matrix{A}),
L(\vector{u}) = \{(i,\vector{u}(i)) : \vector{m}(i) = {\sf true} \}
\rangle$ is created.  The value of each of its elements is computed
by $\vector{u}(i) = \bigoplus_{j \in \vector{i}(\vector{v}) \cap
\vector{i}(\matrix{A}(:,i))} (\vector{v}(j) \otimes \matrix{A}(j,i))$,
where $\oplus$ and $\otimes$ are the additive and multiplicative
operations of space {\sf s}, respectively.
If $\vector{i}(\vector{v}) \cap \vector{i}(\matrix{A}(:,i)) = \emptyset$ then
the pair $(i,\vector{u}(i))$ is not included in $L(\vector{u})$.

Finally, output parameter {\sf u} is computed from vector $\vector{u}$ as
specified by descriptor {\sf d}. (Again, in the simplest case this is just
a copy, but additional postprocessing, including casting and accumulation
of result values, can be specified.)  A consistency check is performed
to verify that $n({\sf u}) = n(\vector{u})$. If the consistency check
fails, the operation is aborted and the method return {\sf GrB\_MISMATCH}.

\scott{We need a more explicit discussion/specification regarding masks and accumulation and their interaction (perhaps the diagram Manoj projected
at the SC15 BoF.}

\scott{Is accumulation restricted to the use of the $\oplus$ operation of the {\sf s} argument?  If so, add to the specification.}

\subsection{Create new descriptor ({\sf Descriptor\_new})}

Creates a new (empty) descriptor.

\paragraph{C99 Syntax}

\begin{verbatim}
#include "GraphBLAS.h"
GrB_info GrB_Descriptor_new(GrB_Descriptor *d)
\end{verbatim}

\paragraph{Output Parameters}

\begin{itemize}
	\item[{\sf d}] Identifier of new descriptor created.
\end{itemize}

\paragraph{Return Value}

\begin{tabular}{rl} 
{\sf GrB\_SUCCESS} 	& operation completed successfully \\
{\sf GrB\_PANIC}	& unknown internal error \\
{\sf GrB\_OUTOFMEM}	& not enough memory available for operation \\
{\sf GrB\_MISMATCH}	& mismatch between field and new value
\end{tabular}

\paragraph{Description}

Returns in {\sf d} the identifier of a newly created empty descriptor.
A newly created descriptor can be populated with calls to
{\sf Descriptor\_add}.

\subsection{Add content to descriptor ({\sf Descriptor\_add})}

\scott{Naming nit: I propose {\sf Descriptor\_set}.  "adding" implies accumulation (OR) of flags across many calls.  Allowing only set which
overwrites any existing values is simpler.}

Adds additional content (details of an operation) to an existing descriptor.

\paragraph{C99 Syntax}

\begin{verbatim}
#include "GraphBLAS.h"
GrB_info GrB_Descriptor_add(GrB_Descriptor d,GrB_Field f,GrB_Value v)
\end{verbatim}

\paragraph{Input Parameters}

\begin{itemize}
	\item[{\sf d}] The descriptor being modified by this method.
	\item[{\sf f}] The field of the descriptor being modified.
	\item[{\sf v}] New value for the field being modified.
\end{itemize}

\paragraph{Return Value}

\begin{tabular}{rl} 
{\sf GrB\_SUCCESS} 	& operation completed successfully \\
{\sf GrB\_PANIC}	& unknown internal error \\
{\sf GrB\_OUTOFMEM}	& not enough memory available for operation \\
{\sf GrB\_MISMATCH}	& mismatch between field and new value
\end{tabular}

\paragraph{Description}

The fields of a descriptor include: {\sf GrB\_OUTP} for the 
output parameter (result) of a method; {\sf GrB\_MASK} for the mask
argument to a method; {\sf GrB\_ARG0} through {\sf GrB\_ARG9} for
the input parameters (from first to last) of a method.

Valid values for a field of a descriptor are as follows:

\begin{tabular}{rl} 
{\sf GrB\_NOP} 	& no operation to be performed for the corresponding parameter \\
{\sf GrB\_LNOT}	& compute the logical inverse \scott{structural complement?} of the corresponding parameter  \\
{\sf GrB\_TRAN}	& compute the transpose of the corresponding parameter (for matrices) \\
{\sf GrB\_ACC}  & accumulate result of operation to current values in destination (for output parameter) \\
{\sf GrB\_CAST} & \parbox[t]{5in}{cast values \scott{"allow casting of values..."} from input parameters to input domains of operation or from output domain of operation
		  to output parameter. (Otherwise, incorrect domain will cause a run-time error.) \scott{trying to understand if exact domain match is required or if language type is what we are talking about.  Not sure how this would be implemented.}}
\end{tabular}

\scott{GrB\_LNOT clashes with operator in Table 2} 
\ajy{GrB\_LNOT: logical inverse of non-mask parameters can be implemented by modifying the operators; therefore consider restricting this to masks only.}

It is possible to specify a combination of values for a field. For 
example, if a matrix is to be both transposed and logically inverted
(element by element), one would use the field value
${\sf GrB\_TRAN} \mid {\sf GrB\_LNOT}$. 

\subsection{Create new space ({\sf Space\_new})}

Creates a new space with specified domain, operations and identities.

\paragraph{C99 Syntax}

\begin{verbatim}
#include "GraphBLAS.h"
GrB_info GrB_Space_new(GrB_Space *s,GrB_type t1, GrB_type t2, GrB_type t3,
                       GrB_operation a,GrB_operation m, t3 z[, t3 o]))
\end{verbatim}

\scott{This signature is a little confusing partially because we have not been explicit with the domain/type specifications earlier. 
If the GrB\_type is the name of an enum for domains (not types) then I would suggest calling it GrB\_domain.  Not sure how {\sf ti} 
can be used as both a type and a parameter and may be related to the domain/type issue.}

\paragraph{Input Parameters}

\begin{itemize}
	\item[{\sf t1}] The type defining the first domain of the space being created. Should be one of the predefined
	GraphBLAS types in Table~\ref{Tab:PredefinedTypes}, or a user created type.
	\item[{\sf t2}] The type defining the second domain of the space being created. Should be one of the predefined
	GraphBLAS types in Table~\ref{Tab:PredefinedTypes}, or a user created type.
	\item[{\sf t3}] The type defining the third domain of the space being created. Should be one of the predefined
	GraphBLAS types in Table~\ref{Tab:PredefinedTypes}, or a user created type.
	\item[{\sf a}] The additive operation of the space.
	\item[{\sf m}] The multiplicative operation of the space.
	\item[{\sf z}] The additive identity of the space.
	\item[{\sf o}] The multiplicative identify of the space.
\end{itemize}

\paragraph{Output Parameter}

\begin{itemize}
	\item[{\sf s}] Identifier of the newly created space.
\end{itemize}

\paragraph{Return Value}

\begin{tabular}{rl} 
{\sf GrB\_SUCCESS} 	& operation completed successfully \\
{\sf GrB\_PANIC}	& unknown internal error \\
{\sf GrB\_OUTOFMEM}	& not enough memory available for operation \\
\end{tabular}

\paragraph{Description}

Creates a new space $S = \langle D({\sf t}), {\sf a}, {\sf m}, {\sf z}, {\sf o} \rangle$ and
returns its identifier in {\sf s}.

\subsection{Create new vector ({\sf Vector\_new})}

Creates a new vector with specified domain and size.

\paragraph{C99 Syntax}

\begin{verbatim}
#include "GraphBLAS.h"
GrB_info GrB_Vector_new(GrB_Vector *v,GrB_type t,GrB_index n)
\end{verbatim}

\paragraph{Input Parameters}

\begin{itemize}
	\item[{\sf t}] The type defining the domain of the vector being created. Should be one of the predefined
	GraphBLAS types in Table~\ref{Tab:PredefinedTypes}, or a user created type.
	\item[{\sf n}] The size of the vector being created.
\end{itemize}

\paragraph{Output Parameter}

\begin{itemize}
	\item[{\sf v}] Identifier of the newly created vector.
\end{itemize}

\paragraph{Return Value}

\begin{tabular}{rl} 
{\sf GrB\_SUCCESS} 	& operation completed successfully \\
{\sf GrB\_PANIC}	& unknown internal error \\
{\sf GrB\_OUTOFMEM}	& not enough memory available for operation \\
\end{tabular}

\paragraph{Description}

Creates a new vector $\vector{v}$ of domain $D({\sf t})$, size {\sf n}, and
empty $L(\vector{v})$. It return in {\sf v} this vector $\vector{v}$.

\subsection{Number of rows in a matrix ({\sf Matrix\_nrows})}

Retrieve the number of rows in a matrix.

\paragraph{C99 Syntax}

\begin{verbatim}
#include "GraphBLAS.h"
GrB_info GrB_Matrix_nrows(GrB_index *m,GrB_Matrix A)
\end{verbatim}

\paragraph{Input Parameters}

\begin{itemize}
	\item[{\sf A}] Matrix being queried.
\end{itemize}

\paragraph{Output Parameters}
\begin{itemize}
	\item[{\sf m}] The number of rows in the matrix.
\end{itemize}

\paragraph{Return Value}

\begin{tabular}{rl}
{\sf GrB\_SUCCESS}	& operation completed successfully \\
{\sf GrB\_PANIC}	& unknown internal error \\
{\sf GrB\_NOMATRIX}	& matrix does not exist \\
\end{tabular}

\paragraph{Description}

Return in {\sf m} the number of rows (parameter $M$ in Section~\ref{Sec:Matrices}) in matrix {\sf A}.

\subsection{Assign values to the elements of an object ({\sf assign})}

\scott{these variants need to be discussed perhaps in the large group; not currently part of any prior document.}

\subsubsection{Flat variant}

Set all the elements of a vector to a given value.

\begin{verbatim}
#include "GraphBLAS.h"
GrB_info GrB_assign(GrB_Vector *v,scalar s[,GrB_Vector m])
\end{verbatim}

\paragraph{Input Parameters}

\begin{itemize}
	\item[{\sf v}] Vector to be assigned.
	\item[{\sf s}] Scalar value for the elements.
	\item[{\sf m}] (Optional) mask for assignment. \aydin{Maybe say in the document that GrB\_Vector's domain could only be GrB\_Index for this function}
\end{itemize}

\paragraph{Return Value}

\begin{tabular}{rl}
{\sf GrB\_SUCCESS}	& operation completed successfully \\
{\sf GrB\_PANIC}	& unknown internal error \\
{\sf GrB\_NOVECTOR}	& vector does not exist \\
{\sf GrB\_MISMATCH}	& mismatch between vector domain and scalar type \\
\end{tabular}

\subsubsection{Indexed variant}

Set some of the elements of a vector to a given value.
\scott{Set one element of...}

\paragraph{C99 Syntax}

\begin{verbatim}
#include "GraphBLAS.h"
GrB_info GrB_assign(GrB_Vector *v,scalar s,GrB_index i)
\end{verbatim}

\paragraph{Input Parameters}

\begin{itemize}
	\item[{\sf v}] Vector to be assigned.
	\item[{\sf s}] Scalar value for the elements.
	\item[{\sf i}] Index of element to be assigned
\end{itemize}

\paragraph{Return Value}

\begin{tabular}{rl}
{\sf GrB\_SUCCESS}	& operation completed successfully \\
{\sf GrB\_PANIC}	& unknown internal error \\
{\sf GrB\_NOVECTOR}	& vector does not exist \\
{\sf GrB\_MISMATCH}	& mismatch between vector domain and scalar type \\
\end{tabular}

\subsection{Perform of a reduction across the elements of an object ({\sf reduce})}

Computes the reduction of the values of the elements of a vector or matrix.

\subsubsection{Vector variant}

\paragraph{C99 Syntax}

\begin{verbatim}
#include "GraphBLAS.h"
GrB_info GrB_reduce(scalar *s,GrB_Vector v,GrB_operations f)
\end{verbatim}

\scott{Should we use the space/algebra in place of the {\sf f} parameter and just use the $\oplus$ or if an algebra consists of monoids this
is another place where a Monoid is appropriate.  Note that we must know the identity value for the operation in order to store the correct
value in the scalar if the vector that you are reducing has not stored values.}


\paragraph{Input Parameters}

\begin{itemize}
	\item[{\sf v}] Vector to be reduced.
	\item[{\sf f}] Operation to be applied in the reduction.
\end{itemize}

\paragraph{Output Parameters}

\begin{itemize}
	\item[{\sf s}] Initial and final value of the reduction.
\end{itemize}

\paragraph{Return Value}

\begin{tabular}{rl}
{\sf GrB\_SUCCESS}	& operation completed successfully \\
{\sf GrB\_PANIC}	& unknown internal error \\
{\sf GrB\_NOVECTOR}	& vector does not exist \\
{\sf GrB\_MISMATCH}	& mismatch between vector domain and scalar type \\
\end{tabular}

\subsection{Destroy object ({\sf free})}

Destroys a previously created GraphBLAS object.

\paragraph{C99 Syntax}

\begin{verbatim}
#include "GraphBLAS.h"
GrB_info GrB_free(GrB_Object o)
\end{verbatim}

\ajy{polymorphic?  is there where \_Generic is specified?}

\paragraph{Input Parameter}

\begin{itemize}
	\item[{\sf o}] GraphBLAS object to be destroyed. Can be a matrix, vector or descriptor.
\end{itemize}

\paragraph{Return Value}

\begin{tabular}{rl}
{\sf GrB\_SUCCESS}	& operation completed successfully \\
{\sf GrB\_PANIC}	& unknown internal error \\
{\sf GrB\_NOOBJECT}	& object does not exist \\
\end{tabular}

\aydin{add \ewiseadd and \ewisemult}

\scott{add matrix\_new, matrix\_ncols, matrix\_nelts}

\appendix

\pagebreak

\section{Example: breadth first search with GraphBLAS}
{\scriptsize
\lstinputlisting[language=C,numbers=left]{BFS5.c}
}

\pagebreak

\section{Example: betweenness centrality with GraphBLAS}
{\scriptsize
\lstinputlisting[language=C,numbers=left]{BC1.c}
}
\end{document}
