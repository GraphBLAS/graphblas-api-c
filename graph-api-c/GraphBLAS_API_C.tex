\documentclass[11pt]{extarticle}

\usepackage{fancyvrb}
\usepackage{color}
\usepackage{graphicx}
\usepackage{fullpage}
\usepackage{verbatim}
\usepackage{tikz}
\usepackage{listings}
\usepackage[yyyymmdd,hhmmss]{datetime}
\usepackage{rotating}
\usepackage{authblk}
\usepackage{amsfonts}
\usepackage{todonotes}
\usepackage{titlesec}
\usepackage{lineno}
\usepackage{tabularx}
\usepackage{enumitem}
\usepackage{hyperref}

\setcounter{secnumdepth}{3}

\usepackage{draftwatermark}
\SetWatermarkText{DRAFT}
\SetWatermarkScale{2}

\titleformat{\paragraph}
{\normalfont\normalsize\bfseries}{\theparagraph}{1em}{}
\titlespacing*{\paragraph}
{0pt}{3.25ex plus 1ex minus .2ex}{1.5ex plus .2ex}



\newcommand{\qg}{\u{g}}
\newcommand{\qG}{\u{G}}
\newcommand{\qc}{\c{c} }
\newcommand{\qC}{\c{C}}
\newcommand{\qs}{\c{s}}
\newcommand{\qS}{\c{S}}
\newcommand{\qu}{\"{u}}
\newcommand{\qU}{\"{U}}
\newcommand{\qo}{\"{o}}
\newcommand{\qO}{\"{O}}
\newcommand{\qI}{\.{I}}
\newcommand{\wa}{\^{a}}
\newcommand{\wA}{\^{A}}


\begin{document}

\linenumbers

\title{Proposal for a GraphBLAS C API\\ (\emph{\large Working document from the \emph{GraphBLAS} Signatures Subgroup})}
%\author{Ayd\i n Bulu\c{c}, Timothy Mattson, Scott McMillan, Jos\'e Moreira, Carl Yang}
\author{The GraphBLAS Forum}
\date{Generated on \today\ at \currenttime\ EDT}

\renewcommand{\vector}[1]{{\bf #1}}
\renewcommand{\matrix}[1]{{\bf #1}}
\newcommand{\zip}{{\mbox{zip}}}
\newcommand{\zap}{{\mbox{zap}}}
\newcommand{\ewiseadd}{{\mbox{\bf ewiseadd}}}
\newcommand{\ewisemult}{{\mbox{\bf ewisemult}}}
\newcommand{\mxm}{{\mbox{\bf mxm}}}
\newcommand{\vxm}{{\mbox{\bf vxm}}}
\newcommand{\mxv}{{\mbox{\bf mxv}}}
\newcommand{\gpit}[1]{{\sf #1}}
\newcommand{\ie}{\emph{i.e.}}
\newcommand{\eg}{\emph{e.g.}}
\newcommand{\nan}{{\sf NaN}}
\newcommand{\nil}{{\bf nil}}
\newcommand{\ifif}{{\bf if}}
\newcommand{\ifthen}{{\bf then}}
\newcommand{\ifelse}{{\bf else}}
\newcommand{\ifendif}{{\bf endif}}
\newcommand{\zero}{{\bf 0}}
\newcommand{\one}{{\bf 1}}
\newcommand{\true}{{\sf true}}
\newcommand{\false}{{\sf false}}

\newcommand{\aydin}[1]{{{\color{orange}[Aydin: #1]}}}
\newcommand{\scott}[1]{{{\color{violet}[Scott: #1]}}}
\newcommand{\tim}[1]{{{\color{teal}[Tim: #1]}}}
\newcommand{\jose}[1]{{{\color{red}[Jose: #1]}}}
\newcommand{\carl}[1]{{{\color{blue}[Carl: #1]}}}
\newcommand{\ajy}[1]{{{\color{brown}[Yzelman: #1]}}}

% AYDIN: Comment these out or remove them to bring back the "comments"
%\renewcommand{\aydin}[1]{{{\color{orange}}}}
%\renewcommand{\scott}[1]{{{\color{violet}}}}
%\renewcommand{\tim}[1]{{{\color{teal}}}}
%\renewcommand{\jose}[1]{{{\color{red}}}}
%\renewcommand{\carl}[1]{{{\color{blue}}}}
%\renewcommand{\ajy}[1]{{{\color{brown}}}}

\renewcommand{\comment}[1]{{}}

\setlength{\parskip}{0.5\baselineskip}
\setlength{\parindent}{0ex}

\maketitle

\vfill

Copyright 2016 Carnegie Mellon University, The Regents of the University of California, through Lawrence Berkeley National Laboratory (subject to receipt of any required approvals from the U.S. Dept. of Energy), the Regents of the University of California (U.C. Davis), and the IBM Corporation. \scott{Need to add Intel Corporation when Tim Mattson returns from walkabout.}

Any opinions, findings and conclusions or recommendations expressed in this material are those of the author(s) and do not necessarily reflect the views of the United States Department of Defense, the United States Department of Energy, Carnegie Mellon University, the Regents of the University of California, or the IBM Corporation.  \scott{Need to add Intel Corporation after Tim Mattson returns from walkabout.}

NO WARRANTY. THIS MATERIAL IS FURNISHED ON AN AS-IS BASIS. THE COPYRIGHT OWNERS AND/OR AUTHORS MAKE NO WARRANTIES OF ANY KIND, EITHER EXPRESSED OR IMPLIED, AS TO ANY MATTER INCLUDING, BUT NOT LIMITED TO, WARRANTY OF FITNESS FOR PURPOSE OR MERCHANTABILITY, EXCLUSIVITY, OR RESULTS OBTAINED FROM USE OF THE MATERIAL. THE COPYRIGHT OWNERS AND/OR AUTHORS DO NOT MAKE ANY WARRANTY OF ANY KIND WITH RESPECT TO FREEDOM FROM PATENT, TRADE MARK, OR COPYRIGHT INFRINGEMENT.

This material is only for limited release only and review purposes.

%[Distribution Statement A] This material has been approved for public release and unlimited distribution.  Please see Copyright notice for non-US Government use and distribution.

Except as otherwise noted, this material is licensed under a Creative Commons Attribution 4.0 license (\href{http://creativecommons.org/licenses/by/4.0/legalcode}{http://creativecommons.org/licenses/by/4.0/legalcode}), and examples are licensed under the BSD License (\href{https://opensource.org/licenses/BSD-3-Clause}{https://opensource.org/licenses/BSD-3-Clause}).

%\begin{abstract}
%\end{abstract}

\pagebreak
\tableofcontents
\pagebreak

%-----------------------------------------------------------------------------

\section*{Acknowledgments}
\addcontentsline{toc}{section}{Acknowledgments}

\scott{Done in the style of the MPI Acknowledgments}

This document represents the work of many people who have served on the C API
Subcommittee of the GraphBLAS Forum.

Those who served as C API Subcommittee members for GraphBLAS 1.0 are (in alphabetical order):
\begin{itemize}
\item Ayd\i n Bulu\c{c} (Lawrence Berkeley National Laboratory)
\item Timothy Mattson (Intel Corporation)
\item Scott McMillan (Software Engineering Institute at Carnegie Mellon University)
\item Jos\'e Moreira (IBM Corporation)
\item Carl Yang (UC Davis)
\end{itemize}

The following list are the members of the GraphBLAS forum at large:
\begin{itemize}
\item John Gilbert, UCSB
\item Jeremy Kepner, MIT/LLSC
\item Manoj Kumar, IBM Corporation
\item Michael Wolf, Sandia National Laboratory
\item Albert Yzelman, Hauwei
\item Marcin Zalewski, Indiana University
\item \scott{Add many others (all folks on the big group calls?)}
\end{itemize}

The GraphBLAS 1.0 specification is based upon work funded and supported in part by:
\begin{itemize}
\item AGENCY under contract No. \#\#\#\#\#\#\# with ORGANIZATION.
\item the Department of Energy Office of Advanced Scientific Computing Research under contract number DE-AC02-05CH11231
\item placeholder for UC Davis?
\item placeholder for IBM
\item placeholder for Intel
\item Department of Defense under Contract No. FA8721-05-C-0003 with Carnegie Mellon University. [DM-0003727]
\end{itemize}
\pagebreak

%-----------------------------------------------------------------------------

%\carl{testing}
%\scott{testing}
%\aydin{testing}
%\tim{testing}
%\jose{testing}
%\ajy{testing}


\section{Introduction}

This is a proposal for the C programming language binding of the GraphBLAS
interface. We adopt C99 as the standard definition of the C programming
language. Furthermore, the interface makes use of static type-based and
number of parameters-based function polymorphism, and we require language
extensions at least on par with the {\tt \_Generic} construct from C11.
After establishing some basic concepts, we proceed by describing the
objects in GraphBLAS: functions, monoids, semirings, vectors, matrices
and descriptors. We then describe the various methods that operate on
those objects. The appendix includes examples of GraphBLAS in C.

%-----------------------------------------------------------------------------

\chapter{Basic concepts}

\section{Domains}

%
% I will add a comment about floating point arithmetic and
% associativity
%

GraphBLAS defines two kinds of collections: matrices and vectors.
For any given collection, the elements of the collection belong to
a \emph{domain}, which is the set of valid values for the element.
In GraphBLAS, domains correspond to the valid values for types from
the host language (in our case, the C programming language).  For any
variable or object $V$ in GraphBLAS we denote as $\bold{D}(V)$ the
domain of $V$. That is, the set of possible values that elements of
$V$ can take.  The predefined types, and corresponding domains, of
GraphBLAS are shown in Table~\ref{Tab:PredefinedTypes}.  The Boolean
type is defined in {\tt stdbool.h}, the integral types are defined in
{\tt stdint.h}, and the floating-point types are native to the language.
GraphBLAS also supports user defined types. In that case, the domain is
the set of valid values for a variable of that type.

\begin{table}
\hrule
\begin{center}
\caption{Predefined {\sf GrB\_Type} values, the corresponding C type (for scalar
parameters, and domains for GraphBLAS. 
        \scott{Consider name change from GrB\_Type to GrB\_Domain}
        \aydin{There will be a way to introduce new GraphBLAS identifiers,
        similar to MPI\_Type\_Commit, these are just predefined stuff}
        \scott{An example would be nice, 
        especially where there is not a 1-to-1 correspondence to a built-in
        type, e.g. \{0,1\}.}}
\label{Tab:PredefinedTypes}
\begin{tabular}{l|l|l}
{\sf GrB\_Type values} & C type            & domain \\
\hline
{\sf GrB\_BOOL}        & {\tt bool}        & $\{ {\tt false}, {\tt true} \}$  \\
{\sf GrB\_INT8}        & {\tt int8\_t}     & $\mathbb{Z} \cap [-2^{7},2^{7})$  \\
{\sf GrB\_UINT8}       & {\tt uint8\_t}    & $\mathbb{Z} \cap [0,2{^8})$  \\
{\sf GrB\_INT16}       & {\tt int16\_t}    & $\mathbb{Z} \cap [-2^{15},2^{15})$ \\
{\sf GrB\_UINT16}      & {\tt uint16\_t}   & $\mathbb{Z} \cap [0,2^{16})$ \\
{\sf GrB\_INT32}       & {\tt int32\_t}    & $\mathbb{Z} \cap [-2^{31},2^{31})$ \\
{\sf GrB\_UINT32}      & {\tt uint32\_t}   & $\mathbb{Z} \cap [0,2^{32})$ \\
{\sf GrB\_INT64}       & {\tt int64\_t}    & $\mathbb{Z} \cap [-2^{63},2^{63})$ \\
{\sf GrB\_UINT64}      & {\tt uint64\_t}   & $\mathbb{Z} \cap [0,2^{64})$ \\
{\sf GrB\_FLOAT}       & {\tt float}       & IEEE 754 {\sf binary32}  \\
{\sf GrB\_DOUBLE}      & {\tt double}      & IEEE 754 {\sf binary64}  \\
\end{tabular}
\end{center}
\hrule
\end{table}

\section{Functions}

In GraphBLAS, a \emph{binary function} is a function that maps two input
values to one output value. A \emph{unary function} is a function that 
maps one input value to one output value. The value of the output is uniquely
determined by the value of the input(s).  Binary functions are defined over 
two input domains and produce an output from a (possibly different) third 
domain. Unary functions are specified over one input domain and produce an 
output from a (possibly different) second domain.  The predefined functions 
of GraphBLAS are listed in Table~\ref{Tab:PredefinedFunctions}.

\begin{table}
\hrule
\begin{center}
\caption{Predefined unary and binary functions for GraphBLAS in C.}
\label{Tab:PredefinedFunctions}

\vspace{1\baselineskip}
(a) Valid suffixes and corresponding C type ($T$ in table (b)).
\vspace{1\baselineskip}

\begin{tabular}{l|l}
Suffix		& C type \\ \hline
{\sf B}		& {\tt bool} \\
{\sf I8}	& {\tt int8\_t} \\
{\sf U8}	& {\tt uint8\_t} \\
{\sf I16}	& {\tt int16\_t} \\
{\sf U16}	& {\tt uint16\_t} \\
{\sf I32}	& {\tt int32\_t} \\
{\sf U32}	& {\tt uint32\_t} \\
{\sf I64}	& {\tt int64\_t} \\
{\sf U64}	& {\tt uint64\_t} \\
{\sf F32}	& {\tt float} \\
{\sf F64}	& {\tt double} \\
\end{tabular}

\vspace{1\baselineskip}
(b) Predefined functions.
\vspace{1\baselineskip}

\begin{tabular}{l|l|l|l|l}
Function & GraphBLAS             &                                                              & \\
type     & identifier            & Domains                                                      & Description \\ \hline
         & {\sf GrB\_NOP}        &                                                              & no operation \\
unary    & {\sf GrB\_LNOT}       & ${\tt bool} \rightarrow {\tt bool}$                          & logical inverse \\
unary    & {\sf GrB\_SCMP}       & ${\tt bool} \rightarrow {\tt bool}$                          & structural complement \\
binary   & {\sf GrB\_LAND}       & ${\tt bool} \times {\tt bool} \rightarrow {\tt bool}$	& logical AND \\
binary   & {\sf GrB\_LOR}        & ${\tt bool} \times {\tt bool} \rightarrow {\tt bool}$        & logical OR \\
binary   & {\sf GrB\_LXOR}       & ${\tt bool} \times {\tt bool} \rightarrow {\tt bool}$        & logical XOR \\
binary   & {\sf GrB\_MIN\_$T$}   & $T \times T \rightarrow T$                                  & minimum \\
binary   & {\sf GrB\_MAX\_$T$}   & $T \times T \rightarrow T$                                  & maximum \\
binary   & {\sf GrB\_PLUS\_$T$}	 & $T \times T \rightarrow T$    				& addition \\
binary   & {\sf GrB\_MINUS\_$T$} & $T \times T \rightarrow T$    				& subtraction \\
binary   & {\sf GrB\_TIMES\_$T$} & $T \times T \rightarrow T$    				& multiplication \\
binary   & {\sf GrB\_DIV\_$T$}   & $T \times T \rightarrow T$    				& division \\
binary   & {\sf GrB\_EQ\_$T$}    & $T \times T \rightarrow {\tt bool}$        			& equal \\
binary   & {\sf GrB\_NE\_$T$}    & $T \times T \rightarrow {\tt bool}$        			& not equal \\
binary   & {\sf GrB\_GT\_$T$}    & $T \times T \rightarrow {\tt bool}$        			& greater than  \\
binary   & {\sf GrB\_LT\_$T$}    & $T \times T \rightarrow {\tt bool}$        			& less than  \\
binary   & {\sf GrB\_GE\_$T$}    & $T \times T \rightarrow {\tt bool}$        			& greater than or equal \\
binary   & {\sf GrB\_LE\_$T$}    & $T \times T \rightarrow {\tt bool}$        			& less than or equal \\
\end{tabular}
\end{center}
\hrule
\end{table}

\section{Index, Index Arrays and Scalar Arrays}

In order to interface with third-party software packages, operations like
buildMatrix (\S~\ref{Sec:buildMatrix})
and extractTuples (\S~\ref{Sec:extractTuples}) need to specify how the data should be
laid out in  non-opaque data structures.  To this end we define, explicitly
the types for indices and the arrays used by these operations.

For indices a typedef is used to give a GraphBLAS name to a concrete type. We define it as follows:

\begin{verbatim}
    typedef uint64_t GrB_Index;
\end{verbatim}

An index array is a pointer to a set of {\sf GrB\_Index} values that are stored in a contiguous block of memory (\ie, {\sf GrB\_Index*})

Likewise a scalar array is a pointer to a contiguous block of memory storing a number of scalar values as specified by the user.

\section{Execution model}

The purpose of most GraphBLAS operations is to manipulate GraphBLAS vectors and matrices
(the input arguments) and produce new content for another GraphBLAS vector or matrix (the output argument).
Each GraphBLAS operation uniquely and unambigiously defines the contents of its output argument.
Any later call to a GraphBLAS method that uses this matrix or vector will use its defined content, until
that content is redefined by another method.

GraphBLAS matrices and vectors are \emph{opaque} objects. That is, they can only be manipulated
by GraphBLAS methods. This opens a wide spectrum of optimizations that can be exploited by GraphBLAS implementations.
Deferred execution, lazy evaluation, chaining and fusion of operations, are all examples of optimizations
that can be used by a GraphBLAS implementation in a manner that is oblivious to the application program.

However, there are situations when this flexibility of the implementation to decide when (and if) an
operation is executed needs to be controlled. First, there are a few methods that use or produce transparent
(nonopaque) data structures. (See \S~\ref{Sec:buildMatrix}, \S~\ref{Sec:buildVector}, \S~\ref{Sec:extractTuples}, \S~\ref{Sec:extract_single_element}.) 
The application program needs to know that it is safe to modify or inspect
these data structures. (In other words, that the GraphBLAS run-time is done with them.)
Second, there are situations when one wants to make performance measurements of specific
GraphBLAS operations. Making measurements when the actual computation can be postponed or even 
eliminated can lead to unreliable and even misleading observations.

To address these situations, GraphBLAS includes a set of {\sf wait} methods. (See \S~\ref{Sec:wait}.)
When a call to {\sf wait} returns successfully, the calling program is guaranteed that all associated
pending operations have indeed completed. The set of pending operations addressed depends on the
particular variant of {\sf wait} method and is described in the appropriate section.


\chapter{Objects}
\label{Chp:Objects}

The following \emph{algebraic objects} (operators, monoids, and semirings) are presented in increasing generality.
The ``algebra generality rule'' of GraphBLAS states that a more general object can always be passed to
any method which requires a less general object. The restriction rules are explained in the respective sections of those objects.

Once algebraic objects (operators, monoids and semirings) are described, we introduce \emph{collections} (vectors, matrices and masks) that algebraic objects operate on. Finally, we introduce \emph{descriptors}, which are a simple way to do modify how algebraic objects operate on collections. More concretely, descriptors can be used (among other things) to perform multiplication with transpose of matrix without the user having to manually transpose the collection. A complete list of what descriptors are capable of can be found in the section.

\section{Operators}

A GraphBLAS \emph{binary operators} $F_b = \langle D_1, D_2, D_3, \odot \rangle$
is defined by three domains, $D_1$, $D_2$, $D_3$, and an operation
$\odot: D_1 \times D_2 \rightarrow D_3$.  For a given GraphBLAS operators
$F_b=\langle D_1, D_2, D_3,\odot \rangle$ we define $\bold{D}_1(F_b) = D_1$,
$\bold{D}_2(F_b) = D_2$, $\bold{D}_3(F_b) = D_3$, and $\bold{\bigodot}(F_b)
= \odot$.  Note that $\odot$ could be used in place of either $\oplus$ or $\otimes$.

A GraphBLAS \emph{unary operators} $F_u = \langle D_1, D_2, f\rangle$
is defined by two domains, $D_1$, $D_2$, and an operation
$f: D_1 \rightarrow D_2$.  For a given GraphBLAS operators
$F_u=\langle D_1, D_2, f \rangle$ we define $\bold{D}_1(F_u) = D_1$,
$\bold{D}_2(F_u) = D_2$, and $\bold{f}(F)
= f$.

\begin{table}
	\hrule
	\begin{center}
		\caption{Properties and recipes for building GraphBLAS algebraic objects: Unary Operator, Binary Operator, Monoid and Semiring (composed of operations Add and Times).\newline
			\hspace{\textwidth}Note 1: Output domain of Semiring Times must be same as domain of Semiring Add. This ensures 3 domains in total for Semiring rather than 4.}
		\label{Tab:Operator}
		
		\vspace{1\baselineskip}
		(a) Properties of algebraic objects.
		\vspace{1\baselineskip}
		
		\begin{tabular}{l|l|l|l}
			Object & Must Be Associative & Identity Must Exist & Number of Domains  \\
                        \hline
			Unary Operator & no & no & 2 \\
			Binary Operator & no & no & 3  \\
			Monoid & yes & yes & 1  \\
			Semiring Add & yes & yes  & 1  \\
			Semiring Times & no & no & 3  (Note 1) \\
		\end{tabular}
		
		\vspace{1\baselineskip}
		(b) Recipes for algebraic objects.
		\vspace{1\baselineskip}
		
		\begin{tabular}{l|l|l}
			Object          & Recipe                & Number of Domains  \\ 
                        \hline
			Unary Operator  & Function Pointer      & 2 \\				
			Binary Operator & Function Pointer      & 3  \\	
			Monoid          & Associative Binary Operator with Identity & 1  \\
			Semiring        & Associative Binary Operator with Identity $+$ &3 \\
                                        & Binary Operator &  \\
                        
		\end{tabular}
		
	\end{center}
	\hrule
\end{table}

\section{Monoids}

A GraphBLAS \emph{generalized monoid} (or \emph{monoid} for short) $M =
\langle D_1,\odot,0 \rangle$ is defined by a single domain $D_1$, an 
\emph{associative}\footnote{It is expected that implementations 
will utilize IEEE-754 floating point arithmetic which is not 
strictly associative.} 
operation $\odot: D_1 \times D_1 \rightarrow D_1$,
and an identity element $0 \in D_1$.  For a given GraphBLAS monoid $M=\langle
D_1,\odot,0 \rangle$ we define $\bold{D}_1(M) = D_1$, $\bold{\bigodot}(M) =
\odot$ and $\bold{0}(M) = 0$.  A GraphBLAS monoid is equivalent to 
the conventional \emph{monoid} algebraic structure.

Let $F = \langle D_1,D_1,D_1,\odot \rangle$ be a GraphBLAS binary operator
with element $0 \in D_1$.  Then $M = \langle F,0 \rangle = \langle
D_1,\odot,0 \rangle$ is a GraphBLAS monoid.

\section{Semirings}

A GraphBLAS \emph{semiring} (or \emph{semiring} for short)
$S=\langle D_1,D_2,D_3,\oplus,\otimes,0 \rangle$ is defined by
three domains $D_1$, $D_2$ and $D_3$, an \emph{associative}\footnote{It 
is expected that implementations will utilize IEEE-754 floating 
point arithmetic which is not strictly associative.} 
additive operation $\oplus : D_3 \times D_3 \rightarrow D_3$, 
a multiplicative operation $\otimes : D_1 \times D_2 \rightarrow
D_3$, and an element $0 \in D_3$.
For a given GraphBLAS semiring $S=\langle D_1,
D_2, D_3,\oplus,\otimes,0 \rangle$ we define $\bold{D}_1(S) = D_1$,
$\bold{D}_2(S) = D_2$, $\bold{D}_3(S) = D_3$, $\bold{\bigoplus}(S) =
\oplus$, $\bold{\bigotimes}(S) = \otimes$, and $\zero(S) = 0$. 

Let $F = \langle D_1,D_2,D_3,\otimes \rangle$ be an operator
and let $A = \langle D_3,\oplus,0 \rangle$ be a monoid,
then $S= \langle A,F \rangle = \langle D_1,D_2,D_3,\oplus,\otimes,0 \rangle$
is a semiring.

Note: There must be one GraphBLAS monoid in every semiring which 
serves as the semiring's additive operator and  
specifies the same domain for its inputs and output parameters. 

A UML diagram of the conceptual hierarchy of object classes in GraphBLAS
algebra (binary operators, monoids and semirings) is shown in 
Figure~\ref{Fig:AlgebraHierarchy}.

\begin{figure}[htb]
    \hrule
    \begin{center}
        \includegraphics[width=1.0\linewidth,trim=3in 2in 0.5in 2in]{Algebra_Hierarchy_v2.pdf}
    \end{center}
    \caption{Hierarchy of algebraic object classes in GraphBLAS. GraphBLAS semirings consist of a conventional monoid with one domain for the 'add' function, and a binary operator with three domains for the 'multiply' function.}
    \label{Fig:AlgebraHierarchy}
    \hrule
\end{figure}

\begin{table}
    \hrule
    \begin{center}
        \caption{Proposed operator input for relevant GraphBLAS operations. 
        The semiring add and times are shown if applicable.}
        \label{Tab:OperatorInputType}
        \begin{tabular}{l|l}
        Operation           & Operator Input  \\ \hline
        {\sf mxm, mxv, vxm} & Semiring \\ \hline
        {\sf eWiseAdd}      & Binary Operator   \\
                            & Monoid           \\
                            & Semiring          \\ \hline
        {\sf eWiseMult}     & Binary Operator   \\
                            & Monoid          \\
                            & Semiring         \\ \hline
  {\sf reduce} (to vector)  & Binary Operator            \\ 
                            & Monoid           \\ \hline
  {\sf reduce} (to scalar)  & Monoid           \\ \hline
        {\sf apply}         & Unary Operator   \\ \hline
  {\sf buildMatrix} (dups)  & Binary Operator   \\
                            & Monoid           \\ \hline
{\sf accum} param, any op   & Binary Operator  \\
                            & Monoid            \\ 
        \end{tabular}
    \end{center}
    \hrule
\end{table}

\section{Vectors}
\label{Sec:Vectors}

A vector $\vector{v} = \langle D, N, \{ (i,v_i) \} \rangle$ is defined by
a domain $D$, a size $N>0$ and a set of tuples $(i,v_i)$ where $0 \leq
i < N$ and $v_i \in D$. A particular value of $i$ can only appear at
most once in $\vector{v}$. We define $\bold{size}(\vector{v}) = N$ and
$\bold{L}(\vector{v}) = \{ (i,v_i) \}$. The set $\bold{L}(\vector{v})$ is
called the \emph{content} of vector $\vector{v}$. We also define the set
$\vector{ind(\vector{v})} = \{ i : (i,v_i) \in \bold{L}(\vector{v}) \}$
(called the \emph{structure} of $\vector{v}$), and $\bold{D}(\vector{v})
= D$. For a vector $\vector{v}$, $\vector{v}(i)$ is a reference to $v_i$
if $(i,v_i) \in \bold{L}(\vector{v})$ and is undefined otherwise.

\section{Matrices}
\label{Sec:Matrices}

A matrix $\matrix{A} = \langle D, M, N, \{ (i,j,A_{ij}) \} \rangle$ is
defined by a domain $D$, its number of rows $M>0$, its number of columns
$N>0$ and a set of tuples $(i,j,A_{ij})$ where $0 \leq i < M$, $0 \leq
j < N$, and $A_{ij} \in D$. A particular pair of values $i,j$ can only
appear at most once in $\matrix{A}$. We define $\bold{ncols}(\matrix{A})
= N$,  $\bold{nrows}(\matrix{A}) = M$ and $\bold{L}(\matrix{A}) =
\{ (i,j,A_{ij}) \}$.  The set $\bold{L}(\matrix{A})$ is called the
\emph{content} of matrix $\matrix{A}$.  We also define the sets
$\vector{indrow(\matrix{A})} = \{ i : \exists (i,j,A_{ij}) \in
\matrix{A} \}$ and $\vector{indcol(\matrix{A})} = \{ j : \exists
(i,j,A_{ij}) \in \matrix{A} \}$.  (These are the sets of nonempty
rows and columns of $\matrix{A}$, respectively.)  The \emph{structure}
of matrix $\matrix{A}$ is the set $\bold{ind}(\matrix{A}) = \{ (i,j) :
(i,j,A_{ij}) \in \bold{L}(\matrix{A}) \}$, and $\bold{D}(\matrix{A}) = D$.
For a matrix $\matrix{A}$, $\matrix{A}(i,j)$ is a reference to $A_{ij}$
if $(i,j,A_{ij}) \in \bold{L}(\matrix{A})$ and is undefined otherwise.

If $\matrix{A}$ is a matrix and $0 \leq j < N$, then $\matrix{A}(:,j)
= \langle D, M, \{(i,A_{ij}) : (i,j,A_{ij}) \in \bold{L}(\matrix{A})
\} \rangle$ is a vector called the $j$-th \emph{column}
of $\matrix{A}$. Correspondingly, if $\matrix{A}$ is a matrix and
$0 \leq i < M$, then $\matrix{A}(i,:) = \langle D, N, \{(j,A_{ij}) :
(i,j,A_{ij}) \in \bold{L}(\matrix{A}) \} \rangle$ is a vector called
the $i$-th \emph{row} of $\matrix{A}$.

Given a matrix $\matrix{A} = \langle D, M, N, \{ (i,j,A_{ij}) \} \rangle$,
its \emph{transpose} is another matrix $\matrix{A}^T = \langle D, N, M, \{
(j,i,A_{ij}) : (i,j,A_{ij}) \in \bold{L}(\matrix{A}) \} \rangle$.

\section{Masks}
\label{Sec:Masks}

A mask can be either a one- or a two-dimensional construct.  One- and
two-dimensional masks, described more formally below, are similar to
vectors and matrices, respectively, except that they have structure
(indices) but no values. Masks are used to perform fine-grain control
and optimization of GraphBLAS operations.

A one-dimensional mask $\vector{m} = \langle N, \{ i \} \rangle$ is
defined by its number of elements $N>0$ and a set $\bold{L}(\vector{m})$
of indices $\{ i \}$ where $0 \leq i < N$.  A particular value of $i$ can
only appear at most once in $\vector{m}$. We define $\bold{size}(\vector{m})
= N$.  We also define the set $\vector{ind(\vector{m})} = \{ i : i \in
\bold{L}(\vector{m}) \}$. Note that for one-dimensional masks, 
$\bold{ind}(\matrix{m})$ and  $\bold{L}(\matrix{m})$ define the same set. 
\scott{If this last statement is accurate then we should ditch the L notation in 
later descriptions.}

A two-dimensional mask $\matrix{M} = \langle M, N, \{ (i,j) \}
\rangle$, is defined by its number of rows $M>0$, its number of
columns $N>0$ and a set $\bold{L}(\matrix{M})$ of tuples $(i,j)$
where $0 \leq i < M$, $0 \leq j < N$.   A particular pair of values
$i,j$ can only appear at most once in $\matrix{M}$.  We define
$\bold{ncols}(\matrix{M}) = N$, and $\bold{nrows}(\matrix{M}) = M$.
We also define the sets $\vector{indrow(\matrix{M})} = \{ i : \exists
(i,j) \in \bold{L}(\matrix{M}) \}$ and $\vector{indcol(\matrix{M})}
= \{ j : \exists (i,j) \in \bold{L}(\matrix{M}) \}$.  These are
the sets of nonempty rows and columns of $\matrix{M}$, respectively.
The \emph{structure} of a two-dimensional mask $\matrix{M}$ is the set
$\bold{ind}(\matrix{M}) = \{ (i,j) : (i,j) \in \bold{L}(\matrix{M}) \}$.
Note that for two-dimensional masks, $\bold{ind}(\matrix{M})$ and 
$\bold{L}(\matrix{M})$ define the same set.
\scott{If this last statement is accurate then we should ditch the L notation in 
later descriptions including the next paragraph.}

One common operation on masks is the \emph{structural complement}.
For a one-dimensional mask $\vector{m}$ this is denoted as
$\neg\vector{m}$. For a two-dimensional masks this is denoted as
$\neg\matrix{M}$.  The structure of the complement of an one-dimensional
mask $\vector{m}$ is defined as $\bold{L}(\neg\vector{m}) = \{i : 0
\leq i < N, i \notin \bold{L}(\vector{m}) \}$.  It is the set of all
possible indices that do not appear in $\vector{m}$.  The structure
of the complement of a two-dimensional mask $\matrix{M}$ is defined as the set
$\bold{ind}(\neg\matrix{M}) = \{(i,j)$ : $0 \leq i < M$, $0 \leq j < N$,
$(i,j) \notin \bold{L}(\matrix{M}) \}$.  It is the set of all possible
indices that do not appear in $\matrix{M}$.

\section{Descriptors}

Descriptors, the last argument in all GraphBLAS methods, are used to
provide more details for the operation to be performed by those methods.
In particular, descriptors specify how the other input arguments that 
correspond to collections -- vectors, matrices and masks -- should
be processed (modified) before the main operation of a method is performed.

The descriptor is a lightweight object.  It pairs a set of flags
representing the possible modifiers with each collection argument of the 
GraphBLAS method.  For example, a descriptor may specify that a particular 
input matrix needs to be transposed or that a mask needs to be structurally 
complemented (defined in Section~\ref{Sec:Masks}) before using it in the operation.

For the purpose of constructing descriptors, the arguments of a method
that can be modified are identified by specific field names. The output parameter (typically
the first parameter in a GraphBLAS method) is indicated by the field name, 
{\sf GrB\_OUTP}.  The mask is indicated {\sf GrB\_MASK} field name. The input parameters
corresponding to the input vectors and matrices are indicated by {\sf GrB\_INP0} and
{\sf GrB\_INP1}, in the order they appear in the signature of the GraphBLAS method.


\section{Methods}
\label{Sec:Methods}

\subsection{Algebra methods}
\label{Sec:AlgebraMethods}

%-----------------------------------------------------------------------------

\subsubsection{{\sf Type\_new}: Construct a new GraphBLAS (user-defined) type}
\label{Sec:TypeNew}

Creates a new user-defined GraphBLAS type. This type can then be used to create new
operators, monoids, semirings, vectors and matrices.

\paragraph{\syntax}

\begin{verbatim}
        GrB_Info GrB_Type_new(GrB_Type  *utype,
                              size_t     sizeof(ctype));
\end{verbatim}

\paragraph{Parameters}

\begin{itemize}[leftmargin=1.1in]
    \item[{\sf utype}] ({\sf INOUT}) On successful return, contains a handle 
                                     to the newly created user-defined GraphBLAS 
                                     type object.
	\item[{\sf ctype}] ({\sf IN})    A C type that defines the new GraphBLAS 
                                     user-defined type.
\end{itemize}

\paragraph{Return Values}

\begin{itemize}[leftmargin=2.1in]
\item[{\sf GrB\_SUCCESS}]           operation completed successfully.
\item[{\sf GrB\_PANIC}]             unknown internal error.
\item[{\sf GrB\_OUT\_OF\_MEMORY}]          not enough memory available for operation.
\item[{\sf GrB\_NULL\_POINTER}]    {\sf utype} pointer is {\sf NULL}.
\end{itemize}

\paragraph{Description}

Given a C type {\sf ctype}, the {\sf Type\_new} method returns in {\sf utype} a handle to
a new GraphBLAS type that is equivalent to the C type.  Variables of this {\sf ctype} 
must be a struct, union, or fixed-size array. In particular, given two variables, 
{\tt src} and {\tt dst}, of type {\sf ctype}, the following operation must be a 
valid way to copy the contents of {\tt src} to {\tt dst}:

\begin{center}
{\tt memcpy(\&dst, \&src, sizeof({\sf ctype}))}
\end{center}

A new, user-defined type {\sf utype} should be destroyed with a call to 
{\sf GrB\_free(utype)} when no longer needed.

It is not an error to call this method more than once on the same variable;  
however, the handle to the previously created object will be overwritten. 

%-----------------------------------------------------------------------------
\subsubsection{{\sf UnaryOp\_new}: Construct a new GraphBLAS unary operator}

Initializes a new GraphBLAS unary operator with a specified user-defined 
function and its types (domains).

\paragraph{\syntax}

\begin{verbatim}
        GrB_Info GrB_UnaryOp_new(GrB_UnaryOp *unary_op,
                                 void       (*unary_func)(void*, const void*),
                                 GrB_Type     d_out,
                                 GrB_Type     d_in);
\end{verbatim}

\paragraph{Parameters}

\begin{itemize}[leftmargin=1.1in]
    \item[{\sf unary\_op}] ({\sf INOUT}) On successful return, contains a
                           handle to the newly created GraphBLAS unary operator object.
    \item[{\sf unary\_func}] ({\sf IN})  a pointer to a user-defined function that takes 
                           one input parameter of {\sf d\_in}'s type
			   and returns a value of {\sf d\_out}'s type, both passed as {\sf void} pointers.
                           Specifically the signature of the function is expected to 
                           be of the form:
          \begin{verbatim}
          void func(void *out, const void *in);
          \end{verbatim}
    \item[{\sf d\_out}] ({\sf IN})  The {\sf GrB\_Type} of the return value of the unary 
                           operator being created.  Should be one of the predefined 
                           GraphBLAS types in Table~\ref{Tab:PredefinedTypes}, or a 
                           user-defined GraphBLAS type.
    \item[{\sf d\_in}] ({\sf IN})  The {\sf GrB\_Type} of the input 
                           argument of the unary operator being created.  Should be 
                           one of the predefined GraphBLAS types in 
                           Table~\ref{Tab:PredefinedTypes}, or a user-defined GraphBLAS type.
\end{itemize}

\paragraph{Return Values}

\begin{itemize}[leftmargin=2.1in]
\item[{\sf GrB\_SUCCESS}]           operation completed successfully.
\item[{\sf GrB\_PANIC}]             unknown internal error.
\item[{\sf GrB\_OUT\_OF\_MEMORY}]          not enough memory available for operation.
\item[{\sf GrB\_UNINITIALIZED\_OBJECT}]          any {\sf GrB\_Type} parameter (for
                                    user-defined types) has not been
                                    initialized by a call to {\sf GrB\_Type\_new}.
\item[{\sf GrB\_NULL\_POINTER}]    {\sf unary\_op} or {\sf unary\_func}
                                    pointers are {\sf NULL}.

%\item[{\sf GrB\_DOMAIN\_MISMATCH}]  the types in the function pointer signature
%                                    are not compatible with the {\sf GrB\_Type}
%                                    parameters specified when user-defined types
%                                    are specified.
\end{itemize}

\paragraph{Description}

\newenvironment{code}{\tt}{}

The {\sf UnaryOp\_new} method creates a new GraphBLAS unary operator
\begin{quote}
$f_u = \langle \mathbf{D}({\sf d\_out}), \mathbf{D}({\sf d\_in}), {\sf unary\_func} \rangle$
\end{quote}
and returns a handle to it in {\sf unary\_op}.

The implementation of {\sf unary\_func} must be such that it works
even if the {\sf d\_out} and {\sf d\_in} arguments are aliased.
In other words, for all invocations of the function:
\begin{quote}
\begin{verbatim}
unary_func(out,in);
\end{verbatim}
\end{quote}
the value of {\sf out} must be the same as if the following code
was executed:

\begin{quote}
\begin{code}
    $\mathbf{D}({\sf d\_in})$ *tmp = malloc(sizeof($\mathbf{D}({\sf d\_in}$))); \\
    memcpy(tmp,in,sizeof($\mathbf{D}({\sf d\_in}$))); \\
    unary\_func(out,tmp); \\
    free(tmp);
\end{code}
\end{quote}

It is not an error to call this method more than once on the same variable;  
however, the handle to the previously created object will be overwritten. 

%-----------------------------------------------------------------------------

\subsubsection{{\sf BinaryOp\_new}: Construct a new GraphBLAS binary operator}

Initializes a new GraphBLAS binary operator with a specified user-defined 
function and its types (domains).

\paragraph{\syntax}

\begin{verbatim}
        GrB_Info GrB_BinaryOp_new(GrB_BinaryOp *binary_op,
                                  void        (*binary_func)(void*,
                                                             const void*,
                                                             const void*),
                                  GrB_Type      d_out,
                                  GrB_Type      d_in1,
                                  GrB_Type      d_in2);
\end{verbatim}

\paragraph{Parameters}

\begin{itemize}[leftmargin=1.1in]
    \item[{\sf binary\_op}] ({\sf INOUT}) On successful return, contains a 
          handle to the newly created GraphBLAS binary operator object.
    \item[{\sf binary\_func}] ({\sf IN}) A pointer to a user-defined function that 
          takes two input parameters of types {\sf d\_in1} and {\sf d\_in2} and returns a value of
		type {\sf d\_out}, all passed as {\sf void} pointers.
          Specifically the signature of the function is expected to 
          be of the form:
      \begin{verbatim}
      void func(void *out, const void *in1, const void *in2);
      \end{verbatim}
    \item[{\sf d\_out}]  ({\sf IN}) The {\sf GrB\_Type} of the return
          value of the binary operator being created. Should be one of the
          predefined GraphBLAS types in Table~\ref{Tab:PredefinedTypes}, or a 
          user-defined GraphBLAS type.
    \item[{\sf d\_in1}]  ({\sf IN}) The {\sf GrB\_Type} of the left hand 
          argument of the binary operator being created. Should be one of the
          predefined GraphBLAS types in Table~\ref{Tab:PredefinedTypes}, or a
          user-defined GraphBLAS type.
    \item[{\sf d\_in2}]  ({\sf IN}) The {\sf GrB\_Type} of the right hand 
          argument of the binary operator being created. Should be one of the
          predefined GraphBLAS types in Table~\ref{Tab:PredefinedTypes}, or a 
          user-defined GraphBLAS type.
\end{itemize}

\paragraph{Return Values}

\begin{itemize}[leftmargin=2.1in]
\item[{\sf GrB\_SUCCESS}]           operation completed successfully.
\item[{\sf GrB\_PANIC}]             unknown internal error.
\item[{\sf GrB\_OUT\_OF\_MEMORY}]          not enough memory available for operation.
\item[{\sf GrB\_UNINITIALIZED\_OBJECT}]          the {\sf GrB\_Type} (for user-defined types)
                                    has not been initialized by a call to {\sf GrB\_Type\_new}.
\item[{\sf GrB\_NULL\_POINTER}]    {\sf binary\_op} or {\sf binary\_func} pointer is {\sf NULL}.

%\item[{\sf GrB\_DOMAIN\_MISMATCH}]  the types in the function pointer signature are not   
%                                    compatible with the {\sf GrB\_Type} parameters specified.
\end{itemize}

\paragraph{Description}

The {\sf BinaryOp\_new} methods creates a new GraphBLAS binary operator
\begin{quote}
$f_b = \langle \mathbf{D}({\sf d\_out}), \mathbf{D}({\sf d\_in1}), \mathbf{D}({\sf d\_in2}), {\sf binary\_func} \rangle$
\end{quote}
and returns a handle to it in {\sf binary\_op}.

The implementation of {\sf binary\_func} must be such that it works
even if any of the {\sf d\_out}, {\sf d\_in1}, and {\sf d\_in2} arguments are aliased to each other.
In other words, for all invocations of the function:
\begin{quote}
\begin{verbatim}
binary_func(out,in1,in2);
\end{verbatim}
\end{quote}
the value of {\sf out} must be the same as if the following code
was executed:

\begin{quote}
\begin{code}
    $\mathbf{D}({\sf d\_in1})$ *tmp1 = malloc(sizeof($\mathbf{D}({\sf d\_in1}$))); \\
    $\mathbf{D}({\sf d\_in2})$ *tmp2 = malloc(sizeof($\mathbf{D}({\sf d\_in2}$))); \\
    memcpy(tmp1,in1,sizeof($\mathbf{D}({\sf d\_in1}$))); \\
    memcpy(tmp2,in2,sizeof($\mathbf{D}({\sf d\_in2}$))); \\
    binary\_func(out,tmp1,tmp2); \\
    free(tmp2); \\
    free(tmp1);
\end{code}
\end{quote}

It is not an error to call this method more than once on the same variable;  
however, the handle to the previously created object will be overwritten. 

%-----------------------------------------------------------------------------

\subsubsection{{\sf Monoid\_new}: Construct a new GraphBLAS monoid}

Creates a new monoid with specified binary operator and identity value.

\paragraph{\syntax}

\begin{verbatim}
        GrB_Info GrB_Monoid_new(GrB_Monoid    *monoid,
                                GrB_BinaryOp   binary_op,
                                <type>         identity);
\end{verbatim}

\paragraph{Parameters}

\begin{itemize}[leftmargin=1.1in]
    \item[{\sf monoid}] ({\sf INOUT}) On successful return, contains a
                         handle to the newly created GraphBLAS monoid object.
    \item[{\sf binary\_op}] ({\sf IN}) An existing GraphBLAS associative binary 
                         operator whose input and output types are the same.
    \item[{\sf identity}]  ({\sf IN}) The value of the identity element of the 
                         monoid. Must be the same type as the type used by the
                         {\sf binary\_op} operator.
\end{itemize}

\paragraph{Return Values}

\begin{itemize}[leftmargin=2.1in]
\item[{\sf GrB\_SUCCESS}]           operation completed successfully.
\item[{\sf GrB\_PANIC}]             unknown internal error.
\item[{\sf GrB\_OUT\_OF\_MEMORY}]   not enough memory available for operation.
\item[{\sf GrB\_UNINITIALIZED\_OBJECT}]  the {\sf GrB\_BinaryOp} (for user-defined operators) has not been
                                    initialized by a call to {\sf GrB\_BinaryOp\_new}.
\item[{\sf GrB\_NULL\_POINTER}]     {\sf monoid} pointer is {\sf NULL}.
\item[{\sf GrB\_DOMAIN\_MISMATCH}]  all three argument types of the binary operator and
                                    the type of the identity value are not the same.
\end{itemize}

\paragraph{Description}

The {\sf Monoid\_new} method creates a new monoid $M = \langle \mathbf{D}({\sf binary\_op}), {\sf binary\_op}, 
{\sf identity} \rangle$ and returns a handle to it in {\sf monoid}.

If {\sf binary\_op} is not associative, the results of GraphBLAS operations that
require associativity of this monoid will be undefined.

It is not an error to call this method more than once on the same variable;  
however, the handle to the previously created object will be overwritten. 

%-----------------------------------------------------------------------------
\subsubsection{{\sf Semiring\_new}: Construct a new GraphBLAS semiring}

Creates a new semiring with specified domain, operators, and elements.

\paragraph{\syntax}

\begin{verbatim}
        GrB_Info GrB_Semiring_new(GrB_Semiring  *semiring,
                                  GrB_Monoid     add_op,
                                  GrB_BinaryOp   mul_op);
\end{verbatim}

\paragraph{Parameters}

\begin{itemize}[leftmargin=1.1in]
    \item[{\sf semiring}] ({\sf INOUT}) On successful return, contains a 
    handle to the newly created GraphBLAS semiring.
    \item[{\sf add\_op}]  ({\sf IN}) An existing GraphBLAS commutative monoid that 
    specifies the addition operator and its identity.
    \item[{\sf mul\_op}]  ({\sf IN}) An existing GraphBLAS binary operator that 
    specifies the semiring's multiplication operator. In addition, {\sf mul\_op}'s
    output domain, $\bDout({\sf mul\_op})$, must be the same as the {\sf add\_op}'s
    domain $\mathbf{D}(\mbox{\sf add\_op})$.
\end{itemize}


\paragraph{Return Values}

\begin{itemize}[leftmargin=2.1in]
\item[{\sf GrB\_SUCCESS}]           operation completed successfully.
\item[{\sf GrB\_PANIC}]             unknown internal error.
\item[{\sf GrB\_OUT\_OF\_MEMORY}]   not enough memory available for this method to complete.
\item[{\sf GrB\_UNINITIALIZED\_OBJECT}]   the {\sf add\_op} (for user-define monoids) object has not been
                                    initialized with a call to {\sf GrB\_Monoid\_new}
				    or the {\sf mul\_op} (for user-defined operators) object has not been
                                    not been initialized by a call to 
                                    {\sf GrB\_BinaryOp\_new}.
\item[{\sf GrB\_NULL\_POINTER}]    {\sf semiring} pointer is {\sf NULL}.
\item[{\sf GrB\_DOMAIN\_MISMATCH}]  the output domain of {\sf mul\_op} does not
                                    match the domain of the {\sf add\_op} monoid.
\end{itemize}

\paragraph{Description}

The {\sf Semiring\_new} method creates a new semiring:
\begin{quote}
$S = \langle \bDout({\sf mul\_op}), 
\bDin1({\sf mul\_op}), \bDin2({\sf mul\_op}), {\sf add\_op}, 
{\sf mul\_op}, \mathbf{0}({\sf add\_op})\rangle$
\end{quote}
and returns a handle to it in 
{\sf semiring}.  Note that $\bDout({\sf mul\_op})$ must be the same as 
$\mathbf{D}({\sf add\_op})$.

If {\sf add\_op} is not commutative, then GraphBLAS operations using this semiring
will be undefined.

It is not an error to call this method more than once on the same variable;  
however, the handle to the previously created object will be overwritten. 

%-----------------------------------------------------------------------------

\subsubsection{{\sf IndexUnaryOp\_new}: Construct a new GraphBLAS index unary operator}

Initializes a new GraphBLAS index unary operator with a specified user-defined 
function and its types (domains).

\paragraph{\syntax}

\begin{verbatim}
    GrB_Info GrB_IndexUnaryOp_new(GrB_IndexUnaryOp   *index_unary_op,
                                  void (*index_unary_func)(void*,
                                                           const void*,
                                                           GrB_Index,
                                                           GrB_Index,
                                                           const void*),
                                  GrB_Type            d_out,
                                  GrB_Type            d_in1,
                                  GrB_Type            d_in2);
\end{verbatim}

\paragraph{Parameters}

\begin{itemize}[leftmargin=1.2in]
    \item[{\sf index\_unary\_op}] ({\sf INOUT}) On successful return, contains a 
          handle to the newly created GraphBLAS index unary operator object.
    \item[{\sf index\_unary\_func}] ({\sf IN}) A pointer to a user-defined 
          function that takes input parameters of types {\sf d\_in1}, 
          {\sf GrB\_Index}, {\sf GrB\_Index} and {\sf d\_in2}
          and returns a value of type {\sf d\_out}.  Except for the {\sf GrB\_Index}
          parameters, all are passed as {\sf void} pointers.
          Specifically the signature of the function is expected to 
          be of the form:
      \begin{verbatim}
      void func(void       *out,
                const void *in1,
                GrB_Index   row_index,
                GrB_Index   col_index, 
                const void *in2);
      \end{verbatim}
    \item[{\sf d\_out}]  ({\sf IN}) The {\sf GrB\_Type} of the return
          value of the index unary operator being created. Should be one of the
          predefined GraphBLAS types in Table~\ref{Tab:PredefinedTypes}, or a 
          user-defined GraphBLAS type.
    \item[{\sf d\_in1}]  ({\sf IN}) The {\sf GrB\_Type} of the first input 
          argument of the index unary operator being created and corresponds to
          the stored values of the {\sf GrB\_Vector} or {\sf GrB\_Matrix} being
          operated on. Should be one of the predefined GraphBLAS types in
          Table~\ref{Tab:PredefinedTypes}, or a user-defined GraphBLAS type.
    \item[{\sf d\_in2}]  ({\sf IN}) The {\sf GrB\_Type} of the last input
          argument of the index unary operator being created and corresponds to
          a scalar provided by the GraphBLAS operation that uses this operator.
          Should be one of the predefined GraphBLAS types in 
          Table~\ref{Tab:PredefinedTypes}, or a user-defined GraphBLAS type.
\end{itemize}

\paragraph{Return Values}

\begin{itemize}[leftmargin=2.1in]
\item[{\sf GrB\_SUCCESS}]           operation completed successfully.
\item[{\sf GrB\_PANIC}]             unknown internal error.
\item[{\sf GrB\_OUT\_OF\_MEMORY}]          not enough memory available for operation.
\item[{\sf GrB\_UNINITIALIZED\_OBJECT}]          the {\sf GrB\_Type} (for user-defined types)
                                    has not been initialized by a call to {\sf GrB\_Type\_new}.
\item[{\sf GrB\_NULL\_POINTER}]    {\sf index\_unary\_op} or {\sf index\_unary\_func} pointer is {\sf NULL}.

%\jose{Domain mistmatch not possible.}
%\item[{\sf GrB\_DOMAIN\_MISMATCH}]  the types in the function pointer signature are not   
%                                    compatible with the {\sf GrB\_Type} parameters specified.
\end{itemize}

\paragraph{Description}

The {\sf IndexUnaryOp\_new} methods creates a new GraphBLAS index unary operator
\begin{quote}
$f_{i} = \langle \mathbf{D}({\sf d\_out}), \mathbf{D}({\sf d\_in1}), \mathbf{D}({\sf GrB\_Index}), \mathbf{D}({\sf GrB\_Index}), \mathbf{D}({\sf d\_in2}), {\sf index\_unary\_func} \rangle$
\end{quote}
and returns a handle to it in {\sf index\_unary\_op}.

The implementation of {\sf index\_unary\_func} must be such that it works
even if any of the {\sf d\_out}, {\sf d\_in1}, and {\sf d\_in2} arguments are aliased to each other.
In other words, for all invocations of the function:
\begin{quote}
\begin{verbatim}
index_unary_func(out,in1,row_index,col_index,n,in2);
\end{verbatim}
\end{quote}
the value of {\sf out} must be the same as if the following code
was executed (shown here for matrices):

\begin{quote}
\begin{code}
    GrB\_Index row\_index = ...;\\
    GrB\_Index col\_index = ...;\\
    $\mathbf{D}({\sf d\_in1})$ *tmp1 = malloc(sizeof($\mathbf{D}({\sf d\_in1}$))); \\
    $\mathbf{D}({\sf d\_in2})$ *tmp2 = malloc(sizeof($\mathbf{D}({\sf d\_in2}$))); \\
    memcpy(tmp1,in1,sizeof($\mathbf{D}({\sf d\_in1}$))); \\
    memcpy(tmp2,in2,sizeof($\mathbf{D}({\sf d\_in2}$))); \\
    index\_unary\_func(out,tmp1,row\_index,col\_index,tmp2); \\
    free(tmp2); \\
    free(tmp1);
\end{code}
\end{quote}

It is not an error to call this method more than once on the same variable;  
however, the handle to the previously created object will be overwritten. 

\subsection{Vector Methods}

%All methods can be defined in use programs by including the {\tt GraphBLAS.h} header file.

%\scott{As with all *\_new operations, what happens when I new an object a second time?}

%-----------------------------------------------------------------------------
\subsubsection{{\sf Vector\_new}: Create new vector}

Creates a new vector with specified domain and size.

\paragraph{\syntax}

\begin{verbatim}
        GrB_Info GrB_Vector_new(GrB_Vector *v,
                                GrB_Type    d,
                                GrB_Index   nsize);
\end{verbatim}

\paragraph{Parameters}

\begin{itemize}[leftmargin=1.1in]
    \item[{\sf v}] ({\sf INOUT}) On successful return, contains the identifier 
                                 of the newly created GraphBLAS vector.
    \item[{\sf d}] ({\sf IN})    The type corresponding to the domain of the 
                                 vector being created.  Can be one of the 
                                 predefined GraphBLAS types in 
                                 Table~\ref{Tab:PredefinedTypes}, or an existing 
                                 user-defined GraphBLAS type.
    \item[{\sf nsize}] ({\sf IN}) The size of the vector being created.
\end{itemize}

\paragraph{Return Values}

\begin{itemize}[leftmargin=2.1in]
\item[{\sf GrB\_SUCCESS}]    operation completed successfully.
\item[{\sf GrB\_PANIC}]      unknown internal error.
\item[{\sf GrB\_OUTOFMEM}]   not enough memory available for operation.
\item[{\sf GrB\_NOOBJECT}]   the {\sf GrB\_Type} parameter (for user-defined
                             types) has not been initialized by a
                             call to {\sf new}.
\item[{\sf GrB\_INVALID\_VALUE}]    {\sf v} pointer is {\sf NULL}.
\item[{\sf GrB\_INVALID\_VALUE}]    {\sf nsize} is zero.
\item[{\sf GrB\_INVALID\_VALUE}]    {\sf v} object is already initialized.
\end{itemize}

\paragraph{Description}

Creates a new vector $\vector{v}$ of domain $\bold{D}({\sf d})$, size {\sf nsize}, 
and empty $\bold{L}(\vector{v})$. It returns in {\sf v} this vector $\vector{v}$.

%-----------------------------------------------------------------------------
\subsubsection{{\sf Vector\_clear}: Clear a vector}

Removes all the elements from a vector.

\paragraph{\syntax}

\begin{verbatim}
        GrB_Info GrB_Vector_clear(GrB_Vector *v);
\end{verbatim}

\paragraph{Parameters}

\begin{itemize}[leftmargin=1.1in]
    \item[{\sf v}] ({\sf IN}) An existing GraphBLAS vector to clear.
\end{itemize}

\paragraph{Return Values}

\begin{itemize}[leftmargin=2.1in]
\item[{\sf GrB\_SUCCESS}]   operation completed successfully.
\item[{\sf GrB\_PANIC}]     unknown internal error.
\item[{\sf GrB\_NOOBJECT}]  the vector has not been initialized with a call to new.
\item[{\sf GrB\_INVALID\_VALUE}]    {\sf v} pointer is {\sf NULL}.
\end{itemize}

\paragraph{Description}

Removes all tuples from an existing vector.

%-----------------------------------------------------------------------------
\subsubsection{{\sf Vector\_size}: Size of a vector}

Retrieve the size of a vector.

\paragraph{\syntax}

\begin{verbatim}
        GrB_Info GrB_Vector_size(GrB_Index        *nsize,
                                 const GrB_Vector  v);
\end{verbatim}

\paragraph{Parameters}

\begin{itemize}[leftmargin=1.1in]
    \item[{\sf nsize}] ({\sf OUT}) On successful return, is set to the size ($N$) 
                                   of the vector.
    \item[{\sf v}]     ({\sf IN})  An existing GraphBLAS vector being queried.
\end{itemize}

\paragraph{Return Values}

\begin{itemize}[leftmargin=2.1in]
\item[{\sf GrB\_SUCCESS}]   operation completed successfully.
\item[{\sf GrB\_PANIC}]     unknown internal error.
\item[{\sf GrB\_NOOBJECT}]  vector has not been initialized with a call to {\sf new}.
\item[{\sf GrB\_INVALID\_VALUE}]    {\sf nsize} pointer is {\sf NULL}.
\end{itemize}

\paragraph{Description}

Return in {\sf nsize} the size (parameter $N$ in Section~\ref{Sec:Vectors}) in vector $\vector{v}$.

%-----------------------------------------------------------------------------
\subsubsection{{\sf Vector\_nvals}: Number of stored elements in a vector}

Retrieve the number of stored elements (tuples) in a vector.

\paragraph{\syntax}

\begin{verbatim}
        GrB_Info GrB_Vector_nvals(GrB_Index        *nvals,
                                  const GrB_Vector  v);
\end{verbatim}

\paragraph{Parameters}

\begin{itemize}[leftmargin=1.1in]
    \item[{\sf nvals}] ({\sf OUT}) On successful return, is set to the number of 
                                   stored elements (tuples) in the vector.
    \item[{\sf v}]     ({\sf IN})  An existing GraphBLAS vector being queried.
\end{itemize}


\paragraph{Return Values}

\begin{itemize}[leftmargin=2.1in]
\item[{\sf GrB\_SUCCESS}]   operation completed successfully.
\item[{\sf GrB\_PANIC}]     unknown internal error.
\item[{\sf GrB\_NOOBJECT}]  vector has not been initialized with a call to {\sf new}.
\item[{\sf GrB\_INVALID\_VALUE}]    {\sf nvals} pointer is {\sf NULL}.
\end{itemize}

\paragraph{Description}

Return in {\sf nvals} the number of stored elements (the size of $\bold{L}(\vector{v})$
in Section~\ref{Sec:Vectors}) in vector {\sf v}.

%-----------------------------------------------------------------------------

\subsubsection{{\sf Vector\_build}: Store elements from tuples into a vector}
\label{Sec:Vector_build}

\paragraph{\syntax}

\begin{verbatim}
        GrB_Info GrB_Vector_build(GrB_Vector            *w,
                                  const GrB_Index       *indices,
                                  const <type>          *values,
                                  GrB_Index              nvals,
                                  const GrB_BinaryOp     dup);
\end{verbatim}

\paragraph{Parameters}

\begin{itemize}[leftmargin=1.1in]
    \item[{\sf w}]       ({\sf INOUT}) An existing Vector object to store the result.
    \item[{\sf indices}] ({\sf IN}) Pointer to an array of indices. 
    \item[{\sf values}]  ({\sf IN}) Pointer to an array of scalars of a type that
                                     is compatible with the domain of vector {\sf w}.
    \item[{\sf nvals}]   ({\sf IN}) The number of entries contained in each array (the same for \arg{indices} and \arg{values}.
    \item[{\sf dup}]     ({\sf IN}) A binary function to apply when duplicate values for
                         the same location are present in the input arrays.
\end{itemize}

\paragraph{Return Values}

\begin{itemize}[leftmargin=2.1in]
\item[{\sf GrB\_SUCCESS}]     operation completed successfully.
\item[{\sf GrB\_PANIC}]       unknown internal error.
\item[{\sf GrB\_OUTOFMEM}]    not enough memory available for operation.
\item[{\sf GrB\_NOOBJECT}]    one or more GraphBLAS objects -- {\sf w} or {\sf dup} -- 
                            have not been initialized by a call to {\sf new}.
\item[{\sf GrB\_INVALID\_VALUE}]  {\sf indices} or {\sf values} pointer is {\sf NULL}.
\item[{\sf GrB\_INDEX\_OUTOFBOUNDS}]
                            A value in {\sf indices} is outside the allowed range for \arg{w}.
\item[\sf GrB\_DOMAIN\_MISMATCH]  
                       Mismatch between value type and vector domain
                       (only when user-defined types are used).
\item[\sf GrB\_DIMENSION\_MISMATCH]  
                       mismatch between dimensions of vector and mask. 
\end{itemize}

\paragraph{Description}

For $i = 0,\ldots,\arg{nvals}-1$, do the following:
\begin{enumerate}
    \item If $\arg{indices}[i] \notin \bold{i}(\arg{w})$, then $\bold{L}(\arg{w}) \leftarrow \bold{L}(\arg{w}) \cup (\arg{indices}[i], \arg{values}[i])$.
    \item If $\arg{indices}[i] \in \bold{i}(\arg{w})$, then replace the tuple $(\arg{indices}[i], v_{\arg{indices}[i]}) \in \bold{L}(\arg{u})$ with the tuple \\ $(\arg{indices}[i], \arg{dup}(v_{\arg{indices}[i]},\arg{values}[i]))$.
\end{enumerate}

\scott{questionable {\sf dup} behaviour.}

\scott{Is the following statement still accurate?}

After a call to {\sf GrB\_Vector\_build}, the program should perform a 
{\sf GrB\_wait} on vector \arg{u} before
modifying or deleting arrays \arg{indices} and \arg{values}.

%-----------------------------------------------------------------------------

\subsubsection{{\sf Vector\_extractTuples}: Extract tuples from a vector}
\label{Sec:Vector_extractTuples}

Extract the contents of a GraphBLAS vector into non-opaque data structures.

\paragraph{\syntax}

\begin{verbatim}
        GrB_Info GrB_Vector_extractTuples(GrB_Index            *indices,
                                          <type>               *values, 
                                          const GrB_Vector      v,
                                          char                 *err);

\end{verbatim}

\begin{itemize}[leftmargin=1.1in]
    \item[{\sf indices}] ({\sf OUT}) Pointer to an array of indices that is sufficient to
                        hold all of the stored values' indices (no checking is performed).
    \item[{\sf values}] ({\sf OUT}) Pointer to an array of scalars of a type that is sufficient to
                        hold all of the stored values (no checking is performed) whose
                        type is compatible with $\bold{D}(\vector{v})$.
    \item[{\sf v}]      ({\sf IN})  An existing GraphBLAS vector.
    \item[{\sf err}]     A null terminated string containing additional error information.
\end{itemize}

\paragraph{Return Values}

\begin{itemize}[leftmargin=2.1in]
\item[{\sf GrB\_SUCCESS}]     operation completed successfully.
\item[{\sf GrB\_PANIC}]       unknown internal error.
\item[{\sf GrB\_NOOBJECT}]    The GraphBLAS vector, {\sf v}, has not been
                       initialized by a call to {\sf new}.
\item[{\sf GrB\_INVALID\_VALUE}]  {\sf indices} or {\sf values} pointer is {\sf NULL}.
\item[\sf GrB\_DIMENSION\_MISMATCH]  
                       Mismatch between dimensions of vector and mask. 
\item[\sf GrB\_DOMAIN\_MISMATCH]  
                       Mismatch between value type and vector domain (only when 
                       user-defined types are used).
\end{itemize}


\paragraph{Description}
\scott{DESCRIPTION MISSING}

\scott{Does allocation occur within function -- then we need OUTOFMEMORY error.
The alternative is that the user is expected to call *\_nvals() function on the
vector to determine how much memory to allocate and pass pointers to the pre 
allocated memory.  If this is the case then INVALID\_VALUE can be returned (as 
shown above) if NULL pointers are passed.  I prefer the latter.  Note, if we go
with the latter, then *\_nvals() should also be able to take a mask so that the
correct amount of memory can be allocated for this call when the same mask is used.}

%==============================================================================================
\subsection{Matrix Methods}

%-----------------------------------------------------------------------------
\subsubsection{{\sf Matrix\_new}: Create new matrix}

Creates a new matrix with specified domain and dimensions.

\paragraph{\syntax}

\begin{verbatim}
        GrB_Info GrB_Matrix_new(GrB_Matrix *A,
                                GrB_Type    d,
                                GrB_Index   nrows,
                                GrB_Index   ncols);
\end{verbatim}

\paragraph{Parameters}

\begin{itemize}[leftmargin=1.1in]
    \item[{\sf A}] ({\sf INOUT}) On successful return, contains the identifier of 
                                 the newly created GraphBLAS matrix.
    \item[{\sf d}] ({\sf IN})    The type corresponding to the domain of the matrix 
                                 being created. Can be one of the predefined
                                 GraphBLAS types in Table~\ref{Tab:PredefinedTypes}, 
                                 or an existing user-defined GraphBLAS type.
    \item[{\sf nrows}] ({\sf IN}) The number of rows of the matrix being created.
    \item[{\sf ncols}] ({\sf IN}) The number of columns of the matrix being created.
\end{itemize}


\paragraph{Return Values}

\begin{itemize}[leftmargin=2.1in]
\item[{\sf GrB\_SUCCESS}]   operation completed successfully.
\item[{\sf GrB\_PANIC}]     unknown internal error.
\item[{\sf GrB\_OUTOFMEM}]  not enough memory available for operation.
\item[{\sf GrB\_NOOBJECT}]   the {\sf GrB\_Type} parameter (for user-defined
                             types) has not been initialized by a
                             call to {\sf new}.
\item[{\sf GrB\_INVALID\_VALUE}]    {\sf nrows} or {\sf ncols} is zero.
\item[{\sf GrB\_INVALID\_VALUE}]    {\sf A} pointer is {\sf NULL}.
\item[{\sf GrB\_INVALID\_VALUE}]    {\sf A} object is already initialized.
\end{itemize}

\paragraph{Description}

Creates a new matrix $\matrix{A}$ of domain $\bold{D}({\sf d})$, size {\sf nrows $\times$ ncols}, and
empty $\bold{L}(\matrix{A})$. It returns, in {\sf A}, this matrix $\matrix{A}$.

%-----------------------------------------------------------------------------
\subsubsection{{\sf Matrix\_clear}: Clear a matrix}

Removes all elements from a matrix.

\paragraph{\syntax}

\begin{verbatim}
        GrB_Info GrB_Matrix_clear(GrB_Matrix *A);
\end{verbatim}

\paragraph{Parameters}

\begin{itemize}[leftmargin=1.1in]
    \item[{\sf A}] ({\sf IN}) An exising GraphBLAS matrix to clear.
\end{itemize}

\paragraph{Return Values}

\begin{itemize}[leftmargin=2.1in]
\item[{\sf GrB\_SUCCESS}]   operation completed successfully.
\item[{\sf GrB\_PANIC}]     unknown internal error.
\item[{\sf GrB\_NOOBJECT}]  the matrix has not been initialized with a call to new.
\item[{\sf GrB\_INVALID\_VALUE}]    {\sf A} pointer is {\sf NULL}.
\end{itemize}

\paragraph{Description}

Removes all elements (tuples) from an existing matrix.

%-----------------------------------------------------------------------------
\subsubsection{{\sf Matrix\_nrows}: Number of rows in a matrix}

Retrieve the number of rows in a matrix.

\paragraph{\syntax}

\begin{verbatim}
        GrB_Info GrB_Matrix_nrows(GrB_Index        *nrows,
                                  const GrB_Matrix  A);
\end{verbatim}

\paragraph{Parameters}

\begin{itemize}[leftmargin=1.1in]
    \item[{\sf nrows}] ({\sf OUT}) On successful return, contains the number of rows in the matrix.
    \item[{\sf A}] ({\sf IN}) An existing GraphBLAS matrix being queried.
\end{itemize}


\paragraph{Return Values}

\begin{itemize}[leftmargin=2.1in]
\item[{\sf GrB\_SUCCESS}]   operation completed successfully.
\item[{\sf GrB\_PANIC}]     unknown internal error.
\item[{\sf GrB\_NOOBJECT}]  matrix has not been initialized with a call to {\sf new}.
\item[{\sf GrB\_INVALID\_VALUE}]    {\sf nrows} pointer is {\sf NULL}.
\end{itemize}

\paragraph{Description}

Return in {\sf nrows} the number of rows (parameter $M$ in Section~\ref{Sec:Matrices}) in matrix {\sf A}.

%-----------------------------------------------------------------------------
\subsubsection{{\sf Matrix\_ncols}: Number of columns in a matrix}

Retrieve the number of columns in a matrix.

\paragraph{\syntax}

\begin{verbatim}
        GrB_Info GrB_Matrix_ncols(GrB_Index        *ncols,
                                  const GrB_Matrix  A);
\end{verbatim}

\paragraph{Parameters}

\begin{itemize}[leftmargin=1.1in]
    \item[{\sf ncols}] ({\sf OUT}) On successful return, contains the number of columns in the matrix.
    \item[{\sf A}] ({\sf IN}) An existing GraphBLAS matrix being queried.
\end{itemize}

\paragraph{Return Values}

\begin{itemize}[leftmargin=2.1in]
\item[{\sf GrB\_SUCCESS}]   operation completed successfully.
\item[{\sf GrB\_PANIC}]     unknown internal error.
\item[{\sf GrB\_NOOBJECT}]  matrix has not been initialized with a call to {\sf new}.
\item[{\sf GrB\_INVALID\_VALUE}]    {\sf ncols} pointer is {\sf NULL}.
\end{itemize}

\paragraph{Description}

Return in {\sf ncols} the number of columns (parameter $N$ in Section~\ref{Sec:Matrices}) in matrix {\sf A}.

%-----------------------------------------------------------------------------
\subsubsection{{\sf Matrix\_nvals}: Number of stored elements in a matrix}

Retrieve the number of stored elements (tuples) in a matrix.

\paragraph{\syntax}

\begin{verbatim}
        GrB_Info GrB_Matrix_nvals(GrB_Index        *nvals,
                                  const GrB_Matrix  A);
\end{verbatim}

\paragraph{Parameters}

\begin{itemize}[leftmargin=1.1in]
    \item[{\sf nvals}] ({\sf OUT}) On successful return, contains the number of 
    stored elements (tuples) in the matrix.
    \item[{\sf A}] ({\sf IN}) An existing GraphBLAS matrix being queried.
\end{itemize}

\paragraph{Return Values}

\begin{itemize}[leftmargin=2.1in]
\item[{\sf GrB\_SUCCESS}]   operation completed successfully.
\item[{\sf GrB\_PANIC}]     unknown internal error.
\item[{\sf GrB\_NOOBJECT}]  matrix has not been initialized with a call to {\sf new}.
\item[{\sf GrB\_INVALID\_VALUE}]    {\sf nvals} pointer is {\sf NULL}.
\end{itemize}

\paragraph{Description}

Return in {\sf nvals} the number of tuples (the size of $\bold{L}(\matrix{A})$
in Section~\ref{Sec:Matrices}) stored in matrix {\sf A}.

%-----------------------------------------------------------------------------

\subsubsection{{\sf Matrix\_build}: Store elements from tuples into a matrix}
\label{Sec:Matrix_build}

\paragraph{\syntax}

% AYDIN: Avoid page break due to preceding table
\begin{Verbatim}[samepage=true]    
        GrB_Info GrB_Matrix_build(GrB_Matrix            *C,
                                  const GrB_Index       *rowIDs,
                                  const GrB_Index       *colIDs, 
                                  const <type>          *values,
                                  GrB_Index              nvals,
                                  const GrB_BinaryOp     dup);
\end{Verbatim}

\paragraph{Parameters}

\begin{itemize}[leftmargin=1.1in]
    \item[{\sf C}]      ({\sf INOUT}) An existing Matrix object to store the result.
    \item[{\sf rowIDs}] ({\sf IN}) Pointer to an array of row indices. 
    \item[{\sf colIDs}] ({\sf IN}) Pointer to an array of column indices. 
    \item[{\sf values}] ({\sf IN}) Pointer to an array of scalars of a type that
                                   is compatible with the domain of matrix, {\sf C}.
    \item[{\sf nvals}]  ({\sf IN}) The number of values contained in each array.
    \item[{\sf dup}]    ({\sf IN}) A binary function to apply when duplicate values 
                        for the same location are present in the input arrays. \scott{Is {\sf GrB\_NULL} allowed?}
\end{itemize}

\paragraph{Return Values}

\begin{itemize}[leftmargin=2.1in]
\item[{\sf GrB\_SUCCESS}]      operation completed successfully.
\item[{\sf GrB\_PANIC}]        unknown internal error.
\item[{\sf GrB\_OUTOFMEM}]     not enough memory available for operation.
\item[{\sf GrB\_NOOBJECT}]     one or more GraphBLAS objects -- {\sf C} or  {\sf dup} -- have not been initialized by a call to {\sf new}.
\item[{\sf GrB\_INVALID\_VALUE}]  {\sf rowIDs}, {\sf colIDs}, or {\sf values} pointer is {\sf NULL}.
\item[{\sf GrB\_INDEX\_OUTOFBOUNDS}]
        A value in i references a nonexistent row in C, or
        the value in j references a nonexistent column in C.
\item[\sf GrB\_DOMAIN\_MISMATCH]  
       mismatch between value type and matrix domain (only when user-defined types are used).
\item[\sf GrB\_DIMENSION\_MISMATCH]  
                       mismatch between dimensions of matrix and mask. 
\end{itemize}


\paragraph{Description}
Each tuple $\{ {\sf rowIDs[i]}, {\sf colIDs[i]}, {\sf values[i]}\}$ is a contribution to the output in the form of 

$$\matrix{C}[{\sf rowIDs[i]}, {\sf colIDs[i]}] = {\sf values[i]}.$$

If multiple values for the same location are present in the input arrays, the 
{\sf dup} binary operand is used to reduce them before assignment n into {\sf C}.
 
{\sf rowIDs}, {\sf colIDs}, and {\sf values} should be of the same length. 

%-----------------------------------------------------------------------------

\subsubsection{{\sf Matrix\_extractTuples}: Extract tuples from a matrix}
\label{Sec:Matrix_extractTuples}

Extract the contents of a GraphBLAS matrix into non-opaque data structures.

\paragraph{\syntax}

\begin{verbatim}
        GrB_Info GrB_Matrix_extractTuples(GrB_Index            *rowIDs,
                                          GrB_Index            *colIDs,
                                          <type>               *values, 
                                          const GrB_Matrix      A,
                                          char                 *err);
\end{verbatim}

\paragraph{Parameters}

\begin{itemize}[leftmargin=1.1in]
    \item[{\sf rowIDs}] ({\sf OUT}) Pointer to an array of row indices that is sufficient to
                        hold all of the row indices (no checking is performed).
    \item[{\sf colIDs}] ({\sf OUT}) Pointer to an array of column indices that is sufficient to
                        hold all of the column indices (no checking is performed). 
    \item[{\sf values}] ({\sf OUT}) Pointer to an array of scalars of a type that is sufficient to
                        hold all of the stored values (no checking is performed) whose
                        type is compatible with $\bold{D}(\matrix{A})$.
    \item[{\sf A}]      ({\sf IN}) An existing GraphBLAS matrix.
    \item[{\sf err}]     A null terminated string containing additional error information.
\end{itemize}

\paragraph{Return Values}

\begin{itemize}[leftmargin=2.1in]
\item[{\sf GrB\_SUCCESS}]     operation completed successfully.
\item[{\sf GrB\_PANIC}]       unknown internal error.
\item[{\sf GrB\_NOOBJECT}]    The GraphBLAS matrix, {\sf A}, has not been 
                       initialized by a call to {\sf new}.
\item[{\sf GrB\_INVALID\_VALUE}]  {\sf rowIDs}, {\sf colIDs} or {\sf values} pointer is {\sf NULL}.
\item[\sf GrB\_DIMENSION\_MISMATCH]  
                       mismatch between dimensions of matrix and mask. 
\item[\sf GrB\_DOMAIN\_MISMATCH]  
                       mismatch between value type and matrix domain (only when 
                       user-defined types are used).
\end{itemize}

\paragraph{Description}
\scott{DESCRIPTION MISSING}

\scott{Does allocation of non-opaque arrays occur within function -- then we need OUTOFMEMORY error.
The alternative is that the user is expected to call *\_nvals() function on the
matrix to determine how much memory to allocate and pass pointers to the pre 
allocated memory.  If this is the case then INVALID\_VALUE can be returned (as 
shown above) if NULL pointers are passed.  I prefer the latter.  Note, if we go
with the latter, then *\_nvals() should also be able to take a mask so that the
correct amount of memory can be allocated for this call when the same mask is used.}

\subsection{Descriptor Methods}

\subsubsection{Create new descriptor ({\sf Descriptor\_new})}

Creates a new (empty) descriptor.

\paragraph{C99 Syntax}

\begin{verbatim}
#include "GraphBLAS.h"
GrB_info GrB_Descriptor_new(GrB_Descriptor *d)
\end{verbatim}

\paragraph{Output Parameters}

\begin{itemize}
	\item[{\sf d}] Identifier of new descriptor created.
\end{itemize}

\paragraph{Return Value}

\begin{tabular}{rl} 
{\sf GrB\_SUCCESS} 	& operation completed successfully \\
{\sf GrB\_PANIC}	& unknown internal error \\
{\sf GrB\_OUTOFMEM}	& not enough memory available for operation \\
{\sf GrB\_MISMATCH}	& mismatch between field and new value
\end{tabular}

\paragraph{Description}

Returns in {\sf d} the identifier of a newly created empty descriptor.
A newly created descriptor can be populated with calls to {\sf
Descriptor\_set}.

\subsubsection{Set content of descriptor ({\sf Descriptor\_set})}

\comment{
\scott{Naming nit: I propose {\sf Descriptor\_set}.  "adding" implies
accumulation (OR) of flags across many calls.  Allowing only set which
overwrites any existing values is simpler.} \jose{Agreed and modified.}  \scott{OK TO REMOVE}
}

Sets the content (details of an operation) for a field of an existing
descriptor.

\paragraph{C99 Syntax}

\begin{verbatim}
#include "GraphBLAS.h"
GrB_info GrB_Descriptor_set(GrB_Descriptor d,GrB_Field f,GrB_Value v)
\end{verbatim}

\paragraph{Input Parameters}

\begin{itemize}
	\item[{\sf d}] The descriptor being modified by this method.
	\item[{\sf f}] The descriptor field being set.
	\item[{\sf v}] New value for the field being set.
\end{itemize}

\paragraph{Return Value}

\begin{tabular}{rl} 
{\sf GrB\_SUCCESS} 	& operation completed successfully \\
{\sf GrB\_PANIC}	& unknown internal error \\
{\sf GrB\_OUTOFMEM}	& not enough memory available for operation \\
{\sf GrB\_MISMATCH}	& mismatch between field and new value
\end{tabular}

\paragraph{Description}

The fields of a descriptor include: {\sf GrB\_OUTP} for the 
output parameter (result) of a method; {\sf GrB\_MASK} for the mask
argument to a method; {\sf GrB\_ARG0} through {\sf GrB\_ARG9} for
the input parameters (from first to last) of a method.

Valid values for a field of a descriptor are as follows:

\begin{tabular}{rl} 
{\sf GrB\_NOP} 	& no operation to be performed for the corresponding parameter \\
{\sf GrB\_LNOT}	& \parbox[t]{5in}{compute the logical inverse \scott{structural complement?} of the corresponding parameter}  \\
{\sf GrB\_TRAN}	& compute the transpose of the corresponding parameter (for matrices) \\
{\sf GrB\_ACC}  & accumulate result of operation to current values in destination (for output parameter) \\
{\sf GrB\_CAST} & \parbox[t]{5in}{allow casting of values from input parameters to input domains of operation
                  or from output domain of operation to output parameter. (Otherwise, mismatching domains will cause a run-time error.)}
\end{tabular}

\scott{GrB\_LNOT clashes with operator in Table 2}

\ajy{GrB\_LNOT: logical inverse of non-mask parameters can be implemented by modifying the operators; therefore consider restricting this to masks only.}
\jose{I am OK with restricting {\sf GrB\_LNOT} to masks only.}

It is possible to specify a combination of values for a field. For 
example, if a matrix is to be both transposed and logically inverted
(element by element), one would use the field value
${\sf GrB\_TRAN} \mid {\sf GrB\_LNOT}$. 




%=============================================================================
%=============================================================================

\section{GraphBLAS Operations}
\label{Sec:Operations}

\begin{table*}[h]
\hrule
\begin{center}
\caption{A Mathematical overview of the fundamental GraphBLAS operations supported.}
\label{Tab:GraphBLASOps}
\begin{tabular}{l|rrl}
{\sf Operation Name} & \multicolumn{3}{c}{Mathematical Description}  \\
\hline
{\sf mxm}          & $\matrix{C}(\neg\matrix{M})$ & $\oplus=$ & $\matrix{A}^T \oplus.\otimes \matrix{B}^T$  \\
{\sf mxv}          & $\vector{c}(\neg\vector{m})$ & $\oplus=$ & $\matrix{A}^T \oplus.\otimes \vector{b}$  \\
{\sf vxm}          & $\vector{c}(\neg\vector{m})$ & $\oplus=$ & $\vector{b} \oplus.\otimes \matrix{A}^T$  \\
{\sf eWiseMult}    & $\matrix{C}(\neg\matrix{M})$ & $\oplus=$ & $\matrix{A}^T \otimes \matrix{B}^T$  \\
{\sf eWiseAdd}     & $\matrix{C}(\neg\matrix{M})$ & $\oplus=$ & $\matrix{A}^T \oplus  \matrix{B}^T$  \\
{\sf reduce} (row) & $\vector{c}(\neg\vector{m})$ & $\oplus=$ & $\oplus_j\matrix{A}^T(:,j)$  \\
{\sf apply}        & $\matrix{C}(\neg\matrix{M})$ & $\oplus=$ & $f(\matrix{A}^T)$ \\
{\sf transpose}    & $\matrix{C}(\neg\matrix{M})$ & $\oplus=$ & $\matrix{A}^T$ \\
{\sf extract}      & $\matrix{C}(\neg\matrix{M})$ & $\oplus=$ & $\matrix{A}^T(\vector{i},\vector{j})$ \\
{\sf assign}       & $\matrix{C}(\neg\matrix{M})(\vector{i},\vector{j})$ & $\oplus=$ & $\matrix{A}^T$ \\
{\sf buildMatrix}  & $\matrix{C}(\neg\matrix{M})$ & $\oplus=$ & $\mathbb{S}^{m\times n}(\vector{i},\vector{j},\vector{v},\oplus_{dup})$ \\
{\sf buildVector}  & $\vector{u}(\neg\vector{m})$ & $\oplus=$ & $\mathbb{S}^{n}(\vector{i},\vector{v})$ \\
{\sf extractTuples}& $(\vector{i},\vector{j},\vector{v})$ & $=$ & $\matrix{A}(\neg\matrix{M})$ \\
\end{tabular}
\end{center}
\hrule
\end{table*}


A mathematical overview of the the fundamental GraphBLAS operations that are
discussed in this section are shown in Table~\ref{Tab:GraphBLASOps}.  This
section also specifies variants to some of these operations where they have
been found especially useful in algorithm development.

When a GraphBLAS operation supports the use of an optional mask, that mask is
specified through a GraphBLAS vector (for one-dimensional masks) or
a GraphBLAS matrix (for two-dimensional masks).

Given a GraphBLAS vector $\vector{v} = \langle D,N, \{ (i,v_i) \} \rangle$, a
one-dimensional mask $m = \langle N, \{ i : \mbox{\tt <bool>}v_i = \true \} \rangle$
is derived for use in the operation, where $\mbox{\tt <bool>}v_i$ denotes
casting the value $v_i$ to a Boolean value (\true\ or \false).
We note that, if cast is disallowed for the mask by the operation descriptor, then
$\bold{D}(\vector{v})$ must be {\sf GrB\_BOOL}.

Given a GraphBLAS matrix $\matrix{A} = \langle D, M, N, \{ (i,j,A_{ij}) \} \rangle$,
a two-dimensional mask $\matrix{M} = \langle M,N, \{ (i,j) : \mbox{\tt <bool>}A_{ij} = \true \} \rangle$
is derived for use in the operation, where $\mbox{\tt <bool>}v_i$ denotes
casting the value $A_{ij}$ to a Boolean value (\true\ or \false).
We note that, if cast is disallowed for the mask by the operation descriptor, then
$\bold{D}(\matrix{A})$ must be {\sf GrB\_BOOL}.

In both the one- and two-dimensional cases, the mask may go through a structural
complement operation ($\S$~\ref{Sec:Masks}) as specified in the descriptor, before a final
mask is generated for use in the operation.

\subsection{{\sf wait}: Force completion of pending operations}
\label{Sec:wait}

Guarantees that pending GraphBLAS operations are fully executed.

\subsubsection{Broad variant}

\paragraph{C99 syntax}

\begin{verbatim}
        GrB_info GrB_wait(char* err)
\end{verbatim}

\paragraph{Parameters}
\begin{itemize}
\item[{\sf err}]     A null terminated string containing additional error
information.
\end{itemize}


\paragraph{Return values}
\begin{itemize}[leftmargin=2.1in]
\item[{\sf GrB\_SUCCESS}]	operation completed successfully
\item[{\sf GrB\_PANIC}]		unknown internal error
\end{itemize}

\paragraph{Description}

Upon return, all previously called GraphBLAS methods have fully completed their execution.
Any (transparent or opaque) data structures produced or manipulated by those methods can be safely touched.

\subsubsection{Matrix variant}

\paragraph{C99 syntax}

\begin{verbatim}
        GrB_info GrB_wait(GrB_Matrix A, char *err)
\end{verbatim}

\paragraph{Parameters}
\begin{itemize}
\item[{\sf err}]     A null terminated string containing additional error
information.
\end{itemize}

\begin{itemize}[leftmargin=1.1in]
	\item[{\sf A}]	({\sf ARG0}) A GraphBLAS matrix.
\end{itemize}

\paragraph{Return values}
\begin{itemize}[leftmargin=2.1in]
\item[{\sf GrB\_SUCCESS}]	operation completed successfully
\item[{\sf GrB\_PANIC}]		unknown internal error
\end{itemize}

\paragraph{Description}

Upon return, all previously called GraphBLAS methods that used {\sf A} either as input or output have fully completed their execution.
Any (transparent or opaque) data structures produced or manipulated by those methods can be safely touched.

\subsubsection{Vector variant}

\paragraph{C99 syntax}

\begin{verbatim}
        GrB_info GrB_wait(GrB_Vector v, char *err)
\end{verbatim}

\paragraph{Parameters}
\begin{itemize}
\item[{\sf err}]     A null terminated string containing additional error
information.
\end{itemize}

\begin{itemize}[leftmargin=1.1in]
	\item[{\sf v}]	({\sf ARG0}) A GraphBLAS vector.
\end{itemize}

\paragraph{Return values}
\begin{itemize}[leftmargin=2.1in]
\item[{\sf GrB\_SUCCESS}]	operation completed successfully
\item[{\sf GrB\_PANIC}]		unknown internal error
\end{itemize}

\paragraph{Description}

Upon return, all previously called GraphBLAS methods that used {\sf v} either as input or output have fully completed their execution.
Any (transparent or opaque) data structures produced or manipulated by those methods can be safely touched.


\subsubsection{Store elements from tuples into a matrix ({\sf buildMatrix})}

\paragraph{C99 Syntax}

\begin{verbatim}
#include "GraphBLAS.h"
GrB_info GrB_buildMatrix(GrB_Matrix * A, const GrB_IndexArray rowids, const GrB_IndexArray colids, 
                 const GrB_Vector values[, GrB_Function accum])
\end{verbatim}



\paragraph{Input Parameters}

\begin{itemize}
	\item[{\sf rowids}] ({\sf ARG0}) Index array holding row indices. 
	multiply.

	\item[{\sf colids}] ({\sf ARG1}) Index array holding column indices. 

	\item[{\sf values}] ({\sf ARG2}) Vector holding values.  

	\item[{\sf accum}] ({\sf ARG3}) Function used for accumulating entries into \matrix{A}. (optional). 
\end{itemize}

\paragraph{Description}
Each tuple $\{ {\sf rowids(i)}, {\sf colids(i)}, {\sf values(i)}\}$ is a contribution to the output in the form of 

$$\matrix{A}({\sf rowids(i)}, {\sf colids(i)}) = {\sf accum}({\sf values(i)}, \matrix{A}({\sf rowids(i)}, {\sf colids(i)}).$$ 

If {\sf accum} parameter is not provided, then the contribution is of the form 

$$\matrix{A}({\sf rowids(i)}, {\sf colids(i)}) \mathrel{+}= {\sf values(i)}.$$
 
{\sf rowids}, {\sf colids}, and {\sf values} should be of the same length. 

\subsubsection{Extract tuples from a matrix ({\sf extractTuples})}

Placeholder

\aydin{Aydin to fill}


\comment{
\pagebreak
\subsection{{\sf mxm}: (simple) Matrix-matrix multiply}

Multiplies a matrix with another matrix on a semiring. The result is a matrix.
This is the simple version: no mask, no accumulate, no user-defined types.
Descriptor only controls transposition of inputs.
Blocking mode only.

\paragraph{\syntax}

\begin{verbatim}
        GrB_info GrB_mxm(GrB_Matrix              *C,
                         const GrB_Semiring       op,
                         const GrB_Matrix         A, 
                         const GrB_Matrix         B,
                         const GrB_Descriptor     desc);
\end{verbatim}

\paragraph{Parameters}

\begin{itemize}[leftmargin=1.1in]
    \item[{\sf C}]    An existing matrix (of dimensions $m \times n$) to store the result. 

    \item[{\sf op}]   Semiring used in the matrix-matrix multiply: ${\sf op}=\langle D_1,D_2,D_3,\oplus,\otimes,0 \rangle$.
    \item[{\sf A}]    The left-hand matrix to be multiplied.  
    \item[{\sf B}]    The right-hand matrix to be multiplied. 

    \item[{\sf desc}]  Operation descriptor (optional). If a
    \emph{default} descriptor is desired, {\sf GrB\_NULL} should be
    used. Valid fields are as follows: \\ ~\\
    \begin{tabular}{lllp{2.75in}}
    Var. & Field  & Value & Description \\
    \hline
    {\sf A}    & {\sf GrB\_INP0} & {\sf GrB\_TRAN}   & Use transpose of {\sf A} for operation. \\
    {\sf B}    & {\sf GrB\_INP1} & {\sf GrB\_TRAN}   & Use transpose of {\sf B} for operation. \\
    \end{tabular}
\end{itemize}

\paragraph{Return Values}

\begin{itemize}[leftmargin=2.1in]
\item[{\sf GrB\_SUCCESS}]             operation completed successfully
\item[{\sf GrB\_PANIC}]               unknown internal error
\item[{\sf GrB\_OUTOFMEM}]            not enough memory available for operation
\item[{\sf GrB\_DIMENSION\_MISMATCH}] matrix dimensions are incompatible
\end{itemize}

\raggedbottom
\pagebreak

\paragraph{Description}

Matrix $\matrix{A}$ is computed from input parameter {\sf A} as follows:
\begin{enumerate}
\item	If {\sf desc[GrB\_INP0].GrB\_TRAN} is \false, then $\matrix{A} = \langle \bold{D}({\sf A}), \bold{m}({\sf A}), \bold{n}({\sf A}),
        \bold{L}(\matrix{A}) = \{(i,j,A_{ij}) : (i,j,A_{ij}) \in \bold{L}({\sf A})\} \rangle$.
\item	If {\sf desc[GrB\_INP0].GrB\_TRAN} is \true,  then $\matrix{A} = \langle \bold{D}({\sf A}), \bold{n}({\sf A}), \bold{m}({\sf A}), 
        \bold{L}(\matrix{A}) = \{(j,i,A_{ij}) : (i,j,A_{ij}) \in \bold{L}({\sf A})\} \rangle$.
\end{enumerate}

Matrix $\matrix{B}$ is computed from input parameter {\sf B} as follows:
\begin{enumerate}
\item	If {\sf desc[GrB\_INP1].GrB\_TRAN} is \false, then $\matrix{B} = \langle \bold{D}({\sf B}), \bold{m}({\sf B}), \bold{n}({\sf B}),
        \bold{L}(\matrix{B}) = \{(i,j,B_{ij}) : (i,j,B_{ij}) \in \bold{L}({\sf B})\} \rangle$.
\item	If {\sf desc[GrB\_INP1].GrB\_TRAN} is \true,  then $\matrix{B} = \langle \bold{D}({\sf B}), \bold{n}({\sf B}), \bold{m}({\sf B}), 
        \bold{L}(\matrix{B}) = \{(j,i,B_{ij}) : (i,j,B_{ij}) \in \bold{L}({\sf B})\} \rangle$.
\end{enumerate}

If $\bold{n}(\matrix{A}) \neq \bold{m}(\matrix{B})$, the method exits with return code {\sf GrB\_DIMENSION\_MISMATCH}.

A new matrix $\matrix{C} = \langle \bold{D}_3({\sf op}),
\bold{m}(\matrix{A}), \bold{n}(\matrix{B}), \bold{L}(\vector{C}) =
\{(i,j,C_{ij}) : \vector{i}(\matrix{A}(i,:)) \cap
\vector{i}(\matrix{B}(:,j)) \neq \emptyset \} \rangle$ is created.
The value of each of its elements is computed by 
\[ 
C_{ij} = \bigoplus_{k
\in \vector{i}(\matrix{A}(i,:)) \cap \vector{i}(\matrix{B}(:,j))}
(((D_1)\matrix{A}(i,k)) \otimes ((D_2)\matrix{B}(k,j))),
\]
where $\oplus$
and $\otimes$ are the additive and multiplicative operators of
semiring {\sf op}, respectively.  

If $\bold{m}(\matrix{C}) \neq \bold{m}({\sf C})$ or $\bold{n}(\matrix{C}) \neq \bold{n}({\sf C})$,
the method exits with return code {\sf GrB\_DIMENSION\_MISMATCH}.
Otherwise, set $\bold{L}({\sf C}) = \{(i,j,(\bold{D}({\sf C}))C_{ij}) : (i,j,C_{ij}) \in \bold{L}(\matrix{C}) \}$, and return {\sf GrB\_SUCCESS}.

}
\raggedbottom
\pagebreak
%-----------------------------------------------------------------------------
\subsection{{\sf mxm}: Matrix-matrix multiply}

Multiplies a matrix with another matrix on a semiring. The result is a matrix.

\paragraph{\syntax}

\begin{verbatim}
        GrB_info GrB_mxm(GrB_Matrix              *C,
                         const GrB_Matrix         Mask,
                         const GrB_BinaryOp       accum,
                         const GrB_Semiring       op,
                         const GrB_Matrix         A, 
                         const GrB_Matrix         B,
                         const GrB_Descriptor     desc);
\end{verbatim}

\paragraph{Parameters}

\begin{itemize}[leftmargin=1.1in]
    \item[{\sf C}]    ({\sf INOUT}) An existing GraphBLAS matrix. On
    input, the matrix provides values that may be accumulated with the
    result of the matrix product.   On output, the matrix holds the
    results of this operation.

    \item[{\sf Mask}] ({\sf IN}) A ``write'' mask that controls which
    results from this operation are stored into the output matrix
    ${\sf C}$ (optional).  If no mask is desired (\ie, all elements
    of result are copied into the output matrix), {\sf GrB\_NULL}
    should be specified. The Mask dimensions must match those of the
    matrix {\sf C} and the domain of the {\sf Mask} matrix must be
    of type {\sf bool} or any of the predefined ``built-in'' types in
    Table~\ref{Tab:PredefinedTypes}.

    \item[{\sf accum}] ({\sf IN}) Operator used for accumulating entries
    into existing \matrix{\sf C} entries: ${\sf accum} = \langle D_x,
    D_y, D_z,\odot \rangle$. If assignment rather than accumulation is
    desired, {\sf GrB\_NULL} should be specified.

    \item[{\sf op}] ({\sf IN}) Semiring used in the matrix-matrix
    multiply: ${\sf op}=\langle D_1,D_2,D_3,\oplus,\otimes,0 \rangle$.

    \item[{\sf A}] ({\sf IN}) The GraphBLAS matrix holding the values
    for the left-hand matrix in the multiplication.

    \item[{\sf B}] ({\sf IN}) The GraphBLAS matrix holding the values
    for the right-hand matrix in the multiplication.

    \item[{\sf desc}] ({\sf IN}) Operation descriptor (optional). If
    a \emph{default} descriptor is desired, {\sf GrB\_NULL} should be
    used. Valid fields are as follows: 

    \begin{tabular}{lllp{2.75in}}
    Argument   & Field           & Value               & Description \\ \hline
    {\sf C}    & {\sf GrB\_OUTP} & {\sf GrB\_REPLACE}  & Output matrix {\sf C} is cleared (all elements removed) before result is stored in it. \\
    {\sf Mask} & {\sf GrB\_MASK} & {\sf GrB\_SCMP}     & Use the structural complement of {\sf Mask}. \\
    {\sf A}    & {\sf GrB\_INP0} & {\sf GrB\_TRAN}     & Use transpose of {\sf A} for operation. \\
    {\sf B}    & {\sf GrB\_INP1} & {\sf GrB\_TRAN}     & Use transpose of {\sf B} for operation. \\
    \end{tabular}
\end{itemize}

\paragraph{Return Values}

\begin{itemize}[leftmargin=2.1in]

	\item[{\sf GrB\_SUCCESS}]	      In blocking mode, operation
	completed successfully. In non-blocking mode, this indicates
	that the consistency tests on dimensions and domains for the
	input arguments passed successfully. Either way, output matrix
	{\sf C} is ready to be used in the next method of the sequence.

	\item[{\sf GrB\_PANIC}]		      Unknown internal error

	\item[{\sf GrB\_OUTOFMEM}]	      Not enough memory available
	for operation

	\item[{\sf GrB\_DIMENSION\_MISMATCH}] Matrix dimensions are
	incompatible.

	\item[{\sf GrB\_DOMAIN\_MISMATCH}]    The domains of the various
	matrices are incompatible with the corresponding domains of the
	accumulating operation, semiring, or mask.

\end{itemize}

\comment{
\paragraph{Description}

Matrices $\matrix{A}, \matrix{B}$, and $\matrix{Mask}$ are computed from
input parameters {\sf A}, {\sf B} and {\sf Mask}, respectively, as specified
by descriptor {\sf desc}. (See below for the properties of a descriptor. In
the simplest form, these are just copies, but additional preprocessing,
including casting and transposing, can be performed.)  $\bold{D}(\matrix{A}) =
\bold{D}_1({\sf op})$ and $\bold{D}(\matrix{B}) = \bold{D}_2({\sf op})$.
$\bold{D}(\matrix{Mask}) = {\sf GrB\_BOOL}$.  If {\sf Mask} is {\sf GrB\_NULL} or omitted,
then $\matrix{Mask}$ is a Boolean matrix of the same dimensions as $\matrix{C}$
and with all elements set to {\sf true}.

If either $\matrix{A}, \matrix{B}$, and $\matrix{Mask}$  cannot be computed
from the input parameters as described above, the method returns {\sf
GrB\_DOMAIN\_MISMATCH}.

A consistency check is performed to verify that dimensions of $\matrix{A}$, $\matrix{B}$
and $\matrix{C}$ are compatible, considering possible transpositions.  If a consistency check fails, the operation is
aborted and the method returns {\sf GrB\_DIMENSION\_MISMATCH}.

A consistency check is performed to verify the dimensions of $\matrix{C}$
and $\matrix{Mask}$ agree.  If a consistency check fails, the operation is
aborted and the method returns {\sf GrB\_DIMENSION\_MISMATCH}.

A new matrix $\matrix{C} = \langle \bold{D}_3({\sf op}),
\bold{m}(\matrix{A}), \bold{n}(\matrix{B}), \bold{L}(\vector{C}) = \{(i,j,C_{ij}) : \matrix{Mask}(i,j)
= {\sf true} \} \rangle$ is created.  The value of each of its elements
is computed by $C_{ij} = \bigoplus_{k \in \vector{i}(\matrix{A}(i,:)) \cap
\vector{i}(\matrix{B}(:,j))} (\matrix{A}(i,j) \otimes \matrix{B}(k,j))$,
where $\oplus$ and $\otimes$ are the additive and multiplicative
operators of semiring {\sf op}, respectively.  If $\vector{i}(\matrix{A}(i,:)) \cap \vector{i}(\matrix{B}(:,j)) = \emptyset $ then the tuple $(i,j,C_{ij})$
is not included in $\bold{L}(\matrix{C})$.

Finally, output parameter {\sf C} is computed from matrix $\matrix{C}$
as specified by descriptor {\sf desc}. (In the simplest case this
is just a copy, but additional postprocessing, including casting and
accumulation of result values, can be specified.)  A consistency check is
performed to verify that $\bold{m}({\sf C}) = \bold{m}(\matrix{C})$ and $\bold{n}({\sf C}) = \bold{n}(\matrix{C})$. If
the consistency check fails, the operation is aborted and the method
return {\sf GrB\_DIMENSION\_MISMATCH}.
}

\paragraph{Description}

{\sf GrB\_mxm} computes the matrix product $\matrix{\sf C} = \matrix{\sf
A} \otimes . \oplus \matrix{\sf B}$ or, if an optional binary accumulation
operator ($\odot$) is provided, $\matrix{\sf C} = \matrix{\sf C} \odot
\matrix{\sf A} \otimes . \oplus \matrix{\sf B}$.  (Matrices {\sf A}
and {\sf B} can be optionally transposed.)  Logically, this operation
occurs in three steps:
\begin{enumerate}
\item The internal matrices and mask used in the computation are formed and their domains/dimensions are tested for consistency.
\item The indicated computations are carried out.
\item The result is written into the output matrix, possibly under control of a mask.
\end{enumerate}

Up to four argument matrices are used in the {\sf GrB\_mxm} operation:
\begin{enumerate}
\item $\matrix{\sf C} = \langle \bold{D}({\sf C}),\bold{nrows}(\matrix{\sf C}),\bold{ncols}(\matrix{\sf C}),\bold{L}(\matrix{\sf C}) = \{(i,j,C_{ij}) \} \rangle$
\item $\matrix{\sf Mask} = \langle \bold{D}({\sf Mask}),\bold{nrows}(\matrix{\sf Mask}),\bold{ncols}(\matrix{\sf Mask}),\bold{L}(\matrix{\sf Mask}) = \{(i,j,{\sf Mask}_{ij}) \} \rangle$ (optional)
\item $\matrix{\sf A} = \langle \bold{D}({\sf A}),\bold{nrows}(\matrix{\sf A}), \bold{ncols}(\matrix{\sf A}),\bold{L}(\matrix{\sf A}) = \{(i,j,A_{ij}) \} \rangle$
\item $\matrix{\sf B} = \langle \bold{D}({\sf B}),\bold{nrows}(\matrix{\sf B}), \bold{ncols}(\matrix{\sf B}),\bold{L}(\matrix{\sf B}) = \{(i,j,B_{ij}) \} \rangle$
\end{enumerate}

The argument matrices, the semiring, and the accumulator operator (if provided) are tested for domain consistency
as follows:
\begin{enumerate}

	\item The domain of {\sf Mask} (if not {\sf GrB\_NULL}) must be from one of the pre-defined types of Table~\ref{Tab:PredefinedTypes}.

	\item $\bold{D}(\matrix{\sf A})$ must be compatible with $D_1$ of the semiring.

	\item $\bold{D}(\matrix{\sf B})$ must be compatible with $D_2$ of the semiring.

	\item If {\sf accum} is {\sf GrB\_NULL}, then $\bold{D}(\matrix{\sf C})$ must be compatible with $D_3$ of the semiring.

	\item If {\sf accum} is not {\sf GrB\_NULL}, then $\bold{D}(\matrix{\sf C})$ must be compatible with $D_x$ and $D_z$ of the 
	accumulator operator and $D_3$ of the semiring must be compatible with $D_y$ of the accumulator operator.

\end{enumerate}
Two domains are compatible with each other if values from one domain can be cast to values in the other domain as per the rules of the C language.
In particualr, domains from Table~\ref{Tab:PredefinedTypes} are all compatible with each other. A domain from a user-defined type is only compatible with itself.
If any consistency rule above is violated, execution of {\sf GrB\_mxm} ends and the domain mismatch error listed above is returned.

From the argument matrices, the internal matrices and mask used in the computation are formed ($\leftarrow$ denotes copy):
\begin{enumerate}

	\item Matrix $\matrix{\widetilde{C}} \leftarrow {\sf C}$.

	\item Two-dimensional mask $\matrix{\widetilde{Mask}}$ is computed from argument {\sf Mask} as follows:
	\begin{enumerate}

		\item	If ${\sf Mask} = {\sf GrB\_NULL}$, then $\matrix{\widetilde{Mask}} = \langle \bold{nrows}({\sf C}), \bold{ncols}({\sf C}), \{(i,j), \forall i,j : 0 \leq i <  \bold{nrows}({\sf C}), 0 \leq j < \bold{ncols}({\sf C}) \} \rangle$.

		\item	Otherwise, $\matrix{\widetilde{Mask}} = \langle \bold{nrows}({\sf Mask}), \bold{ncols}({\sf Mask}), \{(i,j) : ({\sf bool}){\sf Mask}(i,j) = \true\} \rangle$.

		\item	If ${\sf desc[GrB\_MASK].GrB\_SCMP}$ is \true, then $\matrix{\widetilde{Mask}} \leftarrow \neg \matrix{\widetilde{Mask}}$.

	\end{enumerate}

	\item Matrix $\matrix{\widetilde{A}} \leftarrow {\sf desc[GrB\_INP0].GrB\_TRAN} \ ? \ {\sf A}^T : {\sf A}$.

	\item Matrix $\matrix{\widetilde{B}} \leftarrow {\sf desc[GrB\_INP1].GrB\_TRAN} \ ? \ {\sf B}^T : {\sf B}$.
\end{enumerate}

The internal matrices and masks are checked for shape consistency. The following conditions must hold:
\begin{enumerate}
	
	\item $\bold{nrows}(\matrix{\widetilde{C}}) = \bold{nrows}(\matrix{\widetilde{Mask}})$.

	\item $\bold{ncols}(\matrix{\widetilde{C}}) = \bold{ncols}(\matrix{\widetilde{Mask}})$.

	\item $\bold{nrows}(\matrix{\widetilde{C}}) = \bold{nrows}(\matrix{\widetilde{A}})$.

	\item $\bold{ncols}(\matrix{\widetilde{C}}) = \bold{ncols}(\matrix{\widetilde{B}})$.

	\item $\bold{ncols}(\matrix{\widetilde{A}}) = \bold{nrows}(\matrix{\widetilde{B}})$.

\end{enumerate}
If any consistency rule above is violated, execution of {\sf GrB\_mxm} ends and the dimension mismatch error listed above is returned.

%
%  TGM:   We need to help the reader follow the flow of this description. 
%

We are now ready to carry out the matrix multiplication and any additional associated operations.    We describe
this in terms of two intermediate matrices:
\begin{itemize}
\item $\matrix{\widetilde{T}}$: The matrix holding the product of matrices $\matrix{\widetilde{A}}$ and $\matrix{\widetilde{B}}$.
\item $\matrix{\widetilde{Z}}$: The matrix holding the result after application of the (optional) accumulator.
\end{itemize}

The intermediate matrix $\matrix{\widetilde{T}} = \langle
D_3, \bold{nrows}(\matrix{\widetilde{A}}),
\bold{ncols}(\matrix{\widetilde{B}}), \bold{L}(\vector{T}) =
\{(i,j,T_{ij}) : \bold{ind}(\matrix{\widetilde{A}}(i,:)) \cap
\bold{ind}(\matrix{\widetilde{B}}(:,j)) \neq \emptyset \} \rangle$
is created.  The value of each of its elements is computed by \[T_{ij}
= \bigoplus_{k \in \bold{ind}(\matrix{\widetilde{A}}(i,:)) \cap
\bold{ind}(\matrix{\widetilde{B}}(:,j))} (\matrix{\widetilde{A}}(i,k)
\otimes \matrix{\widetilde{B}}(k,j)),\] where $\oplus$ and $\otimes$
are the additive and multiplicative operators of semiring {\sf op},
respectively.

The intermediate  matrix $\matrix{\widetilde{Z}}$ is created as follows:
If ${\sf accum} = {\sf GrB\_NULL}$, then $\matrix{\widetilde{Z}} = \matrix{\widetilde{T}}$.
If \[ {\sf accum} = \langle D_x, D_y, D_z, \odot \rangle, \] then matrix $\matrix{\widetilde{Z}}$ is defined as $\langle D_z, \bold{nrows}(\matrix{\widetilde{C}}), \bold{ncols}(\matrix{\widetilde{C}}), \bold{L}(\matrix{\widetilde{Z}}) 
		= \{(i,j,Z_{ij})  \forall (i,j) \in \bold{ind}(\matrix{\widetilde{C}}) \cup \bold{ind}(\matrix{\widetilde{T}}) \} \rangle$.  
The values of the elements of $\matrix{\widetilde{Z}}$ are computed based on the relationships between the sets of indices in $\matrix{\widetilde{C}}$ and $\matrix{\widetilde{T}}$.
\[
Z_{ij} = \matrix{\widetilde{C}}(i,j) \odot \matrix{\widetilde{T}}(i,j), \ \mbox{if}\  (i,j) \in  (\bold{ind}(\matrix{\widetilde{T}}) \cap \bold{ind}(\matrix{\widetilde{C}})),
\]
\[
Z_{ij} = \matrix{\widetilde{C}}(i,j) \ \mbox{if}\  (i,j) \in  (\bold{ind}(\matrix{\widetilde{C}}) - (\bold{ind}(\matrix{\widetilde{T}}) \cap \bold{ind}(\matrix{\widetilde{C}}))),
\]
\[
Z_{ij} = \matrix{\widetilde{T}}(i,j) \ \mbox{if}\  (i,j) \in  (\bold{ind}(\matrix{\widetilde{T}}) - (\bold{ind}(\matrix{\widetilde{T}}) \cap \bold{ind}(\matrix{\widetilde{C}}))).
\]
where the difference operator in the previous expressions refers to set difference.

Finally, the set of output values that make up the $\matrix{\widetilde{Z}}$ matrix are written into the final result matrix, $\matrix{C}$. 
This is carried out under the influence of the mask which acts as a ``write mask''.
If {\sf desc[GrB\_OUTP].GrB\_REPLACE} is \true, then any values in $\matrix{C}$ on input to {\sf GrB\_mxm()} are deleted and the new
output matrix $\matrix{C}$ is,
\[ \bold{L}({\sf C}) = \{(i,j,Z_{ij}) : (i,j) \in (\bold{L}(\matrix{\widetilde{Mask}}) \cap \bold{ind}(\matrix{\widetilde{Z}})) \}. \]
If {\sf desc[GrB\_OUTP].GrB\_REPLACE} is not given or is equal to any value other than \true, the elements of $\matrix{\widetilde{Z}}$ indicated by the mask
are copied into the result matrix, $\matrix{C}$ and elements of $\matrix{C}$  that fall outside the set indicated by the mask are unchanged:
\[ \bold{L}({\sf C}) = \{(i,j,C_{ij}) : (i,j) \in (\bold{ind}(\matrix{\sf C}) \cap \bold{L}(\neg \matrix{\widetilde{Mask}})) \} \cup \{(i,j,Z_{ij}) : (i,j) \in (\bold{L}(\matrix{\widetilde{Mask}}) \cap \bold{ind}(\matrix{\widetilde{Z}})) \}. \]

In {\sf GrB\_Blocking} mode, the method exits with return value {\sf GrB\_SUCCESS} and the new content of matrix {\sf C} is as defined above and fully computed.
In {\sf GrB\_NonBlocking} mode, the method exits with return value {\sf GrB\_SUCCESS} and the new content of matrix {\sf C} is as defined above but may not be fully computed. (It can be used in the next GraphBLAS method call in a sequence.)

\raggedbottom
\pagebreak
\comment{
\paragraph{Alternate Description (WORK IN PROGRESS)}

A consistency check is performed to verify that dimensions of {\sf A}, {\sf B}, {\sf C}
and {\sf Mask} are compatible, considering possible transpositions of {\sf A} and 
{\sf B}.  As an example without transposes, it is verified that 
$\bold{m}({\sf C}) = \bold{m}(\sf A)$ and $\bold{n}({\sf C}) = \bold{n}(\sf B)$. 
A consistency check is also performed to verify that the dimensions of $\sf C$ and 
$\sf Mask$ are the same.  Note, if {\sf Mask} is {\sf GrB\_NULL} or omitted,
then $\matrix{Mask}$ is a Boolean matrix of the same dimensions as $\sf C$
and with all elements set to {\sf true}. If either consistency check fails, the operation is
aborted and the method returns {\sf GrB\_DIMENSION\_MISMATCH}.

A check is also performed to verify that the domains of the matrices are consistent
with the operations to be performed.  Matrices $\matrix{A}, \matrix{B}$, and 
$\matrix{Mask}$ are computed from input parameters {\sf A}, {\sf B} and {\sf Mask},
respectively, where additional preprocessing such as casting, transposition, and
structural complement could be performed as specified by descriptor, {\sf desc}.
Note that at this point, $\bold{D}(\matrix{A}) = \bold{D}_1({\sf op})$ and 
$\bold{D}(\matrix{B}) = \bold{D}_2({\sf op})$.  Also, $\bold{D}(\matrix{Mask}) = 
{\sf GrB\_BOOL}$.  The operation is aborted and the method returns a {\sf GrB\_DOMAIN\_MISMATCH}
in the following situations:
\begin{itemize}
\item $\bold{D}({\sf A})$ does not equal $\bold{D}_1({\sf op})$ and the descriptor prohibits casting of {\sf GrB\_INP0},
\item $\bold{D}({\sf B})$ does not equal $\bold{D}_2({\sf op})$ and the descriptor prohibits casting of {\sf GrB\_INP1},
\item $\bold{D}({\sf Mask})$ does not equal ${\sf GrB\_BOOL}$ and the descriptor prohibits casting of {\sf GrB\_MASK} (Mask is not {\sf GrB\_NULL},
\item If accumulation operator is NOT present:
        \begin{itemize}
        \item $\bold{D}({\sf C})$ does not equal $\bold{D}_3({\sf op})$ and the descriptor prohibits casting of {\sf GrB\_OUTP},
        \end{itemize}
\item If accumulation operator is present:
        \begin{itemize}
        \item $\bold{D}({\sf C})$ does not equal $\langle \bold{D}_3({\sf accum})$ and the descriptor prohibits casting of {\sf GrB\_OUTP}
        \item $\bold{D}({\sf C})$ does not equal $\langle \bold{D}_1({\sf accum})$ and the descriptor prohibits casting of {\sf GrB\_OUTP}
        \item $\bold{D}_3({\sf op})$ does not equal $\langle \bold{D}_2({\sf accum})$
        \end{itemize}
\item Casting is allowed and necessary, for any argument but there is no "defined cast operation" between the two domains.
\end{itemize}



A new matrix $\matrix{C} = \langle \bold{D}_3({\sf op}),
\bold{m}(\matrix{A}), \bold{n}(\matrix{B}), \bold{L}(\vector{C}) = \{(i,j,C_{ij}) : \matrix{Mask}(i,j)
= {\sf true} \} \rangle$ is created.  The value of each of its elements
is computed by $C_{ij} = \bigoplus_{k \in \vector{i}(\matrix{A}(i,:)) \cap
\vector{i}(\matrix{B}(:,j))} (\matrix{A}(i,j) \otimes \matrix{B}(k,j))$,
where $\oplus$ and $\otimes$ are the additive and multiplicative
operators of semiring {\sf op}, respectively.  If $\vector{i}(\matrix{A}(i,:)) \cap \vector{i}(\matrix{B}(:,j)) = \emptyset $ then the tuple $(i,j,C_{ij})$
is not included in $\bold{L}(\matrix{C})$.

Finally, output parameter {\sf C} is computed from matrix $\matrix{C}$. In the 
simplest case this is just a copy, but the {\sf accum} and {\sf desc} parameters can
specify additional postprocessing, including casting and accumulation of result
values.
}

%-----------------------------------------------------------------------------

\subsection{{\sf vxm}: Vector-matrix multiply}

Multiplies a vector by a matrix on an semiring. The result is a vector.

\paragraph{\syntax}

\begin{verbatim}
        GrB_info GrB_vxm(GrB_Vector            *u,
                         const GrB_Vector       mask,
                         const GrB_BinaryOp     accum,
                         const GrB_Semiring     op,
                         const GrB_Vector       v, 
                         const GrB_Matrix       A,
                         const GrB_Descriptor   desc);
\end{verbatim}

\paragraph{Parameters}

\begin{itemize}[leftmargin=1.1in]
    \item[{\sf u}]    ({\sf GrB\_OUTP}) An existing vector to store the result.

    \item[{\sf mask}] ({\sf GrB\_MASK}) Output mask vector . The mask
    specifies which elements of the result vector can be modified.
    If no mask is necessary (i.e., compute all elements of result
    vector), {\sf GrB\_NULL} should be specified.

    \item[{\sf accum}]  Operator used for accumulating entries into existing
                        \vector{u} entries. If no accumulation is desired,
                        {\sf GrB\_NULL} should be specified.

    \item[{\sf op}]   Semiring used in the vector-matrix multiply.
    \item[{\sf v}]    ({\sf GrB\_INP0}) The left-hand vector to be multiplied.
    \item[{\sf A}]    ({\sf GrB\_INP1}) The right-hand matrix to be multiplied.

    \item[{\sf desc}]  Operation descriptor (optional). If a
    \emph{default} descriptor is desired, {\sf GrB\_NULL} can be
    used or the descriptor can be omitted. Valid fields are as follows: \\
    \begin{tabular}{llp{3in}}
    Field  & Value & Description \\
    \hline
    {\sf GrB\_OUTP} & {\sf GrB\_NOCAST} & $\bold{D}({\sf u})$ must equal $\bold{D}_3({\sf op})$
                                          when {\sf accum} is {\sf GrB\_NULL}; otherwise, $\bold{D}({\sf u})$
                                          must equal both $\bold{D}_1({\sf accum})$ and $\bold{D}_3({\sf accum})$.\\
    {\sf GrB\_MASK} & {\sf GrB\_SCMP}   & Use the structural complement of {\sf mask}. \\
    {\sf GrB\_MASK} & {\sf GrB\_NOCAST} & $\bold{D}({\sf mask})$ must equal {\sf GrB\_BOOL}. \\
    {\sf GrB\_INP0} & {\sf GrB\_NOCAST} & $\bold{D}({\sf v})$ must equal $\bold{D}_1({\sf op})$. \\
    {\sf GrB\_INP1} & {\sf GrB\_TRAN}   & Use transpose of {\sf A} for operation. \\
    {\sf GrB\_INP1} & {\sf GrB\_NOCAST} & $\bold{D}({\sf A})$ must equal $\bold{D}_2({\sf op})$. \\
    \end{tabular}
\end{itemize}

\paragraph{Return Values}

\begin{itemize}[leftmargin=2.1in]
\item[{\sf GrB\_SUCCESS}]             operation completed successfully
\item[{\sf GrB\_PANIC}]               unknown internal error
\item[{\sf GrB\_OUTOFMEM}]            not enough memory available for operation
\item[{\sf GrB\_DIMENSION\_MISMATCH}] mismatch among vectors, matrix dimensions.
\item[{\sf GrB\_DOMAIN\_MISMATCH}]    mismatch among domains of vectors, matrix, and/or semiring, for which the descriptor did not explicitly allow casting.
\end{itemize}

\paragraph{Description}

Vectors $\vector{v}, \vector{mask}$ and matrix $\matrix{A}$ are computed from
input parameters {\sf v}, {\sf mask} and {\sf A}, respectively, as specified
by descriptor {\sf desc}. (See below for the properties of a descriptor. In
the simplest form, these are just copies, but additional preprocessing,
including casting and transposition, can be performed.)  $\bold{D}(\vector{v}) =
\bold{D}_1({\sf op})$ and $\bold{D}(\matrix{A}) = \bold{D}_2({\sf op})$.
$\bold{D}(\vector{mask}) = {\sf GrB\_BOOL}$.  If {\sf mask} is {\sf GrB\_NULL}
then $\vector{mask}$ acts as a Boolean vector of size $\bold{n}(\vector{A})$
and with all elements set to {\sf true}.

If either $\vector{v}, \vector{mask}$ or $\matrix{A}$ cannot be computed
from the input parameters as described above, the method returns {\sf
GrB\_DOMAIN\_MISMATCH}.

A consistency check is performed to verify that $\bold{n}(\vector{v})
= \bold{m}(\matrix{A})$ and $\bold{n}(\vector{m}) =
\bold{n}(\vector{v})$. If a consistency check fails, the operation is
aborted and the method returns {\sf GrB\_DIMENSION\_MISMATCH}.

A new vector $\vector{u} = \langle \bold{D}_3({\sf op}),
\bold{n}(\matrix{A}), \bold{L}(\vector{u}) = \{(i,u_i) : \vector{mask}(i)
= {\sf true} \} \rangle$ is created.  The value of each of its elements
is computed by $u_i = \bigoplus_{j \in \vector{i}(\vector{v}) \cap
\vector{i}(\matrix{A}(:,i))} (\vector{v}(j) \otimes \matrix{A}(j,i))$,
where $\oplus$ and $\otimes$ are the additive and multiplicative
operator of semiring {\sf op}, respectively.  If $\vector{i}(\vector{v})
\cap \vector{i}(\matrix{A}(:,i)) = \emptyset$ then the pair $(i,u_i)$
is not present in $\bold{L}(\vector{u})$.

Finally, output parameter {\sf u} is computed from vector $\vector{u}$
as specified by descriptor {\sf desc}. (Again, in the simplest case this
is just a copy, but additional postprocessing, including casting and
accumulation of result values, can be specified.)  A consistency check is
performed to verify that $\bold{n}({\sf u}) = \bold{n}(\vector{u})$. If
the consistency check fails, the operation is aborted and the method
return {\sf GrB\_DIMENSION\_MISMATCH}.

\scott{We need a more explicit discussion/specification regarding
masks and accumulation and their interaction (perhaps the diagram Manoj
projected at the SC15 BoF.} \jose{Agree. Need to find the proper place
for it.}

%-----------------------------------------------------------------------------

\subsection{{\sf mxv}: Matrix-vector multiply}

Multiplies a matrix by a vector within a semiring. The result is a vector.

\paragraph{\syntax}

\begin{verbatim}
        GrB_info GrB_mxv(GrB_Vector            *u,
                         const GrB_Vector       mask,
                         const GrB_BinaryOp     accum,
                         const GrB_Semiring     op, 
                         const GrB_Matrix       A,
                         const GrB_Vector       v,
                         const GrB_Descriptor   desc);
\end{verbatim}

\paragraph{Parameters}

\begin{itemize}[leftmargin=1.1in]
    \item[{\sf u}]    ({\sf GrB\_OUTP}) An existing vector to store the result.
    
    \item[{\sf mask}] ({\sf GrB\_MASK}) Output mask vector . The mask
    specifies which elements of the result vector can be modified.
    If no mask is necessary (i.e., compute all elements of result
    vector), {\sf GrB\_NULL} should be specified.

	\item[{\sf accum}]  Operator used for accumulating entries into existing
                        \vector{u} entries. If no accumulation is desired,
	                    {\sf GrB\_NULL} should be specified.

    \item[{\sf op}]   Semiring used in the matrix-vector multiply.
    \item[{\sf A}]    ({\sf GrB\_INP0}) The left-hand matrix to be multiplied.
    \item[{\sf v}]    ({\sf GrB\_INP1}) The right-hand vector to be multiplied.

    \item[{\sf desc}]  Operation descriptor (optional). If a
    \emph{default} descriptor is desired, {\sf GrB\_NULL} is to be
    used. Valid fields are as follows: \\
    \begin{tabular}{llp{3in}}
    Field  & Value & Description \\
    \hline
    {\sf GrB\_OUTP} & {\sf GrB\_NOCAST} & $\bold{D}({\sf u})$ must equal $\bold{D}_3({\sf op})$
                                          when {\sf accum} is {\sf GrB\_NULL}; otherwise, $\bold{D}({\sf u})$
                                          must equal both $\bold{D}_1({\sf accum})$ and $\bold{D}_3({\sf accum})$.\\
    {\sf GrB\_MASK} & {\sf GrB\_SCMP}   & Use the structural complement of {\sf mask}. \\
    {\sf GrB\_MASK} & {\sf GrB\_NOCAST} & $\bold{D}({\sf mask})$ must equal {\sf GrB\_BOOL}. \\
    {\sf GrB\_INP0} & {\sf GrB\_TRAN}   & Use transpose of {\sf A} for operation. \\
    {\sf GrB\_INP0} & {\sf GrB\_NOCAST} & $\bold{D}({\sf A})$ must equal $\bold{D}_1({\sf op})$. \\
    {\sf GrB\_INP1} & {\sf GrB\_NOCAST} & $\bold{D}({\sf v})$ must equal $\bold{D}_2({\sf op})$. \\
    \end{tabular}
\end{itemize}

\paragraph{Return Values}

\begin{itemize}[leftmargin=2.1in]
\item[{\sf GrB\_SUCCESS}]             operation completed successfully
\item[{\sf GrB\_PANIC}]               unknown internal error
\item[{\sf GrB\_OUTOFMEM}]            not enough memory available for operation
\item[{\sf GrB\_DIMENSION\_MISMATCH}] mismatch among vectors, matrix dimensions.
\item[{\sf GrB\_DOMAIN\_MISMATCH}]    mismatch among domains of vectors, matrix, and/or semiring, for which the descriptor did not explicitly allow casting.
\end{itemize}


\paragraph{Description}

Vectors $\vector{v}, \vector{mask}$ and matrix $\matrix{A}$ are computed from
input parameters {\sf v}, {\sf mask} and {\sf A}, respectively, as specified
by descriptor {\sf desc}. (See below for the properties of a descriptor. In
the simplest form, these are just copies, but additional preprocessing,
including casting and transposition, can be performed.)  $\bold{D}(\vector{v}) =
\bold{D}_1({\sf op})$ and $\bold{D}(\matrix{A}) = \bold{D}_2({\sf op})$.
$\bold{D}(\vector{mask}) = {\sf GrB\_BOOL}$.  If {\sf mask} is {\sf GrB\_NULL} or omitted,
then $\vector{mask}$ is a Boolean vector of size $\bold{m}(\vector{A})$
and with all elements set to {\sf true}.

If either $\vector{v}, \vector{mask}$ or $\matrix{A}$ cannot be computed
from the input parameters as described above, the method returns {\sf
GrB\_DOMAIN\_MISMATCH}.

A consistency check is performed to verify that $\bold{n}(\vector{v})
= \bold{n}(\matrix{A})$ and $\bold{n}(\vector{mask}) =
\bold{n}(\vector{u}) = \bold{m}(\matrix{A})$. If a consistency check fails, the operation is
aborted and the method returns {\sf GrB\_DIMENSION\_MISMATCH}.

A new vector $\vector{u} = \langle \bold{D}_3({\sf op}),
\bold{m}(\matrix{A}), \bold{L}(\vector{u}) = \{(i,u_i) : \vector{mask}(i)
= {\sf true} \} \rangle$ is created.  The value of each of its elements
is computed by $u_i = \bigoplus_{j \in \vector{i}(\vector{v}) \cap
\vector{i}(\matrix{A}(i,:))} (\matrix{A}(i,j)) \otimes \vector{v}(j)$,
where $\oplus$ and $\otimes$ are the additive and multiplicative
operators of semiring {\sf op}, respectively.  If $\vector{i}(\vector{v})
\cap \vector{i}(\matrix{A}(i,:)) = \emptyset$ then the pair $(i,u_i)$
is not included in $\bold{L}(\vector{u})$.

Finally, output parameter {\sf u} is computed from vector $\vector{u}$
as specified by descriptor {\sf desc}. (Again, in the simplest case this
is just a copy, but additional postprocessing, including casting and
accumulation of result values, can be specified.)  A consistency check is
performed to verify that $\bold{n}({\sf u}) = \bold{n}(\vector{u})$. If
the consistency check fails, the operation is aborted and the method
return {\sf GrB\_DIMENSION\_MISMATCH}.

\subsubsection{{\sf eWiseMult}: Element-wise multiplication}

{\bf Note:} The difference between {\sf eWiseAdd} and {\sf eWiseMult} is not 
about the semiring operation but how the index sets are treated.
{\sf eWiseAdd} returns an object whose indices are the ``union'' of the 
indices of the inputs whereas {\sf eWiseMult} returns an object whose indices 
are the ``intersection'' of the indices of the inputs. In both cases, the 
passed monoid (or function) operates on the set of values from the 
intersection set. 
 
\paragraph{Vector variant}

Perform element-wise (general) multiplication on the elements of two vectors,
producing a third vector as result.

\subparagraph{C99 Syntax}

\begin{verbatim}
        GrB_info GrB_eWiseMult(GrB_Vector               w,
                               const GrB_Monoid         op, 
                               const GrB_Vector         u,
                               const GrB_Vector         v
                            [, const GrB_MaskVector     mask
                            [, const GrB_Descriptor     desc
                            [, const GrB_BinaryFunction accum]]]);
                            
        GrB_info GrB_eWiseMult(GrB_Vector               w,
                               const GrB_BinaryFunction op, 
                               const GrB_Vector         u,
                               const GrB_Vector         v
                            [, const GrB_MaskVector     mask
                            [, const GrB_Descriptor     desc
                            [, const GrB_BinaryFunction accum]]]);
\end{verbatim}

\subparagraph{Parameters}

\begin{itemize}[leftmargin=1.1in]
    \item[{\sf w}]     ({\sf OUTP}) An existing vector to hold the result.
    \item[{\sf op}]    ({\sf ARG0}) monoid/function used in the element-wise multiplication.
    \item[{\sf u}]     ({\sf ARG1}) Left hand operand.
    \item[{\sf v}]     ({\sf ARG2}) Right hand operand.

    \item[{\sf mask}] ({\sf MASK}) Output mask vector (optional). The mask
    specifies which elements of the result vector can be modified.
    If no mask is necessary (i.e., compute all elements of result
    vector), {\sf GrB\_NULL} can be used or the mask can be omitted.

    \item[{\sf desc}]  Operation descriptor (optional). If a
    \emph{default} descriptor is desired, {\sf GrB\_NULL} can be
    used or the descriptor can be omitted. Valid fields are as follows: \\
    \begin{tabular}{lll}
    Field  & Value & Description \\
    \hline
    {\sf OUTP} & {\sf GrB\_CAST} & Allow casting from $\bold{D}_3({\sf op})$ to $\bold{D}(\vector{w})$. \\
    {\sf ARG1} & {\sf GrB\_CAST} & Allow casting from $\bold{D}(\vector{u})$ to $\bold{D}_1({\sf op})$. \\
    {\sf ARG2} & {\sf GrB\_CAST} & Allow casting from $\bold{D}(\vector{v})$ to $\bold{D}_2({\sf op})$. \\
    {\sf MASK} & {\sf GrB\_SCMP} & Use the structural complement of {\sf mask}. \\
    {\sf ACCUM}& {\sf GrB\_ACC}  & Use the {\sf accum} function to add to existing values in {\sf C}.\\
    \end{tabular}

  	\item[{\sf accum}]  Function used for accumulating entries with existing \vector{w} entries. (optional).
\end{itemize}

\subparagraph{Return Value}

\begin{itemize}[leftmargin=2.1in]
\item[{\sf GrB\_SUCCESS}]             operation completed successfully
\item[{\sf GrB\_PANIC}]               unknown internal error
\item[{\sf GrB\_OUTOFMEM}]            not enough memory available for operation
\item[{\sf GrB\_DIMENSION\_MISMATCH}] mismatch among vectors dimensions.
\item[{\sf GrB\_DOMAIN\_MISMATCH}]    mismatch among domains of vectors, and operation, for which the descriptor did not explicitly allow casting.
\end{itemize}

\subparagraph{Description}

%If {\sf op} is a semiring, then $\otimes = \bigotimes({\sf op})$. 
%If {\sf op} is a monoid or function, then $\otimes = \bigoplus({\sf op})$.

The binary operation, {\sf op}, whether it is a monoid or a binary function
is generically referred to as $\otimes$ in this discussion.

Vectors $\vector{v}, \vector{m}$ and $\vector{u}$ are computed from
input parameters {\sf v}, {\sf m} and {\sf u}, respectively, as specified
by descriptor {\sf d}. (See below for the properties of a descriptor. In
the simplest form, these are just copies, but additional preprocessing,
including casting, can be specified.)  $\bold{D}(\vector{u}) =
\bold{D}_1({\sf op})$ and $\bold{D}(\vector{v}) = \bold{D}_2({\sf op})$.
$\bold{D}(\vector{m}) = {\sf GrB\_BOOL}$.  If {\sf m} is {\sf GrB\_NULL} or omitted,
then $\vector{m}$ is a Boolean vector of size $\bold{n}(\vector{u})$
and with all elements set to {\sf true}.

If either $\vector{v}, \vector{m}$ or $\vector{u}$ cannot be computed
from the input parameters as described above, the method returns {\sf
GrB\_DOMAIN\_MISMATCH}.

A consistency check is performed to verify that $\bold{n}(\vector{v})
= \bold{n}(\vector{u}) = \bold{n}(\vector{m})$. If a consistency
check fails, the operation is aborted and the method returns {\sf
GrB\_DIMENSION\_MISMATCH}.

A new vector $\vector{w} = \langle \bold{D}_3({\sf op}),
\bold{n}(\vector{u}), \bold{L}(\vector{w}) = \{(i,w_i)  \forall i \in
\vector{i}(\vector{v}) \cap \vector{i}(\vector{u}) : \vector{m}(i)
= {\sf true} \} \rangle$ is created.  The value of each of its
elements is computed by $w_i = \vector{u}(i) \otimes \vector{v}(i)$,
where $\otimes$ is as defined above from {\sf op}.
If $\vector{i}(\vector{v}) \cap \vector{i}(\vector{u}) = \emptyset$
then $\bold{L}(\vector{w}) = \emptyset$.

Finally, output parameter {\sf w} is computed from vector $\vector{w}$
as specified by descriptor {\sf d}. (Again, in the simplest case this
is just a copy, but additional postprocessing, including casting and
accumulation of result values, can be specified.)  A consistency check is
performed to verify that $\bold{n}({\sf w}) = \bold{n}(\vector{w})$. If
the consistency check fails, the operation is aborted and the method
return {\sf GrB\_DIMENSION\_MISMATCH}.

%-----------------------------------------------------------------------------

\paragraph{Matrix variant}

Placeholder

\aydin{Aydin to fill}


\subparagraph{C99 Syntax}

\begin{verbatim}
        GrB_info GrB_eWiseMult(GrB_Matrix               C,
                               const GrB_Monoid         op, 
                               const GrB_Matrix         A,
                               const GrB_Matrix         B
                            [, const GrB_MaskMatrix     mask
                            [, const GrB_Descriptor     desc
                            [, const GrB_BinaryFunction accum]]]);
                            
        GrB_info GrB_eWiseMult(GrB_Matrix               C,
                               const GrB_BinaryFunction op, 
                               const GrB_Matrix         A,
                               const GrB_Matrix         B
                            [, const GrB_MaskMatrix     mask
                            [, const GrB_Descriptor     desc
                            [, const GrB_BinaryFunction accum]]]);
\end{verbatim}


%-----------------------------------------------------------------------------


\subsubsection{{\sf eWiseAdd}: Element-wise addition}

{\bf Note:} The difference between {\sf eWiseAdd} and {\sf eWiseMult} is not about the semiring operation but how the index sets are treated.
 {\sf eWiseAdd} returns an object whose indices are the ``union'' of the indices of the inputs whereas  
 {\sf eWiseMult} returns an object whose indices are the ``intersection'' of the indices of the inputs. In both cases, the passed monoid (or function) operates on the 
 set of values from the intersection set. 

\paragraph{Vector variant}

Perform element-wise (general) addition on the elements of two vectors,
producing a third vector as result.

\subparagraph{C99 Syntax}

\begin{verbatim}
        GrB_info GrB_eWiseAdd(GrB_Vector               w,
                              const GrB_Monoid         op, 
                              const GrB_Vector         u,
                              const GrB_Vector         v
                           [, const GrB_MaskVector     mask
                           [, const GrB_Descriptor     desc
                           [, const GrB_BinaryFunction accum]]]);
                            
        GrB_info GrB_eWiseAdd(GrB_Vector               w,
                              const GrB_BinaryFunction op, 
                              const GrB_Vector         u,
                              const GrB_Vector         v
                           [, const GrB_MaskVector     mask
                           [, const GrB_Descriptor     desc
                           [, const GrB_BinaryFunction accum]]]);
\end{verbatim}

\subparagraph{Parameters}

\begin{itemize}[leftmargin=1.1in]
    \item[{\sf w}]     ({\sf OUTP}) An existing vector to hold the result.
    \item[{\sf op}]    ({\sf ARG0}) monoid/function used in the element-wise addition.
    \item[{\sf u}]     ({\sf ARG1}) Left hand operand.
    \item[{\sf v}]     ({\sf ARG2}) Right hand operand.

    \item[{\sf mask}] ({\sf MASK}) Output mask vector (optional). The mask
    specifies which elements of the result vector can be modified.
    If no mask is necessary (i.e., compute all elements of result
    vector), {\sf GrB\_NULL} can be used or the mask can be omitted.

    \item[{\sf desc}]  Operation descriptor (optional). If a
    \emph{default} descriptor is desired, {\sf GrB\_NULL} can be
    used or the descriptor can be omitted. Valid fields are as follows: \\
    \begin{tabular}{lll}
    Field  & Value & Description \\
    \hline
    {\sf OUTP} & {\sf GrB\_CAST} & Allow casting from $\bold{D}_3({\sf op})$ to $\bold{D}(\vector{w})$. \\
    {\sf ARG1} & {\sf GrB\_CAST} & Allow casting from $\bold{D}(\vector{u})$ to $\bold{D}_1({\sf op})$. \\
    {\sf ARG2} & {\sf GrB\_CAST} & Allow casting from $\bold{D}(\vector{v})$ to $\bold{D}_2({\sf op})$. \\
    {\sf MASK} & {\sf GrB\_SCMP} & Use the structural complement of {\sf mask}. \\
    {\sf ACCUM}& {\sf GrB\_ACC}  & Use the {\sf accum} function to add to existing values in {\sf C}.\\
    \end{tabular}

  	\item[{\sf accum}]  Function used for accumulating entries with existing \vector{w} entries. (optional).
\end{itemize}

\subparagraph{Return Values}

\begin{itemize}[leftmargin=2.1in]
\item[{\sf GrB\_SUCCESS}]             operation completed successfully
\item[{\sf GrB\_PANIC}]               unknown internal error
\item[{\sf GrB\_OUTOFMEM}]            not enough memory available for operation
\item[{\sf GrB\_DIMENSION\_MISMATCH}] mismatch among vectors dimensions.
\item[{\sf GrB\_DOMAIN\_MISMATCH}]    mismatch among domains of vectors, and operation, for which the descriptor did not explicitly allow casting.
\end{itemize}

\subparagraph{Description}

%If {\sf op} is a semiring, then $\oplus = \bigoplus({\sf op})$. 
%If {\sf op} is a monoid or function, then $\oplus = \bigoplus({\sf op})$.

The binary operation, {\sf op}, whether it is a monoid or a binary function
is generically referred to as $\oplus$ in this discussion.

Vectors $\vector{v}, \vector{m}$ and $\vector{u}$ are computed from
input parameters {\sf v}, {\sf m} and {\sf u}, respectively, as specified
by descriptor {\sf d}. (See below for the properties of a descriptor. In
the simplest form, these are just copies, but additional preprocessing,
including casting, can be specified.)  $\bold{D}(\vector{u}) =
\bold{D}_3({\sf op})$ and $\bold{D}(\vector{v}) = \bold{D}_3({\sf op})$.
$\bold{D}(\vector{m}) = {\sf GrB\_BOOL}$.  If {\sf m} is {\sf GrB\_NULL} or omitted,
then $\vector{m}$ is a Boolean vector of size $\bold{n}(\vector{u})$
and with all elements set to {\sf true}.

If either $\vector{v}, \vector{m}$ or $\vector{u}$ cannot be computed
from the input parameters as described above, the method returns {\sf
GrB\_DOMAIN\_MISMATCH}.

A consistency check is performed to verify that $\bold{n}(\vector{v})
= \bold{n}(\vector{u}) = \bold{n}(\vector{m})$. If a consistency check fails, the operation is
aborted and the method returns {\sf GrB\_DIMENSION\_MISMATCH}.

A new vector $\vector{w} = \langle \bold{D}_3({\sf op}),
\bold{n}(\vector{u}), \bold{L}(\vector{w}) = \{(i,w_i)  \forall i \in
\vector{i}(\vector{v}) \cup \vector{i}(\vector{u}) : \vector{m}(i)
= {\sf true} \} \rangle$ is created.  The value of each of its
elements is computed by 
\[
w_i = \vector{u}(i) \oplus \vector{v}(i), \ \mbox{if}\  i \in  \vector{i}(\vector{v}) \cap \vector{i}(\vector{u})
\]
\[
w_i = \vector{u}(i) \ \mbox{if}\  i \in  \vector{i}(\vector{u}) - (\vector{i}(\vector{v}) \cap \vector{i}(\vector{u}))
\]
\[
w_i = \vector{v}(i) \ \mbox{if}\  i \in  \vector{i}(\vector{v}) - (\vector{i}(\vector{v}) \cap \vector{i}(\vector{u}))
\]
where $\oplus$ is as defined above for {\sf op}.
If $\vector{i}(\vector{v}) \cup \vector{i}(\vector{u}) = \emptyset$
then $\bold{L}(\vector{w}) = \emptyset$.

Finally, output parameter {\sf w} is computed from vector $\vector{w}$
as specified by descriptor {\sf d}. (Again, in the simplest case this
is just a copy, but additional postprocessing, including casting and
accumulation of result values, can be specified.)  A consistency check is
performed to verify that $\bold{n}({\sf w}) = \bold{n}(\vector{w})$. If
the consistency check fails, the operation is aborted and the method
return {\sf GrB\_DIMENSION\_MISMATCH}.

%-----------------------------------------------------------------------------

\paragraph{Matrix variant}

Placeholder

\aydin{Aydin to fill}


\subparagraph{C99 Syntax}

\begin{verbatim}
        GrB_info GrB_eWiseAdd(GrB_Matrix               C,
                              const GrB_Monoid         op, 
                              const GrB_Matrix         A,
                              const GrB_Matrix         B
                           [, const GrB_MaskMatrix     mask
                           [, const GrB_Descriptor     desc
                           [, const GrB_BinaryFunction accum]]]);
                            
        GrB_info GrB_eWiseAdd(GrB_Matrix               C,
                              const GrB_BinaryFunction op, 
                              const GrB_Matrix         A,
                              const GrB_Matrix         B
                           [, const GrB_MaskMatrix     mask
                           [, const GrB_Descriptor     desc
                           [, const GrB_BinaryFunction accum]]]);
\end{verbatim}


\subsubsection{{\sf extract}: Selecting Sub-Graphs}

Extract a sub-matrix from a larger matrix. 

%-----------------------------------------------------------------------------
\paragraph{Standard Matrix and Vector Versions}

In the standard version of {\sf extract} GraphBLAS index arrays (dense vectors)
specify the locations in the source vector/matrix that should be copied to the
destination.  If a structural zero exists at a specified location in the source,
the corresponding location in dst will be cleared.  For vectors, only one index array is used to specify
locations, and for matrices two index arrays (for row and column indices are needed).
The size of the destination vector is the same size as the one index array provided.
For matrices, the size of the destination matrix has the same number of rows as the
{\sf rows} index array and the same number of columns as the {\sf cols} index array.

Normally elements selected from source will be replicated in the destination.  If the
destination is not empty, this operation WILL NOT create structural zeros where there
are stored values in the destination even if the corresponding location in source has
a structural zero.

If the {\sf accum} function is specified and there are stored locations in correspondning
locations in both source and destination, then the
{\sf accum} is used to combine both values before overwriting the destination value
with the result.  If source and destination values collide and the {\sf accum} function is
not specified, then the destination value will be replaced with the source value (the
default behaviour of {\sf accum}).

\subparagraph{C99 Syntax}

\begin{verbatim}
GrB_info GrB_extract(GrB_Vector          *dst,
                     const GrB_Function   accum,
                     const GrB_Vector     src,
                     const GrB_IndexArray rows
                  [, const GrB_Vector     mask
                  [, const GrB_Descriptor desc]]);
                  
GrB_info GrB_extract(GrB_Matrix          *dst,
                     const GrB_Function   accum,
                     const GrB_Matrix     src,
                     const GrB_IndexArray rows,
                     const GrB_IndexArray cols
                  [, const GrB_Matrix     mask
                  [, const GrB_Descriptor desc]]);
\end{verbatim}

\subparagraph{Parameters}

\begin{itemize}[leftmargin=1in]
	\item[{\sf dst (ARG0)}]   The matrix or vector to assign the extracted subgraph.
	\item[{\sf accum (ARG1)}] Function used for accumulation into dst.  {\sf GrB\_NULL}
                              can be used if no accumulation into dst is used.
	\item[{\sf src} (ARG2)]   The matrix or vector from which to extract the subgraph.
	\item[{\sf rows} (ARG3)]  The set of row indices specifying locations from src that
                              are assigned to dst. Can
                              be set to {\sf GrB\_ALL} if all rows are
                              to be extracted.
	\item[{\sf cols} (ARG4)]  (Matrix version only) The set of column indices specifying
                              locations from src that are assigned to dst. Can
                              be set to {\sf GrB\_ALL} if all columns are
                              to be extracted.

	\item[{\sf mask} (MASK)]  Operation mask (optional). The mask
	specifies which elements of the result vector can be assigned.
	If no mask is necessary (i.e., compute all elements of result
	vector), {\sf GrB\_NULL} can be used or the mask can be omitted.

	\item[{\sf desc}] Operation descriptor (optional). The descriptor
    is used to specify details of the operation. Valid options are transpose
    of src ({\sf ARG2}), and negate (structural complement) of mask ({\sf MASK}). If
    a \emph{default} descriptor is desired,	{\sf GrB\_NULL} can be
    used or the descriptor can be omitted. {\scott if we negate src what values
    are we storing in dst?}
\end{itemize}

\subparagraph{Return Values}

\scott{Are invalid/unused descriptors an error or ignored?}

\begin{itemize}[leftmargin=2.1in]
\item[{\sf GrB\_SUCCESS}] 	operation completed successfully.
\item[{\sf GrB\_PANIC}]	    unknown internal error.
\item[{\sf GrB\_OUTOFMEM}]	not enough memory available for operation.
\item[{\sf GrB\_DIMENSION\_MISMATCH}] 
        The size of rows is not equal to the number of rows in dst, or
        the size of cols is not equal to the number of columns in dst (matrix version).
\item[{\sf GrB\_INDEX\_OUTOFBOUNDS}]
        A value in rows references a non-existent row in src, or
	    the value in cols references a non-existent column in src (matrix version).
\item[\sf GrB\_DOMAIN\_MISMATCH]  
	   mismatch among elements of vectors or matrices or the accum function. \scott{elaborate}
\end{itemize}


\subparagraph{Description}

TBD

%-----------------------------------------------------------------------------
\paragraph{Row and Column Variants}

Extract from one column or one row of a matrix into a vector. 

\scott{Depending on how we treat Vectors we may not need both. If a vector is
strictly column oriented, the to extract a row, we should use the column version
and set the Descriptor for ARG2 (src) to transpose.}


\subparagraph{C99 Syntax}

\begin{verbatim}
// extract a row
GrB_info GrB_extract(GrB_Vector          *dst, 
                     GrB_Function const   accum,
                     GrB_Matrix const     src,
                     GrB_Index            row, // row index
                     GrB_IndexArray       cols
                  [, GrB_Vector const     mask
                  [, GrB_Descriptor const desc]]);

// extract a column
GrB_info GrB_extract(GrB_Vector          *dst, 
                     GrB_Function const   accum,
                     GrB_Matrix const     src, 
                     GrB_IndexArray       rows
                     GrB_Index            col // column index
                  [, GrB_Vector const     mask
                  [, GrB_Descriptor const desc]]);
\end{verbatim}

\subparagraph{Parameters}

\begin{itemize}[leftmargin=1in]
	\item[{\sf dst} (ARG0)]   The scalar into which to assign the extracted value.
	\item[{\sf accum} (ARG1)] Function used for accumulation into dst.  {\sf GrB\_NULL}
                              can be used if no accumulation into dst is used.
	\item[{\sf src} (ARG2)]   The matrix from which to extract the column.

    \item[{\sf row} (ARG3)]   (extract row version) The index of the row to extract.
    \item[{\sf cols (ARG4}]   (extract row version) The set of column indices to extract. Can
                              be set to {\sf GrB\_ALL} if all columns are
                              to be extracted.
   
    \item[{\sf rows (ARG3}]   (extract col version) The set of row indices to extract. Can
                              be set to {\sf GrB\_ALL} if all rows are
                              to be extracted.
	\item[{\sf col} (ARG4)]   (extract col version) The index of the column to extract.

	\item[{\sf mask} (MASK)]  Operation mask (optional). The mask
	specifies which elements of the result vector can be assigned.
	If no mask is necessary (i.e., compute all elements of result
	vector), {\sf GrB\_NULL} can be used or the mask can be omitted.

	\item[{\sf desc}] Operation descriptor (optional). The descriptor
    is used to specify details of the operation. Valid options are transpose
    of src ({\sf ARG2}), and invert (structural complement) of mask ({\sf MASK}). If
    a \emph{default} descriptor is desired,	{\sf GrB\_NULL} can be
    used or the descriptor can be omitted.
\end{itemize}

\subparagraph{Return Values}

\begin{itemize}[leftmargin=2.1in]
\item[{\sf GrB\_SUCCESS}] 	          Operation completed successfully.
\item[{\sf GrB\_PANIC}]	              Unknown internal error.
\item[{\sf GrB\_INDEX\_OUTOFBOUNDS}]  The indexes specify a position that outside the dimensions of src.
\item[{\sf GrB\_DOMAIN\_MISMATCH}]    Mismatch between vector/matrix domain and scalar type.
\item[{\sf GrB\_DIMENSION\_MISMATCH}] 
        The size of rows is greater than the size of dst (column version), or
        the size of cols is greater than the size in dst (row version).
\end{itemize}

\subparagraph{Description}

TBD

%-----------------------------------------------------------------------------
\paragraph{Single Element Variants}

Extract one element of a vector/matrix into a scalar.  

\scott{This is an attempt to mirror the single value assign.}
\scott{If the indexed position is a structural zero, I have opted to return an error
in lieu of deciding where the additive identity is coming from.}  


\subparagraph{C99 Syntax}

\begin{verbatim}
GrB_info GrB_extract(scalar              *dst, 
                     GrB_Function const   accum,
                     GrB_Vector const     src,
                     GrB_Index            row
                  [, GrB_Descriptor const desc]]);

GrB_info GrB_extract(scalar              *dst,
                     GrB_Function const   accum,
                     GrB_Matrix const     src,
                     GrB_Index            row,
                     GrB_Index            col
                  [, GrB_Descriptor const desc]]);

\end{verbatim}

\subparagraph{Parameters}

\begin{itemize}[leftmargin=1in]
	\item[{\sf dst (ARG0)}]   The scalar into which to assign the extracted value.
	\item[{\sf accum (ARG1)}] Function used for accumulation into dst.  {\sf GrB\_NULL}
                              can be used if no accumulation into dst is used.
	\item[{\sf src (ARG2)}]   The matrix or vector from which to extract the scalar.
	\item[{\sf row (ARG3)}]   The row index of the location to extract.
	\item[{\sf col (ARG4)}]   (Matrix version only) The column index of location to extract.
    \item[{\sf desc}]         Operation descriptor (optional). The descriptor
                              is used to specify details of the operation. Valid option is
                              transpose of src ({\sf ARG2}). If a \emph{default} descriptor
                              is desired,	{\sf GrB\_NULL} can be used or the descriptor
                              can be omitted.  \scott{I don't know if a descriptor is necessary here.}
\end{itemize}

\subparagraph{Return Values}

\begin{itemize}[leftmargin=2.1in]
\item[{\sf GrB\_SUCCESS}] 	          Operation completed successfully.
\item[{\sf GrB\_PANIC}]	              Unknown internal error.
\item[{\sf GrB\_NO\_VALUE}]           No stored value at specified location (is it an error?).
\item[{\sf GrB\_INDEX\_OUTOFBOUNDS}]  The indexes specify a position that does not exist in src.
\item[{\sf GrB\_DOMAIN\_MISMATCH}]    Mismatch between vector/matrix domain and scalar type.
\end{itemize}

\subparagraph{Description}

{\scott I don't think masks need to be supported for this one}
%-----------------------------------------------------------------------------



%-----------------------------------------------------------------------------
%-----------------------------------------------------------------------------
\subsubsection{{\sf assign}: Modifying Sub-Graphs}

Assign a matrix to a set of indices (sub-matrix) of a larger matrix

\scott{the variants beyond the standard version need to be discussed perhaps in the large group; not currently part of any prior document.}

%-----------------------------------------------------------------------------
\paragraph{Standard Matrix and Vector Versions}

In the standard version of {\sf assign} GraphBLAS index arrays (dense vectors)
specify the locations in the destination vector/matrix that should be assign from
the source.  If a structural zero exists at a specified location in the source,
no value will be copied.  For vectors, only one index array is used to specify
locations, and for matrices two index arrays (for row and column indices are needed).
The size of the source vector is the same size as the one index array provided.
For matrices, the size of the source matrix has the same number of rows as the
{\sf i} index array and the same number of columns as the {\sf j} index array.

Normally elements selected from source will be replicated in the destination.  If the
destination is not empty, this operation WILL NOT create structural zeros where there
are stored values in the destination even if the corresponding location in source 
has a structural zero.

If the {\sf accum} function is specified and there are stored locations in correspondning
locations in both source and destination, then the
{\sf accum} is used to combine both values before overwriting the destination value
with the result.  If source and destination values collide and the {\sf accum} function is
not specified, then the destination value will be replaced with the source value (the
default behaviour of {\sf accum}).

\subparagraph{C99 Syntax}

\begin{verbatim}
GrB_info GrB_assign(GrB_Vector *dst, const GrB_Function accum, const GrB_Vector src,
                    const GrB_IndexArray i
                    [, const GrB_Vector mask[, const GrB_Descriptor desc]]);
GrB_info GrB_assign(GrB_Matrix *dst, const GrB_Function accum, const GrB_Matrix src,
                    const GrB_IndexArray i, const GrB_IndexArray j
                    [, const GrB_Vector mask[, const GrB_Descriptor desc]]);
\end{verbatim}

\subparagraph{Input Parameters}

\begin{itemize}
	\item[{\sf dst}]   ({\sf ARG0}) The matrix or vector into which to assign the subgraph.
	\item[{\sf accum}] ({\sf ARG1}) Function used for accumulation into dst.  {\sf GrB\_NULL}
                       can be used if no accumulation into dst is desired.
	\item[{\sf src}]   ({\sf ARG2}) The matrix or vector containing the subgraph.
	\item[{\sf i}]     ({\sf ARG3}) The set of row indices specifying locations in dst that
                       are assigned from src.
	\item[{\sf j}]     ({\sf ARG4}) (Matrix version only) The set of column indices specifying
                       locations in dst that are assigned from src.

	\item[{\sf mask}]  ({\sf MASK}) Operation mask (optional). The mask
	specifies which elements of the result vector can be assigned.
	If no mask is necessary (i.e., compute all elements of result
	vector), {\sf GrB\_NULL} can be used or the mask can be omitted.

	\item[{\sf desc}]  Operation descriptor (optional). The descriptor
    is used to specify details of the operation. Valid options are transpose
    of src ({\sf ARG2}), and invert (structural complement) of src ({\sf ARG2}). If
    a \emph{default} descriptor is desired,	{\sf GrB\_NULL} can be
    used or the descriptor can be omitted.
\end{itemize}

\subparagraph{Return Value}

\scott{Are invalid/unused descriptors an error?}

\scott{TODO: fix indent}

\begin{itemize}
\item[{\sf GrB\_SUCCESS}] 	operation completed successfully.
\item[{\sf GrB\_PANIC}]	    unknown internal error.
\item[{\sf GrB\_OUTOFMEM}]	not enough memory available for operation.
\item[{\sf GrB\_DIMENSION\_MISMATCH}] 
        The size of i is greater than the number of rows in src, or
        the size of j is greater than the number of columns in src (matrix version).
\item[{\sf GrB\_INDEX\_OUTOFBOUNDS}]
        A value in i references a nonexistent row in dst, or
	    the value in j references a nonexistent column in dst (matrix version).
\item[\sf GrB\_MISMATCH]  
	   mismatch among vectors, matrix and/or semiring \scott{elaborate}
\end{itemize}


%-----------------------------------------------------------------------------
\paragraph{{\sf assign}: Single-Value Variant (was Indexed Variant)}

Set one element of a vector/matrix to a given value.
\scott{I have tried to make this more consistent with the standard version by
adding things like accum. If the destination location is a structural zero, is
a stored value created?.
This could be construed as inconsistent behaviour with standard version.}

\subparagraph{C99 Syntax}

\begin{verbatim}
#include "GraphBLAS.h"
GrB_info GrB_assign(GrB_Vector *dst, const GrB_Function accum, scalar src,
                    GrB_index i);
GrB_info GrB_assign(GrB_Matrix *dst, const GrB_Function accum, scalar src,
                    GrB_index i, GrB_index j);
\end{verbatim}

\subparagraph{Input Parameters}

\begin{itemize}
	\item[{\sf dst}] Vector/matrix for which an element is to be assigned.
	\item[{\sf accum}] Function used for accumulation into dst.  {\sf GrB\_NULL}
                       can be used if no accumulation into dst is desired.
	\item[{\sf src}] Scalar value to assign to the element.
	\item[{\sf i}]   Row index of element to be assigned
	\item[{\sf j}]   Column index of element to be assigned (matrix version)
\end{itemize}

\subparagraph{Return Value}~

\begin{tabular}{rl}
{\sf GrB\_SUCCESS}	& operation completed successfully \\
{\sf GrB\_PANIC}	& unknown internal error \\
{\sf GrB\_NOVECTOR}	& vector does not exist \\
{\sf GrB\_INDEX\_OUTOFBOUNDS} & The indexes specify a position that does not exist in dst. \\
{\sf GrB\_DOMAIN\_MISMATCH}	& mismatch between vector domain and scalar type \\
\end{tabular}

%-----------------------------------------------------------------------------
\paragraph{{\sf assign}: Constant variant (was Flat Variant)}

Set ALL(?) of stored elements of a vector/matrix to a specified constant value.
This only affects stored values in dst

{\scott added accum functionality}

\begin{verbatim}
#include "GraphBLAS.h"
GrB_info GrB_assign(GrB_Vector *dst, const GrB_Function accum, scalar s[, const GrB_Vector mask])
GrB_info GrB_assign(GrB_Matrix *dst, const GrB_Function accum, scalar s[, const GrB_Mask mask])
\end{verbatim}

\subparagraph{Input Parameters}

\begin{itemize}
	\item[{\sf dst}]   Vector/Matrix to be assigned.
	\item[{\sf accum}] Function used for accumulation into dst.  {\sf GrB\_NULL}
                       can be used if no accumulation into dst is desired.
	\item[{\sf s}]     Scalar value for the elements.
	\item[{\sf mask}]  (Optional) mask for assignment. \aydin{Maybe say in the document that GrB\_Vector's domain could only be GrB\_Index for this function} \jose{Any domain that can be cast to {\sf GrB\_BOOL} will do.}
\end{itemize}

\subparagraph{Return Value}~

\begin{tabular}{rl}
{\sf GrB\_SUCCESS}	& operation completed successfully \\
{\sf GrB\_PANIC}	& unknown internal error \\
{\sf GrB\_NOVECTOR}	& vector does not exist \\
{\sf GrB\_MISMATCH}	& mismatch between vector domain and scalar type \\
\end{tabular}

\subsubsection{{\sf apply}: Apply a unary function to the elements of a matrix}


\subparagraph{C99 Syntax}

\begin{verbatim}
        // Vector version
        GrB_info GrB_apply(GrB_Vector                *dst,
                           const GrB_Vector           src,
                           GrB_UnaryFunction          func
                        [, const GrB_MaskVector       mask
                        [, const GrB_Descriptor       desc
                        [, const GrB_BinaryFunction   accum]]]);

        // Matrix version
        GrB_info GrB_apply(GrB_Matrix                *dst,
                           const GrB_Matrix           src,
                           GrB_UnaryFunction          func
                        [, const GrB_MaskMatrix       mask
                        [, const GrB_Descriptor       desc
                        [, const GrB_BinaryFunction   accum]]]);
\end{verbatim}

\subparagraph{Input Parameters}

\begin{itemize}[leftmargin=1.1in]
    \item[{\sf dst}]   ({\sf OUTP}) The matrix/vector to assign the result.
    \item[{\sf src}]   ({\sf ARG0}) The matrix to transpose.
    \item[{\sf func}]  ({\sf ARG1}) The unary function to apply to each stored element in src.

    \item[{\sf mask}]  ({\sf MASK}) Output mask (optional). The mask
    specifies which elements of {\sf dst} can be assigned.
    If no mask is necessary (i.e., compute all elements of result),
    {\sf GrB\_NULL} can be used or the mask can be omitted.

    \item[{\sf desc}]   Operation descriptor (optional). If a
    \emph{default} descriptor is desired, {\sf GrB\_NULL} can be
    used or the descriptor can be omitted.  Valid fields and values are as follows: \\
    \begin{tabular}{lll}
    Field  & Value & Description \\
    \hline
    {\sf ARG0} & {\sf GrB\_CAST} & Allow casting from $\bold{D}({\sf src})$ to $\bold{D}({\sf dst})$ \\
    {\sf ARG0} & {\sf GrB\_TRAN} & Transpose {\sf src} \\
    {\sf MASK} & {\sf GrB\_SCMP} & Use the structural complement of {\sf mask}. \\
    {\sf OUTP}& {\sf GrB\_ACC}  & Use the {\sf accum} function to combine with existing values in {\sf dst}.\\
    \end{tabular}

    \item[{\sf accum}] Function used for accumulation into dst.  {\sf GrB\_NULL}
                       can be used if no accumulation into dst is desired.
\end{itemize}

\subparagraph{Return Value}

\scott{Are invalid descriptors an error or ignored?}

\begin{itemize}[leftmargin=2.1in]
\item[{\sf GrB\_SUCCESS}]     operation completed successfully.
\item[{\sf GrB\_PANIC}]        unknown internal error.
\item[{\sf GrB\_DIMENSION\_MISMATCH}]            
        If the size/shape of dst is not the same as either mask or
        of the transpose of src.
\item[{\sf GrB\_DOMAIN\_MISMATCH}]  
        domain mismatch among matrices, unary function, and/or
        accum function \scott{elaborate}
\end{itemize}

\subparagraph{Description}

The dst matrix must have already been created with the proper dimensions
prior to calling this function.  The unary function operators on stored values in src and the results are assigned to corresponding location in dst.
Src can be optionally transposed first.  The results of the unary function can be optionally accumulated with existing values in dst using accum.  Assignment into dst can be optionally masked.

%=========================================================================

\subsubsection{{\sf reduce}: Perform a reduction across the elements of an object}

Computes the reduction of the values of the elements of a vector or matrix.  There are a number of different variants.

%-----------------------------------------------------------------------------
\paragraph{Matrix to vector variants}

Placeholder

\aydin{Aydin to fill}

\begin{verbatim}
        // reduce rows to a column vector
        GrB_info GrB_rowReduce(GrB_Vector               *dst,
                               const GrB_BinaryFunction  op,   // or monoid
                               const GrB_Matrix          A
                            [, const GrB_MaskVector      mask
                            [, const GrB_Descriptor      desc
                            [, const GrB_BinaryFunction  accum]]])
                               
        // reduce columns to a row vector (not necessary if use transpose descriptor)
        //GrB_info GrB_colReduce(GrB_Vector               *dst,
        //                       const GrB_BinaryFunction  op,   // or monoid
        //                       const GrB_Matrix          A
        //                    [, const GrB_MaskVector      mask
        //                    [, const GrB_Descriptor      desc
        //                    [, const GrB_BinaryFunction  accum]]])
\end{verbatim}

%-----------------------------------------------------------------------------
\paragraph{Reduce to scalar variants}

\subparagraph{C99 Syntax}

\begin{verbatim}
        // semiring version, to be removed?
        //GrB_info GrB_reduce(scalar               *dst,
        //                    const GrB_Semiring    sr,
        //                    const GrB_Vector      src
        //                 [, const GrB_Descriptor  desc]);
                         
        // vector version
        GrB_info GrB_reduce(scalar               *dst
                            const GrB_Monoid      op
                            const GrB_Vector      src
                         [, const GrB_MaskVector  mask
                         [, const GrB_Descriptor  desc]]);
        // matrix version
        GrB_info GrB_reduce(scalar               *dst
                            const GrB_Monoid      op
                            const GrB_Matrix      src
                         [, const GrB_MaskVector  mask
                         [, const GrB_Descriptor  desc]]);
\end{verbatim}

\comment{
\scott{Should we use the space/semiring in place of the {\sf f} parameter
and just use the $\oplus$ or if an semiring consists of monoids this is
another place where a Monoid is appropriate.  Note that we must know the
identity value for the operation in order to store the correct value in
the scalar if the vector that you are reducing has not stored values.}
\jose{Yes, we should. Changed.}\scott{OK TO REMOVE}
}

\scott{Now we shift to the conversation about whether s is replaced with a
"relaxed monoid" or "binary function + identity"}

\subparagraph{Input Parameters}

\begin{itemize}
    \item[{\sf v}] Vector to be reduced.
    \item[{\sf s}] Semiring/monoid defining the reduction.
    \item[{\sf d}] Operation descriptor (optional).
\end{itemize}

\subparagraph{Output Parameters}

\begin{itemize}
    \item[{\sf t}] Value of the reduction. It must
    be a pointer to one of the types in 
    the left column of Table~\ref{Tab:PredefinedTypes} or
    {\tt void*}.
\end{itemize}

\subparagraph{Return Value}

\begin{itemize}[leftmargin=2.1in]
\item[{\sf GrB\_SUCCESS}]     operation completed successfully.
\item[{\sf GrB\_PANIC}]        unknown internal error.
\item[{\sf GrB\_NOVECTOR}]    vector does not exist
\item[{\sf GrB\_MISMATCH}]    mismatch between vector domain, scalar type or semiring/monoid
\end{itemize}

\subparagraph{Description}

Let $0 = \bold{0}({\sf s})$, whether ${\sf s}$ is a semiring or monoid.
Let $\oplus = \bigoplus({\sf s})$.

We must have $\bold{D}_3({\sf s}) = \bold{D}_1({\sf s})$.
Otherwise, the method returns {\sf GrB\_MISMATCH}.

Vector $\vector{v}$ is computed from input parameter ${\sf v}$ as
specified by descriptor {\sf d}. $\bold{D}(\vector{v}) = \bold{D}_2({\sf s})$
and $\bold{n}(\vector{v}) = \bold{n}({\sf v})$. If $\vector{v}$ cannot be computed
from the input parameters, the method returns {\sf GrB\_MISMATCH}.

A scalar variable $t$ such that $\bold{D}(t) = \bold{D_1}({\sf s})$ is
created and initialized $t \leftarrow \bold{0}({\sf s})$. 
We then compute the recurrence $t \leftarrow t \oplus v_i, \forall i \in \vector{i}(\vector{v})$.

Finally, output parameter {\sf t} is computed from scalar $t$.

%=========================================================================

\subsubsection{{\sf transpose}: Transpose rows and columns of a matrix}

\scott{ If we adopt the concept of a vector having only one index and conventional orientation of column, then there is no materialized transpose for vector. For vector transpose only makes sense in the context of other operations.}

%-----------------------------------------------------------------------------
\paragraph{Standard Matrix Variant}

This version materializes a new matrix that is the transpose of the source matrix.

\subparagraph{C99 Syntax}

\begin{verbatim}
        GrB_info GrB_transpose(GrB_Matrix           *dst,
                               const GrB_Matrix      src
                            [, const GrB_MaskMatrix  mask
                            [, const GrB_Descriptor  desc
                            [, const GrB_Function    accum]]]);
\end{verbatim}

\subparagraph{Input Parameters}

\begin{itemize}
    \item[{\sf dst}]   ({\sf OUTP}) The matrix to assign the result.
    \item[{\sf src}]   ({\sf ARG0}) The matrix to transpose.

    \item[{\sf desc}]  Operation descriptor (optional). The descriptor
    is used to specify details of the operation. Valid options are 
    invert (structural complement) of mask ({\sf ARG3}). If
    a \emph{default} descriptor is desired,    {\sf GrB\_NULL} can be
    used or the descriptor can be omitted.

    \item[{\sf mask}]  (MASK) Output mask (optional). The mask
    specifies which elements of {\sf dst} can be assigned.
    If no mask is necessary (i.e., compute all elements of result),
    {\sf GrB\_NULL} can be used or the mask can be omitted.

    \item[{\sf desc}]   Operation descriptor (optional). If a
    \emph{default} descriptor is desired, {\sf GrB\_NULL} can be
    used or the descriptor can be omitted.  Valid fields and values are as follows: \\
    \begin{tabular}{lll}
    Field  & Value & Description \\
    \hline
    {\sf ARG0} & {\sf GrB\_CAST} & Allow casting from $\bold{D}({\sf src})$ to $\bold{D}({\sf dst})$ \\
    {\sf MASK} & {\sf GrB\_SCMP} & Use the structural complement of {\sf mask}. \\
    {\sf OUTP}& {\sf GrB\_ACC}  & Use the {\sf accum} function to add to existing values in {\sf dst}.\\
    \end{tabular}

    \item[{\sf accum}] Function used for accumulation into dst.  {\sf GrB\_NULL}
                       can be used if no accumulation into dst is used.
\end{itemize}

\subparagraph{Return Value}

\scott{Are invalid descriptors an error or ignored?}

\begin{itemize}[leftmargin=2.1in]
\item[{\sf GrB\_SUCCESS}]     operation completed successfully.
\item[{\sf GrB\_PANIC}]        unknown internal error.
\item[{\sf GrB\_DIMENSION\_MISMATCH}]      
        If the size/shape of dst is not the same as either mask or
        of the transpose of src.
\item[{\sf GrB\_DOMAIN\_MISMATCH}]  
        domain mismatch among matrices and/or
        accum function \scott{elaborate}
\end{itemize}

\subparagraph{Description}

The dst matrix must have already been created with the proper dimensions
prior to calling this function.  Stored values are inserted into dst to
create a transpose of the src matrix where 
$dst(j,i) = src(i,j) \forall i,j where src(i,j) \neq 0$.
If accum is specified then the function is used to combine with existing values in dst using ewiseadd semantics:
$dst(j,i) \oplus = src(i,j) \forall i,j where src(i,j) \neq 0$.
When the mask is specified the destination location will only be assigned if the corresponding location in the mask has a stored value.

\scott{What happens when src is negated? It can't be. Only masks can be negated.}

\scott{What happens if accum is not used and dst is not empty? dst is overwritten.}

\scott{What happens if src and dst refer to the same matrix?  With immutable dimensions and non-square matrices this should be disallowed.}




\section{Utility Methods}

\scott{TODO: there are many useful functions possible.  Need to get input.}

%-----------------------------------------------------------------------------
\subsection{{\sf init}: Initialize a GraphBLAS context}

Creates and initializes a GraphBLAS C API context.  The argument
to {\sf GrB\_init} defines the mode for the context.  The two
available modes are:

\begin{itemize}
\item {\sf GrB\_Blocking}: Methods in a sequence return after
computations in the method have completed and output arguments
are available to subsequent statements in an application.  When
executing in {\sf GrB\_Blocking} mode, the methods execute 
in program order.

\item {\sf GrB\_NonBlocking}: Methods in a sequence return after
arguments in the method have been tested for consistency with the
method but potentially before computations complete or output 
arguments are available to subsequent statements in an application.
When executing in {\sf GrB\_NonBlocking} mode, the methods 
in a sequence may execute in any order that preserves the 
mathematically result defined by the squence.

\end{itemize}

{\sf GrB\_init()} may be called with {\sf GrB\_NULL} to select 
the default mode.  This mode is implementation defined.

\paragraph{C99 Syntax}

\begin{verbatim}
        GrB_info GrB_init(GrB_Mode m)
\end{verbatim}


\paragraph{Parameters}

\begin{itemize}
	\item[{\sf m}] Mode for the GraphBLAS context.
\end{itemize}

\paragraph{Return Values}

\begin{tabular}{rl}
{\sf GrB\_SUCCESS}	& operation completed successfully \\
{\sf GrB\_PANIC}	& unknown internal error \\
{\sf GrB\_NOMODE}	& mode does not exist \\
\end{tabular}


%-----------------------------------------------------------------------------
\subsection{{\sf finalize}: finalize a GraphBLAS context}

Terminates and frees any internal resources created to 
support the GraphBLAS C API context.
An application may not create a new context after 
{\sf GrB\_finalilze} has been called.

\paragraph{C99 Syntax}

\begin{verbatim}
        GrB_info GrB_finalize()
\end{verbatim}

\paragraph{Return Values}

\begin{tabular}{rl}
{\sf GrB\_SUCCESS}	& operation completed successfully \\
{\sf GrB\_PANIC}	& unknown internal error \\
\end{tabular}

%-----------------------------------------------------------------------------
\subsection{{\sf free}: Destroy object}

Destroys a previously created GraphBLAS object.

\paragraph{C99 Syntax}

\begin{verbatim}
        GrB_info GrB_free(GrB_Object o)
\end{verbatim}

\ajy{polymorphic?  is there where \_Generic is specified?}
\jose{This is \emph{one} of the polymorphic functions in the C API.
I changed the introduction to say that polymorphism must be supported by an
extensio of C99. {\tt \_Generic} is just one way of accomplishing that.}

\paragraph{Parameters}

\begin{itemize}
	\item[{\sf o}] GraphBLAS object to be destroyed. Can be a matrix, vector or descriptor.
\end{itemize}

\paragraph{Return Values}

\begin{tabular}{rl}
{\sf GrB\_SUCCESS}	& operation completed successfully \\
{\sf GrB\_PANIC}	& unknown internal error \\
{\sf GrB\_NOOBJECT}	& object does not exist \\
\end{tabular}


%=============================================================================
%=============================================================================

\appendix
\chapter{Examples}

\pagebreak
\nolinenumbers
\section{Example: breadth first search with GraphBLAS}
\jose{This needs to be fixed.}
{\scriptsize
\lstinputlisting[language=C,numbers=left]{BFS5M.c}
}

\pagebreak
\nolinenumbers
\section{Example: betweenness centrality (BC) with GraphBLAS}
{\scriptsize
\lstinputlisting[language=C,numbers=left]{BC1M.c}
}

\pagebreak
\nolinenumbers
\section{Example: BC with GraphBLAS standard definitions}
{\scriptsize
\lstinputlisting[language=C,numbers=left]{BC1Mstd.c}
}

\pagebreak
\nolinenumbers
\section{Example: maximal independent set with GraphBLAS}
\jose{This needs to be fixed.}
{\scriptsize
\lstinputlisting[language=C,numbers=left]{MIS1.c}
}

\pagebreak


%\def\IEEEbibitemsep{3pt plus .5pt}
%\bibliographystyle{IEEEtran}
%\bibliography{refs}

\end{document}
