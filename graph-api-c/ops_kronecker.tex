%-----------------------------------------------------------------------------
\subsection{{\sf kronecker}: Kronecker product of two matrices}

Computes the Kronecker product of two matrices. The result is a matrix.

\paragraph{\syntax}

\begin{verbatim}
        GrB_Info GrB_kronecker(GrB_Matrix             C,
                               const GrB_Matrix       Mask,
                               const GrB_BinaryOp     accum,
                               const GrB_Semiring     op,
                               const GrB_Matrix       A,
                               const GrB_Matrix       B,
                               const GrB_Descriptor   desc);

        GrB_Info GrB_kronecker(GrB_Matrix             C,
                               const GrB_Matrix       Mask,
                               const GrB_BinaryOp     accum,
                               const GrB_Monoid       op,
                               const GrB_Matrix       A,
                               const GrB_Matrix       B,
                               const GrB_Descriptor   desc);

        GrB_Info GrB_kronecker(GrB_Matrix             C,
                               const GrB_Matrix       Mask,
                               const GrB_BinaryOp     accum,
                               const GrB_BinaryOp     op,
                               const GrB_Matrix       A,
                               const GrB_Matrix       B,
                               const GrB_Descriptor   desc);
\end{verbatim}

\paragraph{Parameters}

\begin{itemize}[leftmargin=1.1in]
    \item[{\sf C}]    ({\sf INOUT}) An existing GraphBLAS matrix. On input,
    the matrix provides values that may be accumulated with the result of the
    Kronecker product.  On output, the matrix holds the results of the
    operation.

    \item[{\sf Mask}] ({\sf IN}) An optional ``write'' mask that controls which
    results from this operation are stored into the output matrix {\sf C}. The 
    mask dimensions must match those of the matrix {\sf C}. If the 
    {\sf GrB\_STRUCTURE} descriptor is {\em not} set for the mask, the domain of the 
    {\sf Mask} matrix must be of type {\sf bool} or any of the predefined 
    ``built-in'' types in Table~\ref{Tab:PredefinedTypes}.  If the default
    mask is desired (\ie, a mask that is all {\sf true} with the dimensions of {\sf C}), 
    {\sf GrB\_NULL} should be specified.

    \item[{\sf accum}] ({\sf IN}) An optional binary operator used for accumulating
    entries into existing {\sf C} entries.
    %: ${\sf accum} = \langle \bDout({\sf accum}),\bDin1({\sf accum}),
    %\bDin2({\sf accum}), \odot \rangle$. 
    If assignment rather than accumulation is
    desired, {\sf GrB\_NULL} should be specified.

    \item[{\sf op}]   ({\sf IN}) The semiring, monoid, or binary operator 
    used in the element-wise ``product'' operation.  Depending on which type is
    passed, the following defines the binary operator, 
    $F_b=\langle \bDout({\sf op}),\bDin1({\sf op}),\bDin2({\sf op}),\otimes\rangle$, used:
    \begin{itemize}[leftmargin=1.1in]
    \item[BinaryOp:] $F_b = \langle \bDout({\sf op}), \bDin1({\sf op}),
    \bDin2({\sf op}),\bold{\bigodot}({\sf op})\rangle$.  
    \item[Monoid:] $F_b = \langle \bold{D}({\sf op}), \bold{D}({\sf op}),
    \bold{D}({\sf op}),\bold{\bigodot}({\sf op})\rangle$;
    the identity element is ignored. 
    \item[Semiring:] $F_b = \langle \bDout({\sf op}), \bDin1({\sf op}),
    \bDin2({\sf op}),\bold{\bigotimes}({\sf op})\rangle$; the
    additive monoid is ignored.
    \end{itemize}

    \item[{\sf A}]    ({\sf IN}) The GraphBLAS matrix holding the values
    for the left-hand matrix in the product.

    \item[{\sf B}]    ({\sf IN}) The GraphBLAS matrix holding the values for
    the right-hand matrix in the product.

    \item[{\sf desc}] ({\sf IN}) An optional operation descriptor. If
    a \emph{default} descriptor is desired, {\sf GrB\_NULL} should be
    specified. Non-default field/value pairs are listed as follows:  \\

    \hspace*{-2em}\begin{tabular}{lllp{2.7in}}
        Param & Field  & Value & Description \\
        \hline
        {\sf C}    & {\sf GrB\_OUTP} & {\sf GrB\_REPLACE} & Output matrix {\sf C}
        is cleared (all elements removed) before the result is stored in it.\\

        {\sf Mask} & {\sf GrB\_MASK} & {\sf GrB\_STRUCTURE}   & The write mask is
        constructed from the structure (pattern of stored values) of the input
        {\sf Mask} matrix. The stored values are not examined.\\

        {\sf Mask} & {\sf GrB\_MASK} & {\sf GrB\_COMP}   & Use the 
        complement of {\sf Mask}. \\

        {\sf A}    & {\sf GrB\_INP0} & {\sf GrB\_TRAN}   & Use transpose of {\sf A}
        for the operation. \\

        {\sf B}    & {\sf GrB\_INP1} & {\sf GrB\_TRAN}   & Use transpose of {\sf B}
        for the operation. \\
    \end{tabular}
\end{itemize}

\paragraph{Return Values}

\begin{itemize}[leftmargin=2.1in]
    \item[{\sf GrB\_SUCCESS}]         In blocking mode, the operation completed
    successfully. In non-blocking mode, this indicates that the compatibility 
    tests on dimensions and domains for the input arguments passed successfully. 
    Either way, output matrix {\sf C} is ready to be used in the next method of
    the sequence.

    \item[{\sf GrB\_PANIC}]           Unknown internal error.

    \item[{\sf GrB\_INVALID\_OBJECT}] This is returned in any execution mode 
    whenever one of the opaque GraphBLAS objects (input or output) is in an invalid 
    state caused by a previous execution error.  Call {\sf GrB\_error()} to access 
    any error messages generated by the implementation.

    \item[{\sf GrB\_OUT\_OF\_MEMORY}] Not enough memory available for the operation.

    \item[{\sf GrB\_UNINITIALIZED\_OBJECT}] One or more of the GraphBLAS objects 
    has not been initialized by a call to {\sf new} (or {\sf Matrix\_dup} for matrix
    parameters).

    \item[{\sf GrB\_DIMENSION\_MISMATCH}] Mask and/or matrix
    dimensions are incompatible.

    \item[{\sf GrB\_DOMAIN\_MISMATCH}]    The domains of the various matrices are
    incompatible with the corresponding domains of the binary operator ({\sf op}) or
    accumulation operator, or the mask's domain is not compatible with {\sf bool}
    (in the case where {\sf desc[GrB\_MASK].GrB\_STRUCTURE} is not set).
\end{itemize}

\paragraph{Description}

{\sf GrB\_kronecker} computes the Kronecker product ${\sf C} = {\sf
A} \kron {\sf B}$ or, if an optional binary accumulation
operator ($\odot$) is provided, ${\sf C} = {\sf C} \odot
\left({\sf A} \kron {\sf B}\right)$ (where matrices {\sf A}
and {\sf B} can be optionally transposed).  The Kronecker product is defined as
follows: \\
\[
    {\sf C} = {\sf A} ~ \kron ~ {\sf B} = \left[
    \text{
    \begin{tabular}{rrcr}
        $A_{0,0} \otimes {\sf B}$ & $A_{0,1} \otimes {\sf B}$ & ... & $A_{0,n_A-1} \otimes {\sf B}$ \\
        $A_{1,0} \otimes {\sf B}$ & $A_{1,1} \otimes {\sf B}$ & ... & $A_{1,n_A-1} \otimes {\sf B}$ \\
        $\vdots$ & $\vdots$ & $\ddots$ & $\vdots$ \\
        $A_{m_A-1,0} \otimes {\sf B}$ & $A_{m_A-1,1} \otimes {\sf B}$ & ... & $A_{m_A-1,n_A-1} \otimes {\sf B}$ \\
    \end{tabular}
    }
    \right]
\]
where ${\sf A} : {\mathbb S}^{m_A \times n_A}$, ${\sf B} : {\mathbb S}^{m_B \times n_B}$, and
${\sf C} : {\mathbb S}^{m_A m_B \times n_A n_B}$.  More explicitly, the elements of
the Kronecker product are defined as
\[
    {\sf C}(i_A m_B + i_B, j_A n_B + j_B) = A_{i_A,j_A} \otimes B_{i_B,j_B},
\]
where $\otimes$ is the multiplicative operator specified by the {\sf op} parameter.

Logically, this operation
occurs in three steps:
\begin{enumerate}[leftmargin=0.85in]
\item[\bf Setup] The internal matrices and mask used in the computation are formed and their 
domains and dimensions are tested for compatibility.
\item[\bf Compute] The indicated computations are carried out.
\item[\bf Output] The result is written into the output matrix, possibly under control of a mask.
\end{enumerate}

Up to four argument matrices are used in the {\sf GrB\_kronecker} operation:
\begin{enumerate}
	\item ${\sf C} = \langle \bold{D}({\sf C}),\bold{nrows}({\sf C}),\bold{ncols}({\sf C}),\bold{L}({\sf C}) = \{(i,j,C_{ij}) \} \rangle$
	\item ${\sf Mask} = \langle \bold{D}({\sf Mask}),\bold{nrows}({\sf Mask}),\bold{ncols}({\sf Mask}),\bold{L}({\sf Mask}) = \{(i,j,M_{ij}) \} \rangle$ (optional)
	\item ${\sf A} = \langle \bold{D}({\sf A}),\bold{nrows}({\sf A}), \bold{ncols}({\sf A}),\bold{L}({\sf A}) = \{(i,j,A_{ij}) \} \rangle$
	\item ${\sf B} = \langle \bold{D}({\sf B}),\bold{nrows}({\sf B}), \bold{ncols}({\sf B}),\bold{L}({\sf B}) = \{(i,j,B_{ij}) \} \rangle$
\end{enumerate}

The argument matrices, the "product" operator ({\sf op}), and the accumulation
operator (if provided) are tested for domain compatibility as follows:
\begin{enumerate}
	\item If {\sf Mask} is not {\sf GrB\_NULL}, and ${\sf desc[GrB\_MASK].GrB\_STRUCTURE}$
    is not set, then $\bold{D}({\sf Mask})$ must be from one of the pre-defined types of 
    Table~\ref{Tab:PredefinedTypes}.

	\item $\bold{D}({\sf A})$ must be compatible with $\bDin1({\sf op})$.

	\item $\bold{D}({\sf B})$ must be compatible with $\bDin2({\sf op})$.

	\item $\bold{D}({\sf C})$ must be compatible with $\bDout({\sf op})$.

	\item If {\sf accum} is not {\sf GrB\_NULL}, then $\bold{D}({\sf C})$ must be
    compatible with $\bDin1({\sf accum})$ and $\bDout({\sf accum})$ of the 
    accumulation operator and $\bDout({\sf op})$ of {\sf op} must be compatible 
    with $\bDin2({\sf accum})$ of the accumulation operator.
\end{enumerate}
Two domains are compatible with each other if values from one domain can be cast 
to values in the other domain as per the rules of the C language.
In particular, domains from Table~\ref{Tab:PredefinedTypes} are all compatible 
with each other. A domain from a user-defined type is only compatible with itself.
If any compatibility rule above is violated, execution of {\sf GrB\_kronecker} ends and 
the domain mismatch error listed above is returned.

From the argument matrices, the internal matrices and mask used in 
the computation are formed ($\leftarrow$ denotes copy):
\begin{enumerate}
	\item Matrix $\matrix{\widetilde{C}} \leftarrow {\sf C}$.

	\item Two-dimensional mask, $\matrix{\widetilde{M}}$, is computed from
    argument {\sf Mask} as follows:
	\begin{enumerate}
		\item If ${\sf Mask} = {\sf GrB\_NULL}$, then $\matrix{\widetilde{M}} = 
        \langle \bold{nrows}({\sf C}), \bold{ncols}({\sf C}), \{(i,j), 
        \forall i,j : 0 \leq i <  \bold{nrows}({\sf C}), 0 \leq j < 
        \bold{ncols}({\sf C}) \} \rangle$.

		\item If {\sf Mask} $\ne$ {\sf GrB\_NULL},
        \begin{enumerate}
            \item If ${\sf desc[GrB\_MASK].GrB\_STRUCTURE}$ is set, then 
            $\matrix{\widetilde{M}} = \langle \bold{nrows}({\sf Mask}), 
            \bold{ncols}({\sf Mask}), \{(i,j) : (i,j) \in \bold{ind}({\sf Mask}) \} \rangle$,
            \item Otherwise, $\matrix{\widetilde{M}} = \langle \bold{nrows}({\sf Mask}), 
            \bold{ncols}({\sf Mask}), \\ \{(i,j) : (i,j) \in \bold{ind}({\sf Mask}) \wedge 
            ({\sf bool}){\sf Mask}(i,j) = \true\} \rangle$.
        \end{enumerate}

		\item	If ${\sf desc[GrB\_MASK].GrB\_COMP}$ is set, then 
        $\matrix{\widetilde{M}} \leftarrow \neg \matrix{\widetilde{M}}$.
	\end{enumerate}

	\item Matrix $\matrix{\widetilde{A}} \leftarrow
    {\sf desc[GrB\_INP0].GrB\_TRAN} \ ? \ {\sf A}^T : {\sf A}$.

	\item Matrix $\matrix{\widetilde{B}} \leftarrow
    {\sf desc[GrB\_INP1].GrB\_TRAN} \ ? \ {\sf B}^T : {\sf B}$.
\end{enumerate}

The internal matrices and masks are checked for dimension compatibility. The following
conditions must hold:
\begin{enumerate}
	\item $\bold{nrows}(\matrix{\widetilde{C}}) = \bold{nrows}(\matrix{\widetilde{M}})$.

	\item $\bold{ncols}(\matrix{\widetilde{C}}) = \bold{ncols}(\matrix{\widetilde{M}})$.

	\item $\bold{nrows}(\matrix{\widetilde{C}}) = \bold{nrows}(\matrix{\widetilde{A}}) \cdot \bold{nrows}(\matrix{\widetilde{B}})$.

	\item $\bold{ncols}(\matrix{\widetilde{C}}) = \bold{ncols}(\matrix{\widetilde{A}}) \cdot \bold{ncols}(\matrix{\widetilde{B}})$.
\end{enumerate}
If any compatibility rule above is violated, execution of {\sf GrB\_kronecker} ends and
the dimension mismatch error listed above is returned.

From this point forward, in {\sf GrB\_NONBLOCKING} mode, the method can 
optionally exit with {\sf GrB\_SUCCESS} return code and defer any computation 
and/or execution error codes.

We are now ready to carry out the Kronecker product and any additional 
associated operations.  We describe this in terms of two intermediate matrices:
\begin{itemize}
    \item $\matrix{\widetilde{T}}$: The matrix holding the Kronecker product of matrices 
    $\matrix{\widetilde{A}}$ and $\matrix{\widetilde{B}}$.
    \item $\matrix{\widetilde{Z}}$: The matrix holding the result after 
    application of the (optional) accumulation operator.
\end{itemize}

The intermediate matrix $\matrix{\widetilde{T}} = \langle
\bDout({\sf op}), 
\bold{nrows}(\matrix{\widetilde{A}}) \times \bold{nrows}(\matrix{\widetilde{B}}), 
\bold{ncols}(\matrix{\widetilde{A}}) \times \bold{ncols}(\matrix{\widetilde{B}}),
\{(i,j,T_{ij}) \text{~where~} i = i_A \cdot m_B + i_B, ~ j = j_A \cdot n_B + j_B, ~ \forall ~
(i_A, j_A) = \bold{ind}(\matrix{\widetilde{A}}), ~
(i_B, j_B) = \bold{ind}(\matrix{\widetilde{B}}) \rangle$
is created.  The value of each of its elements is computed by 
\[T_{i_A \cdot m_B + i_B,~ j_A \cdot n_B + j_B} = \matrix{\widetilde{A}}(i_A,j_A) \otimes
\matrix{\widetilde{B}}(i_B,j_B)),\] where $\otimes$ is the multiplicative 
operator specified by the {\sf op} parameter.


%\emph{Standard Matrix Accumulate Option}

The intermediate matrix $\matrix{\widetilde{Z}}$ is created as follows:
\begin{itemize}
    \item If ${\sf accum} = {\sf GrB\_NULL}$, then $\matrix{\widetilde{Z}} = \matrix{\widetilde{T}}$.

    \item If ${\sf accum}$ is a binary operator, then $\matrix{\widetilde{Z}}$ is defined as
        \[ \langle \bDout({\sf accum}), \bold{nrows}(\matrix{\widetilde{C}}), \bold{ncols}(\matrix{\widetilde{C}}),
        %\bold{L}(\matrix{\widetilde{Z}}) =
        \{(i,j,Z_{ij})  \forall (i,j) \in \bold{ind}(\matrix{\widetilde{C}}) \cup 
        \bold{ind}(\matrix{\widetilde{T}}) \} \rangle.\]

    The values of the elements of $\matrix{\widetilde{Z}}$ are computed based on the
    relationships between the sets of indices in $\matrix{\widetilde{C}}$ and 
    $\matrix{\widetilde{T}}$.
\[
    Z_{ij} = \matrix{\widetilde{C}}(i,j) \odot \matrix{\widetilde{T}}(i,j), \ \mbox{if}\  
    (i,j) \in  (\bold{ind}(\matrix{\widetilde{T}}) \cap \bold{ind}(\matrix{\widetilde{C}})),
\]
\[
    Z_{ij} = \matrix{\widetilde{C}}(i,j), \ \mbox{if}\  
    (i,j) \in (\bold{ind}(\matrix{\widetilde{C}}) - (\bold{ind}(\matrix{\widetilde{T}})
    \cap \bold{ind}(\matrix{\widetilde{C}}))),
\]
\[
    Z_{ij} = \matrix{\widetilde{T}}(i,j), \ \mbox{if}\  (i,j) \in  
    (\bold{ind}(\matrix{\widetilde{T}}) - (\bold{ind}(\matrix{\widetilde{T}})
    \cap \bold{ind}(\matrix{\widetilde{C}}))),
\]
where $\odot  = \bigodot({\sf accum})$, and the difference operator refers to set difference.
\end{itemize}



Finally, the set of output values that make up matrix $\matrix{\widetilde{Z}}$ 
are written into the final result matrix {\sf C}, using what is called
a \emph{standard matrix mask and replace}. 
This is carried out under control of the mask which acts as a ``write mask''.
\begin{itemize}
\item If {\sf desc[GrB\_OUTP].GrB\_REPLACE} is set, then any values in {\sf C} on
input to this operation are deleted and the contents of the new output matrix,
{\sf C}, is defined as,
\[
\bold{L}({\sf C}) = \{(i,j,Z_{ij}) : (i,j) \in (\bold{ind}(\matrix{\widetilde{Z}}) 
\cap \bold{ind}(\matrix{\widetilde{M}})) \}. 
\]

\item If {\sf desc[GrB\_OUTP].GrB\_REPLACE} is not set, the elements of 
$\matrix{\widetilde{Z}}$ indicated by the mask are copied into the result 
matrix, {\sf C}, and elements of {\sf C} that fall outside the set 
indicated by the mask are unchanged:
\[
\bold{L}({\sf C}) = \{(i,j,C_{ij}) : (i,j) \in (\bold{ind}({\sf C}) 
\cap \bold{ind}(\neg \matrix{\widetilde{M}})) \} \cup \{(i,j,Z_{ij}) : (i,j) \in 
(\bold{ind}(\matrix{\widetilde{Z}}) \cap \bold{ind}(\matrix{\widetilde{M}})) \}.
\]
\end{itemize}

In {\sf GrB\_BLOCKING} mode, the method exits with return value 
{\sf GrB\_SUCCESS} and the new content of matrix {\sf C} is as defined above
and fully computed.
In {\sf GrB\_NONBLOCKING} mode, the method exits with return value 
{\sf GrB\_SUCCESS} and the new content of matrix {\sf C} is as defined above
but may not be fully computed; however, it can be used in the next GraphBLAS 
method call in a sequence.
s
