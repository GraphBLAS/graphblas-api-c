\subsubsection{{\sf extract}: Selecting Sub-Graphs}

Extract a sub-matrix from a larger matrix. 

\paragraph{Standard matrix and vector variants}

In the standard version of {\sf extract}, GraphBLAS index arrays
specify the elements in the source vector/matrix that are copied to
the destination.  For vectors, only one index array is used to specify
elements, and for matrices two index arrays (for row and column indices)
are needed.  The size of the destination vector is the same size as
the one index array provided.  For matrices, the destination matrix has
the same number of rows as the size of the row index array and the same
number of columns as the size of the column index array.

If an accumulation function is specified,
then that function is
used to combine current elements in the destination with those elements
extracted from the source. If no accumulation function is specified, then
the destination is overwritten with the extracted elements.

\paragraph{C99 Syntax (vector variant)}

\begin{verbatim}
        GrB_info GrB_extract(GrB_Vector                *v,
                             const GrB_Vector           mask,
                             const GrB_BinaryFunction   accum,
                             const GrB_Vector           u,
                             const GrB_Index           *indices,
                             const GrB_Index            n,
                          [, const GrB_Descriptor       desc]);
\end{verbatim}

\paragraph{C99 Syntax (matrix variant)}

\begin{verbatim}                 
        GrB_info GrB_extract(GrB_Matrix                *B,
                             const GrB_Matrix           Mask,
                             const GrB_BinaryFunction   accum,
                             const GrB_Matrix           A,
                             const GrB_Index           *rows,
                             const GrB_Index            m,
                             const GrB_Index           *cols,
                             const GrB_Index            n,
                          [, GrB_Descriptor const       desc]);
\end{verbatim}

\paragraph{Parameters (vector variant)}

\begin{itemize}[leftmargin=1in]
    \item[{\sf v}]   ({\sf OUTP}) The vector to store the extracted subgraph.

    \item[{\sf mask}] ({\sf MASK}) Output mask. The mask
    specifies which elements of {\sf dst} can be modified.
    If no mask is necessary (i.e., compute all elements of result
    vector), {\sf GrB\_NULL} should be specified.

    \item[{\sf accum}]  Function used for accumulating entries into existing {\sf v} entries. 
			If no accumulation is desired, {\sf GrB\_NULL} should be specified.

    \item[{\sf u}]   ({\sf ARG0}) The vector from which to extract the subgraph.
    \item[{\sf indices}]     ({\sf ARG1}) The set of element indices specifying elements from source that
                              are extracted. Can
                              be set to {\sf GrB\_ALL} if all elements are
                              to be extracted.
    \item[{\sf n}]     ({\sf ARG2}) The number of indices in array {\sf indices}.

    \item[{\sf desc}]   Operation descriptor (optional). If a
    \emph{default} descriptor is desired, {\sf GrB\_NULL} can be
    used or the descriptor can be omitted.  Valid fields and values are as follows: \\
    \begin{tabular}{lll}
    Field  & Value & Description \\
    \hline
    {\sf MASK} & {\sf GrB\_NOCAST} & Prohibit casting from $\bold{D}({\sf mask})$ to {\sf bool} domain. \\
    {\sf MASK} & {\sf GrB\_SCMP}   & Use the structural complement of {\sf mask}. \\
    {\sf ARG0} & {\sf GrB\_NOCAST} & Prohibit casting from $\bold{D}({\sf u})$ to $\bold{D}({\sf v})$ \\
    \end{tabular}
\end{itemize}

\paragraph{Parameters (matrix variant)}

\begin{itemize}[leftmargin=1in]
    \item[{\sf B}]   ({\sf OUTP}) The matrix to store the extracted subgraph.

    \item[{\sf Mask}] ({\sf MASK}) Output mask. The mask
    specifies which elements of {\sf dst} can be modified.
    If no mask is necessary (i.e., compute all elements of result
    vector), {\sf GrB\_NULL} should be specified.

    \item[{\sf accum}]  Function used for accumulating entries into existing {\sf B} entries. 
			If no accumulation is desired, {\sf GrB\_NULL} should be specified.

    \item[{\sf A}]   ({\sf ARG0}) The matrix from which to extract the subgraph.
    \item[{\sf rows}]     ({\sf ARG1}) The set of row indices specifying rows from source that
                              are extracted. Can
                              be set to {\sf GrB\_ALL} if all rows are
                              to be extracted.
    \item[{\sf m}]     ({\sf ARG2}) The number of indices in array {\sf rows}.
    \item[{\sf cols}]     ({\sf ARG3}) The set of column indices specifying
                              columns from source that are extracted. Can
                              be set to {\sf GrB\_ALL} if all columns are
                              to be extracted.
    \item[{\sf n}]     ({\sf ARG4}) The number of indices in array {\sf cols}.

    \item[{\sf desc}]   Operation descriptor (optional). If a
    \emph{default} descriptor is desired, {\sf GrB\_NULL} can be
    used or the descriptor can be omitted.  Valid fields and values are as follows: \\
    \begin{tabular}{lll}
    Field  & Value & Description \\
    \hline
    {\sf MASK} & {\sf GrB\_NOCAST} & Prohibit casting from $\bold{D}({\sf mask})$ to {\sf bool} domain. \\
    {\sf MASK} & {\sf GrB\_SCMP}   & Use the structural complement of {\sf mask}. \\
    {\sf ARG0} & {\sf GrB\_NOCAST} & Prohibit casting from $\bold{D}({\sf src})$ to $\bold{D}({\sf dst})$ \\
    {\sf ARG0} & {\sf GrB\_TRAN}   & Apply transpose to {\sf A} before extract.) \\
    \end{tabular}
\end{itemize}

\paragraph{Return Values (vector version)}

\begin{itemize}[leftmargin=2.1in]
\item[{\sf GrB\_SUCCESS}]     operation completed successfully.
\item[{\sf GrB\_PANIC}]        unknown internal error.
\item[{\sf GrB\_OUTOFMEM}]    not enough memory available for operation.
\item[{\sf GrB\_INDEX\_OUTOFBOUNDS}]
        A value in {\sf indices} references a non-existent element in {\sf u}.
\item[{\sf GrB\_DIMENSION\_MISMATCH}] 
        The size of {\sf indices} is not equal to the size of {\sf v}, or
        the dimensions of the mask (if specified) do not match {\sf v}.
\item[{\sf GrB\_DOMAIN\_MISMATCH}]    Mismatch between $\bold{D}({\sf u})$ and $\bold{D}({\sf v})$, and/or the domains of the 
                                      {\sf accum} operation (if used) when {\sf NOCAST} has
                                      been specified or if $\bold{D}(\sf mask)$ is incorrect.
\end{itemize}

\paragraph{Return Values (matrix version)}

\scott{Are invalid/unused descriptors an error or ignored?}

\begin{itemize}[leftmargin=2.1in]
\item[{\sf GrB\_SUCCESS}]     operation completed successfully.
\item[{\sf GrB\_PANIC}]        unknown internal error.
\item[{\sf GrB\_OUTOFMEM}]    not enough memory available for operation.
\item[{\sf GrB\_INDEX\_OUTOFBOUNDS}]
        A value in {\sf rows} references a non-existent row in {\sf A}, or
        the value in {\sf cols} references a non-existent column in {\sf A}.
\item[{\sf GrB\_DIMENSION\_MISMATCH}] 
        The size of {\sf rows} is not equal to the number of rows in {\sf B}, or
        the size of {\sf cols} is not equal to the number of columns in {\sf B}, or
        the dimensions of the mask (if specified) do not match {\sf B}.
\item[{\sf GrB\_DOMAIN\_MISMATCH}]    Mismatch between $\bold{D}({\sf A})$ and $\bold{D}({\sf B})$, and/or the domains of the 
                                      {\sf accum} operation (if used) when {\sf NOCAST} has
                                      been specified or if $\bold{D}(\sf Mask)$ is incorrect.
\end{itemize}


\paragraph{Description (vector variant)}

Vectors $\vector{u}$ and $\vector{mask}$ are computed from input
parameters {\sf u} and {\sf mask}, respectively, as specified by
descriptor {\sf desc}.  If either $\vector{u}$ or $\vector{mask}$
cannot be computed from the input parameters, the method returns {\sf
GrB\_DOMAIN\_MISMATCH}.

A consistency check is performed to verify that all indices in array
{\sf indices} are valid. That is, they must fall within the range of allowed
indices for vector $\vector{u}$ ($0 \leq {\sf indices}[i] < \bold{n}(\vector{u})
\forall i = 0,\ldots,{\sf n}-1$).  Otherwise, the method returns {\sf
GrB\_INDEX\_OUTOFBOUNDS}.

A new vector $\vector{v} = \langle \bold{D}({\sf v}),{\sf n},
\bold{L}(\vector{v}) = \{ (i,v_i), i = 0,\ldots,{\sf n}-1 : {\sf indices}[i]
\in \bold{i}(\vector{u}) \wedge \vector{mask}[i] = \true \} \rangle$
is created.  The value $v_i$ is set to $\vector{u}[{\sf indices}[i]]$. If
casting from $\bold{D}(\vector{u})$ to $\bold{D}(\vector{v})$ is not
allowed, the method returns {\sf GrB\_DOMAIN\_MISMATCH}.

Finally, output parameter {\sf v} is computed from vector $\vector{v}$ as
specified by descriptor {\sf desc} and accumulation function {\sf accum}.

\paragraph{Description (matrix variant)}

Matrices $\matrix{A}$ and $\matrix{Mask}$ are computed from input
parameters {\sf A} and {\sf Mask}, respectively, as specified by
descriptor {\sf desc}.  If either $\matrix{A}$ or $\matrix{Mask}$
cannot be computed from the input parameters, the method returns {\sf
GrB\_DOMAIN\_MISMATCH}.

A consistency check is performed to verify that all indices in arrays
{\sf rows} and {\sf cols} are valid. That is, they must fall within the range of allowed
indices for matrix $\matrix{A}$ ($0 \leq {\sf rows}[i] < \bold{m}(\matrix{A})
\forall i = 0,\ldots,{\sf m}-1$ and
$0 \leq {\sf cols}[j] < \bold{n}(\matrix{A}) \forall j = 0,\ldots,{\sf n}-1$).  Otherwise, the method returns {\sf
GrB\_INDEX\_OUTOFBOUNDS}.

A new matrix $\matrix{B} = \langle \bold{D}({\sf B}),{\sf m},{\sf n},
\bold{L}(\vector{B}) = \{ (i,j,B_{ij}), i = 0,\ldots,{\sf m}-1, j = 0, \ldots,{\sf n}-1 : {\sf rows}[i]
\in \bold{i}(\matrix{A}) \wedge {\sf cols}[j] \in \bold{j}(\matrix{A}) \wedge \vector{Mask}[i,j] = \true \} \rangle$
is created.  The value $B_{ij}$ is set to $\matrix{A}[{\sf rows}[i],{\sf cols}[j]]$. If
casting from $\bold{D}(\vector{A})$ to $\bold{D}(\vector{B})$ is not
allowed, the method returns {\sf GrB\_DOMAIN\_MISMATCH}.

Finally, output parameter {\sf B} is computed from matrix $\matrix{B}$ as
specified by descriptor {\sf desc} and accumulation function {\sf accum}.

%-----------------------------------------------------------------------------
\paragraph{Column (and Row) Variant}

Extract from one column of a matrix into a vector.  Note that with the transpose
descriptor for the {\sf src} matrix, elements of an arbitrary row of the matrix
can be extracted with this function as well.

\subparagraph{C99 Syntax}

\begin{verbatim}
        GrB_info GrB_extract(GrB_Vector                *dst,
                             const GrB_Vector           mask,
                             const GrB_BinaryFunction   accum,
                             const GrB_Matrix           src,
                             const GrB_Index           *i,
                             GrB_Index                  j
                          [, const GrB_Descriptor       desc]); 
\end{verbatim}

\subparagraph{Parameters}

\begin{itemize}[leftmargin=1in]
    \item[{\sf dst}]   ({\sf OUTP}) The vector into which to assign the extracted values.

    \item[{\sf mask}] ({\sf MASK}) Output mask vector. The mask
    specifies which elements of {\sf dst} can be modified.
    If no mask is necessary (i.e., compute all elements of result
    vector), {\sf GrB\_NULL} should be specified.

    \item[{\sf accum}]  Function used for accumulating entries into existing {\sf dst} entries. 
			If no accumulation is desired, {\sf GrB\_NULL} should be specified.

    \item[{\sf src}]   ({\sf ARG0}) The matrix from which to extract the column.

    \item[{\sf i}]     ({\sf ARG1}) An array of row indices to extract. Can
                              be set to a special array, {\sf GrB\_ALL}, if all elements
                              are to be extracted from the columns.
    \item[{\sf j}]     ({\sf ARG2}) The index of the column to extract.

    \item[{\sf desc}]   Operation descriptor (optional). If a
    \emph{default} descriptor is desired, {\sf GrB\_NULL} can be
    used or the descriptor can be omitted.  Valid fields and values are as follows: \\
    \begin{tabular}{lll}
    Field  & Value & Description \\
    \hline
    {\sf MASK} & {\sf GrB\_NOCAST} & Prohibit casting from $\bold{D}({\sf mask})$ to {\sf bool} domain. \\
    {\sf MASK} & {\sf GrB\_SCMP} & Use the structural complement of {\sf mask}. \\
    {\sf ARG0} & {\sf GrB\_NOCAST} & Prohibit casting from $\bold{D}({\sf src})$ to $\bold{D}({\sf dst})$ \\
    {\sf ARG0} & {\sf GrB\_TRAN} & Apply transpose to {\sf src} before extract. \\
    \end{tabular}

    \item[{\sf accum}] Function used for accumulation into dst.  {\sf GrB\_NULL}
                       can be used if no accumulation into dst is desired.
\end{itemize}

\subparagraph{Return Values}

\begin{itemize}[leftmargin=2.1in]
\item[{\sf GrB\_SUCCESS}]             Operation completed successfully.
\item[{\sf GrB\_PANIC}]               Unknown internal error.
\item[{\sf GrB\_INDEX\_OUTOFBOUNDS}]  The indexes specify a position that outside the dimensions of src.
\item[{\sf GrB\_DIMENSION\_MISMATCH}] 
       The size of {\sf i} is greater than the size of {\sf dst} (or transposed {\sf dst}), 
       or the dimensions of the mask (if specified) do not match dst.
\item[{\sf GrB\_DOMAIN\_MISMATCH}]    Mismatch between $\bold{D}(\sf dst)$, 
                                      $\bold{D}({\sf src})$, and/or the domains of the 
                                      {\sf accum} operation (if used) when {\sf NOCAST} has
                                      been specified or if $\bold{D}(\sf mask)$ is incorrect.
\end{itemize}

\subparagraph{Description}

A convenience function for specifying a single column or row (transposed column) from which to extract.

%-----------------------------------------------------------------------------
\paragraph{Single Element Variants}

Extract one element of a vector/matrix into a scalar. 

\subparagraph{C99 Syntax}

\begin{verbatim}
        GrB_info GrB_extract(<type>                    *dst,
                             const GrB_BinaryFunction   accum,
                             const GrB_Vector           src,
                             GrB_Index                  i,
                          [, const GrB_Descriptor       desc]); 

        GrB_info GrB_extract(<type>                    *dst,
                             const GrB_BinaryFunction   accum,
                             const GrB_Matrix           src,
                             GrB_Index                  i,
                             GrB_Index                  j
                          [, const GrB_Descriptor       desc]); 

\end{verbatim}

\subparagraph{Parameters}

\begin{itemize}[leftmargin=1in]
    \item[{\sf dst}]   ({\sf OUTP}) The scalar into which to assign the extracted value.  The specific type should be in domain $\bold{D}({\sf src})$ if {\sf NOCAST} is specified in {\sf desc}.
    \item[{\sf accum}] Function used for accumulation into dst. If no accumulation is desired,
                        {\sf GrB\_NULL} should be specified.
    \item[{\sf src}]   ({\sf ARG0}) The matrix or vector from which to extract the scalar.
    \item[{\sf i}]     ({\sf ARG1}) (Vector version only) The index of location to extract
    \item[{\sf i}]     ({\sf ARG1}) (Matrix version only) The row index of location to extract.
    \item[{\sf j}]     ({\sf ARG2}) (Matrix version only) The column index of location to extract.

    \item[{\sf desc}]   Operation descriptor (optional). If a
    \emph{default} descriptor is desired, {\sf GrB\_NULL} can be
    used or the descriptor can be omitted.  Valid fields and values are as follows: \\
    \begin{tabular}{lll}
    Field  & Value & Description \\
    \hline
    {\sf ARG0} & {\sf GrB\_NOCAST} & Prohibit casting from $\bold{D}({\sf src})$ to $\bold{D}({\sf dst})$ \\
    {\sf ARG0} & {\sf GrB\_TRAN} & Not needed. \scott{should we remove?}\\
    \end{tabular}

\end{itemize}


\subparagraph{Return Values}

\begin{itemize}[leftmargin=2.1in]
\item[{\sf GrB\_SUCCESS}]          Operation completed successfully.
\item[{\sf GrB\_PANIC}]            Unknown internal error.
\item[{\sf GrB\_NO\_VALUE}]        No stored value at specified location (is it an error?).
\item[{\sf GrB\_INDEX\_OUTOFBOUNDS}]  In the vector version, {\sf i} is out of 
                                      bounds of the vector.  In the matrix version,
                                      {\sf i} or {\sf j} is out of matrix bounds.
\item[{\sf GrB\_DOMAIN\_MISMATCH}]    Mismatch between $\bold{D}(\sf dst)$, 
                                      $\bold{D}({\sf src})$, and/or the domains of the 
                                      {\sf accum} operation (if used) when {\sf NOCAST} has
                                      been specified.
\end{itemize}

\subparagraph{Description}

Extract the stored value at the specified location in the source vector or matrix and assign (or accumulate) to the destination scalar.  If the location specified is a structural zero then an error code is returned.

\scott{TODO: needs more precise description.}

%-----------------------------------------------------------------------------
%-----------------------------------------------------------------------------
\subsubsection{{\sf assign}: Modifying Sub-Graphs}

Assign a matrix to a set of indices (sub-matrix) of a larger matrix

%-----------------------------------------------------------------------------
\paragraph{Standard Matrix and Vector Versions}

In the standard version of {\sf assign} GraphBLAS index arrays (dense vectors)
specify the locations in the destination vector/matrix that should be assign from
the source.  If a structural zero exists at a specified location in the source,
no value will be copied.  For vectors, only one index array is used to specify
locations, and for matrices two index arrays (for row and column indices are needed).
The size of the source vector is the same size as the one index array provided.
For matrices, the size of the source matrix has the same number of rows as the
{\sf i} index array and the same number of columns as the {\sf j} index array.

Normally elements selected from source will be replicated in the destination.  If the
destination is not empty, this operation WILL NOT create structural zeros where there
are stored values in the destination even if the corresponding location in source 
has a structural zero.

If the {\sf accum} function is specified and there are stored locations in correspondning
locations in both source and destination, then the
{\sf accum} is used to combine both values before overwriting the destination value
with the result.  If source and destination values collide and the {\sf accum} function is
not specified, then the destination value will be replaced with the source value (the
default behaviour of {\sf accum}).

\subparagraph{C99 Syntax}

\begin{verbatim}
        GrB_info GrB_assign(GrB_Vector                *dst,
                            const GrB_Vector           mask,
                            const GrB_BinaryFunction   accum,
                            const GrB_Vector           src,
                            const GrB_Index           *i
                         [, const GrB_Descriptor       desc])

        GrB_info GrB_assign(GrB_Matrix                *dst,
                            const GrB_Matrix           mask,
                            const GrB_BinaryFunction   accum,
                            const GrB_Matrix           src,
                            const GrB_Index           *i,
                            const GrB_Index           *j
                         [, const GrB_Descriptor       desc
\end{verbatim}

\subparagraph{Parameters}

\begin{itemize}[leftmargin=1.1in]
    \item[{\sf dst}]   ({\sf OUTP}) The matrix or vector into which to assign the subgraph.

    \item[{\sf mask}] ({\sf MASK}) Output mask. The mask specifies which elements
    of {\sf dst} can be modified. If no mask is necessary (i.e., compute all
    elements of {\sf dst}), {\sf GrB\_NULL} should be specified.

    \item[{\sf accum}] Function used for accumulation into {\sf dst}.  If no accumulation
                        is desired, {\sf GrB\_NULL} should be specified.

    \item[{\sf src}]   ({\sf ARG0}) The matrix or vector containing the subgraph.
    \item[{\sf i}]     ({\sf ARG1}) An array of row indices specifying locations in dst that
                       are assigned from src.
    \item[{\sf j}]     ({\sf ARG2}) (Matrix version only) An array of column indices 
                       specifying locations in dst that are assigned from src.


    \item[{\sf desc}]   Operation descriptor (optional). If a
    \emph{default} descriptor is desired, {\sf GrB\_NULL} can be
    used or the descriptor can be omitted.  Valid fields and values are as follows: \\
    \begin{tabular}{lll}
    Field  & Value & Description \\
    \hline
    {\sf MASK} & {\sf GrB\_NOCAST} & Prohibit casting from $\bold{D}({\sf mask})$ to {\sf bool} domain. \\
    {\sf MASK} & {\sf GrB\_SCMP} & Use the structural complement of {\sf mask}. \\
    {\sf ARG0} & {\sf GrB\_NOCAST} & Prohibit casting from $\bold{D}({\sf src})$ to $\bold{D}({\sf dst})$ \\
    {\sf ARG0} & {\sf GrB\_TRAN} & Apply transpose to {\sf src} before extract (Matrix version only.) \\
    \end{tabular}

\end{itemize}

\subparagraph{Return Value}

\scott{Are invalid/unused descriptors an error?}

\begin{itemize}[leftmargin=2.1in]
\item[{\sf GrB\_SUCCESS}]      operation completed successfully.
\item[{\sf GrB\_PANIC}]        unknown internal error.
\item[{\sf GrB\_OUTOFMEM}]     not enough memory available for operation.
\item[{\sf GrB\_INDEX\_OUTOFBOUNDS}]
        A value in i references a nonexistent row in dst, or
        the value in j references a nonexistent column in dst (matrix version).
\item[{\sf GrB\_DIMENSION\_MISMATCH}] 
        The size of i is greater than the number of rows in src, or
        the size of j is greater than the number of columns in src (matrix version), or
        the dimensions of the mask (if specified) do not match dst.
\item[\sf GrB\_DOMAIN\_MISMATCH]  
       domain mismatch among elements of vectors or matrices or the accum function where the cast descriptor was not used, or a mismatch with {\sf accum}, if specified
\end{itemize}

\subparagraph{Description}
\scott{TODO}

%-----------------------------------------------------------------------------
\paragraph{Column (and Row) Variant}

Assign to one column of a matrix from a vector.  \scott{Note we cannot use transpose on dst to assign a row, so in this case we have two functions.  Should I add the second function back into extract variant and ditch the transpose descriptor?}

\subparagraph{C99 Syntax}

\begin{verbatim}
        // assign a column
        GrB_info GrB_assign(GrB_Matrix                *dst,
                            const GrB_Matrix           mask,
                            const GrB_BinaryFunction   accum,
                            const GrB_Vector           src,
                            const GrB_Index           *i,
                            GrB_Index                  j
                         [, const GrB_Descriptor       desc]); 

        // assign a row
        GrB_info GrB_assign(GrB_Matrix                *dst,
                            const GrB_Matrix           mask,
                            const GrB_BinaryFunction   accum,
                            const GrB_Vector           src,
                            GrB_Index                  i,
                            const GrB_Index           *j
                         [, const GrB_Descriptor       desc]); 
\end{verbatim}

\subparagraph{Parameters}

\begin{itemize}[leftmargin=1.1in]
    \item[{\sf dst}]   ({\sf OUTP}) The matrix in which to assign column/row of values.

    \item[{\sf mask}] ({\sf MASK}) Output mask. The mask specifies which elements
    of {\sf dst} can be modified. If no mask is necessary (i.e., compute all
    elements of {\sf dst}), {\sf GrB\_NULL} should be specified.

    \item[{\sf accum}] Function used for accumulation into {\sf dst}.  If no accumulation
                        is desired, {\sf GrB\_NULL} should be specified.

    \item[{\sf src}]   ({\sf ARG0}) The vector of values to assign in a column/row of dst.

    \item[{\sf i}]     ({\sf ARG1}) (column version) An array of row indices to assign. Can
                              be set to a special array, {\sf GrB\_ALL}, if all elements
                              are to be extracted from the column.
    \item[{\sf j}]     ({\sf ARG2}) (column version) The index of the column to assign.

    \item[{\sf i}]     ({\sf ARG1}) (row version) The index of the row to assign.
    \item[{\sf j}]     ({\sf ARG2}) (row version) An array of column indices to assign. Can
                              be set to a special array, {\sf GrB\_ALL}, if all elements
                              are to be extracted from the row.


    \item[{\sf desc}]   Operation descriptor (optional). If a
    \emph{default} descriptor is desired, {\sf GrB\_NULL} can be
    used or the descriptor can be omitted.  Valid fields and values are as follows: \\
    \begin{tabular}{lll}
    Field  & Value & Description \\
    \hline
    {\sf MASK} & {\sf GrB\_NOCAST} & Prohibit casting from $\bold{D}({\sf mask})$ to {\sf bool} domain. \\
    {\sf MASK} & {\sf GrB\_SCMP} & Use the structural complement of {\sf mask}. \\
    {\sf ARG0} & {\sf GrB\_NOCAST} & Prohibit casting from $\bold{D}({\sf src})$ to $\bold{D}({\sf dst})$ \\
    \end{tabular}

\end{itemize}

\subparagraph{Return Values}

\begin{itemize}[leftmargin=2.1in]
\item[{\sf GrB\_SUCCESS}]             Operation completed successfully.
\item[{\sf GrB\_PANIC}]               Unknown internal error.
\item[{\sf GrB\_OUTOFMEM}]            Not enough memory available for operation.
\item[{\sf GrB\_INDEX\_OUTOFBOUNDS}]  The i and j indexes specify a position that outside the dimensions of dst
\item[{\sf GrB\_DIMENSION\_MISMATCH}] 
        The size of {\sf i} is greater than the size of {\sf src} (column version), or
        the size of {\sf j} is greater than the size of {\sf src} (row version)
        the dimensions of the mask (if specified) do not match {\sf dst}.
\item[{\sf GrB\_DOMAIN\_MISMATCH}]    Mismatch between vector/matrix domain and scalar type,
                                      or a mismatch with {\sf accum}, if specified.
\end{itemize}

\subparagraph{Description}

A convenience function for specifying a single column or row (transposed column) into which to assign.

\scott{TODO: Need more precise description.}

%-----------------------------------------------------------------------------
\paragraph{{\sf assign}: Single-Value Variant (was Indexed Variant)}

Set one element of a vector/matrix to a given value.
\scott{I have tried to make this more consistent with the standard version by
adding things like accum. If the destination location is a structural zero, is
a stored value created?.
This could be construed as inconsistent behaviour with standard version.}

\subparagraph{C99 Syntax}

\begin{verbatim}
        GrB_info GrB_assign(GrB_Vector               *dst,
                            const GrB_BinaryFunction  accum,
                            <type>                    src,
                            GrB_Index                 i
                         [, const GrB_Descriptor      desc]); 

        GrB_info GrB_assign(GrB_Matrix               *dst,
                            const GrB_BinaryFunction  accum,
                            <type>                    src,
                            GrB_Index                 i,
                            GrB_Index                 j
                         [, const GrB_Descriptor      desc]); 
\end{verbatim}

\subparagraph{Parameters}

\begin{itemize}[leftmargin=1.1in]
    \item[{\sf dst}]   ({\sf OUTP}) Vector/matrix for which an element is to be assigned.

    \item[{\sf accum}] Function used for accumulation into {\sf dst}.  If no accumulation
                        is desired, {\sf GrB\_NULL} should be specified.

    \item[{\sf src}]   ({\sf ARG0}) Scalar value to assign to the element.  The type must
                              be consistent with the domain of dst (or castable, if allowed).
    \item[{\sf i}]     ({\sf ARG1}) Row index of element to be assigned
    \item[{\sf j}]     ({\sf ARG2}) Column index of element to be assigned (matrix version)

    \item[{\sf desc}]   Operation descriptor (optional). If a
    \emph{default} descriptor is desired, {\sf GrB\_NULL} can be
    used or the descriptor can be omitted.  Valid fields and values are as follows: \\
    \begin{tabular}{lll}
    Field  & Value & Description \\
    \hline
    {\sf ARG0} & {\sf GrB\_NOCAST} & Prohibit casting from $\bold{D}({\sf src})$ to $\bold{D}({\sf dst})$ \\
    \end{tabular}

\end{itemize}

\subparagraph{Return Values}

\begin{itemize}[leftmargin=2.1in]
\item[{\sf GrB\_SUCCESS}]             Operation completed successfully.
\item[{\sf GrB\_PANIC}]               Unknown internal error.
\item[{\sf GrB\_NOVECTOR}]            Vector does not exist (vector version)
\item[{\sf GrB\_NOMATRIX}]            Matrix does not exist (matrix version)
\item[{\sf GrB\_OUTOFMEM}]            Not enough memory available for operation.
\item[{\sf GrB\_INDEX\_OUTOFBOUNDS}]  The i (or j) indexes specify a position that outside the dimensions of dst
\item[{\sf GrB\_DOMAIN\_MISMATCH}]    Mismatch between vector/matrix domain and scalar type,
                                      or a mismatch with {\sf accum}, if specified.
\end{itemize}

\subparagraph{Description}
\scott{TODO}

%-----------------------------------------------------------------------------
\paragraph{{\sf assign}: Constant variant}

Assign the same value to a specified subgraph.  With use of {\sf GrB\_ALL} the entire
destination vector or matrix can be filled with the constant.

\scott{We should seriously consider a fill method for Vector and Matrix instead.}

\subparagraph{C99 Syntax}

\begin{verbatim}
        GrB_info GrB_assign(GrB_Vector                *dst,
                            const GrB_Vector           mask,
                            const GrB_BinaryFunction   accum,
                            <type>                     src,
                            const GrB_Index           *i
                         [, const GrB_Descriptor       desc]);

        GrB_info GrB_assign(GrB_Matrix                *dst,
                            const GrB_Matrix           Mask,
                            const GrB_BinaryFunction   accum,
                            <type>                     src,
                            const GrB_Index           *i,
                            const GrB_Index           *j
                         [, const GrB_Descriptor       desc]);
\end{verbatim}

\subparagraph{Parameters}

\begin{itemize}[leftmargin=1.1in]
    \item[{\sf dst}]   ({\sf OUTP}) Vector/Matrix to be assigned.

    \item[{\sf mask}] ({\sf MASK}) Output mask. The mask specifies which elements
    of {\sf dst} can be modified. If no mask is necessary (i.e., compute all
    elements of {\sf dst}), {\sf GrB\_NULL} should be specified.

    \item[{\sf accum}] Function used for accumulation into {\sf dst}.  If no accumulation
                        is desired, {\sf GrB\_NULL} should be specified.

    \item[{\sf src}]   ({\sf ARG0}) Scalar value to assign to all elements.
    \item[{\sf i}]     ({\sf ARG1}) An array of row indices specifying locations in dst that
                       are assigned from src.
    \item[{\sf j}]     ({\sf ARG2}) (Matrix version only) An array of column indices 
                       specifying locations in dst that are assigned from src.


    \item[{\sf desc}]   Operation descriptor (optional). If a
    \emph{default} descriptor is desired, {\sf GrB\_NULL} can be
    used or the descriptor can be omitted.  Valid fields and values are as follows: \\
    \begin{tabular}{lll}
    Field  & Value & Description \\
    \hline
    {\sf MASK} & {\sf GrB\_NOCAST} & Prohibit casting from $\bold{D}({\sf mask})$ to {\sf bool} domain. \\
    {\sf MASK} & {\sf GrB\_SCMP} & Use the structural complement of {\sf mask}. \\
    {\sf ARG0} & {\sf GrB\_NOCAST} & Prohibit casting from $\bold{D}({\sf src})$ to $\bold{D}({\sf dst})$ \\
    \end{tabular}

\end{itemize}

\subparagraph{Return Values}

\begin{itemize}[leftmargin=2.1in]
\item[{\sf GrB\_SUCCESS}]             Operation completed successfully.
\item[{\sf GrB\_PANIC}]               Unknown internal error.
\item[{\sf GrB\_NOVECTOR}]            Vector does not exist (vector version)
\item[{\sf GrB\_NOMATRIX}]            Matrix does not exist (matrix version)
\item[{\sf GrB\_OUTOFMEM}]            Not enough memory available for operation.
\item[{\sf GrB\_INDEX\_OUTOFBOUNDS}]
        A value in i references a nonexistent row in dst, or
        the value in j references a nonexistent column in dst (matrix version).
\item[{\sf GrB\_DIMENSION\_MISMATCH}] 
        the dimensions of the mask (if specified) do not match dst.
\item[{\sf GrB\_DOMAIN\_MISMATCH}]    Mismatch between vector/matrix domain and scalar type,
                                      or a mismatch with {\sf accum}, if specified.
\end{itemize}


\subparagraph{Description}
\scott{TODO}
