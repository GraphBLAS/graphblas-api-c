\section{GraphBLAS Operations}
\label{Sec:Operations}

\begin{table*}[h]
\hrule
\begin{center}
\caption{A Mathematical overview of the fundamental GraphBLAS operations supported.}
\label{Tab:GraphBLASOps}
\begin{tabular}{l|rrl}
{\sf Operation Name} & \multicolumn{3}{c}{Mathematical Description}  \\
\hline
{\sf mxm}          & $\matrix{C}(\neg\matrix{M})$ & $\oplus=$ & $\matrix{A}^T \oplus.\otimes \matrix{B}^T$  \\
{\sf mxv}          & $\vector{c}(\neg\vector{m})$ & $\oplus=$ & $\matrix{A}^T \oplus.\otimes \vector{b}$  \\
{\sf vxm}          & $\vector{c}(\neg\vector{m})$ & $\oplus=$ & $\vector{b} \oplus.\otimes \matrix{A}^T$  \\
{\sf eWiseMult}    & $\matrix{C}(\neg\matrix{M})$ & $\oplus=$ & $\matrix{A}^T \otimes \matrix{B}^T$  \\
{\sf eWiseAdd}     & $\matrix{C}(\neg\matrix{M})$ & $\oplus=$ & $\matrix{A}^T \oplus  \matrix{B}^T$  \\
{\sf reduce} (row) & $\vector{c}(\neg\vector{m})$ & $\oplus=$ & $\oplus_j\matrix{A}^T(:,j)$  \\
{\sf apply}        & $\matrix{C}(\neg\matrix{M})$ & $\oplus=$ & $f(\matrix{A}^T)$ \\
{\sf transpose}    & $\matrix{C}(\neg\matrix{M})$ & $\oplus=$ & $\matrix{A}^T$ \\
{\sf extract}      & $\matrix{C}(\neg\matrix{M})$ & $\oplus=$ & $\matrix{A}^T(\vector{i},\vector{j})$ \\
{\sf assign}       & $\matrix{C}(\neg\matrix{M})(\vector{i},\vector{j})$ & $\oplus=$ & $\matrix{A}^T$ \\
{\sf buildMatrix}  & $\matrix{C}(\neg\matrix{M})$ & $\oplus=$ & $\mathbb{S}^{m\times n}(\vector{i},\vector{j},\vector{v},\oplus_{dup})$ \\
{\sf buildVector}  & $\vector{u}(\neg\vector{m})$ & $\oplus=$ & $\mathbb{S}^{n}(\vector{i},\vector{v})$ \\
{\sf extractTuples}& $(\vector{i},\vector{j},\vector{v})$ & $=$ & $\matrix{A}(\neg\matrix{M})$ \\
\end{tabular}
\end{center}
\hrule
\end{table*}


A mathematical overview of the the fundamental GraphBLAS operations that are
discussed in this section are shown in Table~\ref{Tab:GraphBLASOps}.  This
section also specifies variants to some of these operations where they have
been found especially useful in algorithm development.

When a GraphBLAS operation supports the use of an optional mask, that mask is
specified through a GraphBLAS vector (for one-dimensional masks) or
a GraphBLAS matrix (for two-dimensional masks).

Given a GraphBLAS vector $\vector{v} = \langle D,N, \{ (i,v_i) \} \rangle$, a
one-dimensional mask $m = \langle N, \{ i : \mbox{\tt <bool>}v_i = \true \} \rangle$
is derived for use in the operation, where $\mbox{\tt <bool>}v_i$ denotes
casting the value $v_i$ to a Boolean value (\true\ or \false).
We note that, if cast is disallowed for the mask by the operation descriptor, then
$\bold{D}(\vector{v})$ must be {\sf GrB\_BOOL}.

Given a GraphBLAS matrix $\matrix{A} = \langle D, M, N, \{ (i,j,A_{ij}) \} \rangle$,
a two-dimensional mask $\matrix{M} = \langle M,N, \{ (i,j) : \mbox{\tt <bool>}A_{ij} = \true \} \rangle$
is derived for use in the operation, where $\mbox{\tt <bool>}v_i$ denotes
casting the value $A_{ij}$ to a Boolean value (\true\ or \false).
We note that, if cast is disallowed for the mask by the operation descriptor, then
$\bold{D}(\matrix{A})$ must be {\sf GrB\_BOOL}.

In both the one- and two-dimensional cases, the mask may go through a structural
complement operation ($\S$~\ref{Sec:Masks}) as specified in the descriptor, before a final
mask is generated for use in the operation.

\subsection{{\sf wait}: Force completion of pending operations}
\label{Sec:wait}

Guarantees that pending GraphBLAS operations are fully executed.

\subsubsection{Broad variant}

\paragraph{C99 syntax}

\begin{verbatim}
        GrB_info GrB_wait(char* err)
\end{verbatim}

\paragraph{Parameters}
\begin{itemize}
\item[{\sf err}]     A null terminated string containing additional error
information.
\end{itemize}


\paragraph{Return values}
\begin{itemize}[leftmargin=2.1in]
\item[{\sf GrB\_SUCCESS}]	operation completed successfully
\item[{\sf GrB\_PANIC}]		unknown internal error
\end{itemize}

\paragraph{Description}

Upon return, all previously called GraphBLAS methods have fully completed their execution.
Any (transparent or opaque) data structures produced or manipulated by those methods can be safely touched.

\subsubsection{Matrix variant}

\paragraph{C99 syntax}

\begin{verbatim}
        GrB_info GrB_wait(GrB_Matrix A, char *err)
\end{verbatim}

\paragraph{Parameters}
\begin{itemize}
\item[{\sf err}]     A null terminated string containing additional error
information.
\end{itemize}

\begin{itemize}[leftmargin=1.1in]
	\item[{\sf A}]	({\sf ARG0}) A GraphBLAS matrix.
\end{itemize}

\paragraph{Return values}
\begin{itemize}[leftmargin=2.1in]
\item[{\sf GrB\_SUCCESS}]	operation completed successfully
\item[{\sf GrB\_PANIC}]		unknown internal error
\end{itemize}

\paragraph{Description}

Upon return, all previously called GraphBLAS methods that used {\sf A} either as input or output have fully completed their execution.
Any (transparent or opaque) data structures produced or manipulated by those methods can be safely touched.

\subsubsection{Vector variant}

\paragraph{C99 syntax}

\begin{verbatim}
        GrB_info GrB_wait(GrB_Vector v, char *err)
\end{verbatim}

\paragraph{Parameters}
\begin{itemize}
\item[{\sf err}]     A null terminated string containing additional error
information.
\end{itemize}

\begin{itemize}[leftmargin=1.1in]
	\item[{\sf v}]	({\sf ARG0}) A GraphBLAS vector.
\end{itemize}

\paragraph{Return values}
\begin{itemize}[leftmargin=2.1in]
\item[{\sf GrB\_SUCCESS}]	operation completed successfully
\item[{\sf GrB\_PANIC}]		unknown internal error
\end{itemize}

\paragraph{Description}

Upon return, all previously called GraphBLAS methods that used {\sf v} either as input or output have fully completed their execution.
Any (transparent or opaque) data structures produced or manipulated by those methods can be safely touched.
