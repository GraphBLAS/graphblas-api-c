\chapter{Methods}
\section{Object Methods}
\subsection{Algebra Methods}
%-----------------------------------------------------------------------------

\subsubsection{{\sf IndexVectorOp\_new}: Create a new GraphBLAS index operator for vectors}

\scott{I think I like the name IndexVectorOp better than IndexOpVector.}

Initializes a new GraphBLAS vector index operator with a specified user-defined 
function and its types (domains).

\paragraph{\syntax}

\begin{verbatim}
    GrB_Info GrB_IndexVectorOp_new(GrB_IndexVectorOp  *index_vector_op,
                                   void (*index_vector_func)(void*,
                                                             const void*,
                                                             GrB_Index,
                                                             const void*),
                                   GrB_Type            d_out,
                                   GrB_Type            d_in1,
                                   GrB_Type            d_in2);
\end{verbatim}

\paragraph{Parameters}

\begin{itemize}[leftmargin=1.2in]
    \item[{\sf index\_vector\_op}] ({\sf INOUT}) On successful return, contains a 
          handle to the newly created GraphBLAS vector index operator object.
    \item[{\sf index\_vector\_func}] ({\sf IN}) A pointer to a user-defined function that 
          takes input parameters of types {\sf d\_in1}, {\sf GrB\_Index} and {\sf d\_in2}
          and returns a value of type {\sf d\_out}.  Except for the {\sf GrB\_Index}
          parameter, all are passed as {\sf void} pointers.
          Specifically the signature of the function is expected to 
          be of the form:
      \begin{verbatim}
      void func(void *out,
                const void *in1, GrB_index i, 
                const void *in2);
      \end{verbatim}
    \item[{\sf d\_out}]  ({\sf IN}) The {\sf GrB\_Type} of the return
          value of the vector index operator being created. Should be one of the
          predefined GraphBLAS types in Table~\ref{Tab:PredefinedTypes}, or a 
          user-defined GraphBLAS type.
    \item[{\sf d\_in1}]  ({\sf IN}) The {\sf GrB\_Type} of the left hand 
          argument of the vector index operator being created and corresponds to scalars 
          from the {\sf GrB\_Vector} being operated on. Should be one of the
          predefined GraphBLAS types in Table~\ref{Tab:PredefinedTypes}, or a
          user-defined GraphBLAS type.
    \item[{\sf d\_in2}]  ({\sf IN}) The {\sf GrB\_Type} of the right hand 
          argument of the vector index operator being created and corresponds to a
          scalar provided by the GraphBLAS operation that uses this operator. Should be one of the
          predefined GraphBLAS types in Table~\ref{Tab:PredefinedTypes}, or a 
          user-defined GraphBLAS type.
\end{itemize}

\paragraph{Return Values}

\begin{itemize}[leftmargin=2.1in]
\item[{\sf GrB\_SUCCESS}]           operation completed successfully.
\item[{\sf GrB\_PANIC}]             unknown internal error.
\item[{\sf GrB\_OUT\_OF\_MEMORY}]          not enough memory available for operation.
\item[{\sf GrB\_UNINITIALIZED\_OBJECT}]          the {\sf GrB\_Type} (for user-defined types)
                                    has not been initialized by a call to {\sf GrB\_Type\_new}.
\item[{\sf GrB\_NULL\_POINTER}]    {\sf index\_vector\_op} or {\sf index\_vector\_func} pointer is {\sf NULL}.

\jose{Domain mistmatch not possible.}
%\item[{\sf GrB\_DOMAIN\_MISMATCH}]  the types in the function pointer signature are not   
%                                    compatible with the {\sf GrB\_Type} parameters specified.
\end{itemize}

\paragraph{Description}

The {\sf IndexVectorOp\_new} methods creates a new GraphBLAS vector index operator
\begin{quote}
$^{i}f = \langle \mathbf{D}({\sf d\_out}), \mathbf{D}({\sf d\_in1}), \mathbf{D}({\sf GrB\_Index}), \mathbf{D}({\sf d\_in2}), {\sf index\_vector\_func} \rangle$
\end{quote}
and returns a handle to it in {\sf index\_vector\_op}.

The implementation of {\sf index\_vector\_func} must be such that it works
even if any of the {\sf d\_out}, {\sf d\_in1}, and {\sf d\_in2} arguments are aliased to each other.
In other words, for all invocations of the function:
\begin{quote}
\begin{verbatim}
index_vector_func(out,in1,i,in2);
\end{verbatim}
\end{quote}
the value of {\sf out} must be the same as if the following code
was executed:

\begin{quote}
\begin{code}
    $\mathbf{D}({\sf d\_in1})$ tmp1 = malloc(sizeof($\mathbf{D}({\sf d\_in1}$))); \\
    $\mathbf{D}({\sf d\_in2})$ tmp2 = malloc(sizeof($\mathbf{D}({\sf d\_in2}$))); \\
    memcpy(tmp1,in1,sizeof($\mathbf{D}({\sf d\_in1}$))); \\
    memcpy(tmp2,in2,sizeof($\mathbf{D}({\sf d\_in2}$))); \\
    index\_vector\_func(out,tmp1,i,tmp2); \\
    free(tmp2); \\
    free(tmp1);
\end{code}
\end{quote}

It is not an error to call this method more than once on the same variable;  
however, the handle to the previously created object will be overwritten. 

%-----------------------------------------------------------------------------

\subsubsection{{\sf IndexMatrixOp\_new}: Create a new GraphBLAS index operator for matrices}

\scott{I think I like IndexMatrixOp better than IndexOpMatrix.}


Initializes a new GraphBLAS matrix index operator with a specified user-defined 
function and its types (domains).

\paragraph{\syntax}

\begin{verbatim}
    GrB_Info GrB_IndexMatrixOp_new(GrB_IndexMatrixOp  *index_matrix_op,
                                   void (*index_matrix_func)(void*,
                                                             const void*,
                                                             const GrB_Index,
                                                             const GrB_Index,
                                                             const void*),
                                   GrB_Type            d_out,
                                   GrB_Type            d_in1,
                                   GrB_Type            d_in2);
\end{verbatim}

\paragraph{Parameters}

\begin{itemize}[leftmargin=1.1in]
    \item[{\sf index\_matrix\_op}] ({\sf INOUT}) On successful return, contains a 
          handle to the newly created GraphBLAS matrix index operator object.
    \item[{\sf matrix\_index\_func}] ({\sf IN}) A pointer to a user-defined function that 
          takes input parameters of types {\sf d\_in1}, {\sf GrB\_Index} and {\sf d\_in2}
          and returns a value of type {\sf d\_out}.  Except for the {\sf GrB\_Index}
          parameters, all are  passed as {\sf void} pointers.
          Specifically the signature of the function is expected to 
          be of the form:
      \begin{verbatim}
      void func(void *out,
                const void *in1, GrB_index i, GrB_index j, 
                const void *in2);
      \end{verbatim}
    \item[{\sf d\_out}]  ({\sf IN}) The {\sf GrB\_Type} of the return
          value of the matrix index operator being created. Should be one of the
          predefined GraphBLAS types in Table~\ref{Tab:PredefinedTypes}, or a 
          user-defined GraphBLAS type.
    \item[{\sf d\_in1}]  ({\sf IN}) The {\sf GrB\_Type} of the left hand 
          argument of the matrix index operator being created and corresponds to scalars 
          from the {\sf GrB\_Matrix} being operated on. Should be one of the
          predefined GraphBLAS types in Table~\ref{Tab:PredefinedTypes}, or a
          user-defined GraphBLAS type.
    \item[{\sf d\_in2}]  ({\sf IN}) The {\sf GrB\_Type} of the right hand 
          argument of the matrix index operator being created and corresponds to a
          scalar provided by the GraphBLAS operation that uses this operator. Should be one of the
          predefined GraphBLAS types in Table~\ref{Tab:PredefinedTypes}, or a 
          user-defined GraphBLAS type.
\end{itemize}

\paragraph{Return Values}

\begin{itemize}[leftmargin=2.1in]
\item[{\sf GrB\_SUCCESS}]           operation completed successfully.
\item[{\sf GrB\_PANIC}]             unknown internal error.
\item[{\sf GrB\_OUT\_OF\_MEMORY}]          not enough memory available for operation.
\item[{\sf GrB\_UNINITIALIZED\_OBJECT}]          the {\sf GrB\_Type} (for user-defined types)
                                    has not been initialized by a call to {\sf GrB\_Type\_new}.
\item[{\sf GrB\_NULL\_POINTER}]    {\sf index\_matrix\_op} or {\sf index\_matrix\_func} pointer is {\sf NULL}.

\jose{Domain mistmatch not possible.}
%\item[{\sf GrB\_DOMAIN\_MISMATCH}]  the types in the function pointer signature are not   
%                                    compatible with the {\sf GrB\_Type} parameters specified.
\end{itemize}

\paragraph{Description}

The {\sf IndexMatrixOp\_new} methods creates a new GraphBLAS matrix index operator
\begin{quote}
$^{i,j}f = \langle \mathbf{D}({\sf d\_out}), \mathbf{D}({\sf d\_in1}), \mathbf{D}({\sf GrB\_Index}), \mathbf{D}({\sf GrB\_Index}), \mathbf{D}({\sf d\_in2}), {\sf index\_matrix\_func} \rangle$
\end{quote}
and returns a handle to it in {\sf index\_matrix\_op}.

The implementation of {\sf index\_matrix\_func} must be such that it works
even if any of the {\sf d\_out}, {\sf d\_in1}, and {\sf d\_in2} arguments are aliased to each other.
In other words, for all invocations of the function:
\begin{quote}
\begin{verbatim}
index_matrix_func(out,in1,i,j,in2);
\end{verbatim}
\end{quote}
the value of {\sf out} must be the same as if the following code
was executed:

\begin{quote}
\begin{code}
    $\mathbf{D}({\sf d\_in1})$ tmp1 = malloc(sizeof($\mathbf{D}({\sf d\_in1}$))); \\
    $\mathbf{D}({\sf d\_in2})$ tmp2 = malloc(sizeof($\mathbf{D}({\sf d\_in2}$))); \\
    memcpy(tmp1,in1,sizeof($\mathbf{D}({\sf d\_in1}$))); \\
    memcpy(tmp2,in2,sizeof($\mathbf{D}({\sf d\_in2}$))); \\
    index\_matrix\_func(out,tmp1,i,j,tmp2); \\
    free(tmp2); \\
    free(tmp1);
\end{code}
\end{quote}

It is not an error to call this method more than once on the same variable;  
however, the handle to the previously created object will be overwritten. 

%-----------------------------------------------------------------------------

\section{GraphBLAS Operations}

%-----------------------------------------------------------------------------

\subsection{{\sf apply}: Apply a function to the elements of an object}
\subsubsection{{\sf apply}: index operator variants}


\subsection{{\sf select}: }