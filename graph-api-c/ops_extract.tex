\subsection{{\sf extract}: Selecting Sub-Graphs}
\label{Sec:extract}

Extract a subset of a matrix or vector. 

%--------------------------------------------------------------

\subsubsection{{\sf extract}: Standard vector variant}

Extract a sub-vector from a larger vector as specified by a set of indices. 
The result is a vector whose size is equal to the number of indices.

\paragraph{\syntax}

\begin{verbatim}
        GrB_Info GrB_extract(GrB_Vector             *w,
                             const GrB_Vector        mask,
                             const GrB_BinaryOp      accum,
                             const GrB_Vector        u,
                             const GrB_Index        *indices,
                             const GrB_Index         nindices,
                             const GrB_Descriptor    desc);
\end{verbatim}

\paragraph{Parameters}

\begin{itemize}[leftmargin=1in]
    \item[{\sf w}]    ({\sf INOUT}) An existing GraphBLAS vector.  On input,
    the vector provides values that may be accumulated with the result of the
    extract operation.  On output, this vector holds the results of the
    operation.

    \item[{\sf mask}]  ({\sf IN}) An optional ``write'' mask that controls which
    results from this operation are stored into the output vector,
    ${\sf w}$.  If no mask is desired (\ie, all elements
    of result are copied into the output vector), {\sf GrB\_NULL}
    should be specified. The mask dimensions must match those of the
    vector {\sf w} and the domain of {\sf mask} must be
    of type {\sf bool} or any of the predefined ``built-in'' types in
    Table~\ref{Tab:PredefinedTypes}.

    \item[{\sf accum}]    ({\sf IN}) An optional operator used for accumulating
    entries into existing {\sf w} entries: ${\sf accum} = \langle D_x,
    D_y, D_z,\odot \rangle$. If assignment rather than accumulation is
    desired, {\sf GrB\_NULL} should be specified.

    \item[{\sf u}]       ({\sf IN}) The GraphBLAS vector from which the subset
    is extracted.
    
    \item[{\sf indices}]  ({\sf IN}) Pointer to the ordered set (array) of 
    indices corresponding to the locations of elements from {\sf u} that are 
    extracted.  If all elements of {\sf u} are to be extracted in order from $0$ 
    to ${\sf nindices} - 1$, then {\sf GrB\_ALL} should be specified.  Regardless of 
    execution mode and return value, this array may be manipulated by the caller
    after this operation returns without affecting any deferred computations for 
    this operation.
    
    \item[{\sf nindices}] ({\sf IN}) The number of values in {\sf indices} array.
    Must be equal to $\bold{size}({\sf w})$.

    \item[{\sf desc}]     ({\sf IN}) An optional operation descriptor.  If a 
    \emph{default} descriptor is desired, {\sf GrB\_NULL} can be used.  Valid 
    fields are as follows: \\
    
    \begin{tabular}{lllp{2.5in}}
        Param & Field  & Value & Description \\
        \hline
        {\sf w}    & {\sf GrB\_OUTP} & {\sf GrB\_REPLACE} & Output vector {\sf w}
        is cleared (all elements removed) before the result is stored in it. \\
        
        {\sf mask} & {\sf GrB\_MASK} & {\sf GrB\_SCMP}   & Use the structural 
        complement of {\sf mask}. \\
    \end{tabular}
\end{itemize}

\paragraph{Return Values}

\begin{itemize}[leftmargin=2.1in]
    \item[{\sf GrB\_SUCCESS}]         In blocking mode, the operation completed
    successfully. In non-blocking mode, this indicates that the consistency 
    tests on dimensions and domains for the input arguments passed successfully. 
    Either way, output vector {\sf w} is ready to be used in the next method of 
    the sequence.

    \item[{\sf GrB\_PANIC}]            Unknown internal error.
    
    \item[{\sf GrB\_INVALID\_OBJECT}] This is returned in any execution mode 
    whenever one of the opaque GraphBLAS objects (input or output) is in an invalid 
    state caused by a previous execution error.  Call {GrB\_error()} to access 
    any error messages generated by the implementation.

    \item[{\sf GrB\_OUT\_OF\_MEMORY}]  Not enough memory available for operation.
    
    \item[{\sf GrB\_UNINITIALIZED\_OBJECT}] One or more of the GraphBLAS objects 
    has not been initialized by a call to {\sf new} (or {\sf dup} for vector
    parameters).
    
    \item[{\sf GrB\_NULL\_POINTER}]  {\sf w} pointer is {\sf NULL}.

    \item[{\sf GrB\_INDEX\_OUT\_OF\_BOUNDS}]  A value in {\sf indices} is greater 
    than or equal to $\bold{size}({\sf u})$.  In non-blocking mode, this can be
    reported as an execution error.
    
    \item[{\sf GrB\_DIMENSION\_MISMATCH}]  {\sf mask} and {\sf w} dimensions are
    incompatible, or ${\sf nindices} \neq \bold{size}({\sf w})$. 
    
    \item[{\sf GrB\_DOMAIN\_MISMATCH}]    The domains of the various vectors are
	incompatible with the corresponding domains of the accumulating operation or 
    mask.
\end{itemize}

\paragraph{Description}

This variant of {\sf GrB\_extract} computes the result of extracting a subset of
locations from a GraphBLAS vector in a specific order: 
${\sf w} = {\sf u}({\sf indices})$; or, if an optional binary accumulation 
operator ($\odot$) is provided, ${\sf w} = {\sf w} \odot {\sf u}({\sf indices})$.  
More explicitly:
\[
    {\sf w}(i) = {\sf u}({\sf indices}(i)) \mbox{~~or~~} 
    {\sf w}(i) = {\sf w}(i) \odot {\sf u}({\sf indices}(i))
    \ \forall i : 0 \leq i < {\sf nindices}
\]  
Logically, this operation occurs in three steps:
\begin{enumerate}[leftmargin=0.75in]
\item[\bf Setup] The internal vectors and mask used in the computation are formed 
and their domains and dimensions are tested for consistency.
\item[\bf Compute] The indicated computations are carried out.
\item[\bf Output] The result is written into the output vector, possibly under 
control of a mask.
\end{enumerate}

Up to three argument vectors are used in this {\sf GrB\_extract} operation:
\begin{enumerate}
	\item ${\sf w} = \langle \bold{D}({\sf w}),\bold{size}({\sf w}),
    \bold{L}({\sf w}) = \{(i,w_i) \} \rangle$

	\item ${\sf mask} = \langle \bold{D}({\sf mask}),\bold{size}({\sf mask}),
    \bold{L}({\sf mask}) = \{(i,m_i) \} \rangle$ (optional)

	\item ${\sf u} = \langle \bold{D}({\sf u}),\bold{size}({\sf u}),
    \bold{L}({\sf u}) = \{(i,u_i) \} \rangle$
\end{enumerate}

The argument vectors and the accumulation 
operator (if provided) are tested for domain consistency as follows:
\begin{enumerate}
	\item The domain of {\sf mask} (if not {\sf GrB\_NULL}) must be from one of 
    the pre-defined types of Table~\ref{Tab:PredefinedTypes}.

	\item If {\sf accum} is {\sf GrB\_NULL}, then $\bold{D}({\sf w})$ must be 
    compatible with $\bold{D}({\sf u})$.

	\item If {\sf accum} is not {\sf GrB\_NULL}, then $\bold{D}({\sf w})$ must be
    compatible with $D_x$ and $D_z$ of the accumulation operator and 
    $\bold{D}({\sf u})$ must be compatible with $D_y$ of the accumulation operator.
\end{enumerate}
Two domains are compatible with each other if values from one domain can be cast 
to values in the other domain as per the rules of the C language.
In particular, domains from Table~\ref{Tab:PredefinedTypes} are all compatible 
with each other. A domain from a user-defined type is only compatible with itself.
If any consistency rule above is violated, execution of {\sf GrB\_extract} ends
and the domain mismatch error listed above is returned.

From the arguments, the internal vectors, mask, and index array used in 
the computation are formed ($\leftarrow$ denotes copy):
\begin{enumerate}
	\item Vector $\vector{\widetilde{w}} \leftarrow {\sf w}$.

	\item One-dimensional mask, $\vector{\widetilde{m}}$, is computed from 
    argument {\sf mask} as follows:
	\begin{enumerate}
		\item	If ${\sf mask} = {\sf GrB\_NULL}$, then $\vector{\widetilde{m}} = 
        \langle \bold{size}({\sf w}), \{i, \forall i : 0 \leq i < 
        \bold{size}({\sf w}) \} \rangle$.

		\item	Otherwise, $\vector{\widetilde{m}} = 
        \langle \bold{size}({\sf mask}), \{i :  i \in \bold{ind}({\sf mask}) \wedge
        ({\sf bool}){\sf mask}(i) = \true \} \rangle$.

		\item	If ${\sf desc[GrB\_MASK].GrB\_SCMP}$ is \true, then 
        $\vector{\widetilde{m}} \leftarrow \neg \vector{\widetilde{m}}$.
	\end{enumerate}

	\item Vector $\vector{\widetilde{u}} \leftarrow {\sf u}$.
    
    \item The internal index array, $\array{\widetilde{I}}$, is computed from 
    argument {\sf indices} as follows:
	\begin{enumerate}
		\item	If ${\sf indices} = {\sf GrB\_ALL}$, then 
        $\array{\widetilde{I}}[i] = i, \forall i : 0 \leq i < {\sf nindices}$.

		\item	Otherwise, $\array{\widetilde{I}}[i] = {\sf indices}[i], 
        \forall i : 0 \leq i < {\sf nindices}$.
    \end{enumerate}
\end{enumerate}

The internal vectors and masks are checked for for dimension consistency. 
The following conditions must hold:
\begin{enumerate}
	\item $\bold{size}(\vector{\widetilde{w}}) = \bold{size}(\vector{\widetilde{m}})$
    \item ${\sf nindices} = \bold{size}(\vector{\widetilde{w}})$.
\end{enumerate}
If any consistency rule above is violated, execution of {\sf GrB\_extract} ends and 
the dimension mismatch error listed above is returned.

From this point forward, in {\sf GrB\_NONBLOCKING} mode, the method can 
optionally exit with {\sf GrB\_SUCCESS} return code and defer any computation 
and/or execution error codes.

We are now ready to carry out the extract and any additional 
associated operations.  We describe this in terms of two intermediate vectors:
\begin{itemize}
	\item $\vector{\widetilde{t}}$: The vector holding the extraction from
    $\vector{\widetilde{u}}$ in their destination locations relative to
    $\vector{\widetilde{w}}$.
    
	\item $\vector{\widetilde{z}}$: The vector holding the result after 
    application of the (optional) accumulation operator.
\end{itemize}

The intermediate vector, $\vector{\widetilde{t}}$, is created as follows:
\[ 
\vector{\widetilde{t}} = \langle
\bold{D}({\sf u}), \bold{size}(\vector{\widetilde{w}}),
\bold{L}(\vector{\widetilde{t}}) =
\{(i,\vector{\widetilde{u}}(\array{\widetilde{I}}[i])) \forall i, 0 \leq i < {\sf nindices} : 
\array{\widetilde{I}}[i] \in \bold{ind}(\vector{\widetilde{u}}) \} \rangle. 
\]
At this point, if any value of $\array{\widetilde{I}}$ is outside the valid
range of indices for vector $\vector{\widetilde{u}}$, computation ends and the 
method returns the index-out-of-bounds error listed above. In 
{\sf GrB\_NONBLOCKING} mode, the error can be deferred until a 
sequence-terminating {\sf GrB\_wait()} is called.  Regardless, the result 
vector, {\sf w}, is invalid from this point forward in the 
sequence.

The intermediate vector $\vector{\widetilde{z}}$ is created as follows:
\begin{itemize}
    \item If ${\sf accum} = {\sf GrB\_NULL}$, then 
    $\vector{\widetilde{z}} = \vector{\widetilde{t}}$.

    \item If ${\sf accum} = \langle D_x, D_y, D_z, \odot \rangle$, then vector 
    $\vector{\widetilde{z}}$ is defined as 
        \[ \langle D_z, \bold{size}(\vector{\widetilde{w}}), \bold{L}(\vector{\widetilde{z}})
		= \{(i,z_{i})  \forall (i) \in \bold{ind}(\vector{\widetilde{w}}) \cup 
        \bold{ind}(\vector{\widetilde{t}}) \} \rangle.\]
    The values of the elements of $\vector{\widetilde{z}}$ are computed based on the relationships between the sets of indices in $\vector{\widetilde{w}}$ and $\vector{\widetilde{t}}$.
\[
z_{i} = \vector{\widetilde{w}}(i) \odot \vector{\widetilde{t}}(i), \ \mbox{if}\  i \in  (\bold{ind}(\vector{\widetilde{t}}) \cap \bold{ind}(\vector{\widetilde{w}})),
\]
\[
z_{i} = \vector{\widetilde{w}}(i), \ \mbox{if}\  i \in  (\bold{ind}(\vector{\widetilde{w}}) - (\bold{ind}(\vector{\widetilde{t}}) \cap \bold{ind}(\vector{\widetilde{w}}))),
\]
\[
z_{i} = \vector{\widetilde{t}}(i), \ \mbox{if}\  i \in  (\bold{ind}(\vector{\widetilde{t}}) - (\bold{ind}(\vector{\widetilde{t}}) \cap \bold{ind}(\vector{\widetilde{w}}))).
\]
where the difference operator in the previous expressions refers to set difference.
\end{itemize}

Finally, the set of output values that make up the $\vector{\widetilde{z}}$ 
vector are written into the final result vector, {\sf w}. 
This is carried out under control of the mask which acts as a ``write mask''.
\begin{itemize}
\item If {\sf desc[GrB\_OUTP].GrB\_REPLACE} is set then any values in {\sf w} 
on input to {\sf GrB\_extract()} are deleted and the new output vector {\sf w} is,
\[ \bold{L}({\sf w}) = \{(i,z_{i}) : i \in (\bold{ind}(\vector{\widetilde{z}}) 
\cap \bold{ind}(\vector{\widetilde{m}})) \}. \]

\item If {\sf desc[GrB\_OUTP].GrB\_REPLACE} is not set, the elements of 
$\vector{\widetilde{z}}$ indicated by 
the mask are copied into the result vector, {\sf w}, and elements of 
{\sf w} that fall outside the set indicated by the mask are unchanged:
\[ \bold{L}({\sf w}) = \{(i,w_{i}) : i \in (\bold{ind}({\sf w}) 
\cap \bold{ind}(\neg \vector{\widetilde{m}})) \} \cup \{(i,z_{i}) : i \in 
(\bold{ind}(\vector{\widetilde{z}}) \cap \bold{ind}(\vector{\widetilde{m}})) \}. \]
\end{itemize}

In {\sf GrB\_BLOCKING} mode, the method exits with return value 
{\sf GrB\_SUCCESS} and the new content of vector {\sf w} is as defined above
and fully computed.  
In {\sf GrB\_NONBLOCKING} mode, the method exits with return value 
{\sf GrB\_SUCCESS} and the new content of vector {\sf w} is as defined above 
but may not be fully computed; however, it can be used in the next GraphBLAS 
method call in a sequence.

%--------------------------------------------------------------

\subsubsection{{\sf extract}: Standard matrix variant}

Extract a sub-matrix from a larger matrix as specified by a set of row indices
and a set of column indices.  The result is a matrix whose size is equal to size of the sets of indices.

\paragraph{\syntax}

\begin{verbatim}                 
        GrB_Info GrB_extract(GrB_Matrix           *C,
                             const GrB_Matrix      Mask,
                             const GrB_BinaryOp    accum,
                             const GrB_Matrix      A,
                             const GrB_Index      *row_indices,
                             const GrB_Index       nrows,
                             const GrB_Index      *col_indices,
                             const GrB_Index       ncols,
                             GrB_Descriptor const  desc);
\end{verbatim}

\paragraph{Parameters}

\begin{itemize}[leftmargin=1in]
    \item[{\sf C}]     ({\sf INOUT}) An existing GraphBLAS matrix.  On input,
    the matrix provides values that may be accumulated with the result of the
    extract operation.  On output, this matrix holds the results of the
    operation.

    \item[{\sf Mask}]  ({\sf IN}) An optional ``write'' mask that controls which
    results from this operation are stored into the output matrix,
    ${\sf C}$.  If no mask is desired (\ie, all elements
    of result are copied into the output vector), {\sf GrB\_NULL}
    should be specified.  The mask dimensions must match those of the
    matrix {\sf C} and the domain of {\sf Mask} must be
    of type {\sf bool} or any of the predefined ``built-in'' types in
    Table~\ref{Tab:PredefinedTypes}.

    \item[{\sf accum}] ({\sf IN})  An optional operator used for accumulating
    entries into existing {\sf C} entries: ${\sf accum} = \langle D_x,
    D_y, D_z,\odot \rangle$. If assignment rather than accumulation is
    desired, {\sf GrB\_NULL} should be specified.

    \item[{\sf A}]     ({\sf IN})  The GraphBLAS matrix from which the subset
    is extracted.

    \item[{\sf row\_indices}] ({\sf IN}) Pointer to the ordered set (array) of 
    indices corresponding to the rows of {\sf A} from which elements are 
    extracted.  If elements in all rows of {\sf A} are to be extracted in order, 
    {\sf GrB\_ALL} should be specified.  Regardless of execution mode and return
    value, this array may be manipulated by the caller after this operation 
    returns without affecting any deferred computations for this operation.
    
    \item[{\sf nrows}] ({\sf IN}) The number of indices in the {\sf row\_indices}
    array.  Must be equal to $\bold{nrows}({\sf C})$ if {\sf C} is not tranposed,
	or equal to $\bold{ncols}({\sf C})$ if {\sf C} is transposed.
    
    \item[{\sf col\_indices}] ({\sf IN}) Pointer to the ordered set (array) of 
    indices corresponding to the columns of {\sf A} from which elements are 
    extracted.  If elements in all columns of {\sf A} are to be extracted in order, 
    {\sf GrB\_ALL} should be specified.  Regardless of execution mode and return
    value, this array may be manipulated by the caller after this operation 
    returns without affecting any deferred computations for this operation.
    
    \item[{\sf ncols}] ({\sf IN}) The number of indices in the {\sf col\_indices}
    array.  Must be equal to $\bold{ncols}({\sf C})$ if {\sf C} is not tranposed,
	or equal to $\bold{nrows}({\sf C})$ if {\sf C} is transposed.

    \item[{\sf desc}]  ({\sf IN}) An optional operation descriptor.  If a 
    \emph{default} descriptor is desired, {\sf GrB\_NULL} can be used.  Valid 
    fields are as follows: \\
    
    \begin{tabular}{lllp{2.5in}}
        Param & Field  & Value & Description \\
        \hline
        {\sf C}    & {\sf GrB\_OUTP} & {\sf GrB\_REPLACE} & Output matrix {\sf C} 
        is cleared (all elements removed) before result is stored in it. \\
    
        {\sf Mask} & {\sf GrB\_MASK} & {\sf GrB\_SCMP}   & Use the structural 
        complement of {\sf Mask}. \\
    
        {\sf A}    & {\sf GrB\_INP0} & {\sf GrB\_TRAN}   & Apply transpose to 
        {\sf A} before extracting elements. \\
    \end{tabular}
\end{itemize}

\paragraph{Return Values}

\begin{itemize}[leftmargin=2.1in]
    \item[{\sf GrB\_SUCCESS}]         In blocking mode, the operation completed
    successfully. In non-blocking mode, this indicates that the consistency 
    tests on dimensions and domains for the input arguments passed successfully. 
    Either way, output matrix {\sf C} is ready to be used in the next method of 
    the sequence.

    \item[{\sf GrB\_PANIC}]            Unknown internal error.
    
    \item[{\sf GrB\_INVALID\_OBJECT}] This is returned in any execution mode 
    whenever one of the opaque GraphBLAS objects (input or output) is in an invalid 
    state caused by a previous execution error.  Call {GrB\_error()} to access 
    any error messages generated by the implementation.

    \item[{\sf GrB\_OUT\_OF\_MEMORY}]  Not enough memory available for operation.
    
    \item[{\sf GrB\_UNINITIALIZED\_OBJECT}] One or more of the GraphBLAS objects 
    has not been initialized by a call to {\sf new} (or {\sf dup} for matrix
    parameters).

    \item[{\sf GrB\_NULL\_POINTER}]  {\sf C} pointer is {\sf NULL}.

    \item[{\sf GrB\_INDEX\_OUT\_OF\_BOUNDS}]  A value in {\sf row\_indices} 
    is greater than or equal to $\bold{nrows}({\sf A})$, or a value in 
    {\sf col\_indices} is greater than or equal to $\bold{ncols}({\sf A})$.  In 
    non-blocking mode, this can be reported as an execution error.
    
    \item[{\sf GrB\_DIMENSION\_MISMATCH}] {\sf Mask} and {\sf C} dimensions are
    incompatible, ${\sf nrows} \neq \bold{nrows}({\sf C})$, or 
    ${\sf ncols} \neq \bold{ncols}({\sf C})$.

    \item[{\sf GrB\_DOMAIN\_MISMATCH}]     The domains of the various matrices
    are incompatible with the corresponding domains of the accumulating 
    operation or mask.
\end{itemize}

\paragraph{Description}

This variant of {\sf GrB\_extract} computes the result of extracting a subset of
locations from specified rows and columns of a GraphBLAS matrix in a specific 
order: ${\sf C} = {\sf A}({\sf row\_indices},{\sf col\_indices})$; or, if an 
optional binary accumulation operator ($\odot$) is provided, 
${\sf C} = {\sf C} \odot {\sf A}({\sf row\_indices},{\sf col\_indices})$.  
More explicitly (not accounting for an optional transpose of {\sf A}):
\[
\begin{aligned}
    {\sf C}(i,j) = &\ {\sf A}({\sf row\_indices}(i),{\sf col\_indices}(j)) 
    \ \forall \ i,j \ : \ 0 \leq i < {\sf nrows},\ 0 \leq j < {\sf ncols} \mbox{,~or~}
    \\
    {\sf C}(i,j) = &\ {\sf C}(i,j) \odot {\sf A}({\sf row\_indices}(i),{\sf col\_indices}(j))
    \ \forall \ i,j \ : \ 0 \leq i < {\sf nrows},\ 0 \leq j < {\sf ncols}
\end{aligned}
\]  
Logically, this operation occurs in three steps:
\begin{enumerate}[leftmargin=0.75in]
\item[\bf Setup] The internal matrices and mask used in the computation are formed 
and their domains and dimensions are tested for consistency.
\item[\bf Compute] The indicated computations are carried out.
\item[\bf Output] The result is written into the output matrix, possibly under 
control of a mask.
\end{enumerate}

Up to three argument matrices are used in this {\sf GrB\_extract} operation:
\begin{enumerate}
	\item ${\sf C} = \langle \bold{D}({\sf C}),\bold{nrows}({\sf C}),
    \bold{ncols}({\sf C}),\bold{L}({\sf C}) = \{(i,j,C_{ij}) \} \rangle$
    
	\item ${\sf Mask} = \langle \bold{D}({\sf Mask}),\bold{nrows}({\sf Mask}),
    \bold{ncols}({\sf Mask}),\bold{L}({\sf Mask}) = \{(i,j,M_{ij}) \} \rangle$ (optional)

	\item ${\sf A} = \langle \bold{D}({\sf A}),\bold{nrows}({\sf A}),
    \bold{ncols}({\sf A}),\bold{L}({\sf A}) = \{(i,j,A_{ij}) \} \rangle$
\end{enumerate}

The argument matrices and the accumulation 
operator (if provided) are tested for domain consistency as follows:
\begin{enumerate}
	\item The domain of {\sf Mask} (if not {\sf GrB\_NULL}) must be from one of 
    the pre-defined types of Table~\ref{Tab:PredefinedTypes}.

	\item If {\sf accum} is {\sf GrB\_NULL}, then $\bold{D}({\sf C})$ must be 
    compatible with $\bold{D}({\sf A})$.

	\item If {\sf accum} is not {\sf GrB\_NULL}, then $\bold{D}({\sf C})$ must be
    compatible with $D_x$ and $D_z$ of the accumulation operator and 
    $\bold{D}({\sf A})$ must be compatible with $D_y$ of the accumulation operator.
\end{enumerate}
Two domains are compatible with each other if values from one domain can be cast 
to values in the other domain as per the rules of the C language.
In particular, domains from Table~\ref{Tab:PredefinedTypes} are all compatible 
with each other. A domain from a user-defined type is only compatible with itself.
If any consistency rule above is violated, execution of {\sf GrB\_extract} ends
and the domain mismatch error listed above is returned.

From the arguments, the internal matrices, mask, and index arrays used in 
the computation are formed ($\leftarrow$ denotes copy):
\begin{enumerate}
	\item Matrix $\matrix{\widetilde{C}} \leftarrow {\sf C}$.

	\item Two-dimensional mask, $\matrix{\widetilde{M}}$, is computed from 
    argument {\sf Mask} as follows:
	\begin{enumerate}

		\item	If ${\sf Mask} = {\sf GrB\_NULL}$, then $\matrix{\widetilde{M}} = 
        \langle \bold{nrows}({\sf C}), \bold{ncols}({\sf C}), \{(i,j), 
        \forall i,j : 0 \leq i <  \bold{nrows}({\sf C}), 0 \leq j < 
        \bold{ncols}({\sf C}) \} \rangle$.

		\item	Otherwise, $\matrix{\widetilde{M}} = \langle \bold{nrows}({\sf Mask}), 
        \bold{ncols}({\sf Mask}), \{(i,j) : (i,j) \in \bold{ind}({\sf Mask}) \wedge 
        ({\sf bool}){\sf Mask}(i,j) = \true\} \rangle$.

		\item	If ${\sf desc[GrB\_MASK].GrB\_SCMP}$ is set, then 
        $\matrix{\widetilde{M}} \leftarrow \neg \matrix{\widetilde{M}}$.

	\end{enumerate}

	\item Matrix $\matrix{\widetilde{A}} \leftarrow 
    {\sf desc[GrB\_INP0].GrB\_TRAN} \ ? \ {\sf A}^T : {\sf A}$.
    
    \item The internal row index array, $\array{\widetilde{I}}$, is computed from 
    argument {\sf row\_indices} as follows:
	\begin{enumerate}
		\item	If ${\sf row\_indices} = {\sf GrB\_ALL}$, then 
        $\array{\widetilde{I}}[i] = i, \forall i : 0 \leq i < {\sf nrows} $.

		\item	Otherwise, $\array{\widetilde{I}}[i] = {\sf row\_indices}[i], 
        \forall i : 0 \leq i < {\sf nrows}$.
    \end{enumerate}
    
    \item The internal column index array, $\array{\widetilde{J}}$, is computed from 
    argument {\sf col\_indices} as follows:
	\begin{enumerate}
		\item	If ${\sf col\_indices} = {\sf GrB\_ALL}$, then 
        $\array{\widetilde{J}}[j] = j, \forall j : 0 \leq j < {\sf ncols}$.

		\item	Otherwise, $\array{\widetilde{J}}[j] = {\sf col\_indices}[j], 
        \forall j : 0 \leq j < {\sf ncols}$.
    \end{enumerate}
\end{enumerate}

The internal matrices, mask and arrays are checked for dimension consistency. 
The following conditions must hold:
\begin{enumerate}
	\item $\bold{nrows}(\matrix{\widetilde{C}}) = \bold{nrows}(\matrix{\widetilde{M}})$.

	\item $\bold{ncols}(\matrix{\widetilde{C}}) = \bold{ncols}(\matrix{\widetilde{M}})$.

	\item $\bold{nrows}(\matrix{\widetilde{C}}) = {\sf nrows}$.

	\item $\bold{ncols}(\matrix{\widetilde{C}}) = {\sf ncols}$.
\end{enumerate}
If any consistency rule above is violated, execution of {\sf GrB\_extract} ends 
and the dimension mismatch error listed above is returned.

From this point forward, in {\sf GrB\_NONBLOCKING} mode, the method can 
optionally exit with {\sf GrB\_SUCCESS} return code and defer any computation 
and/or execution error codes.

We are now ready to carry out the extract and any additional 
associated operations.  We describe this in terms of two intermediate matrices:
\begin{itemize}
    \item $\matrix{\widetilde{T}}$: The matrix holding the extraction from 
    $\matrix{\widetilde{A}}$.
    
    \item $\matrix{\widetilde{Z}}$: The matrix holding the result after 
    application of the (optional) accumulation operator.
\end{itemize}

The intermediate matrix, $\matrix{\widetilde{T}}$, is created as follows:
\[
\begin{aligned}
\matrix{\widetilde{T}} = \langle & \bold{D}({\sf A}),
                           \bold{nrows}(\matrix{\widetilde{C}}), 
                           \bold{ncols}(\matrix{\widetilde{C}}),  \\
                         & \bold{L}(\matrix{\widetilde{T}}) =
\{(i,j,\matrix{\widetilde{A}}(\array{\widetilde{I}}[i],\array{\widetilde{J}}[j])) 
\ \forall \ (i,j), \ 0 \leq i < {\sf nrows}, \ 0 \leq j < {\sf ncols} :
(\array{\widetilde{I}}[i], \array{\widetilde{J}}[j]) \in 
\bold{ind}(\matrix{\widetilde{A}}) \} \rangle.
\end{aligned}
\]
At this point, if any value in the $\array{\widetilde{I}}$ array is not in
the range $[0,\ \bold{nrows}(\matrix{\widetilde{A}}) )$ or any value in the 
$\array{\widetilde{J}}$ array is not in the range 
$[0,\ \bold{ncols}(\matrix{\widetilde{A}}))$, the execution of {\sf GrB\_extract} 
ends and the index out-of-bounds error listed above is generated.  In 
{\sf GrB\_NONBLOCKING} mode, the error can be deferred until a 
sequence-terminating {\sf GrB\_wait()} is called.  Regardless, the result 
matrix {\sf C} is invalid from this point forward in the sequence.

The intermediate matrix $\matrix{\widetilde{Z}}$ is created as follows:
\begin{itemize}
    \item If ${\sf accum} = {\sf GrB\_NULL}$, then $\matrix{\widetilde{Z}} = \matrix{\widetilde{T}}$.

    \item If ${\sf accum} = \langle D_x, D_y, D_z, \odot \rangle$, then matrix $\matrix{\widetilde{Z}}$ is defined as 
        \[ \langle D_z, \bold{nrows}(\matrix{\widetilde{C}}), \bold{ncols}(\matrix{\widetilde{C}}),
        \bold{L}(\matrix{\widetilde{Z}}) 
		= \{(i,j,Z_{ij})  \forall (i,j) \in \bold{ind}(\matrix{\widetilde{C}}) \cup 
        \bold{ind}(\matrix{\widetilde{T}}) \} \rangle. \]

        The values of the elements of $\matrix{\widetilde{Z}}$ are computed based on the 
        relationships between the sets of indices in $\matrix{\widetilde{C}}$ and 
        $\matrix{\widetilde{T}}$.
\[
Z_{ij} = \matrix{\widetilde{C}}(i,j) \odot \matrix{\widetilde{T}}(i,j), \ \mbox{if}\  (i,j) \in  (\bold{ind}(\matrix{\widetilde{T}}) \cap \bold{ind}(\matrix{\widetilde{C}})),
\]
\[
Z_{ij} = \matrix{\widetilde{C}}(i,j) \ \mbox{if}\  (i,j) \in  (\bold{ind}(\matrix{\widetilde{C}}) - (\bold{ind}(\matrix{\widetilde{T}}) \cap \bold{ind}(\matrix{\widetilde{C}}))),
\]
\[
Z_{ij} = \matrix{\widetilde{T}}(i,j) \ \mbox{if}\  (i,j) \in  (\bold{ind}(\matrix{\widetilde{T}}) - (\bold{ind}(\matrix{\widetilde{T}}) \cap \bold{ind}(\matrix{\widetilde{C}}))).
\]
where the difference operator in the previous expressions refers to set difference.
\end{itemize}

Finally, the set of output values that make up the $\matrix{\widetilde{Z}}$ 
matrix are written into the final result matrix, {\sf C}. 
This is carried out under control of the mask which acts as a ``write mask''.
\begin{itemize}
\item If {\sf desc[GrB\_OUTP].GrB\_REPLACE} is set then any values in {\sf C} 
on input to {\sf GrB\_extract()} are deleted and the new output matrix {\sf C} is,
\[ \bold{L}({\sf C}) = \{(i,j,Z_{ij}) : (i,j) \in (\bold{ind}(\matrix{\widetilde{Z}}) 
\cap \bold{ind}(\matrix{\widetilde{M}})) \}. \]

\item If {\sf desc[GrB\_OUTP].GrB\_REPLACE} is not set, the elements of 
$\matrix{\widetilde{Z}}$ indicated by 
the mask are copied into the result matrix, {\sf C}, and elements of 
$\matrix{C}$ that fall outside the set indicated by the mask are unchanged:
\[ \bold{L}({\sf C}) = \{(i,j,C_{ij}) : (i, j) \in (\bold{ind}({\sf C}) 
\cap \bold{ind}(\neg \matrix{\widetilde{M}})) \} \cup \{(i,j,Z_{ij}) : (i,j) \in 
(\bold{ind}(\matrix{\widetilde{Z}}) \cap \bold{ind}(\matrix{\widetilde{M}})) \}. \]
\end{itemize}

In {\sf GrB\_BLOCKING} mode, the method exits with return value 
{\sf GrB\_SUCCESS} and the new content of matrix {\sf C} is as defined above
and fully computed.  
In {\sf GrB\_NONBLOCKING} mode, the method exits with return value 
{\sf GrB\_SUCCESS} and the new content of matrix {\sf C} is as defined above 
but may not be fully computed; however, it can be used in the next GraphBLAS 
method call in a sequence.

%-----------------------------------------------------------------------------
\subsubsection{{\sf extract}: Column (and row) variant}

Extract from one column of a matrix into a vector.  Note that with the transpose
descriptor for the source matrix, elements of an arbitrary row of the matrix
can be extracted with this function as well.

\paragraph{\syntax}

\begin{verbatim}
        GrB_Info GrB_extract(GrB_Vector             *w,
                             const GrB_Vector        mask,
                             const GrB_BinaryOp      accum,
                             const GrB_Matrix        A,
                             const GrB_Index        *row_indices,
                             GrB_Index               nrows,
                             GrB_Index               col_index,
                             const GrB_Descriptor    desc); 
\end{verbatim}

\paragraph{Parameters}

\begin{itemize}[leftmargin=1in]
    \item[{\sf w}]    ({\sf INOUT}) An existing GraphBLAS vector.  On input,
    the vector provides values that may be accumulated with the result of the
    extract operation.  On output, this vector holds the results of the
    operation.

    \item[{\sf mask}]  ({\sf IN}) An optional ``write'' mask that controls which
    results from this operation are stored into the output vector,
    ${\sf w}$.  If no mask is desired (\ie, all elements
    of result are copied into the output vector), {\sf GrB\_NULL}
    should be specified. The mask dimensions must match those of the
    vector {\sf w} and the domain of {\sf mask} must be
    of type {\sf bool} or any of the predefined ``built-in'' types in
    Table~\ref{Tab:PredefinedTypes}.

    \item[{\sf accum}]    ({\sf IN}) An optional operator used for accumulating
    entries into existing {\sf w} entries: ${\sf accum} = \langle D_x,
    D_y, D_z,\odot \rangle$. If assignment rather than accumulation is
    desired, {\sf GrB\_NULL} should be specified.


    \item[{\sf A}]     ({\sf IN})  The GraphBLAS matrix from which the column
    subset is extracted.

    \item[{\sf row\_indices}] ({\sf IN}) Pointer to the ordered set (array) of 
    indices corresponding to the locations within the specified column of {\sf A} 
    from which elements are extracted.  If elements in all rows of {\sf A} are 
    to be extracted in order, {\sf GrB\_ALL} should be specified. Regardless of
    execution mode and return value, this array may be manipulated by the caller 
    after this operation returns without affecting any deferred computations for 
    this operation.
    
    \item[{\sf nrows}] ({\sf IN}) The number of indices in the {\sf row\_indices}
    array.  Must be equal to $\bold{size}({\sf w})$.
    
    \item[{\sf col\_index}]  ({\sf IN}) The index of the column of {\sf A} from
    which to extract values.  It must be in the range $[0,\ \bold{ncols}({\sf A}))$.

    \item[{\sf desc}]  ({\sf IN}) An optional operation descriptor.  If a 
    \emph{default} descriptor is desired, {\sf GrB\_NULL} can be used.  Valid 
    fields are as follows: \\
    
    \begin{tabular}{lllp{2.5in}}
        Param & Field  & Value & Description \\
        \hline
        {\sf w}    & {\sf GrB\_OUTP} & {\sf GrB\_REPLACE} & Output vector {\sf w}
        is cleared (all elements removed) before result is stored in it. \\
        
        {\sf mask} & {\sf GrB\_MASK} & {\sf GrB\_SCMP} & Use the structural
        complement of {\sf mask}. \\
        
        {\sf A}    & {\sf GrB\_INP0} & {\sf GrB\_TRAN} & Apply transpose to 
        {\sf A} before extract. \\
    \end{tabular}
\end{itemize}

\paragraph{Return Values}

\begin{itemize}[leftmargin=2.1in]
    \item[{\sf GrB\_SUCCESS}]         In blocking mode, the operation completed
    successfully. In non-blocking mode, this indicates that the consistency 
    tests on dimensions and domains for the input arguments passed successfully. 
    Either way, output vector {\sf w} is ready to be used in the next method of 
    the sequence.

    \item[{\sf GrB\_PANIC}]            Unknown internal error.
    
    \item[{\sf GrB\_INVALID\_OBJECT}] This is returned in any execution mode 
    whenever one of the opaque GraphBLAS objects (input or output) is in an invalid 
    state caused by a previous execution error.  Call {GrB\_error()} to access 
    any error messages generated by the implementation.

    \item[{\sf GrB\_OUT\_OF\_MEMORY}]  Not enough memory available for operation.
    
    \item[{\sf GrB\_UNINITIALIZED\_OBJECT}] One or more of the GraphBLAS objects 
    has not been initialized by a call to {\sf new} (or {\sf dup} for vector
    or matrix parameters).
    
    \item[{\sf GrB\_NULL\_POINTER}]    {\sf w} pointer is NULL.
    
    \item[{\sf GrB\_INVALID\_INDEX}]    {\sf col\_index} is outside the allowable 
    range (i.e., greater than $\bold{ncols}({\sf A})$).

    \item[{\sf GrB\_INDEX\_OUT\_OF\_BOUNDS}]  A value in {\sf row\_indices} 
    is greater than or equal to $\bold{nrows}({\sf A})$.  In 
    non-blocking mode, this can be reported as an execution error.
    
    \item[{\sf GrB\_DIMENSION\_MISMATCH}] {\sf mask} and {\sf w} dimensions are
    incompatible, or ${\sf nrows} \neq \bold{size}({\sf w})$.

    \item[{\sf GrB\_DOMAIN\_MISMATCH}]     The domains of the vector or matrix
    are incompatible with the corresponding domains of the accumulating 
    operation or mask.
\end{itemize}

\paragraph{Description}

This variant of {\sf GrB\_extract} computes the result of extracting a subset of
locations (in a specific order) from a specified column of a GraphBLAS matrix: 
${\sf w} = {\sf A}(:,{\sf col\_index})({\sf row\_indices})$; or, if an 
optional binary accumulation operator ($\odot$) is provided, 
${\sf w} = {\sf w} \odot {\sf A}(:,{\sf col\_index})({\sf row\_indices})$.  
More explicitly:
\[
\begin{aligned}
    {\sf w}(i) = &\ {\sf A}({\sf row\_indices}(i),{\sf col\_index}) 
    \ \forall \ i \ : \ 0 \leq i < {\sf nrows}, \mbox{~or~}
    \\
    {\sf w}(i) = &\ {\sf w}(i) \odot {\sf A}({\sf row\_indices}(i),{\sf col\_index})
    \ \forall \ i \ : \ 0 \leq i < {\sf nrows}
\end{aligned}
\]  
Logically, this operation occurs in three steps:
\begin{enumerate}[leftmargin=0.75in]
\item[\bf Setup] The internal matrices, vectors, and mask used in the computation are formed 
and their domains and dimensions are tested for consistency.
\item[\bf Compute] The indicated computations are carried out.
\item[\bf Output] The result is written into the output vector, possibly under 
control of a mask.
\end{enumerate}

Up to three argument vectors and matrices are used in this {\sf GrB\_extract} 
operation:
\begin{enumerate}
	\item ${\sf w} = \langle \bold{D}({\sf w}),\bold{size}({\sf w}),
    \bold{L}({\sf w}) = \{(i,w_{i}) \} \rangle$
    
	\item ${\sf mask} = \langle \bold{D}({\sf mask}),\bold{size}({\sf mask}),
    \bold{L}({\sf mask}) = \{(i,m_{i}) \} \rangle$ (optional)

	\item ${\sf A} = \langle \bold{D}({\sf A}),\bold{nrows}({\sf A}),
    \bold{ncols}({\sf A}),\bold{L}({\sf A}) = \{(i,j,A_{ij}) \} \rangle$
\end{enumerate}

The argument vectors, matrix and the accumulation 
operator (if provided) are tested for domain consistency as follows:
\begin{enumerate}
	\item The domain of {\sf mask} (if not {\sf GrB\_NULL}) must be from one of 
    the pre-defined types of Table~\ref{Tab:PredefinedTypes}.

	\item If {\sf accum} is {\sf GrB\_NULL}, then $\bold{D}({\sf w})$ must be 
    compatible with $\bold{D}({\sf A})$.

	\item If {\sf accum} is not {\sf GrB\_NULL}, then $\bold{D}({\sf w})$ must be
    compatible with $D_x$ and $D_z$ of the accumulation operator and 
    $\bold{D}({\sf A})$ must be compatible with $D_y$ of the accumulation operator.
\end{enumerate}
Two domains are compatible with each other if values from one domain can be cast 
to values in the other domain as per the rules of the C language.
In particular, domains from Table~\ref{Tab:PredefinedTypes} are all compatible 
with each other. A domain from a user-defined type is only compatible with itself.
If any consistency rule above is violated, execution of {\sf GrB\_extract} ends
and the domain mismatch error listed above is returned.

From the arguments, the internal vector, matrix, mask, and index array used in 
the computation are formed ($\leftarrow$ denotes copy):
\begin{enumerate}
	\item Vector $\vector{\widetilde{w}} \leftarrow {\sf w}$.

	\item One-dimensional mask, $\vector{\widetilde{m}}$, is computed from 
    argument {\sf mask} as follows:
	\begin{enumerate}
		\item	If ${\sf mask} = {\sf GrB\_NULL}$, then $\vector{\widetilde{m}} = 
        \langle \bold{size}({\sf w}), \{i, \forall i : 0 \leq i < 
        \bold{size}({\sf w}) \} \rangle$.

		\item	Otherwise, $\vector{\widetilde{m}} = 
        \langle \bold{size}({\sf mask}), \{i :  i \in \bold{ind}({\sf mask}) \wedge
        ({\sf bool}){\sf mask}(i) = \true \} \rangle$.

		\item	If ${\sf desc[GrB\_MASK].GrB\_SCMP}$ is \true, then 
        $\vector{\widetilde{m}} \leftarrow \neg \vector{\widetilde{m}}$.
	\end{enumerate}

	\item Matrix $\matrix{\widetilde{A}} \leftarrow 
    {\sf desc[GrB\_INP0].GrB\_TRAN} \ ? \ {\sf A}^T : {\sf A}$.
    
    \item The internal row index array, $\array{\widetilde{I}}$, is computed from 
    argument {\sf row\_indices} as follows:
	\begin{enumerate}
		\item	If ${\sf indices} = {\sf GrB\_ALL}$, then 
        $\array{\widetilde{I}}[i] = i, \forall i : 0 \leq i < {\sf nrows}$.

		\item	Otherwise, $\array{\widetilde{I}}[i] = {\sf indices}[i], 
        \forall i : 0 \leq i < {\sf nrows}$.
    \end{enumerate}
\end{enumerate}

The internal vector, mask, and index array are checked for dimension 
consistency.  The following conditions must hold:
\begin{enumerate}
	\item $\bold{size}(\vector{\widetilde{w}}) = \bold{size}(\vector{\widetilde{m}})$.

	\item $\bold{size}(\matrix{\widetilde{w}}) = {\sf nrows}$.
\end{enumerate}
If any consistency rule above is violated, execution of {\sf GrB\_extract} ends 
and the dimension mismatch error listed above is returned.

The {\sf col\_index} parameter is checked for a valid value.  The following
condition must hold:
\begin{enumerate}
	\item $0\ \leq\ {\sf col\_index} \ <\ \bold{ncols}({\sf A})$
\end{enumerate}
If the rule above is violated, execution of {\sf GrB\_extract} ends 
and the invalid index error listed above is returned.

From this point forward, in {\sf GrB\_NONBLOCKING} mode, the method can 
optionally exit with {\sf GrB\_SUCCESS} return code and defer any computation 
and/or execution error codes.

We are now ready to carry out the extract and any additional 
associated operations.  We describe this in terms of two intermediate vectors:
\begin{itemize}
	\item $\vector{\widetilde{t}}$: The vector holding the extraction from
    a column of $\matrix{\widetilde{A}}$.
	\item $\vector{\widetilde{z}}$: The vector holding the result after 
    application of the (optional) accumulation operator.
\end{itemize}

The intermediate vector, $\vector{\widetilde{t}}$, is created as follows:
\[ 
\vector{\widetilde{t}} = \langle \bold{D}({\sf A}), {\sf nrows},
\bold{L}(\vector{\widetilde{t}}) = 
\{(i,\matrix{\widetilde{A}}(\array{\widetilde{I}}[i],{\sf col\_index})) 
\ \forall\ i, 0 \leq i < {\sf nrows} : 
(\array{\widetilde{I}}[i], {\sf col\_index}) \in 
\bold{ind}(\matrix{\widetilde{A}}) \} \rangle. 
\]
At this point, if any value in the $\array{\widetilde{I}}$ array is not in
the range $[0,\ \bold{nrows}(\matrix{\widetilde{A}}) )$, the execution of 
{\sf GrB\_extract} ends and the index out-of-bounds error listed above is 
generated.   In 
{\sf GrB\_NONBLOCKING} mode, the error can be deferred until a 
sequence-terminating {\sf GrB\_wait()} is called.  Regardless, the result 
vector, {\sf w}, is invalid from this point forward in the 
sequence.

The intermediate vector $\vector{\widetilde{z}}$ is created as follows:
\begin{itemize}
    \item If ${\sf accum} = {\sf GrB\_NULL}$, then $\vector{\widetilde{z}} = \vector{\widetilde{t}}$.

    \item If ${\sf accum} = \langle D_x, D_y, D_z, \odot \rangle$, then vector $\vector{\widetilde{z}}$ is defined as 
        \[ \langle D_z, \bold{size}(\vector{\widetilde{w}}), \bold{L}(\vector{\widetilde{z}})
		= \{(i,z_{i})  \forall (i) \in \bold{ind}(\vector{\widetilde{w}}) \cup 
        \bold{ind}(\vector{\widetilde{t}}) \} \rangle.\]
    The values of the elements of $\vector{\widetilde{z}}$ are computed based on the relationships between the sets of indices in $\vector{\widetilde{w}}$ and $\vector{\widetilde{t}}$.
\[
z_{i} = \vector{\widetilde{w}}(i) \odot \vector{\widetilde{t}}(i), \ \mbox{if}\  i \in  (\bold{ind}(\vector{\widetilde{t}}) \cap \bold{ind}(\vector{\widetilde{w}})),
\]
\[
z_{i} = \vector{\widetilde{w}}(i), \ \mbox{if}\  i \in  (\bold{ind}(\vector{\widetilde{w}}) - (\bold{ind}(\vector{\widetilde{t}}) \cap \bold{ind}(\vector{\widetilde{w}}))),
\]
\[
z_{i} = \vector{\widetilde{t}}(i), \ \mbox{if}\  i \in  (\bold{ind}(\vector{\widetilde{t}}) - (\bold{ind}(\vector{\widetilde{t}}) \cap \bold{ind}(\vector{\widetilde{w}}))).
\]
where the difference operator in the previous expressions refers to set difference.
\end{itemize}

Finally, the set of output values that make up the $\vector{\widetilde{z}}$ 
vector are written into the final result vector, {\sf w}. 
This is carried out under control of the mask which acts as a ``write mask''.
\begin{itemize}
\item If {\sf desc[GrB\_OUTP].GrB\_REPLACE} is set then any values in {\sf w} 
on input to {\sf GrB\_extract()} are deleted and the new output vector {\sf w} is,
\[ \bold{L}({\sf w}) = \{(i,z_{i}) : i \in (\bold{ind}(\vector{\widetilde{z}}) 
\cap \bold{ind}(\vector{\widetilde{m}})) \}. \]

\item If {\sf desc[GrB\_OUTP].GrB\_REPLACE} is not set, the elements of 
$\vector{\widetilde{z}}$ indicated by 
the mask are copied into the result vector, {\sf w}, and elements of 
{\sf w} that fall outside the set indicated by the mask are unchanged:
\[ \bold{L}({\sf w}) = \{(i,w_{i}) : i \in (\bold{ind}({\sf w}) 
\cap \bold{ind}(\neg \vector{\widetilde{m}})) \} \cup \{(i,z_{i}) : i \in 
(\bold{ind}(\vector{\widetilde{z}}) \cap \bold{ind}(\vector{\widetilde{m}})) \}. \]
\end{itemize}

In {\sf GrB\_BLOCKING} mode, the method exits with return value 
{\sf GrB\_SUCCESS} and the new content of vector {\sf w} is as defined above
and fully computed.  
In {\sf GrB\_NONBLOCKING} mode, the method exits with return value 
{\sf GrB\_SUCCESS} and the new content of vector {\sf w} is as defined above 
but may not be fully computed; however, it can be used in the next GraphBLAS 
method call in a sequence.
