\section{Sequence Termination}

\scott{Can NULL be legally passed for err? Does the implementation manipulate the char pointer? Who allocates and who owns the error string upon return}

%-----------------------------------------------------------------------------
\subsection{{\sf wait}: Waits until pending operations complete}
\label{Sec:wait}

When running in non-blocking mode, this function guarantees that all pending GraphBLAS operations are fully executed.  Note that this can be called in blocking mode without an error, but there should be no pending GraphBLAS operations to complete.

\paragraph{\syntax}

\begin{verbatim}
        GrB_Info GrB_wait();
\end{verbatim}

\paragraph{Parameters}

\paragraph{Return values}
\begin{itemize}[leftmargin=2.1in]
	\item[{\sf GrB\_SUCCESS}]	operation completed successfully.
	\item[{\sf GrB\_INDEX\_OUT\_OF\_BOUNDS}]	an index out-of-bounds execution error happend during completion of pending operations.
	\item[{\sf GrB\_OUT\_OF\_MEMORY}]		and out-of-memory execution error happened during completion of pending operations.
	\item[{\sf GrB\_PANIC}]		unknown internal error.
\end{itemize}

\paragraph{Description}

Upon successful return, all previously called GraphBLAS methods have
fully completed their execution, and 
 any (transparent or opaque) data
structures produced or manipulated by those methods can be safely touched.
If an error occured in any pending GraphBLAS operations, {\sf GrB\_error()}
can be used to retrieve implementation defined
error information about the problem encountered.


%-----------------------------------------------------------------------------
\subsection{{error}: Get an error message regarding internal errors}

\scott{Maybe this does not belong here or maybe this section should change names
or merge with the context methods.}

\begin{verbatim}
        const char *GrB_error();
\end{verbatim}

\paragraph{Parameters}

\paragraph{Return value}
\begin{itemize}[leftmargin=2.1in]
	\item A pointer to a null-terminated string (owned by the library).
\end{itemize}

\paragraph{Description}

\scott{Copied from Basic Concepts}

After a call to any GraphBLAS method, the program can retrieve additional
error information (beyond the error code returned by the method) though a
call to the function {\sf GrB\_error()}. 
The function returns a pointer to a null terminated string and the contents of that string
are implementation dependent. In particular, a null string (not a {\sf NULL} pointer) is always a valid error string.
The pointer is valid until the next call to any GraphBLAS method by the same thread.
{\sf GrB\_error()} is a thread-safe function, in the sense that multiple threads can
call it simultaneously and each will get its own error string back, referring to the
last GraphBLAS method it called.
