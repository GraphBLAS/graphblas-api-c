\subsection{{\sf assign}: Modifying sub-graphs}
\label{Sec:assign}

Assign the contents of a subset of a matrix or vector.

%-----------------------------------------------------------------------------

\subsubsection{{\sf assign}: Standard vector variant}

Assign values from one GraphBLAS vector to a subset of a 
vector as specified by a set of indices. The size of the input vector is the
same size as the index array provided.

\paragraph{\syntax}

\begin{verbatim}
        GrB_Info GrB_assign(GrB_Vector            w,
                            const GrB_Vector      mask,
                            const GrB_BinaryOp    accum,
                            const GrB_Vector      u,
                            const GrB_Index      *indices,
                            GrB_Index             nindices,
                            const GrB_Descriptor  desc);
\end{verbatim}

\paragraph{Parameters}

\begin{itemize}[leftmargin=1.1in]
    \item[{\sf w}]    ({\sf INOUT}) An existing GraphBLAS vector.  On input,
    the vector provides values that may be accumulated with the result of the
    assign operation.  On output, this vector holds the results of the
    operation.

    \item[{\sf mask}] ({\sf IN}) An optional ``write'' mask that controls which
    results from this operation are stored into the output vector {\sf w}. The 
    mask dimensions must match those of the vector {\sf w} If the 
    {\sf GrB\_STRUCTURE} descriptor is {\em not} set for the mask, the domain of the
    {\sf mask} vector must be of type {\sf bool} or any of the predefined 
    ``built-in'' types in Table~\ref{Tab:PredefinedTypes}.  If the default
    mask is desired (\ie, a mask that is all {\sf true} with the dimensions of {\sf w}), 
    {\sf GrB\_NULL} should be specified.

    \item[{\sf accum}] ({\sf IN}) An optional binary operator used for accumulating
    entries into existing {\sf w} entries. If assignment rather than accumulation is
    desired, {\sf GrB\_NULL} should be specified.

    \item[{\sf u}]    ({\sf IN}) The GraphBLAS vector whose contents are 
    assigned to a subset of {\sf w}.

    \item[{\sf indices}]  ({\sf IN}) Pointer to the ordered set (array) of 
    indices corresponding to the locations in {\sf w} that are to be assigned.  
    If all elements of {\sf w} are to be assigned in order from $0$ to 
    ${\sf nindices} - 1$, then {\sf GrB\_ALL} should be specified.  Regardless of 
    execution mode and return value, this array may be manipulated by the caller
    after this operation returns without affecting any deferred computations for 
    this operation.  
    If this array contains duplicate values, it implies in assignment of more 
    than one value to the same location which leads to undefined results.
    
    \item[{\sf nindices}] ({\sf IN}) The number of values in {\sf indices} array.
    Must be equal to $\bold{size}({\sf u})$.

    \item[{\sf desc}] ({\sf IN}) An optional operation descriptor. If
    a \emph{default} descriptor is desired, {\sf GrB\_NULL} should be
    specified. Non-default field/value pairs are listed as follows:  \\

    \hspace*{-2em}\begin{tabular}{lllp{2.7in}}
        Param & Field  & Value & Description \\
        \hline
        {\sf w}    & {\sf GrB\_OUTP} & {\sf GrB\_REPLACE} & Output vector {\sf w}
        is cleared (all elements removed) before the result is stored in it.\\

        {\sf mask} & {\sf GrB\_MASK} & {\sf GrB\_STRUCTURE}   & The write mask is
        constructed from the structure (pattern of stored values) of the input
        {\sf mask} vector. The stored values are not examined.\\

        {\sf mask} & {\sf GrB\_MASK} & {\sf GrB\_COMP}   & Use the 
        complement of {\sf mask}. \\
    \end{tabular}
\end{itemize}

\paragraph{Return Values}

\begin{itemize}[leftmargin=2.3in]
    \item[{\sf GrB\_SUCCESS}]         In blocking mode, the operation completed
    successfully. In non-blocking mode, this indicates that the compatibility 
    tests on dimensions and domains for the input arguments passed successfully. 
    Either way, output vector {\sf w} is ready to be used in the next method of
    the sequence.

    \item[{\sf GrB\_PANIC}]           Unknown internal error.

    \item[{\sf GrB\_INVALID\_OBJECT}] This is returned in any execution mode 
    whenever one of the opaque GraphBLAS objects (input or output) is in an invalid 
    state caused by a previous execution error.  Call {\sf GrB\_error()} to access 
    any error messages generated by the implementation.

    \item[{\sf GrB\_OUT\_OF\_MEMORY}] Not enough memory available for operation.

    \item[{\sf GrB\_UNINITIALIZED\_OBJECT}] One or more of the GraphBLAS objects
    has not been initialized by a call to {\sf new} (or {\sf dup} for vector
    parameters).

    \item[{\sf GrB\_INDEX\_OUT\_OF\_BOUNDS}]  A value in {\sf indices} is greater
    than or equal to $\bold{size}({\sf w})$.  In non-blocking mode, this can be
    reported as an execution error.

    \item[{\sf GrB\_DIMENSION\_MISMATCH}] {\sf mask} and {\sf w} dimensions are
    incompatible, or ${\sf nindices} \neq \bold{size}({\sf u})$. 

    \item[{\sf GrB\_DOMAIN\_MISMATCH}]    The domains of the various vectors are
    incompatible with each other or the corresponding domains of the
    accumulation operator, or the mask's domain is not compatible with {\sf bool}
    (in the case where {\sf desc[GrB\_MASK].GrB\_STRUCTURE} is not set).

    \item[{\sf GrB\_NULL\_POINTER}] Argument {\sf indices} is a {\sf NULL} pointer.
\end{itemize}

\paragraph{Description}

This variant of {\sf GrB\_assign} computes the result of assigning elements from
a source GraphBLAS vector to a destination GraphBLAS vector in a specific order:
${\sf w}({\sf indices}) = {\sf u}$; or, if an optional binary accumulation 
operator ($\odot$) is provided, 
${\sf w}({\sf indices}) = {\sf w}({\sf indices}) \odot {\sf u}$. 
More explicitly:
\[
\begin{aligned}
    {\sf w}({\sf indices}[i]) = &\ \ \ \ \ \ \ \ \ \ \ \ \ \ \ \ \ \ \ \ {\sf u}(i),
    \ \forall \ i \ : \ 0 \leq i < {\sf nindices}, \mbox{~~or~~}
    \\
    {\sf w}({\sf indices}[i]) = &\ {\sf w}({\sf indices}[i]) \odot {\sf u}(i),
    \ \forall \ i \ : \ 0 \leq i < {\sf nindices}.
\end{aligned}
\]
Logically, this operation occurs in three steps:
\begin{enumerate}[leftmargin=0.75in]
\item[\bf Setup] The internal vectors and mask used in the computation are formed 
and their domains and dimensions are tested for compatibility.
\item[\bf Compute] The indicated computations are carried out.
\item[\bf Output] The result is written into the output vector, possibly under 
control of a mask.
\end{enumerate}

Up to three argument vectors are used in the {\sf GrB\_assign} operation:
\begin{enumerate}
	\item ${\sf w} = \langle \bold{D}({\sf w}),\bold{size}({\sf w}),
    \bold{L}({\sf w}) = \{(i,w_i) \} \rangle$

	\item ${\sf mask} = \langle \bold{D}({\sf mask}),\bold{size}({\sf mask}),
    \bold{L}({\sf mask}) = \{(i,m_i) \} \rangle$ (optional)

	\item ${\sf u} = \langle \bold{D}({\sf u}),\bold{size}({\sf u}),
    \bold{L}({\sf u}) = \{(i,u_i) \} \rangle$
\end{enumerate}

The argument vectors and the accumulation 
operator (if provided) are tested for domain compatibility as follows:
\begin{enumerate}
	\item If {\sf mask} is not {\sf GrB\_NULL}, and ${\sf desc[GrB\_MASK].GrB\_STRUCTURE}$
    is not set, then $\bold{D}({\sf mask})$ must be from one of the pre-defined types of 
    Table~\ref{Tab:PredefinedTypes}.

	\item $\bold{D}({\sf w})$ must be
    compatible with $\bold{D}({\sf u})$.

	\item If {\sf accum} is not {\sf GrB\_NULL}, then $\bold{D}({\sf w})$ must be
    compatible with $\bDin1({\sf accum})$ and $\bDout({\sf accum})$ of the accumulation operator and 
    $\bold{D}({\sf u})$ must be compatible with $\bDin2({\sf accum})$ of the accumulation operator.
\end{enumerate}
Two domains are compatible with each other if values from one domain can be cast 
to values in the other domain as per the rules of the C language.
In particular, domains from Table~\ref{Tab:PredefinedTypes} are all compatible 
with each other. A domain from a user-defined type is only compatible with itself.
If any compatibility rule above is violated, execution of {\sf GrB\_assign} ends
and the domain mismatch error listed above is returned.

From the arguments, the internal vectors, mask and index array used in 
the computation are formed ($\leftarrow$ denotes copy):
\begin{enumerate}
	\item Vector $\vector{\widetilde{w}} \leftarrow {\sf w}$.

	\item One-dimensional mask, $\vector{\widetilde{m}}$, is computed from 
    argument {\sf mask} as follows:
	\begin{enumerate}
		\item If ${\sf mask} = {\sf GrB\_NULL}$, then $\vector{\widetilde{m}} = 
        \langle \bold{size}({\sf w}), \{i, \ \forall \ i : 0 \leq i < 
        \bold{size}({\sf w}) \} \rangle$.

		\item If {\sf mask} $\ne$ {\sf GrB\_NULL},  
        \begin{enumerate}
            \item If ${\sf desc[GrB\_MASK].GrB\_STRUCTURE}$ is set, then
            $\vector{\widetilde{m}} = 
            \langle \bold{size}({\sf mask}), \{i : i \in \bold{ind}({\sf mask}) \} \rangle$,
            \item Otherwise, $\vector{\widetilde{m}} = 
            \langle \bold{size}({\sf mask}), \{i : i \in \bold{ind}({\sf mask}) \wedge
            ({\sf bool}){\sf mask}(i) = \true \} \rangle$.
        \end{enumerate}

		\item	If ${\sf desc[GrB\_MASK].GrB\_COMP}$ is set, then 
        $\vector{\widetilde{m}} \leftarrow \neg \vector{\widetilde{m}}$.
	\end{enumerate}

	\item Vector $\vector{\widetilde{u}} \leftarrow {\sf u}$.
    
    \item The internal index array, $\grbarray{\widetilde{I}}$, is computed from 
    argument {\sf indices} as follows:
	\begin{enumerate}
		\item	If ${\sf indices} = {\sf GrB\_ALL}$, then 
        $\grbarray{\widetilde{I}}[i] = i, \ \forall \ i : 0 \leq i < {\sf nindices}$.

		\item	Otherwise, $\grbarray{\widetilde{I}}[i] = {\sf indices}[i], 
        \ \forall \ i : 0 \leq i < {\sf nindices}$.
    \end{enumerate}
\end{enumerate}

The internal vector and mask are checked for dimension compatibility. 
The following conditions must hold:
\begin{enumerate}
    \item $\bold{size}(\vector{\widetilde{w}}) = \bold{size}(\vector{\widetilde{m}})$
    \item ${\sf nindices} = \bold{size}(\vector{\widetilde{u}})$.
\end{enumerate}
If any compatibility rule above is violated, execution of {\sf GrB\_assign} ends and 
the dimension mismatch error listed above is returned.

From this point forward, in {\sf GrB\_NONBLOCKING} mode, the method can 
optionally exit with {\sf GrB\_SUCCESS} return code and defer any computation 
and/or execution error codes.

We are now ready to carry out the assign and any additional 
associated operations.  We describe this in terms of two intermediate vectors:
\begin{itemize}
    \item $\vector{\widetilde{t}}$: The vector holding the elements from
    $\vector{\widetilde{u}}$ in their destination locations relative to 
    $\vector{\widetilde{w}}$.

    \item $\vector{\widetilde{z}}$: The vector holding the result after 
    application of the (optional) accumulation operator.
\end{itemize}

The intermediate vector, $\vector{\widetilde{t}}$, is created as follows:
\[
\vector{\widetilde{t}} = \langle
\bold{D}({\sf u}), \bold{size}(\vector{\widetilde{w}}),
\{(\grbarray{\widetilde{I}}[i],\vector{\widetilde{u}}(i)) \forall i, 0 \leq i < {\sf nindices} : 
i \in \bold{ind}(\vector{\widetilde{u}}) \} \rangle. 
\]
At this point, if any value of $\grbarray{\widetilde{I}}[i]$ is outside the valid
range of indices for vector $\vector{\widetilde{w}}$, computation ends and the 
method returns the index-out-of-bounds error listed above. In 
{\sf GrB\_NONBLOCKING} mode, the error can be deferred until a 
sequence-terminating {\sf GrB\_wait()} is called.  Regardless, the result 
vector, {\sf w}, is invalid from this point forward in the 
sequence.

The intermediate vector $\vector{\widetilde{z}}$ is created as follows:
\begin{itemize}
    \item If ${\sf accum} = {\sf GrB\_NULL}$, then $\vector{\widetilde{z}}$ is defined as 
    \[ 
        \vector{\widetilde{z}} =
		\langle \bold{D}({\sf w}), \bold{size}(\vector{\widetilde{w}}), 
		\{(i,z_{i}), \forall i \in (\bold{ind}(\vector{\widetilde{w}}) - (\{
            \grbarray{\widetilde{I}}[k],
            \forall k\} \cap \bold{ind}(\vector{\widetilde{w}}))) \cup
        \bold{ind}(\vector{\widetilde{t}}) \} \rangle.
    \]
    The above expression defines the structure of vector $\vector{\widetilde{z}}$ as follows:
    We start with the structure of $\vector{\widetilde{w}}$ ($\bold{ind}(\vector{\widetilde{w}})$) and remove from 
    it all the indices of $\vector{\widetilde{w}}$ that are
    in the set of indices being assigned ($\{\grbarray{\widetilde{I}}[k],\forall k\} \cap \bold{ind}(\vector{\widetilde{w}})$). Finally, we
    add the structure of $\vector{\widetilde{t}}$ ($\bold{ind}(\vector{\widetilde{t}})$).

    The values of the elements of $\vector{\widetilde{z}}$ are computed based on the 
    relationships between the sets of indices in $\vector{\widetilde{w}}$ 
    and $\vector{\widetilde{t}}$.
    \[
        z_{i} = \vector{\widetilde{w}}(i), \ \mbox{if}\  i \in  
        (\bold{ind}(\vector{\widetilde{w}}) - (\{\grbarray{\widetilde{I}}[k],\forall k\}
        \cap \bold{ind}(\vector{\widetilde{w}}))),
    \]
    \[
        z_{i} = \vector{\widetilde{t}}(i), \ \mbox{if}\  i \in  
        \bold{ind}(\vector{\widetilde{t}}),
    \]
    where the difference operator refers to set difference.

    \item If ${\sf accum}$ is a binary operator, then $\vector{\widetilde{z}}$ is defined as
        \[ \langle \bDout({\sf accum}), \bold{size}(\vector{\widetilde{w}}),
        \{(i,z_{i}) \ \forall \ i \in \bold{ind}(\vector{\widetilde{w}}) \cup 
        \bold{ind}(\vector{\widetilde{t}}) \} \rangle.\]

    The values of the elements of $\vector{\widetilde{z}}$ are computed based on the 
    relationships between the sets of indices in $\vector{\widetilde{w}}$ and 
    $\vector{\widetilde{t}}$.
\[
    z_{i} = \vector{\widetilde{w}}(i) \odot \vector{\widetilde{t}}(i), \ \mbox{if}\  
    i \in  (\bold{ind}(\vector{\widetilde{t}}) \cap \bold{ind}(\vector{\widetilde{w}})),
\]
\[
    z_{i} = \vector{\widetilde{w}}(i), \ \mbox{if}\  
    i \in (\bold{ind}(\vector{\widetilde{w}}) - (\bold{ind}(\vector{\widetilde{t}})
    \cap \bold{ind}(\vector{\widetilde{w}}))),
\]
\[
    z_{i} = \vector{\widetilde{t}}(i), \ \mbox{if}\  i \in  
    (\bold{ind}(\vector{\widetilde{t}}) - (\bold{ind}(\vector{\widetilde{t}})
    \cap \bold{ind}(\vector{\widetilde{w}}))),
\]
where $\odot  = \bigodot({\sf accum})$, and the difference operator refers to set difference.
\end{itemize}

\input{ops_mask_replace_vector}


%-----------------------------------------------------------------------------

\subsubsection{{\sf assign}: Standard matrix variant}

Assign values from one GraphBLAS matrix to a subset of a 
matrix as specified by a set of indices. The dimensions of the input matrix are
the same size as the row and column index arrays provided.

\paragraph{\syntax}

\begin{verbatim}
        GrB_Info GrB_assign(GrB_Matrix            C,
                            const GrB_Matrix      Mask,
                            const GrB_BinaryOp    accum,
                            const GrB_Matrix      A,
                            const GrB_Index      *row_indices,
                            GrB_Index             nrows,
                            const GrB_Index      *col_indices,
                            GrB_Index             ncols,
                            const GrB_Descriptor  desc);
\end{verbatim}

\paragraph{Parameters}

\begin{itemize}[leftmargin=1.1in]
    \item[{\sf C}]    ({\sf INOUT}) An existing GraphBLAS matrix. On input,
    the matrix provides values that may be accumulated with the result of the
    assign operation.  On output, the matrix holds the results of the
    operation.

    \item[{\sf Mask}] ({\sf IN}) An optional ``write'' mask that controls which
    results from this operation are stored into the output matrix {\sf C}. The 
    mask dimensions must match those of the matrix {\sf C}. If the 
    {\sf GrB\_STRUCTURE} descriptor is {\em not} set for the mask, the domain of the 
    {\sf Mask} matrix must be of type {\sf bool} or any of the predefined 
    ``built-in'' types in Table~\ref{Tab:PredefinedTypes}.  If the default
    mask is desired (\ie, a mask that is all {\sf true} with the dimensions of {\sf C}), 
    {\sf GrB\_NULL} should be specified.

    \item[{\sf accum}] ({\sf IN}) An optional binary operator used for accumulating
    entries into existing {\sf C} entries.
    If assignment rather than accumulation is
    desired, {\sf GrB\_NULL} should be specified.

    \item[{\sf A}]     ({\sf IN}) The GraphBLAS matrix whose contents are 
    assigned to a subset of {\sf C}.
    
    \item[{\sf row\_indices}] ({\sf IN}) Pointer to the ordered set (array) of 
    indices corresponding to the rows of {\sf C} that are assigned.  If all rows
    of {\sf C} are to be assigned in order from $0$ to ${\sf nrows} - 1$, then 
    {\sf GrB\_ALL} can be specified.  Regardless of execution mode and return 
    value, this array may be manipulated by the caller after this operation 
    returns without affecting any deferred computations for this operation.
    If this array contains duplicate values, it implies assignment of more 
    than one value to the same location which leads to undefined results.

    \item[{\sf nrows}] ({\sf IN}) The number of values in the {\sf row\_indices}
    array. Must be equal to $\bold{nrows}({\sf A})$ if {\sf A} is not tranposed,
    or equal to $\bold{ncols}({\sf A})$ if {\sf A} is transposed.

    \item[{\sf col\_indices}] ({\sf IN}) Pointer to the ordered set (array) of 
    indices corresponding to the columns of {\sf C} that are assigned.  If all 
    columns of {\sf C} are to be assigned in order from $0$ to ${\sf ncols} - 1$, 
    then {\sf GrB\_ALL} should be specified.  Regardless of execution mode and return
    value, this array may be manipulated by the caller after this operation 
    returns without affecting any deferred computations for this operation.
    If this array contains duplicate values, it implies assignment of more 
    than one value to the same location which leads to undefined results.
    
    \item[{\sf ncols}] ({\sf IN}) The number of values in {\sf col\_indices}
    array. Must be equal to $\bold{ncols}({\sf A})$ if {\sf A} is not tranposed,
	or equal to $\bold{nrows}({\sf A})$ if {\sf A} is transposed.

    \item[{\sf desc}] ({\sf IN}) An optional operation descriptor. If
    a \emph{default} descriptor is desired, {\sf GrB\_NULL} should be
    specified. Non-default field/value pairs are listed as follows:  \\

    \hspace*{-2em}\begin{tabular}{lllp{2.7in}}
        Param & Field  & Value & Description \\
        \hline
        {\sf C}    & {\sf GrB\_OUTP} & {\sf GrB\_REPLACE} & Output matrix {\sf C}
        is cleared (all elements removed) before the result is stored in it.\\

        {\sf Mask} & {\sf GrB\_MASK} & {\sf GrB\_STRUCTURE}   & The write mask is
        constructed from the structure (pattern of stored values) of the input
        {\sf Mask} matrix. The stored values are not examined.\\

        {\sf Mask} & {\sf GrB\_MASK} & {\sf GrB\_COMP}   & Use the 
        complement of {\sf Mask}. \\

        {\sf A}    & {\sf GrB\_INP0} & {\sf GrB\_TRAN}   & Use transpose of {\sf A}
        for the operation. \\
    \end{tabular}
\end{itemize}

\paragraph{Return Values}

\begin{itemize}[leftmargin=2.3in]
    \item[{\sf GrB\_SUCCESS}]         In blocking mode, the operation completed
    successfully. In non-blocking mode, this indicates that the compatibility 
    tests on dimensions and domains for the input arguments passed successfully. 
    Either way, output matrix {\sf C} is ready to be used in the next method of
    the sequence.

    \item[{\sf GrB\_PANIC}]           Unknown internal error.

    \item[{\sf GrB\_INVALID\_OBJECT}] This is returned in any execution mode 
    whenever one of the opaque GraphBLAS objects (input or output) is in an invalid 
    state caused by a previous execution error.  Call {\sf GrB\_error()} to access 
    any error messages generated by the implementation.

    \item[{\sf GrB\_OUT\_OF\_MEMORY}] Not enough memory available for the operation.

    \item[{\sf GrB\_UNINITIALIZED\_OBJECT}] One or more of the GraphBLAS objects 
    has not been initialized by a call to {\sf new} (or {\sf Matrix\_dup} for matrix
    parameters).

    \item[{\sf GrB\_INDEX\_OUT\_OF\_BOUNDS}]  A value in {\sf row\_indices} 
    is greater than or equal to $\bold{nrows}({\sf C})$, or a value in 
    {\sf col\_indices} is greater than or equal to $\bold{ncols}({\sf C})$.  In 
    non-blocking mode, this can be reported as an execution error.

    \item[{\sf GrB\_DIMENSION\_MISMATCH}] {\sf Mask} and {\sf C}
    dimensions are incompatible, ${\sf nrows} \neq \bold{nrows}({\sf A})$, or 
    ${\sf ncols} \neq \bold{ncols}({\sf A})$.

    \item[{\sf GrB\_DOMAIN\_MISMATCH}]    The domains of the various matrices are
    incompatible with each other or the corresponding domains of the 
    accumulation operator, or the mask's domain is not compatible with {\sf bool}
    (in the case where {\sf desc[GrB\_MASK].GrB\_STRUCTURE} is not set).

    \item[{\sf GrB\_NULL\_POINTER}] Either argument {\sf row\_indices} is a {\sf NULL} pointer,
	argument {\sf col\_indices} is a {\sf NULL} pointer, or both.
\end{itemize}

\paragraph{Description}

This variant of {\sf GrB\_assign} computes the result of assigning the contents
of {\sf A} to a subset of rows and columns in {\sf C} in a specified order:
${\sf C}({\sf row\_indices, col\_indices}) = {\sf A}$; or, if an optional
binary accumulation operator ($\odot$) is provided, 
${\sf C}({\sf row\_indices},{\sf col\_indices}) = 
{\sf C}({\sf row\_indices},{\sf col\_indices}) \odot {\sf A}$.
More explicitly (not accounting for an optional transpose of {\sf A}):
\[
\begin{aligned}
  	{\sf C}({\sf row\_indices}[i],{\sf col\_indices}[j]) = &\ {\sf A}(i,j), 
 	\ \forall \ i,j \ : \ 0 \leq i < {\sf nrows},\ 0 \leq j < {\sf ncols} \mbox{,~or~} \\
  	{\sf C}({\sf row\_indices}[i],{\sf col\_indices}[j]) = &\ 
    {\sf C}({\sf row\_indices}[i],{\sf col\_indices}[j]) \odot {\sf A}(i,j), \\
 	& \ \forall \ (i,j) \ : \ 0 \leq i < {\sf nrows},\ 0 \leq j < {\sf ncols}
\end{aligned}
\]  
Logically, this operation occurs in three steps:
\begin{enumerate}[leftmargin=0.85in]
\item[Setup] The internal matrices and mask used in the computation are formed 
and their domains and dimensions are tested for compatibility.
\item[Compute] The indicated computations are carried out.
\item[Output] The result is written into the output matrix, possibly under 
control of a mask.
\end{enumerate}

Up to three argument matrices are used in the {\sf GrB\_assign} operation:
\begin{enumerate}
	\item ${\sf C} = \langle \bold{D}({\sf C}),\bold{nrows}({\sf C}),
    \bold{ncols}({\sf C}),\bold{L}({\sf C}) = \{(i,j,C_{ij}) \} \rangle$

	\item ${\sf Mask} = \langle \bold{D}({\sf Mask}),\bold{nrows}({\sf Mask}),
    \bold{ncols}({\sf Mask}),\bold{L}({\sf Mask}) = \{(i,j,M_{ij}) \} \rangle$ (optional)

	\item ${\sf A} = \langle \bold{D}({\sf A}),\bold{nrows}({\sf A}),
    \bold{ncols}({\sf A}),\bold{L}({\sf A}) = \{(i,j,A_{ij}) \} \rangle$
\end{enumerate}

The argument matrices and the accumulation 
operator (if provided) are tested for domain compatibility as follows:
\begin{enumerate}
	\item If {\sf Mask} is not {\sf GrB\_NULL}, and ${\sf desc[GrB\_MASK].GrB\_STRUCTURE}$
    is not set, then $\bold{D}({\sf Mask})$ must be from one of the pre-defined types of 
    Table~\ref{Tab:PredefinedTypes}.

	\item $\bold{D}({\sf C})$ must be
    compatible with $\bold{D}({\sf A})$.

	\item If {\sf accum} is not {\sf GrB\_NULL}, then $\bold{D}({\sf C})$ must be
    compatible with $\bDin1({\sf accum})$ and $\bDout({\sf accum})$ of the accumulation operator and 
    $\bold{D}({\sf A})$ must be compatible with $\bDin2({\sf accum})$ of the accumulation operator.
\end{enumerate}
Two domains are compatible with each other if values from one domain can be cast 
to values in the other domain as per the rules of the C language.
In particular, domains from Table~\ref{Tab:PredefinedTypes} are all compatible 
with each other. A domain from a user-defined type is only compatible with itself.
If any compatibility rule above is violated, execution of {\sf GrB\_assign} ends
and the domain mismatch error listed above is returned.

From the arguments, the internal matrices, mask, and index arrays used in 
the computation are formed ($\leftarrow$ denotes copy):
\begin{enumerate}
	\item Matrix $\matrix{\widetilde{C}} \leftarrow {\sf C}$.

	\item Two-dimensional mask $\matrix{\widetilde{M}}$ is computed from
    argument {\sf Mask} as follows:
	\begin{enumerate}
		\item If ${\sf Mask} = {\sf GrB\_NULL}$, then $\matrix{\widetilde{M}} = 
        \langle \bold{nrows}({\sf C}), \bold{ncols}({\sf C}), \{(i,j), 
        \forall i,j : 0 \leq i <  \bold{nrows}({\sf C}), 0 \leq j < 
        \bold{ncols}({\sf C}) \} \rangle$.

		\item If {\sf Mask} $\ne$ {\sf GrB\_NULL},
        \begin{enumerate}
            \item If ${\sf desc[GrB\_MASK].GrB\_STRUCTURE}$ is set, then 
            $\matrix{\widetilde{M}} = \langle \bold{nrows}({\sf Mask}), 
            \bold{ncols}({\sf Mask}), \{(i,j) : (i,j) \in \bold{ind}({\sf Mask}) \} \rangle$,
            \item Otherwise, $\matrix{\widetilde{M}} = \langle \bold{nrows}({\sf Mask}), 
            \bold{ncols}({\sf Mask}), \\ \{(i,j) : (i,j) \in \bold{ind}({\sf Mask}) \wedge 
            ({\sf bool}){\sf Mask}(i,j) = \true\} \rangle$.
        \end{enumerate}

		\item	If ${\sf desc[GrB\_MASK].GrB\_COMP}$ is set, then 
        $\matrix{\widetilde{M}} \leftarrow \neg \matrix{\widetilde{M}}$.
	\end{enumerate}

	\item Matrix $\matrix{\widetilde{A}} \leftarrow
    {\sf desc[GrB\_INP0].GrB\_TRAN} \ ? \ {\sf A}^T : {\sf A}$.

    \item The internal row index array, $\grbarray{\widetilde{I}}$, is computed from 
    argument {\sf row\_indices} as follows:
	\begin{enumerate}
		\item	If ${\sf row\_indices} = {\sf GrB\_ALL}$, then 
        $\grbarray{\widetilde{I}}[i] = i, \forall i : 0 \leq i < {\sf nrows}$.

		\item	Otherwise, $\grbarray{\widetilde{I}}[i] = {\sf row\_indices}[i], 
        \forall i : 0 \leq i < {\sf nrows}$.
    \end{enumerate}
    
    \item The internal column index array, $\grbarray{\widetilde{J}}$, is computed from 
    argument {\sf col\_indices} as follows:
	\begin{enumerate}
		\item	If ${\sf col\_indices} = {\sf GrB\_ALL}$, then 
        $\grbarray{\widetilde{J}}[j] = j, \forall j : 0 \leq j < {\sf ncols}$.

		\item	Otherwise, $\grbarray{\widetilde{J}}[j] = {\sf col\_indices}[j], 
        \ \forall \ j : 0 \leq j < {\sf ncols}$.
    \end{enumerate}
\end{enumerate}

The internal matrices and mask are checked for dimension compatibility. The following
conditions must hold:
\begin{enumerate}
    \item $\bold{nrows}(\matrix{\widetilde{C}}) = \bold{nrows}(\matrix{\widetilde{M}})$.

    \item $\bold{ncols}(\matrix{\widetilde{C}}) = \bold{ncols}(\matrix{\widetilde{M}})$.

    \item $\bold{nrows}(\matrix{\widetilde{A}}) = {\sf nrows}$.

    \item $\bold{ncols}(\matrix{\widetilde{A}}) = {\sf ncols}$.
\end{enumerate}
If any compatibility rule above is violated, execution of {\sf GrB\_assign} ends and
the dimension mismatch error listed above is returned.

From this point forward, in {\sf GrB\_NONBLOCKING} mode, the method can 
optionally exit with {\sf GrB\_SUCCESS} return code and defer any computation 
and/or execution error codes.

We are now ready to carry out the assign and any additional 
associated operations.  We describe this in terms of two intermediate vectors:
\begin{itemize}
    \item $\matrix{\widetilde{T}}$: The matrix holding the contents from
    $\matrix{\widetilde{A}}$ in their destination locations relative to
    $\matrix{\widetilde{C}}$.

    \item $\matrix{\widetilde{Z}}$: The matrix holding the result after 
    application of the (optional) accumulation operator.
\end{itemize}

The intermediate matrix, $\matrix{\widetilde{T}}$, is created as follows:
\[
\begin{aligned}
\matrix{\widetilde{T}} = \langle & \bold{D}({\sf A}),
                           \bold{nrows}(\matrix{\widetilde{C}}), 
                           \bold{ncols}(\matrix{\widetilde{C}}), \\
 & \{ (\grbarray{\widetilde{I}}[i],\grbarray{\widetilde{J}}[j],\matrix{\widetilde{A}}(i,j)) 
\ \forall \ (i,j), \ 0 \leq i < {\sf nrows}, \ 0 \leq j < {\sf ncols} :
(i,j) \in \bold{ind}(\matrix{\widetilde{A}}) \} \rangle. 
\end{aligned}
\]
At this point, if any value in the $\grbarray{\widetilde{I}}$ array is not in
the range $[0,\ \bold{nrows}(\matrix{\widetilde{C}}) )$ or any value in the 
$\grbarray{\widetilde{J}}$ array is not in the range 
$[0,\ \bold{ncols}(\matrix{\widetilde{C}}))$, the execution of {\sf GrB\_assign} 
ends and the index out-of-bounds error listed above is generated.  In 
{\sf GrB\_NONBLOCKING} mode, the error can be deferred until a 
sequence-terminating {\sf GrB\_wait()} is called.  Regardless, the result 
matrix {\sf C} is invalid from this point forward in the sequence.

The intermediate matrix $\matrix{\widetilde{Z}}$ is created as follows:
\begin{itemize}
    \item If ${\sf accum} = {\sf GrB\_NULL}$, then $\matrix{\widetilde{Z}}$ is defined as 
    \begin{eqnarray}
        \matrix{\widetilde{Z}} & = &
		\langle \bold{D}({\sf C}),\bold{nrows}(\matrix{\widetilde{C}}),
        \bold{ncols}(\matrix{\widetilde{C}}), \nonumber \\
    & & \{(i,j,Z_{ij})  \forall (i,j) \in
        (\bold{ind}(\matrix{\widetilde{C}}) - (\{
            (\grbarray{\widetilde{I}}[k],\grbarray{\widetilde{J}}[l]),
            \forall k,l\} \cap \bold{ind}(\matrix{\widetilde{C}}))) \cup
        \bold{ind}(\matrix{\widetilde{T}}) \} \rangle. \nonumber
    \end{eqnarray}
    The above expression defines the structure of matrix $\matrix{\widetilde{Z}}$ as follows:
    We start with the structure of $\matrix{\widetilde{C}}$ ($\bold{ind}(\matrix{\widetilde{C}})$) and remove from 
    it all the indices of $\matrix{\widetilde{C}}$ that are
    in the set of indices being assigned ($\{(\grbarray{\widetilde{I}}[k],\grbarray{\widetilde{J}}[l]),\forall k,l\} \cap \bold{ind}(\matrix{\widetilde{C}})$). Finally, we
    add the structure of $\matrix{\widetilde{T}}$ ($\bold{ind}(\matrix{\widetilde{T}})$).

    The values of the elements of $\matrix{\widetilde{Z}}$ are computed based on the 
    relationships between the sets of indices in $\matrix{\widetilde{C}}$ and 
    $\matrix{\widetilde{T}}$.
    \[
        Z_{ij} = \matrix{\widetilde{C}}(i,j), \ \mbox{if}\  (i,j) \in  
        (\bold{ind}(\matrix{\widetilde{C}}) - (\{ (\grbarray{\widetilde{I}}[k],\grbarray{\widetilde{J}}[l]), \forall k, l\}
        \cap \bold{ind}(\matrix{\widetilde{C}}))),
    \]
    \[
        Z_{ij} = \matrix{\widetilde{T}}(i,j), \ \mbox{if}\ (i,j) \in  
        \bold{ind}(\matrix{\widetilde{T}}),
    \]
    where the difference operator refers to set difference.

    \item If ${\sf accum}$ is a binary operator, then $\matrix{\widetilde{Z}}$ is defined as
        \[ \langle \bDout({\sf accum}), \bold{nrows}(\matrix{\widetilde{C}}), \bold{ncols}(\matrix{\widetilde{C}}),
        \{(i,j,Z_{ij})  \forall (i,j) \in \bold{ind}(\matrix{\widetilde{C}}) \cup 
        \bold{ind}(\matrix{\widetilde{T}}) \} \rangle.\]

    The values of the elements of $\matrix{\widetilde{Z}}$ are computed based on the
    relationships between the sets of indices in $\matrix{\widetilde{C}}$ and 
    $\matrix{\widetilde{T}}$.
\[
    Z_{ij} = \matrix{\widetilde{C}}(i,j) \odot \matrix{\widetilde{T}}(i,j), \ \mbox{if}\  
    (i,j) \in  (\bold{ind}(\matrix{\widetilde{T}}) \cap \bold{ind}(\matrix{\widetilde{C}})),
\]
\[
    Z_{ij} = \matrix{\widetilde{C}}(i,j), \ \mbox{if}\  
    (i,j) \in (\bold{ind}(\matrix{\widetilde{C}}) - (\bold{ind}(\matrix{\widetilde{T}})
    \cap \bold{ind}(\matrix{\widetilde{C}}))),
\]
\[
    Z_{ij} = \matrix{\widetilde{T}}(i,j), \ \mbox{if}\  (i,j) \in  
    (\bold{ind}(\matrix{\widetilde{T}}) - (\bold{ind}(\matrix{\widetilde{T}})
    \cap \bold{ind}(\matrix{\widetilde{C}}))),
\]
where $\odot  = \bigodot({\sf accum})$, and the difference operator refers to set difference.
\end{itemize}

\input{ops_mask_replace_matrix}


%-----------------------------------------------------------------------------

\subsubsection{{\sf assign}: Column variant}

Assign the contents a vector to a subset of elements in one column of a matrix. 
Note that since the output cannot be transposed, a different variant of
{\sf assign} is provided to assign to a row of a matrix.

\paragraph{\syntax}

\begin{verbatim}
        GrB_Info GrB_assign(GrB_Matrix            C,
                            const GrB_Vector      mask,
                            const GrB_BinaryOp    accum,
                            const GrB_Vector      u,
                            const GrB_Index      *row_indices,
                            GrB_Index             nrows,
                            GrB_Index             col_index,
                            const GrB_Descriptor  desc);
\end{verbatim}

\paragraph{Parameters}

\begin{itemize}[leftmargin=1.1in]
    \item[{\sf C}]    ({\sf INOUT}) An existing GraphBLAS matrix.  On input,
    the matrix provides values that may be accumulated with the result of the
    assign operation.  On output, this matrix holds the results of the
    operation.

    \item[{\sf mask}] ({\sf IN}) An optional ``write'' mask that controls which
    results from this operation are stored into the specified column of the output matrix {\sf C}. The 
    mask dimensions must match those of a single column of the matrix {\sf C}. If the 
    {\sf GrB\_STRUCTURE} descriptor is {\em not} set for the mask, the domain of the
    {\sf Mask} matrix must be of type {\sf bool} or any of the predefined 
    ``built-in'' types in Table~\ref{Tab:PredefinedTypes}.  If the default
    mask is desired (\ie, a mask that is all {\sf true} with the dimensions of a column of {\sf C}), 
    {\sf GrB\_NULL} should be specified.

    \item[{\sf accum}] ({\sf IN}) An optional binary operator used for accumulating
    entries into existing {\sf C} entries. If assignment rather than accumulation is
    desired, {\sf GrB\_NULL} should be specified.

    \item[{\sf u}]    ({\sf IN}) The GraphBLAS vector whose contents are 
    assigned to (a subset of) a column of {\sf C}.

    \item[{\sf row\_indices}]  ({\sf IN}) Pointer to the ordered set (array) of 
    indices corresponding to the locations in the specified column of {\sf C} 
    that are to be assigned.  If all elements of the column in {\sf C} are to be 
    assigned in order from index $0$ to ${\sf nrows} - 1$, then {\sf GrB\_ALL} should be 
    specified.  Regardless of execution mode and return value, this array may be
    manipulated by the caller after this operation returns without affecting any 
    deferred computations for this operation.
    If this array contains duplicate values, it implies in assignment of more 
    than one value to the same location which leads to undefined results.
    
    \item[{\sf nrows}] ({\sf IN}) The number of values in {\sf row\_indices} array.
    Must be equal to $\bold{size}({\sf u})$.
    
    \item[{\sf col\_index}]  ({\sf IN}) The index of the column in {\sf C} to 
    assign. Must be in the range $[0, \bold{ncols}({\sf C}))$.

    \item[{\sf desc}] ({\sf IN}) An optional operation descriptor. If
    a \emph{default} descriptor is desired, {\sf GrB\_NULL} should be
    specified. Non-default field/value pairs are listed as follows:  \\

    \hspace*{-2em}\begin{tabular}{lllp{2.7in}}
        Param & Field  & Value & Description \\
        \hline
        {\sf C}    & {\sf GrB\_OUTP} & {\sf GrB\_REPLACE} &  Output column in 
        {\sf C} is cleared (all elements removed) before result is stored in it.\\

        {\sf mask} & {\sf GrB\_MASK} & {\sf GrB\_STRUCTURE}   & The write mask is
        constructed from the structure (pattern of stored values) of the input
        {\sf mask} vector. The stored values are not examined.\\

        {\sf mask} & {\sf GrB\_MASK} & {\sf GrB\_COMP}   & Use the 
        complement of {\sf mask}. \\
    \end{tabular}
\end{itemize}

\paragraph{Return Values}

\begin{itemize}[leftmargin=2.3in]
    \item[{\sf GrB\_SUCCESS}]         In blocking mode, the operation completed
    successfully. In non-blocking mode, this indicates that the compatibility 
    tests on dimensions and domains for the input arguments passed successfully. 
    Either way, output matrix {\sf C} is ready to be used in the next method of 
    the sequence.

    \item[{\sf GrB\_PANIC}]           Unknown internal error.

    \item[{\sf GrB\_INVALID\_OBJECT}] This is returned in any execution mode 
    whenever one of the opaque GraphBLAS objects (input or output) is in an invalid 
    state caused by a previous execution error.  Call {\sf GrB\_error()} to access 
    any error messages generated by the implementation.

    \item[{\sf GrB\_OUT\_OF\_MEMORY}] Not enough memory available for operation.

    \item[{\sf GrB\_UNINITIALIZED\_OBJECT}] One or more of the GraphBLAS objects
    has not been initialized by a call to {\sf new} (or {\sf dup} for vector or
    matrix parameters).

    \item[{\sf GrB\_INVALID\_INDEX}]    {\sf col\_index} is outside the allowable 
    range (i.e., greater than $\bold{ncols}({\sf C})$).

    \item[{\sf GrB\_INDEX\_OUT\_OF\_BOUNDS}]  A value in {\sf row\_indices}
    is greater than or equal to $\bold{nrows}({\sf C})$.  In 
    non-blocking mode, this can be reported as an execution error.

    \item[{\sf GrB\_DIMENSION\_MISMATCH}] {\sf mask} size and number of rows
    in {\sf C} are not the same, or ${\sf nrows} \neq \bold{size}({\sf u})$.

    \item[{\sf GrB\_DOMAIN\_MISMATCH}]    The domains of the matrix and vector are
    incompatible with each other or the corresponding domains of the
    accumulation operator, or the mask's domain is not compatible with {\sf bool}
    (in the case where {\sf desc[GrB\_MASK].GrB\_STRUCTURE} is not set).

    \item[{\sf GrB\_NULL\_POINTER}] Argument {\sf row\_indices} is a {\sf NULL} pointer.
\end{itemize}

\paragraph{Description}

This variant of {\sf GrB\_assign} computes the result of assigning a subset of
locations in a column of a GraphBLAS matrix (in a specific order) from the 
contents of a GraphBLAS vector: \\
${\sf C}(:,{\sf col\_index}) = {\sf u}$; or, if an 
optional binary accumulation operator ($\odot$) is provided, 
${\sf C}(:,{\sf col\_index}) = 
{\sf C}(:,{\sf col\_index}) \odot {\sf u}$. Taking order of {\sf row\_indices}
into account, it is more explicitly written as:
\[
\begin{aligned}
    {\sf C}({\sf row\_indices}[i],{\sf col\_index}) = &\ {\sf u}(i),
    \ \forall \ i \ : \ 0 \leq i < {\sf nrows}, \mbox{~or~}
    \\
    {\sf C}({\sf row\_indices}[i],{\sf col\_index}) = &\ 
    {\sf C}({\sf row\_indices}[i],{\sf col\_index}) \odot {\sf u}(i),
    \ \forall \ i \ : \ 0 \leq i < {\sf nrows}.
\end{aligned}
\]
Logically, this operation occurs in three steps:
\begin{enumerate}[leftmargin=0.75in]
\item[\bf Setup] The internal matrices, vectors and mask used in the computation are formed 
and their domains and dimensions are tested for compatibility.
\item[\bf Compute] The indicated computations are carried out.
\item[\bf Output] The result is written into the output matrix, possibly under 
control of a mask.
\end{enumerate}

Up to three argument vectors and matrices are used in this {\sf GrB\_assign} 
operation:
\begin{enumerate}
	\item ${\sf C} = \langle \bold{D}({\sf C}),\bold{nrows}({\sf C}),
    \bold{ncols}({\sf C}),\bold{L}({\sf C}) = \{(i,j,C_{ij}) \} \rangle$

	\item ${\sf mask} = \langle \bold{D}({\sf mask}),\bold{size}({\sf mask}),
    \bold{L}({\sf mask}) = \{(i,m_i) \} \rangle$ (optional)

	\item ${\sf u} = \langle \bold{D}({\sf u}),\bold{size}({\sf u}),
    \bold{L}({\sf u}) = \{(i,u_i) \} \rangle$
\end{enumerate}

The argument vectors, matrix, and the accumulation 
operator (if provided) are tested for domain compatibility as follows:
\begin{enumerate}
	\item If {\sf mask} is not {\sf GrB\_NULL}, and ${\sf desc[GrB\_MASK].GrB\_STRUCTURE}$
    is not set, then $\bold{D}({\sf mask})$ must be from one of the pre-defined types of 
    Table~\ref{Tab:PredefinedTypes}.

	\item $\bold{D}({\sf C})$ must be 
    compatible with $\bold{D}({\sf u})$.

	\item If {\sf accum} is not {\sf GrB\_NULL}, then $\bold{D}({\sf C})$ must be
    compatible with $\bDin1({\sf accum})$ and $\bDout({\sf accum})$ of the accumulation operator and 
    $\bold{D}({\sf u})$ must be compatible with $\bDin2({\sf accum})$ of the accumulation operator.
\end{enumerate}
Two domains are compatible with each other if values from one domain can be cast 
to values in the other domain as per the rules of the C language.
In particular, domains from Table~\ref{Tab:PredefinedTypes} are all compatible 
with each other. A domain from a user-defined type is only compatible with itself.
If any compatibility rule above is violated, execution of {\sf GrB\_assign} ends
and the domain mismatch error listed above is returned.

The {\sf col\_index} parameter is checked for a valid value.  The following
condition must hold:
\begin{enumerate}
	\item $0\ \leq\ {\sf col\_index} \ <\ \bold{ncols}({\sf C})$
\end{enumerate}
If the rule above is violated, execution of {\sf GrB\_assign} ends 
and the invalid index error listed above is returned.

From the arguments, the internal vectors, mask, and index array used in 
the computation are formed ($\leftarrow$ denotes copy):
\begin{enumerate}
	\item The vector, $\vector{\widetilde{c}}$, is extracted from a column of {\sf C}
    as follows:
    \[
        \vector{\widetilde{c}} = \langle  \bold{D}({\sf C}), \bold{nrows}({\sf C}), 
        \{ (i, C_{ij}) \ \forall \ i : 0 \leq i < \bold{nrows}({\sf C}),
        j = {\sf col\_index}, (i, j) \in \bold{ind}({\sf C})\} \rangle
    \]

	\item One-dimensional mask, $\vector{\widetilde{m}}$, is computed from 
    argument {\sf mask} as follows:
	\begin{enumerate}
		\item If ${\sf mask} = {\sf GrB\_NULL}$, then $\vector{\widetilde{m}} = 
        \langle \bold{nrows}({\sf C}), \{i, \ \forall \ i : 0 \leq i < 
        \bold{nrows}({\sf C}) \} \rangle$.

		\item If {\sf mask} $\ne$ {\sf GrB\_NULL},  
        \begin{enumerate}
            \item If ${\sf desc[GrB\_MASK].GrB\_STRUCTURE}$ is set, then
            $\vector{\widetilde{m}} = 
            \langle \bold{size}({\sf mask}), \{i : i \in \bold{ind}({\sf mask}) \} \rangle$,
            \item Otherwise, $\vector{\widetilde{m}} = 
            \langle \bold{size}({\sf mask}), \{i : i \in \bold{ind}({\sf mask}) \wedge
            ({\sf bool}){\sf mask}(i) = \true \} \rangle$.
        \end{enumerate}

		\item	If ${\sf desc[GrB\_MASK].GrB\_COMP}$ is set, then 
        $\vector{\widetilde{m}} \leftarrow \neg \vector{\widetilde{m}}$.
	\end{enumerate}

	\item Vector $\vector{\widetilde{u}} \leftarrow {\sf u}$.
    
    \item The internal row index array, $\grbarray{\widetilde{I}}$, is computed from 
    argument {\sf row\_indices} as follows:
	\begin{enumerate}
		\item	If ${\sf row\_indices} = {\sf GrB\_ALL}$, then 
        $\grbarray{\widetilde{I}}[i] = i, \ \forall \ i : 0 \leq i < {\sf nrows}$.

		\item	Otherwise, $\grbarray{\widetilde{I}}[i] = {\sf row\_indices}[i], 
        \ \forall \ i : 0 \leq i < {\sf nrows}$.
    \end{enumerate}
\end{enumerate}

The internal vectors, matrices, and masks are checked for dimension compatibility. 
The following conditions must hold:
\begin{enumerate}
    \item $\bold{size}(\vector{\widetilde{c}}) = \bold{size}(\vector{\widetilde{m}})$
    \item ${\sf nrows} = \bold{size}(\vector{\widetilde{u}})$.
\end{enumerate}
If any compatibility rule above is violated, execution of {\sf GrB\_assign} ends and 
the dimension mismatch error listed above is returned.

From this point forward, in {\sf GrB\_NONBLOCKING} mode, the method can 
optionally exit with {\sf GrB\_SUCCESS} return code and defer any computation 
and/or execution error codes.

We are now ready to carry out the assign and any additional 
associated operations.  We describe this in terms of two intermediate vectors:
\begin{itemize}
    \item $\vector{\widetilde{t}}$: The vector holding the elements from
    $\vector{\widetilde{u}}$ in their destination locations relative to 
    $\vector{\widetilde{c}}$.

    \item $\vector{\widetilde{z}}$: The vector holding the result after 
    application of the (optional) accumulation operator.
\end{itemize}

The intermediate vector, $\vector{\widetilde{t}}$, is created as follows:
\[
\vector{\widetilde{t}} = \langle
\bold{D}({\sf u}), \bold{size}(\vector{\widetilde{c}}),
\{(\grbarray{\widetilde{I}}[i],\vector{\widetilde{u}}(i))\ \forall \ i, \ 
0 \leq i < {\sf nrows} : i \in \bold{ind}(\vector{\widetilde{u}}) \} \rangle. 
\]
At this point, if any value of $\grbarray{\widetilde{I}}[i]$ is outside the valid 
range of indices for vector $\vector{\widetilde{c}}$, computation ends and the 
method returns the index out-of-bounds error listed above. In 
{\sf GrB\_NONBLOCKING} mode, the error can be deferred until a 
sequence-terminating {\sf GrB\_wait()} is called.  Regardless, the result 
matrix, {\sf C}, is invalid from this point forward in the 
sequence.

The intermediate vector $\vector{\widetilde{z}}$ is created as follows:
\begin{itemize}
    \item If ${\sf accum} = {\sf GrB\_NULL}$, then $\vector{\widetilde{z}}$ is defined as 
    \[ 
        \vector{\widetilde{z}} =
		\langle \bold{D}({\sf C}), \bold{size}(\vector{\widetilde{c}}), 
		\{(i,z_{i}), \forall i \in (\bold{ind}(\vector{\widetilde{c}})-(\{\grbarray{\widetilde{I}}[k],\forall k\} \cap \bold{ind}(\vector{\widetilde{c}}))) \cup 
        \bold{ind}(\vector{\widetilde{t}}) \} \rangle.
    \]
    The above expression defines the structure of vector $\vector{\widetilde{z}}$ as follows:
    We start with the structure of $\vector{\widetilde{c}}$ ($\bold{ind}(\vector{\widetilde{c}})$) and remove from 
    it all the indices of $\vector{\widetilde{c}}$ that are
    in the set of indices being assigned ($\{\grbarray{\widetilde{I}}[k],\forall k\} \cap \bold{ind}(\vector{\widetilde{c}})$). Finally, we
    add the structure of $\vector{\widetilde{t}}$ ($\bold{ind}(\vector{\widetilde{t}})$).

    The values of the elements of $\vector{\widetilde{z}}$ are computed based on the 
    relationships between the sets of indices in $\vector{\widetilde{c}}$ 
    and $\vector{\widetilde{t}}$.
    \[
        z_{i} = \vector{\widetilde{c}}(i), \ \mbox{if}\  i \in  
        (\bold{ind}(\vector{\widetilde{c}}) - (\{\grbarray{\widetilde{I}}[k],\forall k\}
        \cap \bold{ind}(\vector{\widetilde{c}}))),
    \]
    \[
        z_{i} = \vector{\widetilde{t}}(i), \ \mbox{if}\  i \in  
        \bold{ind}(\vector{\widetilde{t}}),
    \]
    where the difference operator refers to set difference.

    \item If ${\sf accum}$ is a binary operator, then $\vector{\widetilde{z}}$ is defined as
        \[ \langle \bDout({\sf accum}), \bold{size}(\vector{\widetilde{c}}),
        \{(i,z_{i}) \ \forall \ i \in \bold{ind}(\vector{\widetilde{c}}) \cup 
        \bold{ind}(\vector{\widetilde{t}}) \} \rangle.\]

    The values of the elements of $\vector{\widetilde{z}}$ are computed based on the 
    relationships between the sets of indices in $\vector{\widetilde{w}}$ and 
    $\vector{\widetilde{t}}$.
\[
    z_{i} = \vector{\widetilde{c}}(i) \odot \vector{\widetilde{t}}(i), \ \mbox{if}\  
    i \in  (\bold{ind}(\vector{\widetilde{t}}) \cap \bold{ind}(\vector{\widetilde{c}})),
\]
\[
    z_{i} = \vector{\widetilde{c}}(i), \ \mbox{if}\ 
    i \in  (\bold{ind}(\vector{\widetilde{c}}) - (\bold{ind}(\vector{\widetilde{t}})
    \cap \bold{ind}(\vector{\widetilde{c}}))),
\]
\[
    z_{i} = \vector{\widetilde{t}}(i), \ \mbox{if}\  i \in  
    (\bold{ind}(\vector{\widetilde{t}}) - (\bold{ind}(\vector{\widetilde{t}})
    \cap \bold{ind}(\vector{\widetilde{c}}))),
\]
where $\odot  = \bigodot({\sf accum})$, and the difference operator refers to set difference.
\end{itemize}

Finally, the set of output values that make up the $\vector{\widetilde{z}}$ 
vector are written into the column of the final result matrix, ${\sf C}(:,{\sf col\_index})$.  
This is carried out under control of the mask which acts as a ``write mask''.
\begin{itemize}
\item If {\sf desc[GrB\_OUTP].GrB\_REPLACE} is set, then any values in ${\sf C}(:,{\sf col\_index})$ 
on input to this operation are deleted and the new contents of the column is given by:
\[
\bold{L}({\sf C}) = \{ (i,j,C_{ij}) : j \neq {\sf col\_index} \} \cup \{(i,{\sf col\_index},z_{i}) : i \in 
(\bold{ind}(\vector{\widetilde{z}}) 
\cap \bold{ind}(\vector{\widetilde{m}})) \}. 
\]

\item If {\sf desc[GrB\_OUTP].GrB\_REPLACE} is not set, the elements of 
$\vector{\widetilde{z}}$ indicated by the mask are copied into the column 
of the final result matrix, ${\sf C}(:,{\sf col\_index})$, and elements of 
this column that fall outside the set indicated by 
the mask are unchanged:
\begin{eqnarray} 
    \bold{L}({\sf C}) & = & \{ (i,j,C_{ij}) : j \neq {\sf col\_index} \} \cup \nonumber \\
    & & \{(i,{\sf col\_index},\vector{\widetilde{c}}(i)) : i \in (\bold{ind}(\vector{\widetilde{c}}) 
    \cap \bold{ind}(\neg \vector{\widetilde{m}})) \} \cup \nonumber \\
    & & \{(i,{\sf col\_index},z_{i}) : i \in 
    (\bold{ind}(\vector{\widetilde{z}}) \cap \bold{ind}(\vector{\widetilde{m}})) \}. \nonumber
\end{eqnarray}
\end{itemize}

In {\sf GrB\_BLOCKING} mode, the method exits with return value 
{\sf GrB\_SUCCESS} and the new content of vector {\sf w} is as defined above
and fully computed.  
In {\sf GrB\_NONBLOCKING} mode, the method exits with return value 
{\sf GrB\_SUCCESS} and the new content of vector {\sf w} is as defined above 
but may not be fully computed; however, it can be used in the next GraphBLAS 
method call in a sequence.


%-----------------------------------------------------------------------------

\subsubsection{{\sf assign}: Row variant}

Assign the contents a vector to a subset of elements in one row of a matrix. 
Note that since the output cannot be transposed, a different variant of
{\sf assign} is provided to assign to a column of a matrix.

\paragraph{\syntax}

\begin{verbatim}
        GrB_Info GrB_assign(GrB_Matrix            C,
                            const GrB_Vector      mask,
                            const GrB_BinaryOp    accum,
                            const GrB_Vector      u,
                            GrB_Index             row_index,
                            const GrB_Index      *col_indices,
                            GrB_Index             ncols,
                            const GrB_Descriptor  desc);
\end{verbatim}

\paragraph{Parameters}

\begin{itemize}[leftmargin=1.1in]
    \item[{\sf C}]    ({\sf INOUT}) An existing GraphBLAS Matrix.  On input,
    the matrix provides values that may be accumulated with the result of the
    assign operation.  On output, this matrix holds the results of the
    operation.

    \item[{\sf mask}] ({\sf IN}) An optional ``write'' mask that controls which
    results from this operation are stored into the specified row of the output matrix {\sf C}. The 
    mask dimensions must match those of a single row of the matrix {\sf C}. If the 
    {\sf GrB\_STRUCTURE} descriptor is {\em not} set for the mask, the domain of the 
    {\sf Mask} matrix must be of type {\sf bool} or any of the predefined 
    ``built-in'' types in Table~\ref{Tab:PredefinedTypes}.  If the default
    mask is desired (\ie, a mask that is all {\sf true} with the dimensions of a row of {\sf C}), 
    {\sf GrB\_NULL} should be specified.

    \item[{\sf accum}] ({\sf IN}) An optional binary operator used for accumulating
    entries into existing {\sf C} entries. If assignment rather than accumulation is
    desired, {\sf GrB\_NULL} should be specified.

    \item[{\sf u}]       ({\sf IN}) The GraphBLAS vector whose contents are 
    assigned to (a subset of) a row of {\sf C}.

    \item[{\sf row\_index}]  ({\sf IN}) The index of the row in {\sf C} to 
    assign. Must be in the range $[0, \bold{nrows}({\sf C}))$.

    \item[{\sf col\_indices}]  ({\sf IN}) Pointer to the ordered set (array) of 
    indices corresponding to the locations in the specified row of {\sf C} 
    that are to be assigned.  If all elements of the row in {\sf C} are to be 
    assigned in order from index $0$ to ${\sf ncols} - 1$, then {\sf GrB\_ALL} should be 
    specified.  Regardless of execution mode and return value, this array may be
    manipulated by the caller after this operation returns without affecting any 
    deferred computations for this operation.
    If this array contains duplicate values, it implies in assignment of more 
    than one value to the same location which leads to undefined results.
    
    \item[{\sf ncols}] ({\sf IN}) The number of values in {\sf col\_indices} array.
    Must be equal to $\bold{size}({\sf u})$.

    \item[{\sf desc}] ({\sf IN}) An optional operation descriptor. If
    a \emph{default} descriptor is desired, {\sf GrB\_NULL} should be
    specified. Non-default field/value pairs are listed as follows:  \\

    \hspace*{-2em}\begin{tabular}{lllp{2.7in}}
        Param & Field  & Value & Description \\
        \hline
        {\sf C}    & {\sf GrB\_OUTP} & {\sf GrB\_REPLACE} &  Output row in 
        {\sf C} is cleared (all elements removed) before result is stored in it.\\

        {\sf mask} & {\sf GrB\_MASK} & {\sf GrB\_STRUCTURE}   & The write mask is
        constructed from the structure (pattern of stored values) of the input
        {\sf mask} vector. The stored values are not examined.\\

        {\sf mask} & {\sf GrB\_MASK} & {\sf GrB\_COMP}   & Use the 
        complement of {\sf mask}. \\
    \end{tabular}
\end{itemize}

\paragraph{Return Values}

\begin{itemize}[leftmargin=2.3in]
    \item[{\sf GrB\_SUCCESS}]         In blocking mode, the operation completed
    successfully. In non-blocking mode, this indicates that the compatibility 
    tests on dimensions and domains for the input arguments passed successfully. 
    Either way, output matrix {\sf C} is ready to be used in the next method of 
    the sequence.

    \item[{\sf GrB\_PANIC}]           Unknown internal error.

    \item[{\sf GrB\_INVALID\_OBJECT}] This is returned in any execution mode 
    whenever one of the opaque GraphBLAS objects (input or output) is in an invalid 
    state caused by a previous execution error.  Call {\sf GrB\_error()} to access 
    any error messages generated by the implementation.

    \item[{\sf GrB\_OUT\_OF\_MEMORY}] Not enough memory available for operation.

    \item[{\sf GrB\_UNINITIALIZED\_OBJECT}] One or more of the GraphBLAS objects
    has not been initialized by a call to {\sf new} (or {\sf dup} for vector or
    matrix parameters).

    \item[{\sf GrB\_INVALID\_INDEX}]    {\sf row\_index} is outside the allowable 
    range (i.e., greater than $\bold{nrows}({\sf C})$).

    \item[{\sf GrB\_INDEX\_OUT\_OF\_BOUNDS}]  A value in {\sf col\_indices} 
    is greater than or equal to $\bold{ncols}({\sf C})$.  In 
    non-blocking mode, this can be reported as an execution error.

    \item[{\sf GrB\_DIMENSION\_MISMATCH}] {\sf mask} size and number of columns
    in {\sf C} are not the same, or ${\sf ncols} \neq \bold{size}({\sf u})$.

    \item[{\sf GrB\_DOMAIN\_MISMATCH}]    The domains of the matrix and vector are
    incompatible with each other or the corresponding domains of the
    accumulation operator, or the mask's domain is not compatible with {\sf bool}
    (in the case where {\sf desc[GrB\_MASK].GrB\_STRUCTURE} is not set).

    \item[{\sf GrB\_NULL\_POINTER}] Argument {\sf col\_indices} is a {\sf NULL} pointer.
\end{itemize}

\paragraph{Description}

This variant of {\sf GrB\_assign} computes the result of assigning a subset of
locations in a row of a GraphBLAS matrix (in a specific order) from the 
contents of a GraphBLAS vector: \\
${\sf C}({\sf row\_index},:) = {\sf u}$; or, if an 
optional binary accumulation operator ($\odot$) is provided, 
${\sf C}({\sf row\_index},:) = 
{\sf C}({\sf row\_index},:) \odot {\sf u}$. Taking order of {\sf col\_indices}
into account it is more explicitly written as:
\[
\begin{aligned}
    {\sf C}({\sf row\_index},{\sf col\_indices}[j]) = &\ {\sf u}(j),
    \ \forall \ j \ : \ 0 \leq j < {\sf ncols}, \mbox{~or~}
    \\
    {\sf C}({\sf row\_index},{\sf col\_indices}[j]) = &\ 
    {\sf C}({\sf row\_index},{\sf col\_indices}[j]) \odot {\sf u}(j),
    \ \forall \ j \ : \ 0 \leq j < {\sf ncols}
\end{aligned}
\]
Logically, this operation occurs in three steps:
\begin{enumerate}[leftmargin=0.75in]
\item[\bf Setup] The internal matrices, vectors and mask used in the computation are formed 
and their domains and dimensions are tested for compatibility.
\item[\bf Compute] The indicated computations are carried out.
\item[\bf Output] The result is written into the output matrix, possibly under 
control of a mask.
\end{enumerate}

Up to three argument vectors and matrices are used in this {\sf GrB\_assign} 
operation:
\begin{enumerate}
	\item ${\sf C} = \langle \bold{D}({\sf C}),\bold{nrows}({\sf C}),
    \bold{ncols}({\sf C}),\bold{L}({\sf C}) = \{(i,j,C_{ij}) \} \rangle$

	\item ${\sf mask} = \langle \bold{D}({\sf mask}),\bold{size}({\sf mask}),
    \bold{L}({\sf mask}) = \{(i,m_i) \} \rangle$ (optional)

	\item ${\sf u} = \langle \bold{D}({\sf u}),\bold{size}({\sf u}),
    \bold{L}({\sf u}) = \{(i,u_i) \} \rangle$
\end{enumerate}

The argument vectors, matrix, and the accumulation 
operator (if provided) are tested for domain compatibility as follows:
\begin{enumerate}
	\item If {\sf mask} is not {\sf GrB\_NULL}, and ${\sf desc[GrB\_MASK].GrB\_STRUCTURE}$
    is not set, then $\bold{D}({\sf mask})$ must be from one of the pre-defined types of 
    Table~\ref{Tab:PredefinedTypes}.

	\item $\bold{D}({\sf C})$ must be 
    compatible with $\bold{D}({\sf u})$.

	\item If {\sf accum} is not {\sf GrB\_NULL}, then $\bold{D}({\sf C})$ must be
    compatible with $\bDin1({\sf accum})$ and $\bDout({\sf accum})$ of the accumulation operator and 
    $\bold{D}({\sf u})$ must be compatible with $\bDin2({\sf accum})$ of the accumulation operator.
\end{enumerate}
Two domains are compatible with each other if values from one domain can be cast 
to values in the other domain as per the rules of the C language.
In particular, domains from Table~\ref{Tab:PredefinedTypes} are all compatible 
with each other. A domain from a user-defined type is only compatible with itself.
If any compatibility rule above is violated, execution of {\sf GrB\_assign} ends
and the domain mismatch error listed above is returned.

The {\sf row\_index} parameter is checked for a valid value.  The following
condition must hold:
\begin{enumerate}
	\item $0\ \leq\ {\sf row\_index} \ <\ \bold{nrows}({\sf C})$
\end{enumerate}
If the rule above is violated, execution of {\sf GrB\_assign} ends 
and the invalid index error listed above is returned.

From the arguments, the internal vectors, mask, and index array used in 
the computation are formed ($\leftarrow$ denotes copy):
\begin{enumerate}
	\item The vector, $\vector{\widetilde{c}}$, is extracted from a row of {\sf C}
    as follows:
    \[
        \vector{\widetilde{c}} = \langle  \bold{D}({\sf C}), \bold{ncols}({\sf C}), 
        \{ (j, C_{ij}) \ \forall \ j : 0 \leq j < \bold{ncols}({\sf C}),
        i = {\sf row\_index}, (i, j) \in \bold{ind}({\sf C})\} \rangle
    \]

	\item One-dimensional mask, $\vector{\widetilde{m}}$, is computed from 
    argument {\sf mask} as follows:
	\begin{enumerate}
		\item If ${\sf mask} = {\sf GrB\_NULL}$, then $\vector{\widetilde{m}} = 
        \langle \bold{ncols}({\sf C}), \{i, \ \forall \ i : 0 \leq i < 
        \bold{ncols}({\sf C}) \} \rangle$.

		\item If {\sf mask} $\ne$ {\sf GrB\_NULL},  
        \begin{enumerate}
            \item If ${\sf desc[GrB\_MASK].GrB\_STRUCTURE}$ is set, then
            $\vector{\widetilde{m}} = 
            \langle \bold{size}({\sf mask}), \{i : i \in \bold{ind}({\sf mask}) \} \rangle$,
            \item Otherwise, $\vector{\widetilde{m}} = 
            \langle \bold{size}({\sf mask}), \{i : i \in \bold{ind}({\sf mask}) \wedge
            ({\sf bool}){\sf mask}(i) = \true \} \rangle$.
        \end{enumerate}

		\item	If ${\sf desc[GrB\_MASK].GrB\_COMP}$ is set, then 
        $\vector{\widetilde{m}} \leftarrow \neg \vector{\widetilde{m}}$.
	\end{enumerate}

	\item Vector $\vector{\widetilde{u}} \leftarrow {\sf u}$.
    
    \item The internal column index array, $\grbarray{\widetilde{J}}$, is computed from 
    argument {\sf col\_indices} as follows:
	\begin{enumerate}
		\item	If ${\sf col\_indices} = {\sf GrB\_ALL}$, then 
        $\grbarray{\widetilde{J}}[j] = j, \ \forall \ j : 0 \leq j < {\sf ncols}$.

		\item	Otherwise, $\grbarray{\widetilde{J}}[j] = {\sf col\_indices}[j], 
        \ \forall \ j : 0 \leq j < {\sf ncols}$.
    \end{enumerate}
\end{enumerate}

The internal vectors, matrices, and masks are checked for dimension compatibility. 
The following conditions must hold:
\begin{enumerate}
    \item $\bold{size}(\vector{\widetilde{c}}) = \bold{size}(\vector{\widetilde{m}})$
    \item ${\sf ncols} = \bold{size}(\vector{\widetilde{u}})$.
\end{enumerate}
If any compatibility rule above is violated, execution of {\sf GrB\_assign} ends and 
the dimension mismatch error listed above is returned.

From this point forward, in {\sf GrB\_NONBLOCKING} mode, the method can 
optionally exit with {\sf GrB\_SUCCESS} return code and defer any computation 
and/or execution error codes.

We are now ready to carry out the assign and any additional 
associated operations.  We describe this in terms of two intermediate vectors:
\begin{itemize}
    \item $\vector{\widetilde{t}}$: The vector holding the elements from
    $\vector{\widetilde{u}}$ in their destination locations relative to 
    $\vector{\widetilde{c}}$.

    \item $\vector{\widetilde{z}}$: The vector holding the result after 
    application of the (optional) accumulation operator.
\end{itemize}

The intermediate vector, $\vector{\widetilde{t}}$, is created as follows:
\[
\vector{\widetilde{t}} = \langle
\bold{D}({\sf u}), \bold{size}(\vector{\widetilde{c}}),
\{(\grbarray{\widetilde{J}}[j],\vector{\widetilde{u}}(j)) \ \forall \ j, \ 
0 \leq j < {\sf ncols} : j \in \bold{ind}(\vector{\widetilde{u}}) \} \rangle. 
\]
At this point, if any value of $\grbarray{\widetilde{J}}[j]$ is outside the valid 
range of indices for vector $\vector{\widetilde{c}}$, computation ends and the 
method returns the index out-of-bounds error listed above. In 
{\sf GrB\_NONBLOCKING} mode, the error can be deferred until a 
sequence-terminating {\sf GrB\_wait()} is called.  Regardless, the result 
matrix, {\sf C}, is invalid from this point forward in the 
sequence.

The intermediate vector $\vector{\widetilde{z}}$ is created as follows:
\begin{itemize}
    \item If ${\sf accum} = {\sf GrB\_NULL}$, then $\vector{\widetilde{z}}$ is defined as 
    \[ 
        \vector{\widetilde{z}} =
		\langle \bold{D}({\sf C}), \bold{size}(\vector{\widetilde{c}}), 
		\{(i,z_{i}), \forall i \in (\bold{ind}(\vector{\widetilde{c}})-(\{\grbarray{\widetilde{I}}[k],\forall k\} \cap \bold{ind}(\vector{\widetilde{c}}))) \cup 
        \bold{ind}(\vector{\widetilde{t}}) \} \rangle.
    \]
    The above expression defines the structure of vector $\vector{\widetilde{z}}$ as follows:
    We start with the structure of $\vector{\widetilde{c}}$ ($\bold{ind}(\vector{\widetilde{c}})$) and remove from 
    it all the indices of $\vector{\widetilde{c}}$ that are
    in the set of indices being assigned ($\{\grbarray{\widetilde{I}}[k],\forall k\} \cap \bold{ind}(\vector{\widetilde{c}})$). Finally, we
    add the structure of $\vector{\widetilde{t}}$ ($\bold{ind}(\vector{\widetilde{t}})$).

    The values of the elements of $\vector{\widetilde{z}}$ are computed based on the 
    relationships between the sets of indices in $\vector{\widetilde{c}}$ 
    and $\vector{\widetilde{t}}$.
    \[
        z_{i} = \vector{\widetilde{c}}(i), \ \mbox{if}\  i \in  
        (\bold{ind}(\vector{\widetilde{c}}) - (\{\grbarray{\widetilde{I}}[k],\forall k\}
        \cap \bold{ind}(\vector{\widetilde{c}}))),
    \]
    \[
        z_{i} = \vector{\widetilde{t}}(i), \ \mbox{if}\  i \in  
        \bold{ind}(\vector{\widetilde{t}}),
    \]
    where the difference operator refers to set difference.

    \item If ${\sf accum}$ is a binary operator, then $\vector{\widetilde{z}}$ is defined as
        \[ \langle \bDout({\sf accum}), \bold{size}(\vector{\widetilde{c}}),
        \{(j,z_{j}) \ \forall \ j \in \bold{ind}(\vector{\widetilde{c}}) \cup 
        \bold{ind}(\vector{\widetilde{t}}) \} \rangle.\]

    The values of the elements of $\vector{\widetilde{z}}$ are computed based on the 
    relationships between the sets of indices in $\vector{\widetilde{w}}$ and 
    $\vector{\widetilde{t}}$.
\[
    z_{j} = \vector{\widetilde{c}}(j) \odot \vector{\widetilde{t}}(j), \ \mbox{if}\  
    j \in  (\bold{ind}(\vector{\widetilde{t}}) \cap \bold{ind}(\vector{\widetilde{c}})),
\]
\[
    z_{j} = \vector{\widetilde{c}}(j), \ \mbox{if}\  
    j \in  (\bold{ind}(\vector{\widetilde{c}}) - (\bold{ind}(\vector{\widetilde{t}})
    \cap \bold{ind}(\vector{\widetilde{c}}))),
\]
\[
    z_{j} = \vector{\widetilde{t}}(j), \ \mbox{if}\  j \in  
    (\bold{ind}(\vector{\widetilde{t}}) - (\bold{ind}(\vector{\widetilde{t}})
    \cap \bold{ind}(\vector{\widetilde{c}}))),
\]
where $\odot  = \bigodot({\sf accum})$, and the difference operator refers to set difference.
\end{itemize}

Finally, the set of output values that make up the $\vector{\widetilde{z}}$ 
vector are written into the column of the final result matrix, ${\sf C}({\sf row\_index},:)$.  
This is carried out under control of the mask which acts as a ``write mask''.
\begin{itemize}
\item If {\sf desc[GrB\_OUTP].GrB\_REPLACE} is set, then any values in ${\sf C}({\sf row\_index},:)$ 
on input to this operation are deleted and the new contents of the column is given by:
\[
\bold{L}({\sf C}) = \{ (i,j,C_{ij}) : i \neq {\sf row\_index} \} \cup \{({\sf row\_index},j,z_{j}) : j \in 
(\bold{ind}(\vector{\widetilde{z}}) 
\cap \bold{ind}(\vector{\widetilde{m}})) \}. 
\]

\item If {\sf desc[GrB\_OUTP].GrB\_REPLACE} is not set, the elements of 
$\vector{\widetilde{z}}$ indicated by the mask are copied into the column 
of the final result matrix, ${\sf C}({\sf row\_index},:)$, and elements of 
this column that fall outside the set indicated by the mask are unchanged:
\begin{eqnarray} 
    \bold{L}({\sf C}) & = & \{ (i,j,C_{ij}) : i \neq {\sf row\_index} \} \cup \nonumber \\
    & & \{({\sf row\_index},j,\vector{\widetilde{c}}(j)) : j \in (\bold{ind}(\vector{\widetilde{c}}) 
    \cap \bold{ind}(\neg \vector{\widetilde{m}})) \} \cup \nonumber \\
    & & \{({\sf row\_index},j,z_{j}) : j \in 
    (\bold{ind}(\vector{\widetilde{z}}) \cap \bold{ind}(\vector{\widetilde{m}})) \}. \nonumber
\end{eqnarray}
\end{itemize}

In {\sf GrB\_BLOCKING} mode, the method exits with return value 
{\sf GrB\_SUCCESS} and the new content of vector {\sf w} is as defined above
and fully computed.  
In {\sf GrB\_NONBLOCKING} mode, the method exits with return value 
{\sf GrB\_SUCCESS} and the new content of vector {\sf w} is as defined above 
but may not be fully computed; however, it can be used in the next GraphBLAS 
method call in a sequence.


%-----------------------------------------------------------------------------

\subsubsection{{\sf assign}: Constant vector variant}

Assign the same value to a specified subset of vector elements.  With the use of
{\sf GrB\_ALL}, the entire destination vector can be filled with the constant.

\paragraph{\syntax}

\begin{verbatim}
        GrB_Info GrB_assign(GrB_Vector            w,
                            const GrB_Vector      mask,
                            const GrB_BinaryOp    accum,
                            <type>                val,
                            const GrB_Index      *indices,
                            GrB_Index             nindices,
                            const GrB_Descriptor  desc);
\end{verbatim}

\begin{verbatim}
        GrB_Info GrB_assign(GrB_Vector            w,
                            const GrB_Vector      mask,
                            const GrB_BinaryOp    accum,
                            const GrB_Scalar      s,
                            const GrB_Index      *indices,
                            GrB_Index             nindices,
                            const GrB_Descriptor  desc);
\end{verbatim}

\paragraph{Parameters}

\begin{itemize}[leftmargin=1.1in]
    \item[{\sf w}]    ({\sf INOUT}) An existing GraphBLAS vector.  On input,
    the vector provides values that may be accumulated with the result of the
    assign operation.  On output, this vector holds the results of the
    operation.

    \item[{\sf mask}] ({\sf IN}) An optional ``write'' mask that controls which
    results from this operation are stored into the output vector {\sf w}. The 
    mask dimensions must match those of the vector {\sf w}. If the 
    {\sf GrB\_STRUCTURE} descriptor is {\em not} set for the mask, the domain of the
    {\sf mask} vector must be of type {\sf bool} or any of the predefined 
    ``built-in'' types in Table~\ref{Tab:PredefinedTypes}.  If the default
    mask is desired (\ie, a mask that is all {\sf true} with the dimensions of {\sf w}), 
    {\sf GrB\_NULL} should be specified.

    \item[{\sf accum}] ({\sf IN}) An optional binary operator used for accumulating
    entries into existing {\sf w} entries. If assignment rather than accumulation is
    desired, {\sf GrB\_NULL} should be specified.

    \item[{\sf val}]    ({\sf IN}) Scalar value to assign to (a subset of) {\sf w}.

    \item[{\sf s}]    ({\sf IN}) Scalar value to assign to (a subset of) {\sf w}.

    \item[{\sf indices}]  ({\sf IN}) Pointer to the ordered set (array) of 
    indices corresponding to the locations in {\sf w} that are to be assigned.  
    If all elements of {\sf w} are to be assigned in order from $0$ to 
    ${\sf nindices} - 1$, then {\sf GrB\_ALL} should be specified.  Regardless of 
    execution mode and return value, this array may be manipulated by the caller
    after this operation returns without affecting any deferred computations for 
    this operation.  
    In this variant, the specific order of the values in the
    array has no effect on the result.  Unlike other variants, if there are 
    duplicated values in this array the result is still defined.
    
    \item[{\sf nindices}] ({\sf IN}) The number of values in {\sf indices} array.
    Must be in the range: $[0, \bold{size}({\sf w})]$.  If {\sf nindices}
    is zero, the operation becomes a NO-OP.

    \item[{\sf desc}] ({\sf IN}) An optional operation descriptor. If
    a \emph{default} descriptor is desired, {\sf GrB\_NULL} should be
    specified. Non-default field/value pairs are listed as follows:  \\

    \hspace*{-2em}\begin{tabular}{lllp{2.7in}}
        Param & Field  & Value & Description \\
        \hline
        {\sf w}    & {\sf GrB\_OUTP} & {\sf GrB\_REPLACE} & Output vector {\sf w}
        is cleared (all elements removed) before the result is stored in it.\\

        {\sf mask} & {\sf GrB\_MASK} & {\sf GrB\_STRUCTURE}   & The write mask is
        constructed from the structure (pattern of stored values) of the input
        {\sf mask} vector. The stored values are not examined.\\

        {\sf mask} & {\sf GrB\_MASK} & {\sf GrB\_COMP}   & Use the 
        complement of {\sf mask}. \\
    \end{tabular}
\end{itemize}

\paragraph{Return Values}

\begin{itemize}[leftmargin=2.3in]
    \item[{\sf GrB\_SUCCESS}]         In blocking mode, the operation completed
    successfully. In non-blocking mode, this indicates that the compatibility 
    tests on dimensions and domains for the input arguments passed successfully. 
    Either way, output vector {\sf w} is ready to be used in the next method of 
    the sequence.

    \item[{\sf GrB\_PANIC}]           Unknown internal error.

    \item[{\sf GrB\_INVALID\_OBJECT}] This is returned in any execution mode 
    whenever one of the opaque GraphBLAS objects (input or output) is in an invalid 
    state caused by a previous execution error.  Call {\sf GrB\_error()} to access 
    any error messages generated by the implementation.

    \item[{\sf GrB\_OUT\_OF\_MEMORY}] Not enough memory available for operation.

    \item[{\sf GrB\_UNINITIALIZED\_OBJECT}] One or more of the GraphBLAS objects
    has not been initialized by a call to {\sf new} (or {\sf dup} for vector
    parameters).

    \item[{\sf GrB\_INDEX\_OUT\_OF\_BOUNDS}]  A value in {\sf indices} is greater
    than or equal to $\bold{size}({\sf w})$.  In non-blocking mode, this can be
    reported as an execution error.

    \item[{\sf GrB\_DIMENSION\_MISMATCH}] {\sf mask} and {\sf w} dimensions are
    incompatible, or {\sf nindices} is not less than $\bold{size}({\sf w})$. 

    \item[{\sf GrB\_DOMAIN\_MISMATCH}]    The domains of the vector and scalar are
    incompatible with each other or the corresponding domains of the
    accumulation operator, or the mask's domain is not compatible with {\sf bool}
    (in the case where {\sf desc[GrB\_MASK].GrB\_STRUCTURE} is not set).

    \item[{\sf GrB\_NULL\_POINTER}] Argument {\sf indices} is a {\sf NULL} pointer.
\end{itemize}

\paragraph{Description}

This variant of {\sf GrB\_assign} computes the result of assigning a constant
scalar value -- either {\sf val} or {\sf s} -- to locations in a destination GraphBLAS vector.
Either 
${\sf w}({\sf indices}) = {\sf val}$ or ${\sf w}({\sf indices}) = {\sf s}$ is performed.
If an optional binary accumulation 
operator ($\odot$) is provided, then either
${\sf w}({\sf indices}) = {\sf w}({\sf indices}) \odot {\sf val}$ or
${\sf w}({\sf indices}) = {\sf w}({\sf indices}) \odot {\sf s}$ is performed.  
More explicitly, if a non-opaque value {\sf val} is provided:
\[
\begin{aligned}
    {\sf w}({\sf indices}[i]) = &\ {\sf val}, \ 
    \forall \  i : 0 \leq i < {\sf nindices}, \mbox{~~or~~}
    \\
    {\sf w}({\sf indices}[i]) = &\ {\sf w}({\sf indices}[i]) \odot {\sf val}, \ 
    \forall \  i : 0 \leq i < {\sf nindices}.
\end{aligned}
\]
Correspondingly, if a {\sf GrB\_Scalar} {\sf s} is provided:
\[
\begin{aligned}
    {\sf w}({\sf indices}[i]) = &\ {\sf s}, \ 
    \forall \  i : 0 \leq i < {\sf nindices}, \mbox{~~or~~}
    \\
    {\sf w}({\sf indices}[i]) = &\ {\sf w}({\sf indices}[i]) \odot {\sf s}, \ 
    \forall \  i : 0 \leq i < {\sf nindices}.
\end{aligned}
\]

Logically, this operation occurs in three steps:
\begin{enumerate}[leftmargin=0.75in]
\item[\bf Setup] The internal vectors and mask used in the computation are formed 
and their domains and dimensions are tested for compatibility.
\item[\bf Compute] The indicated computations are carried out.
\item[\bf Output] The result is written into the output vector, possibly under 
control of a mask.
\end{enumerate}

Up to two argument vectors are used in the {\sf GrB\_assign} operation:
\begin{enumerate}
	\item ${\sf w} = \langle \bold{D}({\sf w}),\bold{size}({\sf w}),
    \bold{L}({\sf w}) = \{(i,w_i) \} \rangle$

	\item ${\sf mask} = \langle \bold{D}({\sf mask}),\bold{size}({\sf mask}),
    \bold{L}({\sf mask}) = \{(i,m_i) \} \rangle$ (optional)
\end{enumerate}

The argument scalar, vectors, and the accumulation 
operator (if provided) are tested for domain compatibility as follows:
\begin{enumerate}
	\item If {\sf mask} is not {\sf GrB\_NULL}, and ${\sf desc[GrB\_MASK].GrB\_STRUCTURE}$
    is not set, then $\bold{D}({\sf mask})$ must be from one of the pre-defined types of 
    Table~\ref{Tab:PredefinedTypes}.

	\item $\bold{D}({\sf w})$ must be 
    compatible with either $\bold{D}({\sf val})$ or $\bold{D}({\sf s})$, depending
	on the signature of the method.

	\item If {\sf accum} is not {\sf GrB\_NULL}, then $\bold{D}({\sf w})$ must be
    compatible with $\bDin1({\sf accum})$ and $\bDout({\sf accum})$ of the accumulation operator.
	
    \item If {\sf accum} is not {\sf GrB\_NULL}, then  
    either $\bold{D}({\sf val})$ or $\bold{D}({\sf s})$, depending on the signature of the method, must be compatible with $\bDin2({\sf accum})$ of the accumulation operator.
\end{enumerate}
Two domains are compatible with each other if values from one domain can be cast 
to values in the other domain as per the rules of the C language.
In particular, domains from Table~\ref{Tab:PredefinedTypes} are all compatible 
with each other. A domain from a user-defined type is only compatible with itself.
If any compatibility rule above is violated, execution of {\sf GrB\_assign} ends
and the domain mismatch error listed above is returned.

From the arguments, the internal vectors, mask and index array used in 
the computation are formed ($\leftarrow$ denotes copy):
\begin{enumerate}
	\item Vector $\vector{\widetilde{w}} \leftarrow {\sf w}$.

	\item One-dimensional mask, $\vector{\widetilde{m}}$, is computed from 
    argument {\sf mask} as follows:
	\begin{enumerate}
		\item If ${\sf mask} = {\sf GrB\_NULL}$, then $\vector{\widetilde{m}} = 
        \langle \bold{size}({\sf w}), \{i, \ \forall \ i : 0 \leq i < 
        \bold{size}({\sf w}) \} \rangle$.

		\item If {\sf mask} $\ne$ {\sf GrB\_NULL},  
        \begin{enumerate}
            \item If ${\sf desc[GrB\_MASK].GrB\_STRUCTURE}$ is set, then
            $\vector{\widetilde{m}} = 
            \langle \bold{size}({\sf mask}), \{i : i \in \bold{ind}({\sf mask}) \} \rangle$,
            \item Otherwise, $\vector{\widetilde{m}} = 
            \langle \bold{size}({\sf mask}), \{i : i \in \bold{ind}({\sf mask}) \wedge
            ({\sf bool}){\sf mask}(i) = \true \} \rangle$.
        \end{enumerate}

		\item	If ${\sf desc[GrB\_MASK].GrB\_COMP}$ is set, then 
        $\vector{\widetilde{m}} \leftarrow \neg \vector{\widetilde{m}}$.
	\end{enumerate}

    \item Scalar $\scalar{\tilde{s}} \leftarrow {\sf s}$ ({\sf GrB\_Scalar} version only).

    \item The internal index array, $\grbarray{\widetilde{I}}$, is computed from 
    argument {\sf indices} as follows:
	\begin{enumerate}
		\item	If ${\sf indices} = {\sf GrB\_ALL}$, then 
        $\grbarray{\widetilde{I}}[i] = i, \ \forall \ i : 0 \leq i < {\sf nindices}$.

		\item	Otherwise, $\grbarray{\widetilde{I}}[i] = {\sf indices}[i], 
        \ \forall \ i : 0 \leq i < {\sf nindices}$.
    \end{enumerate}
\end{enumerate}

The internal vector and mask are checked for dimension compatibility. 
The following conditions must hold:
\begin{enumerate}
    \item $\bold{size}(\vector{\widetilde{w}}) = \bold{size}(\vector{\widetilde{m}})$
    \item $0 \leq {\sf nindices} \leq \bold{size}(\vector{\widetilde{w}})$.
\end{enumerate}
If any compatibility rule above is violated, execution of {\sf GrB\_assign} ends and 
the dimension mismatch error listed above is returned.

From this point forward, in {\sf GrB\_NONBLOCKING} mode, the method can 
optionally exit with {\sf GrB\_SUCCESS} return code and defer any computation 
and/or execution error codes.

We are now ready to carry out the assign and any additional 
associated operations.  We describe this in terms of two intermediate vectors:
\begin{itemize}
    \item $\vector{\widetilde{t}}$: The vector holding the copies of the scalar, either 
	{\sf val} or $\scalar{\tilde{s}}$, in their destination locations relative to 
    $\vector{\widetilde{w}}$.

    \item $\vector{\widetilde{z}}$: The vector holding the result after 
    application of the (optional) accumulation operator.
\end{itemize}

The intermediate vector, $\vector{\widetilde{t}}$, is created as follows. If a non-opaque scalar {\sf val} is provided:
\[
\vector{\widetilde{t}} = \langle
\bold{D}({\sf val}), \bold{size}(\vector{\widetilde{w}}),
\{(\grbarray{\widetilde{I}}[i],{\sf val})\ \forall \ i,\ 0 \leq i < {\sf nindices} \} \rangle. 
\]
Correspondingly, if a non-empty {\sf GrB\_Scalar} $\scalar{\tilde{s}}$ is provided (\ie, $\mathbf{size}(\scalar{\tilde{s}}) = 1$):
\[
\vector{\widetilde{t}} = \langle
\bold{D}(\scalar{\tilde{s}}), \bold{size}(\vector{\widetilde{w}}),
\{(\grbarray{\widetilde{I}}[i],\mathbf{val}(\scalar{\tilde{s}}))\ \forall \ i,\ 0 \leq i < {\sf nindices} \} \rangle. 
\]
Finally, if an empty {\sf GrB\_Scalar} $\scalar{\tilde{s}}$ is provided (\ie, $\mathbf{size}(\scalar{\tilde{s}}) = 0$):
\[
\vector{\widetilde{t}} = \langle
\bold{D}(\scalar{\tilde{s}}), \bold{size}(\vector{\widetilde{w}}),
\emptyset \rangle. 
\]

If $\grbarray{\widetilde{I}}$ is empty, this operation results in an empty 
vector, $\vector{\widetilde{t}}$.  Otherwise, if any value in the 
$\grbarray{\widetilde{I}}$ array is not in the range 
$[0,\ \bold{size}(\vector{\widetilde{w}}) )$, the execution of {\sf GrB\_assign} 
ends and the index out-of-bounds error listed above is generated. In 
{\sf GrB\_NONBLOCKING} mode, the error can be deferred until a 
sequence-terminating {\sf GrB\_wait()} is called.  Regardless, the result 
vector, {\sf w}, is invalid from this point forward in the 
sequence.

The intermediate vector $\vector{\widetilde{z}}$ is created as follows:
\begin{itemize}
    \item If ${\sf accum} = {\sf GrB\_NULL}$, then $\vector{\widetilde{z}}$ is defined as 
    \[ 
        \vector{\widetilde{z}} =
		\langle \bold{D}({\sf w}), \bold{size}(\vector{\widetilde{w}}), 
		\{(i,z_{i}), \forall i \in (\bold{ind}(\vector{\widetilde{w}})-(\{\grbarray{\widetilde{I}}[k],\forall k\} \cap \bold{ind}(\vector{\widetilde{w}}))) \cup 
        \bold{ind}(\vector{\widetilde{t}}) \} \rangle.
    \]
    The above expression defines the structure of vector $\vector{\widetilde{z}}$ as follows:
    We start with the structure of $\vector{\widetilde{w}}$ ($\bold{ind}(\vector{\widetilde{w}})$) and remove from 
    it all the indices of $\vector{\widetilde{w}}$ that are
    in the set of indices being assigned ($\{\grbarray{\widetilde{I}}[k],\forall k\} \cap \bold{ind}(\vector{\widetilde{w}})$). Finally, we
    add the structure of $\vector{\widetilde{t}}$ ($\bold{ind}(\vector{\widetilde{t}})$).

    The values of the elements of $\vector{\widetilde{z}}$ are computed based on the 
    relationships between the sets of indices in $\vector{\widetilde{w}}$ 
    and $\vector{\widetilde{t}}$.
    \[
        z_{i} = \vector{\widetilde{w}}(i), \ \mbox{if}\  i \in  
        (\bold{ind}(\vector{\widetilde{w}}) - (\{\grbarray{\widetilde{I}}[k],\forall k\}
        \cap \bold{ind}(\vector{\widetilde{w}}))),
    \]
    \[
        z_{i} = \vector{\widetilde{t}}(i), \ \mbox{if}\  i \in  
        \bold{ind}(\vector{\widetilde{t}}),
    \]
    where the difference operator refers to set difference.
    We note that in this case of assigning a constant, 
    $\{\grbarray{\widetilde{I}}[k],\forall k\}$ 
    and $\bold{ind}(\vector{\widetilde{t}})$ are identical.

    \item If ${\sf accum}$ is a binary operator, then $\vector{\widetilde{z}}$ is defined as
        \[ \langle \bDout({\sf accum}), \bold{size}(\vector{\widetilde{w}}),
        \{(i,z_{i}) \ \forall \ i \in \bold{ind}(\vector{\widetilde{w}}) \cup 
        \bold{ind}(\vector{\widetilde{t}}) \} \rangle.\]

    The values of the elements of $\vector{\widetilde{z}}$ are computed based on the 
    relationships between the sets of indices in $\vector{\widetilde{w}}$ and 
    $\vector{\widetilde{t}}$.
\[
    z_{i} = \vector{\widetilde{w}}(i) \odot \vector{\widetilde{t}}(i), \ \mbox{if}\  
    i \in  (\bold{ind}(\vector{\widetilde{t}}) \cap \bold{ind}(\vector{\widetilde{w}})),
\]
\[
    z_{i} = \vector{\widetilde{w}}(i), \ \mbox{if}\  
    i \in (\bold{ind}(\vector{\widetilde{w}}) - (\bold{ind}(\vector{\widetilde{t}})
    \cap \bold{ind}(\vector{\widetilde{w}}))),
\]
\[
    z_{i} = \vector{\widetilde{t}}(i), \ \mbox{if}\  i \in  
    (\bold{ind}(\vector{\widetilde{t}}) - (\bold{ind}(\vector{\widetilde{t}})
    \cap \bold{ind}(\vector{\widetilde{w}}))),
\]
where $\odot  = \bigodot({\sf accum})$, and the difference operator refers to set difference.
\end{itemize}

\input{ops_mask_replace_vector}



%-----------------------------------------------------------------------------

\subsubsection{{\sf assign}: Constant matrix variant}

Assign the same value to a specified subset of matrix elements.  With the use of 
{\sf GrB\_ALL}, the entire destination matrix can be filled with the constant.

\paragraph{\syntax}

\begin{verbatim}
        GrB_Info GrB_assign(GrB_Matrix            C,
                            const GrB_Matrix      Mask,
                            const GrB_BinaryOp    accum,
                            <type>                val,
                            const GrB_Index      *row_indices,
                            GrB_Index             nrows,
                            const GrB_Index      *col_indices,
                            GrB_Index             ncols,
                            const GrB_Descriptor  desc);
\end{verbatim}

\begin{verbatim}
        GrB_Info GrB_assign(GrB_Matrix            C,
                            const GrB_Matrix      Mask,
                            const GrB_BinaryOp    accum,
                            const GrB_Scalar      s,
                            const GrB_Index      *row_indices,
                            GrB_Index             nrows,
                            const GrB_Index      *col_indices,
                            GrB_Index             ncols,
                            const GrB_Descriptor  desc);
\end{verbatim}

\paragraph{Parameters}

\begin{itemize}[leftmargin=1.1in]
    \item[{\sf C}]    ({\sf INOUT}) An existing GraphBLAS matrix. On input,
    the matrix provides values that may be accumulated with the result of the
    assign operation.  On output, the matrix holds the results of the
    operation.

    \item[{\sf Mask}] ({\sf IN}) An optional ``write'' mask that controls which
    results from this operation are stored into the output matrix {\sf C}. The 
    mask dimensions must match those of the matrix {\sf C}. If the 
    {\sf GrB\_STRUCTURE} descriptor is {\em not} set for the mask, the domain of the 
    {\sf Mask} matrix must be of type {\sf bool} or any of the predefined 
    ``built-in'' types in Table~\ref{Tab:PredefinedTypes}.  If the default
    mask is desired (\ie, a mask that is all {\sf true} with the dimensions of {\sf C}), 
    {\sf GrB\_NULL} should be specified.

    \item[{\sf accum}] ({\sf IN}) An optional binary operator used for accumulating
    entries into existing {\sf C} entries.
    If assignment rather than accumulation is
    desired, {\sf GrB\_NULL} should be specified.

    \item[{\sf val}]    ({\sf IN}) Scalar value to assign to (a subset of) {\sf C}.
    
    \item[{\sf s}]    ({\sf IN}) Scalar value to assign to (a subset of) {\sf C}.

    \item[{\sf row\_indices}] ({\sf IN}) Pointer to the ordered set (array) of 
    indices corresponding to the rows of {\sf C} that are assigned.  If all rows
    of {\sf C} are to be assigned in order from $0$ to ${\sf nrows} - 1$, then 
    {\sf GrB\_ALL} can be specified.  Regardless of execution mode and return 
    value, this array may be manipulated by the caller after this operation 
    returns without affecting any deferred computations for this operation.  
    Unlike other variants, if there are duplicated values in this array the 
    result is still defined.

    \item[{\sf nrows}] ({\sf IN}) The number of values in {\sf row\_indices}
    array. Must be in the range: $[0, \bold{nrows}({\sf C})]$.  If
    {\sf nrows} is zero, the operation becomes a NO-OP.

    \item[{\sf col\_indices}] ({\sf IN}) Pointer to the ordered set (array) of 
    indices corresponding to the columns of {\sf C} that are assigned.  If all 
    columns of {\sf C} are to be assigned in order from $0$ to ${\sf ncols} - 1$, 
    then {\sf GrB\_ALL} should be specified.  Regardless of execution mode and return
    value, this array may be manipulated by the caller after this operation 
    returns without affecting any deferred computations for this operation.
    Unlike other variants, if there are duplicated values in this array the 
    result is still defined.

    \item[{\sf ncols}] ({\sf IN}) The number of values in {\sf col\_indices}
    array. Must be in the range: $[0, \bold{ncols}({\sf C})]$.  If
    {\sf ncols} is zero, the operation becomes a NO-OP.

    \item[{\sf desc}] ({\sf IN}) An optional operation descriptor. If
    a \emph{default} descriptor is desired, {\sf GrB\_NULL} should be
    specified. Non-default field/value pairs are listed as follows:  \\

    \hspace*{-2em}\begin{tabular}{lllp{2.7in}}
        Param & Field  & Value & Description \\
        \hline
        {\sf C}    & {\sf GrB\_OUTP} & {\sf GrB\_REPLACE} & Output matrix {\sf C}
        is cleared (all elements removed) before the result is stored in it.\\

        {\sf Mask} & {\sf GrB\_MASK} & {\sf GrB\_STRUCTURE}   & The write mask is
        constructed from the structure (pattern of stored values) of the input
        {\sf Mask} matrix. The stored values are not examined.\\

        {\sf Mask} & {\sf GrB\_MASK} & {\sf GrB\_COMP}   & Use the 
        complement of {\sf Mask}. \\
    \end{tabular}
\end{itemize}

\paragraph{Return Values}

\begin{itemize}[leftmargin=2.3in]
    \item[{\sf GrB\_SUCCESS}]         In blocking mode, the operation completed
    successfully. In non-blocking mode, this indicates that the compatibility 
    tests on dimensions and domains for the input arguments passed successfully. 
    Either way, output matrix {\sf C} is ready to be used in the next method of 
    the sequence.

    \item[{\sf GrB\_PANIC}]           Unknown internal error.

    \item[{\sf GrB\_INVALID\_OBJECT}] This is returned in any execution mode 
    whenever one of the opaque GraphBLAS objects (input or output) is in an invalid 
    state caused by a previous execution error.  Call {\sf GrB\_error()} to access 
    any error messages generated by the implementation.

    \item[{\sf GrB\_OUT\_OF\_MEMORY}] Not enough memory available for the operation.

    \item[{\sf GrB\_UNINITIALIZED\_OBJECT}] One or more of the GraphBLAS objects 
    has not been initialized by a call to {\sf new} (or {\sf dup} for vector
    parameters).

    \item[{\sf GrB\_INDEX\_OUT\_OF\_BOUNDS}]  A value in {\sf row\_indices} 
    is greater than or equal to $\bold{nrows}({\sf C})$, or a value in 
    {\sf col\_indices} is greater than or equal to $\bold{ncols}({\sf C})$.  In 
    non-blocking mode, this can be reported as an execution error.

    \item[{\sf GrB\_DIMENSION\_MISMATCH}] {\sf Mask} and {\sf C}
    dimensions are incompatible, {\sf nrows} is not less than $\bold{nrows}({\sf C})$, or
    {\sf ncols} is not less than $\bold{ncols}({\sf C})$. 

    \item[{\sf GrB\_DOMAIN\_MISMATCH}]    The domains of the matrix and scalar are
    incompatible with each other or the corresponding domains of the 
    accumulation operator, or the mask's domain is not compatible with {\sf bool}
    (in the case where {\sf desc[GrB\_MASK].GrB\_STRUCTURE} is not set).

    \item[{\sf GrB\_NULL\_POINTER}] Either argument {\sf row\_indices} is a {\sf NULL} pointer,
	argument {\sf col\_indices} is a {\sf NULL} pointer, or both.
\end{itemize}

\paragraph{Description}

This variant of {\sf GrB\_assign} computes the result of assigning a constant
scalar value -- either {\sf val} or {\sf s} -- to locations in a destination GraphBLAS matrix: 
Either
${\sf C}({\sf row\_indices, col\_indices}) = {\sf val}$ or ${\sf C}({\sf row\_indices, col\_indices}) = {\sf s}$ is performed.
If an optional binary accumulation operator ($\odot$) is provided, then either
${\sf C}({\sf row\_indices, col\_indices}) = {\sf C}({\sf row\_indices, col\_indices}) \odot {\sf val}$ or
${\sf C}({\sf row\_indices, col\_indices}) = {\sf C}({\sf row\_indices, col\_indices}) \odot {\sf s}$ is performed.
More explicitly, if a non-opaque value {\sf val} is provided:
\[
\begin{aligned}
	{\sf C}({\sf row\_indices}[i],{\sf col\_indices}[j]) =\ & {\sf val} \mbox{,~or~} \\
    {\sf C}({\sf row\_indices}[i],{\sf col\_indices}[j]) =\ & 
    {\sf C}({\sf row\_indices}[i],{\sf col\_indices}[j]) \odot {\sf val} \\
    & \ \forall \ (i,j) \ : \ 0 \leq i < {\sf nrows},\ 0 \leq j < {\sf ncols}
\end{aligned}
\]  
Correspondingly, if a {\sf GrB\_Scalar} {\sf s} is provided:
\[
\begin{aligned}
	{\sf C}({\sf row\_indices}[i],{\sf col\_indices}[j]) =\ & {\sf s} \mbox{,~or~} \\
    {\sf C}({\sf row\_indices}[i],{\sf col\_indices}[j]) =\ & 
    {\sf C}({\sf row\_indices}[i],{\sf col\_indices}[j]) \odot {\sf s} \\
    & \ \forall \ (i,j) \ : \ 0 \leq i < {\sf nrows},\ 0 \leq j < {\sf ncols}
\end{aligned}
\]

Logically, this operation occurs in three steps:
\begin{enumerate}[leftmargin=0.85in]
\item[Setup] The internal vectors and mask used in the computation are formed 
and their domains and dimensions are tested for compatibility.
\item[Compute] The indicated computations are carried out.
\item[Output] The result is written into the output matrix, possibly under 
control of a mask.
\end{enumerate}

Up to two argument matrices are used in the {\sf GrB\_assign} operation:
\begin{enumerate}
	\item ${\sf C} = \langle \bold{D}({\sf C}),\bold{nrows}({\sf C}),
    \bold{ncols}({\sf C}),\bold{L}({\sf C}) = \{(i,j,C_{ij}) \} \rangle$

	\item ${\sf Mask} = \langle \bold{D}({\sf Mask}),\bold{nrows}({\sf Mask}),
    \bold{ncols}({\sf Mask}),\bold{L}({\sf Mask}) = \{(i,j,M_{ij}) \} \rangle$ (optional)
\end{enumerate}

The argument scalar, matrices, and the accumulation 
operator (if provided) are tested for domain compatibility as follows:
\begin{enumerate}
	\item If {\sf Mask} is not {\sf GrB\_NULL}, and ${\sf desc[GrB\_MASK].GrB\_STRUCTURE}$
    is not set, then $\bold{D}({\sf Mask})$ must be from one of the pre-defined types of 
    Table~\ref{Tab:PredefinedTypes}.

	\item $\bold{D}({\sf C})$ must be 
    compatible with either $\bold{D}({\sf val})$ or $\bold{D}({\sf val})$, depending
	on the signature of the method.

	\item If {\sf accum} is not {\sf GrB\_NULL}, then $\bold{D}({\sf C})$ must be
    compatible with $\bDin1({\sf accum})$ and $\bDout({\sf accum})$ of the accumulation operator.
	
\item If {\sf accum} is not {\sf GrB\_NULL}, then  
    either $\bold{D}({\sf val})$ or $\bold{D}({\sf s})$, depending on the signature of the method, must be compatible with $\bDin2({\sf accum})$ of the accumulation operator.
\end{enumerate}
Two domains are compatible with each other if values from one domain can be cast 
to values in the other domain as per the rules of the C language.
In particular, domains from Table~\ref{Tab:PredefinedTypes} are all compatible 
with each other. A domain from a user-defined type is only compatible with itself.
If any compatibility rule above is violated, execution of {\sf GrB\_assign} ends
and the domain mismatch error listed above is returned.

From the arguments, the internal matrices, index arrays, and mask used in 
the computation are formed ($\leftarrow$ denotes copy):
\begin{enumerate}
	\item Matrix $\matrix{\widetilde{C}} \leftarrow {\sf C}$.

	\item Two-dimensional mask $\matrix{\widetilde{M}}$ is computed from
    argument {\sf Mask} as follows:
	\begin{enumerate}
		\item If ${\sf Mask} = {\sf GrB\_NULL}$, then $\matrix{\widetilde{M}} = 
        \langle \bold{nrows}({\sf C}), \bold{ncols}({\sf C}), \{(i,j), 
        \forall i,j : 0 \leq i <  \bold{nrows}({\sf C}), 0 \leq j < 
        \bold{ncols}({\sf C}) \} \rangle$.

		\item If {\sf Mask} $\ne$ {\sf GrB\_NULL},
        \begin{enumerate}
            \item If ${\sf desc[GrB\_MASK].GrB\_STRUCTURE}$ is set, then 
            $\matrix{\widetilde{M}} = \langle \bold{nrows}({\sf Mask}), 
            \bold{ncols}({\sf Mask}), \{(i,j) : (i,j) \in \bold{ind}({\sf Mask}) \} \rangle$,
            \item Otherwise, $\matrix{\widetilde{M}} = \langle \bold{nrows}({\sf Mask}), 
            \bold{ncols}({\sf Mask}), \\ \{(i,j) : (i,j) \in \bold{ind}({\sf Mask}) \wedge 
            ({\sf bool}){\sf Mask}(i,j) = \true\} \rangle$.
        \end{enumerate}

		\item	If ${\sf desc[GrB\_MASK].GrB\_COMP}$ is set, then 
        $\matrix{\widetilde{M}} \leftarrow \neg \matrix{\widetilde{M}}$.
	\end{enumerate}

    \item Scalar $\scalar{\tilde{s}} \leftarrow {\sf s}$ ({\sf GrB\_Scalar} version only).

    \item The internal row index array, $\grbarray{\widetilde{I}}$, is computed from 
    argument {\sf row\_indices} as follows:
	\begin{enumerate}
		\item	If ${\sf row\_indices} = {\sf GrB\_ALL}$, then 
        $\grbarray{\widetilde{I}}[i] = i, \forall i : 0 \leq i < {\sf nrows}$.

		\item	Otherwise, $\grbarray{\widetilde{I}}[i] = {\sf row\_indices}[i], 
        \forall i : 0 \leq i < {\sf nrows}$.
    \end{enumerate}
    
    \item The internal column index array, $\grbarray{\widetilde{J}}$, is computed from 
    argument {\sf col\_indices} as follows:
	\begin{enumerate}
		\item	If ${\sf col\_indices} = {\sf GrB\_ALL}$, then 
        $\grbarray{\widetilde{J}}[j] = j, \forall j : 0 \leq j < {\sf ncols}$.

		\item	Otherwise, $\grbarray{\widetilde{J}}[j] = {\sf col\_indices}[j], 
        \forall j : 0 \leq j < {\sf ncols}$.
    \end{enumerate}
\end{enumerate}

The internal matrix and mask are checked for dimension compatibility. The following
conditions must hold:
\begin{enumerate}
    \item $\bold{nrows}(\matrix{\widetilde{C}}) = \bold{nrows}(\matrix{\widetilde{M}})$.

    \item $\bold{ncols}(\matrix{\widetilde{C}}) = \bold{ncols}(\matrix{\widetilde{M}})$.

    \item $0 \leq {\sf nrows} \leq \bold{nrows}(\matrix{\widetilde{C}})$.

    \item $0 \leq {\sf ncols} \leq \bold{ncols}(\matrix{\widetilde{C}})$.
\end{enumerate}
If any compatibility rule above is violated, execution of {\sf GrB\_assign} ends and 
the dimension mismatch error listed above is returned.

From this point forward, in {\sf GrB\_NONBLOCKING} mode, the method can 
optionally exit with {\sf GrB\_SUCCESS} return code and defer any computation 
and/or execution error codes.

We are now ready to carry out the assign and any additional 
associated operations.  We describe this in terms of two intermediate matrices:
\begin{itemize}
    \item $\matrix{\widetilde{T}}$: The matrix holding the copies of the scalar, either 
	{\sf val} or $\scalar{\tilde{s}}$, in their destination locations relative to 
    $\matrix{\widetilde{C}}$.

    \item $\matrix{\widetilde{Z}}$: The matrix holding the result after 
    application of the (optional) accumulation operator.
\end{itemize}

The intermediate matrix, $\matrix{\widetilde{T}}$, is created as follows. If a non-opaque scalar {\sf val} is provided:
\[
\begin{aligned}
\matrix{\widetilde{T}} = \langle & \bold{D}({\sf val}),
                           \bold{nrows}(\matrix{\widetilde{C}}), 
                           \bold{ncols}(\matrix{\widetilde{C}}), \\
 & \{ (\grbarray{\widetilde{I}}[i],\grbarray{\widetilde{J}}[j], {\sf val})
\ \forall \ (i,j), \ 0 \leq i < {\sf nrows}, \ 0 \leq j < {\sf ncols} \} \rangle. 
\end{aligned}
\]
Correspondingly, if a non-empty {\sf GrB\_Scalar} $\scalar{\tilde{s}}$ is provided (\ie, $\mathbf{size}(\scalar{\tilde{s}}) = 1$):
\[
\begin{aligned}
\matrix{\widetilde{T}} = \langle & \bold{D}(\scalar{\tilde{s}}),
                           \bold{nrows}(\matrix{\widetilde{C}}), 
                           \bold{ncols}(\matrix{\widetilde{C}}), \\
    & \{ (\grbarray{\widetilde{I}}[i],\grbarray{\widetilde{J}}[j], \mathbf{val}(\scalar{\tilde{s}}))
\ \forall \ (i,j), \ 0 \leq i < {\sf nrows}, \ 0 \leq j < {\sf ncols} \} \rangle. 
\end{aligned}
\]
Finally, if an empty {\sf GrB\_Scalar} $\scalar{\tilde{s}}$ is provided (\ie, $\mathbf{size}(\scalar{\tilde{s}}) = 0$):
\[
\begin{aligned}
\matrix{\widetilde{T}} = \langle & \bold{D}(\scalar{\tilde{s}}),
                           \bold{nrows}(\matrix{\widetilde{C}}), 
                           \bold{ncols}(\matrix{\widetilde{C}}), \emptyset \rangle. 
\end{aligned}
\]

If either $\grbarray{\widetilde{I}}$ or $\grbarray{\widetilde{J}}$ is empty, this 
operation results in an empty matrix, $\matrix{\widetilde{T}}$.  Otherwise, if 
any value in the $\grbarray{\widetilde{I}}$ array is not in
the range $[0,\ \bold{nrows}(\matrix{\widetilde{C}}) )$ or any value in the 
$\grbarray{\widetilde{J}}$ array is not in the range 
$[0,\ \bold{ncols}(\matrix{\widetilde{C}}))$, the execution of {\sf GrB\_assign} 
ends and the index out-of-bounds error listed above is generated.  In 
{\sf GrB\_NONBLOCKING} mode, the error can be deferred until a 
sequence-terminating {\sf GrB\_wait()} is called.  Regardless, the result 
matrix {\sf C} is invalid from this point forward in the sequence.

The intermediate matrix $\matrix{\widetilde{Z}}$ is created as follows:
\begin{itemize}
    \item If ${\sf accum} = {\sf GrB\_NULL}$, then $\matrix{\widetilde{Z}}$ is defined as 
    \begin{eqnarray}
        \matrix{\widetilde{Z}} & = &
		\langle \bold{D}({\sf C}),\bold{nrows}(\matrix{\widetilde{C}}),
        \bold{ncols}(\matrix{\widetilde{C}}), \nonumber \\
    & & \{(i,j,Z_{ij})  \forall (i,j) \in
        (\bold{ind}(\matrix{\widetilde{C}}) - (\{
            (\grbarray{\widetilde{I}}[k],\grbarray{\widetilde{J}}[l]),
            \forall k,l\} \cap \bold{ind}(\matrix{\widetilde{C}}))) \cup
        \bold{ind}(\matrix{\widetilde{T}}) \} \rangle. \nonumber
    \end{eqnarray}
    The above expression defines the structure of matrix $\matrix{\widetilde{Z}}$ as follows:
    We start with the structure of $\matrix{\widetilde{C}}$ ($\bold{ind}(\matrix{\widetilde{C}})$) and remove from 
    it all the indices of $\matrix{\widetilde{C}}$ that are
    in the set of indices being assigned ($\{(\grbarray{\widetilde{I}}[k],\grbarray{\widetilde{J}}[l]),\forall k,l\} \cap \bold{ind}(\matrix{\widetilde{C}})$). Finally, we
    add the structure of $\matrix{\widetilde{T}}$ ($\bold{ind}(\matrix{\widetilde{T}})$).

    The values of the elements of $\matrix{\widetilde{Z}}$ are computed based on the 
    relationships between the sets of indices in $\matrix{\widetilde{C}}$ and 
    $\matrix{\widetilde{T}}$.
\[
    Z_{ij} = \matrix{\widetilde{C}}(i,j), \mbox{~if~}  (i,j) \in  
    (\bold{ind}(\matrix{\widetilde{C}}) - (\{ (\grbarray{\widetilde{I}}[k],\grbarray{\widetilde{J}}[l]), \forall k, l\}
    \cap \bold{ind}(\matrix{\widetilde{C}}))),
\]
\[
    Z_{ij} = \matrix{\widetilde{T}}(i,j), \ \mbox{if}\ (i,j) \in  
    \bold{ind}(\matrix{\widetilde{T}}),
\]
where the difference operator refers to set difference.
We note that, in this particular case of assigning a constant to a matrix, the sets
$\{ (\grbarray{\widetilde{I}}[k],\grbarray{\widetilde{J}}[l]), \forall k, l\}$ and
$\bold{ind}(\matrix{\widetilde{T}})$ are identical.

    \item If ${\sf accum}$ is a binary operator, then $\matrix{\widetilde{Z}}$ is defined as
        \[ \langle \bDout({\sf accum}), \bold{nrows}(\matrix{\widetilde{C}}), \bold{ncols}(\matrix{\widetilde{C}}),
        \{(i,j,Z_{ij})  \forall (i,j) \in \bold{ind}(\matrix{\widetilde{C}}) \cup 
        \bold{ind}(\matrix{\widetilde{T}}) \} \rangle.\]

    The values of the elements of $\matrix{\widetilde{Z}}$ are computed based on the
    relationships between the sets of indices in $\matrix{\widetilde{C}}$ and 
    $\matrix{\widetilde{T}}$.
\[
    Z_{ij} = \matrix{\widetilde{C}}(i,j) \odot \matrix{\widetilde{T}}(i,j), \ \mbox{if}\  
    (i,j) \in  (\bold{ind}(\matrix{\widetilde{T}}) \cap \bold{ind}(\matrix{\widetilde{C}})),
\]
\[
    Z_{ij} = \matrix{\widetilde{C}}(i,j), \ \mbox{if}\  
    (i,j) \in (\bold{ind}(\matrix{\widetilde{C}}) - (\bold{ind}(\matrix{\widetilde{T}})
    \cap \bold{ind}(\matrix{\widetilde{C}}))),
\]
\[
    Z_{ij} = \matrix{\widetilde{T}}(i,j), \ \mbox{if}\  (i,j) \in  
    (\bold{ind}(\matrix{\widetilde{T}}) - (\bold{ind}(\matrix{\widetilde{T}})
    \cap \bold{ind}(\matrix{\widetilde{C}}))),
\]
where $\odot  = \bigodot({\sf accum})$, and the difference operator refers to set difference.
\end{itemize}

\input{ops_mask_replace_matrix}
