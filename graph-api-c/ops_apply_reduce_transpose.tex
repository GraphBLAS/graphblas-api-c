\subsection{{\sf apply}: Apply a unary function to the elements of an object}

%-----------------------------------------------------------------------------

\subsubsection{{\sf apply}: Vector variant}

\paragraph{\syntax}

\begin{verbatim}
        GrB_Info GrB_apply(GrB_Vector             w,
                           const GrB_Vector       mask,
                           const GrB_BinaryOp     accum,
                           const GrB_UnaryOp      op,
                           const GrB_Vector       u,
                           const GrB_Descriptor   desc);
\end{verbatim}

\paragraph{Parameters}

\begin{itemize}[leftmargin=1.1in]
    \item[{\sf w}]    ({\sf INOUT}) An existing GraphBLAS vector.  On input,
    the vector provides values that may be accumulated with the result of the
    apply operation.  On output, this vector holds the results of the
    operation.

    \item[{\sf mask}] ({\sf IN}) An optional ``write'' mask that controls which
    results from this operation are stored into the output vector {\sf w}. The 
    mask dimensions must match those of the vector {\sf w} and the domain of the
    {\sf mask} vector must be of type {\sf bool} or any of the predefined 
    ``built-in'' types in Table~\ref{Tab:PredefinedTypes}.  If the default
    vector is desired (\ie, with correct dimensions filled with {\sf true}), 
    {\sf GrB\_NULL} should be specified.

    \item[{\sf accum}] ({\sf IN}) An optional binary operator used for accumulating
    entries into existing {\sf w} entries. If assignment rather than accumulation is
    desired, {\sf GrB\_NULL} should be specified.

    \item[{\sf op}]	({\sf IN}) A unary operator applied to 
    each element of input vector {\sf u}.
        
    \item[{\sf u}] ({\sf IN}) The GraphBLAS vector to which the unary function
    is applied.

    \item[{\sf desc}] ({\sf IN}) An optional operation descriptor. If
    a \emph{default} descriptor is desired, {\sf GrB\_NULL} should be
    specified. Non-default field/value pairs are listed as follows:  \\

    \begin{tabular}{lllp{2.5in}}
        Param & Field  & Value & Description \\
        \hline
        {\sf w}    & {\sf GrB\_OUTP} & {\sf GrB\_REPLACE} & Output vector {\sf w}
        is cleared (all elements removed) before the result is stored in it.\\

        {\sf mask} & {\sf GrB\_MASK} & {\sf GrB\_SCMP}   & Use the structural
        complement of {\sf mask}. \\
    \end{tabular}
\end{itemize}

\paragraph{Return Values}

\begin{itemize}[leftmargin=2.1in]
    \item[{\sf GrB\_SUCCESS}]         In blocking mode, the operation completed
    successfully. In non-blocking mode, this indicates that the compatibility 
    tests on dimensions and domains for the input arguments passed successfully. 
    Either way, output vector, {\sf w}, is ready to be used in the next method of 
    the sequence.

    \item[{\sf GrB\_PANIC}]           Unknown internal error.

    \item[{\sf GrB\_INVALID\_OBJECT}] This is returned in any execution mode 
    whenever one of the opaque GraphBLAS objects (input or output) is in an invalid 
    state caused by a previous execution error.  Call {\sf GrB\_error()} to access 
    any error messages generated by the implementation.

    \item[{\sf GrB\_OUT\_OF\_MEMORY}] Not enough memory available for operation.

    \item[{\sf GrB\_UNINITIALIZED\_OBJECT}] One or more of the GraphBLAS objects
    has not been initialized by a call to {\sf new} (or {\sf dup} for vector
    parameters).

    \item[{\sf GrB\_DIMENSION\_MISMATCH}]  {\sf mask}, {\sf w} and/or {\sf u}
    dimensions are incompatible.

    \item[{\sf GrB\_DOMAIN\_MISMATCH}]    The domains of the various vectors are
    incompatible with the corresponding domains of the accumulating operation, 
    mask, or unary function.
\end{itemize}

\paragraph{Description}

This variant of {\sf GrB\_apply} computes the result of applying a unary function
to the elements of a GraphBLAS vector:
${\sf w} = f({\sf u})$; or, if an optional binary accumulation 
operator ($\odot$) is provided, ${\sf w} = {\sf w} \odot f({\sf u})$.  

Logically, this operation occurs in three steps:
\begin{enumerate}[leftmargin=0.75in]
\item[\bf Setup] The internal vectors and mask used in the computation are formed 
and their domains and dimensions are tested for compatibility.
\item[\bf Compute] The indicated computations are carried out.
\item[\bf Output] The result is written into the output vector, possibly under 
control of a mask.
\end{enumerate}

Up to three argument vectors are used in this {\sf GrB\_apply} operation:
\begin{enumerate}
	\item ${\sf w} = \langle \bold{D}({\sf w}),\bold{size}({\sf w}),
    \bold{L}({\sf w}) = \{(i,w_i) \} \rangle$

	\item ${\sf mask} = \langle \bold{D}({\sf mask}),\bold{size}({\sf mask}),
    \bold{L}({\sf mask}) = \{(i,m_i) \} \rangle$ (optional)

	\item ${\sf u} = \langle \bold{D}({\sf u}),\bold{size}({\sf u}),
    \bold{L}({\sf u}) = \{(i,u_i) \} \rangle$
\end{enumerate}

The argument vectors and the accumulation 
operator (if provided) are tested for domain compatibility as follows:
\begin{enumerate}
	\item The domain of {\sf mask} (if not {\sf GrB\_NULL}) must be from one of 
    the pre-defined types of Table~\ref{Tab:PredefinedTypes}.

	\item If {\sf accum} is {\sf GrB\_NULL}, then $\bold{D}({\sf w})$ must be 
    compatible with $\bDout({\sf op})$ of the unary operator.

	\item If {\sf accum} is not {\sf GrB\_NULL}, then $\bold{D}({\sf w})$ must be
    compatible with $\bDin1({\sf accum})$ and $\bDout({\sf accum})$ of the accumulation operator and 
    $\bDout({\sf op})$ of the unary operator must be compatible with $\bDin2({\sf accum})$ of the accumulation operator.

	\item $\bold{D}({\sf u})$ must be compatible with $\bDinn({\sf op})$.
\end{enumerate}
Two domains are compatible with each other if values from one domain can be cast 
to values in the other domain as per the rules of the C language.
In particular, domains from Table~\ref{Tab:PredefinedTypes} are all compatible 
with each other. A domain from a user-defined type is only compatible with itself.
If any compatibility rule above is violated, execution of {\sf GrB\_apply} ends
and the domain mismatch error listed above is returned.

From the argument vectors, the internal vectors and mask used in 
the computation are formed ($\leftarrow$ denotes copy):
\begin{enumerate}
	\item Vector $\vector{\widetilde{w}} \leftarrow {\sf w}$.

	\item One-dimensional mask, $\vector{\widetilde{m}}$, is computed from 
    argument {\sf mask} as follows:
	\begin{enumerate}
		\item	If ${\sf mask} = {\sf GrB\_NULL}$, then $\vector{\widetilde{m}} = 
        \langle \bold{size}({\sf w}), \{i, \forall i : 0 \leq i < 
        \bold{size}({\sf w}) \} \rangle$.

		\item	Otherwise, $\vector{\widetilde{m}} = 
        \langle \bold{size}({\sf mask}), \{i : ({\sf bool}){\sf mask}(i) = 
        \true\} \rangle$.

		\item	If ${\sf desc[GrB\_MASK].GrB\_SCMP}$ is set, then 
        $\vector{\widetilde{m}} \leftarrow \neg \vector{\widetilde{m}}$.
	\end{enumerate}

	\item Vector $\vector{\widetilde{u}} \leftarrow {\sf u}$.
\end{enumerate}

The internal vectors and masks are checked for dimension compatibility. 
The following conditions must hold:
\begin{enumerate}
	\item $\bold{size}(\vector{\widetilde{w}}) = \bold{size}(\vector{\widetilde{m}})$
    \item $\bold{size}(\vector{\widetilde{u}}) = \bold{size}(\vector{\widetilde{w}})$.
\end{enumerate}
If any compatibility rule above is violated, execution of {\sf GrB\_apply} ends and 
the dimension mismatch error listed above is returned.

From this point forward, in {\sf GrB\_NONBLOCKING} mode, the method can 
optionally exit with {\sf GrB\_SUCCESS} return code and defer any computation 
and/or execution error codes.

We are now ready to carry out the apply and any additional 
associated operations.  We describe this in terms of two intermediate vectors:
\begin{itemize}
    \item $\vector{\widetilde{t}}$: The vector holding the result from applying the unary operator to the input vector
    $\vector{\widetilde{u}}$.
    \item $\vector{\widetilde{z}}$: The vector holding the result after 
    application of the (optional) accumulation operator.
\end{itemize}

The intermediate vector, $\vector{\widetilde{t}}$, is created as follows:
\[
\vector{\widetilde{t}} = \langle
\bDout({\sf op}), \bold{size}(\vector{\widetilde{u}}),
\bold{L}(\vector{\widetilde{t}}) =
\{(i,f(\vector{\widetilde{u}}(i))) \forall i \in \bold{ind}(\vector{\widetilde{u}}) \} \rangle, 
\]
where $f = \mathbf{f}({\sf op})$.


The intermediate vector $\vector{\widetilde{z}}$ is created as follows, using what is called a \emph{standard vector accumulate}:
\begin{itemize}
    \item If ${\sf accum} = {\sf GrB\_NULL}$, then $\vector{\widetilde{z}} = \vector{\widetilde{t}}$.

    \item If ${\sf accum}$ is a binary operator, then $\vector{\widetilde{z}}$ is defined as
        \[ \langle \bDout({\sf accum}), \bold{size}(\vector{\widetilde{w}}),
        %\bold{L}(\vector{\widetilde{z}}) =
        \{(i,z_{i}) \ \forall \ i \in \bold{ind}(\vector{\widetilde{w}}) \cup 
        \bold{ind}(\vector{\widetilde{t}}) \} \rangle.\]

    The values of the elements of $\vector{\widetilde{z}}$ are computed based on the 
    relationships between the sets of indices in $\vector{\widetilde{w}}$ and 
    $\vector{\widetilde{t}}$.
\[
    z_{i} = \vector{\widetilde{w}}(i) \odot \vector{\widetilde{t}}(i), \ \mbox{if}\  
    i \in  (\bold{ind}(\vector{\widetilde{t}}) \cap \bold{ind}(\vector{\widetilde{w}})),
\]
\[
    z_{i} = \vector{\widetilde{w}}(i), \ \mbox{if}\  
    i \in (\bold{ind}(\vector{\widetilde{w}}) - (\bold{ind}(\vector{\widetilde{t}})
    \cap \bold{ind}(\vector{\widetilde{w}}))),
\]
\[
    z_{i} = \vector{\widetilde{t}}(i), \ \mbox{if}\  i \in  
    (\bold{ind}(\vector{\widetilde{t}}) - (\bold{ind}(\vector{\widetilde{t}})
    \cap \bold{ind}(\vector{\widetilde{w}}))),
\]
where $\odot  = \bigodot({\sf accum})$, and the difference operator refers to set difference.
\end{itemize}




Finally, the set of output values that make up vector $\vector{\widetilde{z}}$ 
are written into the final result vector {\sf w}, using
what is called a \emph{standard vector mask and replace}. 
This is carried out under control of the mask which acts as a ``write mask''.
\begin{itemize}
\item If {\sf desc[GrB\_OUTP].GrB\_REPLACE} is set, then any values in {\sf w} 
on input to this operation are deleted and the content of the new output vector,
{\sf w}, is defined as,
\[ 
\bold{L}({\sf w}) = \{(i,z_{i}) : i \in (\bold{ind}(\vector{\widetilde{z}}) 
\cap \bold{ind}(\vector{\widetilde{m}})) \}. 
\]

\item If {\sf desc[GrB\_OUTP].GrB\_REPLACE} is not set, the elements of 
$\vector{\widetilde{z}}$ indicated by the mask are copied into the result 
vector, {\sf w}, and elements of {\sf w} that fall outside the set indicated by 
the mask are unchanged:
\[ 
\bold{L}({\sf w}) = \{(i,w_{i}) : i \in (\bold{ind}({\sf w}) 
\cap \bold{ind}(\neg \vector{\widetilde{m}})) \} \cup \{(i,z_{i}) : i \in 
(\bold{ind}(\vector{\widetilde{z}}) \cap \bold{ind}(\vector{\widetilde{m}})) \}. 
\]
\end{itemize}

In {\sf GrB\_BLOCKING} mode, the method exits with return value 
{\sf GrB\_SUCCESS} and the new content of vector {\sf w} is as defined above
and fully computed.  
In {\sf GrB\_NONBLOCKING} mode, the method exits with return value 
{\sf GrB\_SUCCESS} and the new content of vector {\sf w} is as defined above 
but may not be fully computed. However, it can be used in the next GraphBLAS 
method call in a sequence.



%--------------------------------------------------------------

\subsubsection{{\sf apply}: Matrix variant}

\paragraph{\syntax}

\begin{verbatim}
        GrB_Info GrB_apply(GrB_Matrix            C,
                           const GrB_Matrix      Mask,
                           const GrB_BinaryOp    accum,
                           const GrB_UnaryOp     op,
                           const GrB_Matrix      A,
                           const GrB_Descriptor  desc);
\end{verbatim}

\paragraph{Parameters}

\begin{itemize}[leftmargin=1.1in]
    \item[{\sf C}]    ({\sf INOUT}) An existing GraphBLAS matrix. On input,
    the matrix provides values that may be accumulated with the result of the
    apply operation.  On output, the matrix holds the results of the
    operation.

    \item[{\sf Mask}] ({\sf IN}) An optional ``write'' mask that controls which
    results from this operation are stored into the output matrix {\sf C}. The 
    mask dimensions must match those of the matrix {\sf C} and the domain of the 
    {\sf Mask} matrix must be of type {\sf bool} or any of the predefined 
    ``built-in'' types in Table~\ref{Tab:PredefinedTypes}.  If the default
    matrix is desired (\ie, with correct dimensions filled with {\sf true}), 
    {\sf GrB\_NULL} should be specified.

    \item[{\sf accum}] ({\sf IN}) An optional binary operator used for accumulating
    entries into existing {\sf C} entries.
    If assignment rather than accumulation is
    desired, {\sf GrB\_NULL} should be specified.

    \item[{\sf op}] ({\sf IN}) A unary operator applied to each
	    element of input matrix {\sf A}.

    \item[{\sf A}]     ({\sf IN})  The GraphBLAS matrix to which the unary function 
    is applied.

    \item[{\sf desc}] ({\sf IN}) An optional operation descriptor. If
    a \emph{default} descriptor is desired, {\sf GrB\_NULL} should be
    specified. Non-default field/value pairs are listed as follows:  \\

    \begin{tabular}{lllp{2.5in}}
        Param & Field  & Value & Description \\
        \hline
        {\sf C}    & {\sf GrB\_OUTP} & {\sf GrB\_REPLACE} & Output matrix {\sf C}
        is cleared (all elements removed) before the result is stored in it.\\

        {\sf Mask} & {\sf GrB\_MASK} & {\sf GrB\_SCMP}   & Use the structural
        complement of {\sf Mask}. \\

        {\sf A}    & {\sf GrB\_INP0} & {\sf GrB\_TRAN}   & Use transpose of {\sf A}
        for the operation. \\
    \end{tabular}
\end{itemize}

\paragraph{Return Values}

\begin{itemize}[leftmargin=2.1in]
    \item[{\sf GrB\_SUCCESS}]         In blocking mode, the operation completed
    successfully. In non-blocking mode, this indicates that the compatibility 
    tests on dimensions and domains for the input arguments passed successfully. 
    Either way, output matrix, {\sf C}, is ready to be used in the next method of
    the sequence.

    \item[{\sf GrB\_PANIC}]           Unknown internal error.

    \item[{\sf GrB\_INVALID\_OBJECT}] This is returned in any execution mode 
    whenever one of the opaque GraphBLAS objects (input or output) is in an invalid 
    state caused by a previous execution error.  Call {\sf GrB\_error()} to access 
    any error messages generated by the implementation.

    \item[{\sf GrB\_OUT\_OF\_MEMORY}] Not enough memory available for the operation.

    \item[{\sf GrB\_UNINITIALIZED\_OBJECT}] One or more of the GraphBLAS objects 
    has not been initialized by a call to {\sf new} (or {\sf Matrix\_dup} for matrix
    parameters).

    \item[{\sf GrB\_INDEX\_OUT\_OF\_BOUNDS}]  A value in {\sf row\_indices} 
    is greater than or equal to $\bold{nrows}({\sf A})$, or a value in 
    {\sf col\_indices} is greater than or equal to $\bold{ncols}({\sf A})$.  In 
    non-blocking mode, this can be reported as an execution error.
    
    \item[{\sf GrB\_DIMENSION\_MISMATCH}] {\sf Mask} and {\sf C}
    dimensions are incompatible, ${\sf nrows} \neq \bold{nrows}({\sf C})$, or 
    ${\sf ncols} \neq \bold{ncols}({\sf C})$.

    \item[{\sf GrB\_DOMAIN\_MISMATCH}]    The domains of the various matrices are
    incompatible with the corresponding domains of the 
    accumulation operator, or the mask's domain is not compatible with {\sf bool}.
\end{itemize}

\paragraph{Description}

This variant of {\sf GrB\_apply} computes the result of applying a unary function
to the elements of a 
GraphBLAS matrix: ${\sf C} = f({\sf A})$; or, if an 
optional binary accumulation operator ($\odot$) is provided, 
${\sf C} = {\sf C} \odot f({\sf A})$.  

Logically, this operation occurs in three steps:
\begin{enumerate}[leftmargin=0.85in]
\item[\bf Setup] The internal matrices and mask used in the computation are formed 
and their domains and dimensions are tested for compatibility.
\item[\bf Compute] The indicated computations are carried out.
\item[\bf Output] The result is written into the output matrix, possibly under 
control of a mask.
\end{enumerate}

Up to three argument matrices are used in the {\sf GrB\_apply} operation:
\begin{enumerate}
	\item ${\sf C} = \langle \bold{D}({\sf C}),\bold{nrows}({\sf C}),
    \bold{ncols}({\sf C}),\bold{L}({\sf C}) = \{(i,j,C_{ij}) \} \rangle$

	\item ${\sf Mask} = \langle \bold{D}({\sf Mask}),\bold{nrows}({\sf Mask}),
    \bold{ncols}({\sf Mask}),\bold{L}({\sf Mask}) = \{(i,j,M_{ij}) \} \rangle$ (optional)

	\item ${\sf A} = \langle \bold{D}({\sf A}),\bold{nrows}({\sf A}),
    \bold{ncols}({\sf A}),\bold{L}({\sf A}) = \{(i,j,A_{ij}) \} \rangle$
\end{enumerate}

The argument matrices and the accumulation 
operator (if provided) are tested for domain compatibility as follows:
\begin{enumerate}
	\item The domain of {\sf Mask} (if not {\sf GrB\_NULL}) must be from one of
    the pre-defined types of Table~\ref{Tab:PredefinedTypes}.

	\item If {\sf accum} is {\sf GrB\_NULL}, then $\bold{D}({\sf C})$ must be
    compatible with $\bDout({\sf op})$ of the unary operator.

	\item If {\sf accum} is not {\sf GrB\_NULL}, then $\bold{D}({\sf C})$ must be
    compatible with $\bDin1({\sf accum})$ and $\bDout({\sf accum})$ of the accumulation operator and 
    $\bDout({\sf op})$ of the unary operator must be compatible with $\bDin2({\sf accum})$ of the accumulation operator.

	\item $\bold{D}({\sf A})$ must be compatible with $\bDinn({\sf op})$ of the unary operator.
\end{enumerate}
Two domains are compatible with each other if values from one domain can be cast 
to values in the other domain as per the rules of the C language.
In particular, domains from Table~\ref{Tab:PredefinedTypes} are all compatible 
with each other. A domain from a user-defined type is only compatible with itself.
If any compatibility rule above is violated, execution of {\sf GrB\_apply} ends and 
the domain mismatch error listed above is returned.

From the argument matrices, the internal matrices, mask, and index arrays used in 
the computation are formed ($\leftarrow$ denotes copy):
\begin{enumerate}
	\item Matrix $\matrix{\widetilde{C}} \leftarrow {\sf C}$.

	\item Two-dimensional mask, $\matrix{\widetilde{M}}$, is computed from 
    argument {\sf Mask} as follows:
	\begin{enumerate}
		\item	If ${\sf Mask} = {\sf GrB\_NULL}$, then $\matrix{\widetilde{M}} = 
        \langle \bold{nrows}({\sf C}), \bold{ncols}({\sf C}), \{(i,j), 
        \forall i,j : 0 \leq i <  \bold{nrows}({\sf C}), 0 \leq j < 
        \bold{ncols}({\sf C}) \} \rangle$.

		\item	Otherwise, $\matrix{\widetilde{M}} = \langle \bold{nrows}({\sf Mask}), 
        \bold{ncols}({\sf Mask}), \{(i,j) : 
        ({\sf bool}){\sf Mask}(i,j) = \true\} \rangle$.

		\item	If ${\sf desc[GrB\_MASK].GrB\_SCMP}$ is set, then 
        $\matrix{\widetilde{M}} \leftarrow \neg \matrix{\widetilde{M}}$.
	\end{enumerate}

	\item Matrix $\matrix{\widetilde{A}} \leftarrow
    {\sf desc[GrB\_INP0].GrB\_TRAN} \ ? \ {\sf A}^T : {\sf A}$.
    
\end{enumerate}

The internal matrices and mask are checked for dimension compatibility. 
The following conditions must hold:
\begin{enumerate}
    \item $\bold{nrows}(\matrix{\widetilde{C}}) = \bold{nrows}(\matrix{\widetilde{M}})$.

    \item $\bold{ncols}(\matrix{\widetilde{C}}) = \bold{ncols}(\matrix{\widetilde{M}})$.

    \item $\bold{nrows}(\matrix{\widetilde{C}}) = \bold{nrows}(\matrix{\widetilde{A}})$.

    \item $\bold{ncols}(\matrix{\widetilde{C}}) = \bold{ncols}(\matrix{\widetilde{A}})$.
\end{enumerate}
If any compatibility rule above is violated, execution of {\sf GrB\_apply} ends and 
the dimension mismatch error listed above is returned.

From this point forward, in {\sf GrB\_NONBLOCKING} mode, the method can optionally exit
with {\sf GrB\_SUCCESS} return code and defer any computation and/or execution error codes.

We are now ready to carry out the apply and any additional 
associated operations.  We describe this in terms of two intermediate matrices:
\begin{itemize}
    \item $\matrix{\widetilde{T}}$: The matrix holding the result from applying the unary operator to the input matrix
    $\matrix{\widetilde{A}}$.

    \item $\matrix{\widetilde{Z}}$: The matrix holding the result after 
    application of the (optional) accumulation operator.
\end{itemize}

The intermediate matrix, $\matrix{\widetilde{T}}$, is created as follows:
\[
\begin{aligned}
\matrix{\widetilde{T}} = & \langle \bDout({\sf op}),
                           \bold{nrows}(\matrix{\widetilde{C}}), 
                           \bold{ncols}(\matrix{\widetilde{C}}),  \bold{L}(\matrix{\widetilde{T}}) =
	\{(i,j,f(\matrix{\widetilde{A}}(i,j))) 
\ \forall \ (i,j) \in 
\bold{ind}(\matrix{\widetilde{A}}) \} \rangle,
\end{aligned}
\]
where $f = \mathbf{f}({\sf op})$.

%\emph{Standard Matrix Accumulate Option}

The intermediate matrix $\matrix{\widetilde{Z}}$ is created as follows:
\begin{itemize}
    \item If ${\sf accum} = {\sf GrB\_NULL}$, then $\matrix{\widetilde{Z}} = \matrix{\widetilde{T}}$.

    \item If ${\sf accum}$ is a binary operator, then $\matrix{\widetilde{Z}}$ is defined as
        \[ \langle \bDout({\sf accum}), \bold{nrows}(\matrix{\widetilde{C}}), \bold{ncols}(\matrix{\widetilde{C}}),
        %\bold{L}(\matrix{\widetilde{Z}}) =
        \{(i,j,Z_{ij})  \forall (i,j) \in \bold{ind}(\matrix{\widetilde{C}}) \cup 
        \bold{ind}(\matrix{\widetilde{T}}) \} \rangle.\]

    The values of the elements of $\matrix{\widetilde{Z}}$ are computed based on the
    relationships between the sets of indices in $\matrix{\widetilde{C}}$ and 
    $\matrix{\widetilde{T}}$.
\[
    Z_{ij} = \matrix{\widetilde{C}}(i,j) \odot \matrix{\widetilde{T}}(i,j), \ \mbox{if}\  
    (i,j) \in  (\bold{ind}(\matrix{\widetilde{T}}) \cap \bold{ind}(\matrix{\widetilde{C}})),
\]
\[
    Z_{ij} = \matrix{\widetilde{C}}(i,j), \ \mbox{if}\  
    (i,j) \in (\bold{ind}(\matrix{\widetilde{C}}) - (\bold{ind}(\matrix{\widetilde{T}})
    \cap \bold{ind}(\matrix{\widetilde{C}}))),
\]
\[
    Z_{ij} = \matrix{\widetilde{T}}(i,j), \ \mbox{if}\  (i,j) \in  
    (\bold{ind}(\matrix{\widetilde{T}}) - (\bold{ind}(\matrix{\widetilde{T}})
    \cap \bold{ind}(\matrix{\widetilde{C}}))),
\]
where $\odot  = \bigodot({\sf accum})$, and the difference operator refers to set difference.
\end{itemize}



Finally, the set of output values that make up matrix $\matrix{\widetilde{Z}}$ 
are written into the final result matrix {\sf C}, using what is called
a \emph{standard matrix mask and replace}. 
This is carried out under control of the mask which acts as a ``write mask''.
\begin{itemize}
\item If {\sf desc[GrB\_OUTP].GrB\_REPLACE} is set, then any values in {\sf C} on
input to this operation are deleted and the contents of the new output matrix,
{\sf C}, is defined as,
\[
\bold{L}({\sf C}) = \{(i,j,Z_{ij}) : (i,j) \in (\bold{ind}(\matrix{\widetilde{Z}}) 
\cap \bold{ind}(\matrix{\widetilde{M}})) \}. 
\]

\item If {\sf desc[GrB\_OUTP].GrB\_REPLACE} is not set, the elements of 
$\matrix{\widetilde{Z}}$ indicated by the mask are copied into the result 
matrix, {\sf C}, and elements of {\sf C} that fall outside the set 
indicated by the mask are unchanged:
\[
\bold{L}({\sf C}) = \{(i,j,C_{ij}) : (i,j) \in (\bold{ind}({\sf C}) 
\cap \bold{ind}(\neg \matrix{\widetilde{M}})) \} \cup \{(i,j,Z_{ij}) : (i,j) \in 
(\bold{ind}(\matrix{\widetilde{Z}}) \cap \bold{ind}(\matrix{\widetilde{M}})) \}.
\]
\end{itemize}

In {\sf GrB\_BLOCKING} mode, the method exits with return value 
{\sf GrB\_SUCCESS} and the new content of matrix {\sf C} is as defined above
and fully computed.
In {\sf GrB\_NONBLOCKING} mode, the method exits with return value 
{\sf GrB\_SUCCESS} and the new content of matrix {\sf C} is as defined above
but may not be fully computed; however, it can be used in the next GraphBLAS 
method call in a sequence.



%=========================================================================

\subsection{{\sf reduce}: Perform a reduction across the elements of an object}

Computes the reduction of the values of the elements of a vector or matrix.

%-----------------------------------------------------------------------------

\subsubsection{{\sf reduce}: Standard matrix to vector variant}

This performs a reduction across rows of a matrix to produce a vector.  If column 
reduction across columns is desired, the input matrix should be transposed which can 
be specified using the descriptor.

\paragraph{\syntax}

\begin{verbatim}
        GrB_Info GrB_reduce(GrB_Vector             w,
                            const GrB_Vector       mask,
                            const GrB_BinaryOp     accum,
                            const GrB_Monoid       op,  
                            const GrB_Matrix       A,
                            const GrB_Descriptor   desc);
                            
        GrB_Info GrB_reduce(GrB_Vector             w,
                            const GrB_Vector       mask,
                            const GrB_BinaryOp     accum,
                            const GrB_BinaryOp     op,  
                            const GrB_Matrix       A,
                            const GrB_Descriptor   desc);
\end{verbatim}

\paragraph{Parameters}

\begin{itemize}[leftmargin=1.1in]
    \item[{\sf w}]    ({\sf INOUT}) An existing GraphBLAS vector.  On input,
    the vector provides values that may be accumulated with the result of the
    reduction operation.  On output, this vector holds the results of the
    operation.

    \item[{\sf mask}] ({\sf IN}) An optional ``write'' mask that controls which
    results from this operation are stored into the output vector {\sf w}. The 
    mask dimensions must match those of the vector {\sf w} and the domain of the
    {\sf mask} vector must be of type {\sf bool} or any of the predefined 
    ``built-in'' types in Table~\ref{Tab:PredefinedTypes}.  If the default
    vector is desired (\ie, with correct dimensions filled with {\sf true}), 
    {\sf GrB\_NULL} should be specified.

    \item[{\sf accum}] ({\sf IN}) An optional binary operator used for accumulating
    entries into existing {\sf w} entries. If assignment rather than accumulation is
    desired, {\sf GrB\_NULL} should be specified.

    \item[{\sf op}]    ({\sf IN}) The monoid or binary operator 
    used in the element-wise reduction operation.  Depending on which type is
    passed, the following defines the binary operator with one domain, $F_b=\langle D,D,D,\oplus \rangle$, that is used:
    \begin{itemize}[leftmargin=1.1in]
    \item[BinaryOp:] $F_b = \langle \bDout({\sf op}), \bDin1({\sf op}),
    \bDin2({\sf op}),\bold{\bigodot}({\sf op})\rangle$.  
    \item[Monoid:] $F_b = \langle \bold{D}({\sf op}), \bold{D}({\sf op}),
    \bold{D}({\sf op}),\bold{\bigodot}({\sf op})\rangle$,
    the identity element of the monoid is ignored. 
    \end{itemize}
    If {\sf op} is a {\sf GrB\_BinaryOp}, then all its domains must be the same.
    Furthermore, in both cases $\bold{\bigodot}({\sf op})$ must be commutative and associative. Otherwise, the outcome of the operation is undefined.
    
    \item[{\sf A}]     ({\sf IN}) The GraphBLAS matrix on which
	    reduction will be performed.

    \item[{\sf desc}] ({\sf IN}) An optional operation descriptor. If
    a \emph{default} descriptor is desired, {\sf GrB\_NULL} should be
    specified. Non-default field/value pairs are listed as follows:  \\

    \begin{tabular}{lllp{2.5in}}
        Param & Field  & Value & Description \\
        \hline
        {\sf w}    & {\sf GrB\_OUTP} & {\sf GrB\_REPLACE} & Output vector {\sf w}
        is cleared (all elements removed) before the result is stored in it.\\

        {\sf mask} & {\sf GrB\_MASK} & {\sf GrB\_SCMP}   & Use the structural
        complement of {\sf mask}. \\

        {\sf A}    & {\sf GrB\_INP0} & {\sf GrB\_TRAN}   & Use transpose of {\sf A}
        for the operation. \\
    \end{tabular}
\end{itemize}

\paragraph{Return Values}

\begin{itemize}[leftmargin=2.1in]
    \item[{\sf GrB\_SUCCESS}]         In blocking mode, the operation completed
    successfully. In non-blocking mode, this indicates that the compatibility 
    tests on dimensions and domains for the input arguments passed successfully. 
    Either way, output vector, {\sf w}, is ready to be used in the next method of 
    the sequence.

    \item[{\sf GrB\_PANIC}]           Unknown internal error.

    \item[{\sf GrB\_INVALID\_OBJECT}] This is returned in any execution mode 
    whenever one of the opaque GraphBLAS objects (input or output) is in an invalid 
    state caused by a previous execution error.  Call {\sf GrB\_error()} to access 
    any error messages generated by the implementation.

    \item[{\sf GrB\_OUT\_OF\_MEMORY}] Not enough memory available for the operation.

    \item[{\sf GrB\_UNINITIALIZED\_OBJECT}] One or more of the GraphBLAS objects
    has not been initialized by a call to {\sf new} (or {\sf dup} for vector
    parameters).
    
    \item[{\sf GrB\_DIMENSION\_MISMATCH}]  {\sf mask}, {\sf w} and/or {\sf u} dimensions are
    incompatible. 

    \item[{\sf GrB\_DOMAIN\_MISMATCH}]    Either the domains of the various vectors and matrices are
    incompatible with the corresponding domains of the accumulating operation, 
    mask, and reduce function, or the domains of the GraphBLAS binary operator {\sf op} are not all the same.
\end{itemize}

\paragraph{Description}

This variant of {\sf GrB\_reduce} computes the result of performing a reduction 
across each of the rows of an input matrix: ${\sf w}(i) = \bigoplus {\sf A}(i,:) \forall i$; 
or, if an optional binary accumulation operator is provided, 
${\sf w}(i) = {\sf w}(i) \odot (\bigoplus {\sf A}(i,:)) \forall i$, 
where $\bigoplus = \bigodot({F_b})$ and $\odot = \bigodot({\sf accum})$.

Logically, this operation occurs in three steps:
\begin{enumerate}[leftmargin=0.75in]
\item[\bf Setup] The internal vector, matrix and mask used in the computation are formed 
and their domains and dimensions are tested for compatibility.
\item[\bf Compute] The indicated computations are carried out.
\item[\bf Output] The result is written into the output vector, possibly under 
control of a mask.
\end{enumerate}

Up to two vector and one matrix argument are used in this {\sf GrB\_reduce} operation:
\begin{enumerate}
	\item ${\sf w} = \langle \bold{D}({\sf w}),\bold{size}({\sf w}),
    \bold{L}({\sf w}) = \{(i,w_i) \} \rangle$

	\item ${\sf mask} = \langle \bold{D}({\sf mask}),\bold{size}({\sf mask}),
    \bold{L}({\sf mask}) = \{(i,m_i) \} \rangle$ (optional)

	\item ${\sf A} = \langle \bold{D}({\sf A}),\bold{nrows}({\sf A}),
    \bold{ncols}({\sf A}),\bold{L}({\sf A}) = \{(i,j,A_{ij}) \} \rangle$
\end{enumerate}

The argument vector, matrix, reduction operator and accumulation 
operator (if provided) are tested for domain compatibility as follows:
\begin{enumerate}
	\item The domain of {\sf mask} (if not {\sf GrB\_NULL}) must be from one of 
    the pre-defined types of Table~\ref{Tab:PredefinedTypes}.

	\item If {\sf accum} is {\sf GrB\_NULL}, then $\bold{D}({\sf w})$ must be 
    compatible with the domain of the reduction binary operator, $\bold{D}(F_b)$.

	\item If {\sf accum} is not {\sf GrB\_NULL}, then $\bold{D}({\sf w})$ must be
    compatible with $\bDin1({\sf accum})$ and $\bDout({\sf accum})$ of the accumulation operator and 
    $\bold{D}(F_b)$, must be compatible with $\bDin2({\sf accum})$ of the accumulation operator.

	\item $\bold{D}({\sf A})$ must be compatible with the domain of the binary reduction operator, $\bold{D}(F_b)$.
\end{enumerate}
Two domains are compatible with each other if values from one domain can be cast 
to values in the other domain as per the rules of the C language.
In particular, domains from Table~\ref{Tab:PredefinedTypes} are all compatible 
with each other. A domain from a user-defined type is only compatible with itself.
If any compatibility rule above is violated, execution of {\sf GrB\_reduce} ends
and the domain mismatch error listed above is returned.

From the argument vectors, the internal vectors and mask used in 
the computation are formed ($\leftarrow$ denotes copy):
\begin{enumerate}
	\item Vector $\vector{\widetilde{w}} \leftarrow {\sf w}$.

	\item One-dimensional mask, $\vector{\widetilde{m}}$, is computed from 
    argument {\sf mask} as follows:
	\begin{enumerate}
		\item	If ${\sf mask} = {\sf GrB\_NULL}$, then $\vector{\widetilde{m}} = 
        \langle \bold{size}({\sf w}), \{i, \forall i : 0 \leq i < 
        \bold{size}({\sf w}) \} \rangle$.

		\item	Otherwise, $\vector{\widetilde{m}} = 
        \langle \bold{size}({\sf mask}), \{i : ({\sf bool}){\sf mask}(i) = 
        \true\} \rangle$.

		\item	If ${\sf desc[GrB\_MASK].GrB\_SCMP}$ is set, then 
        $\vector{\widetilde{m}} \leftarrow \neg \vector{\widetilde{m}}$.
	\end{enumerate}

	\item Matrix $\matrix{\widetilde{A}} \leftarrow 
    {\sf desc[GrB\_INP0].GrB\_TRAN} \ ? \ {\sf A}^T : {\sf A}$.
\end{enumerate}

The internal vectors and masks are checked for dimension compatibility. 
The following conditions must hold:
\begin{enumerate}
	\item $\bold{size}(\vector{\widetilde{w}}) = \bold{size}(\vector{\widetilde{m}})$
    \item $\bold{size}(\vector{\widetilde{w}}) = \bold{nrows}(\matrix{\widetilde{A}})$.
\end{enumerate}
If any compatibility rule above is violated, execution of {\sf GrB\_reduce} ends and 
the dimension mismatch error listed above is returned.

From this point forward, in {\sf GrB\_NONBLOCKING} mode, the method can 
optionally exit with {\sf GrB\_SUCCESS} return code and defer any computation 
and/or execution error codes.

We carry out the reduce and any additional 
associated operations.  We describe this in terms of two intermediate vectors:
\begin{itemize}
    \item $\vector{\widetilde{t}}$: The vector holding the result from reducing along the rows of input matrix
    $\matrix{\widetilde{A}}$.

    \item $\vector{\widetilde{z}}$: The vector holding the result after 
    application of the (optional) accumulation operator.
\end{itemize}

The intermediate vector, $\vector{\widetilde{t}}$, is created as follows:
\[
\vector{\widetilde{t}} = \langle
\bold{D}({\sf op}), \bold{size}(\vector{\widetilde{w}}),
\bold{L}(\vector{\widetilde{t}}) =
\{(i,t_i) : \bold{ind}(A(i,:)) \neq \emptyset   \} \rangle. 
\]
The value of each of its elements is computed by
\[
	t_i = \bigoplus_{j \in \bold{ind}(\matrix{\widetilde{A}}(i,:))} \matrix{\widetilde{A}}(i,j), 
\]
where $\bigoplus = \bigodot(F_b)$.


The intermediate vector $\vector{\widetilde{z}}$ is created as follows, using what is called a \emph{standard vector accumulate}:
\begin{itemize}
    \item If ${\sf accum} = {\sf GrB\_NULL}$, then $\vector{\widetilde{z}} = \vector{\widetilde{t}}$.

    \item If ${\sf accum}$ is a binary operator, then $\vector{\widetilde{z}}$ is defined as
        \[ \langle \bDout({\sf accum}), \bold{size}(\vector{\widetilde{w}}),
        %\bold{L}(\vector{\widetilde{z}}) =
        \{(i,z_{i}) \ \forall \ i \in \bold{ind}(\vector{\widetilde{w}}) \cup 
        \bold{ind}(\vector{\widetilde{t}}) \} \rangle.\]

    The values of the elements of $\vector{\widetilde{z}}$ are computed based on the 
    relationships between the sets of indices in $\vector{\widetilde{w}}$ and 
    $\vector{\widetilde{t}}$.
\[
    z_{i} = \vector{\widetilde{w}}(i) \odot \vector{\widetilde{t}}(i), \ \mbox{if}\  
    i \in  (\bold{ind}(\vector{\widetilde{t}}) \cap \bold{ind}(\vector{\widetilde{w}})),
\]
\[
    z_{i} = \vector{\widetilde{w}}(i), \ \mbox{if}\  
    i \in (\bold{ind}(\vector{\widetilde{w}}) - (\bold{ind}(\vector{\widetilde{t}})
    \cap \bold{ind}(\vector{\widetilde{w}}))),
\]
\[
    z_{i} = \vector{\widetilde{t}}(i), \ \mbox{if}\  i \in  
    (\bold{ind}(\vector{\widetilde{t}}) - (\bold{ind}(\vector{\widetilde{t}})
    \cap \bold{ind}(\vector{\widetilde{w}}))),
\]
where $\odot  = \bigodot({\sf accum})$, and the difference operator refers to set difference.
\end{itemize}




Finally, the set of output values that make up vector $\vector{\widetilde{z}}$ 
are written into the final result vector {\sf w}, using
what is called a \emph{standard vector mask and replace}. 
This is carried out under control of the mask which acts as a ``write mask''.
\begin{itemize}
\item If {\sf desc[GrB\_OUTP].GrB\_REPLACE} is set, then any values in {\sf w} 
on input to this operation are deleted and the content of the new output vector,
{\sf w}, is defined as,
\[ 
\bold{L}({\sf w}) = \{(i,z_{i}) : i \in (\bold{ind}(\vector{\widetilde{z}}) 
\cap \bold{ind}(\vector{\widetilde{m}})) \}. 
\]

\item If {\sf desc[GrB\_OUTP].GrB\_REPLACE} is not set, the elements of 
$\vector{\widetilde{z}}$ indicated by the mask are copied into the result 
vector, {\sf w}, and elements of {\sf w} that fall outside the set indicated by 
the mask are unchanged:
\[ 
\bold{L}({\sf w}) = \{(i,w_{i}) : i \in (\bold{ind}({\sf w}) 
\cap \bold{ind}(\neg \vector{\widetilde{m}})) \} \cup \{(i,z_{i}) : i \in 
(\bold{ind}(\vector{\widetilde{z}}) \cap \bold{ind}(\vector{\widetilde{m}})) \}. 
\]
\end{itemize}

In {\sf GrB\_BLOCKING} mode, the method exits with return value 
{\sf GrB\_SUCCESS} and the new content of vector {\sf w} is as defined above
and fully computed.  
In {\sf GrB\_NONBLOCKING} mode, the method exits with return value 
{\sf GrB\_SUCCESS} and the new content of vector {\sf w} is as defined above 
but may not be fully computed. However, it can be used in the next GraphBLAS 
method call in a sequence.



%-----------------------------------------------------------------------------

\subsubsection{{\sf reduce}: Vector-scalar variant}
\label{Sec:Reduce_vector_scalar}

Reduce all stored values into a single scalar.

\paragraph{\syntax}

\begin{verbatim}
        GrB_Info GrB_reduce(<type>               *val,
                            const GrB_BinaryOp    accum,
                            const GrB_Monoid      op,
                            const GrB_Vector      u,
                            const GrB_Descriptor  desc);
\end{verbatim}

\paragraph{Parameters}

\begin{itemize}[leftmargin=1.1in]
    \item[{\sf val}]    ({\sf INOUT}) Scalar to store final reduced value into. On input,
    the scalar provides a value that may be accumulated with the result of the
    reduction operation.  On output, this scalar holds the results of the
    operation.

    \item[{\sf accum}] ({\sf IN}) An optional binary operator used for accumulating
    entries into existing {\sf val} value. If assignment rather than accumulation is
    desired, {\sf GrB\_NULL} should be specified.

    \item[{\sf op}]    ({\sf IN}) The monoid used in the element-wise reduction
    operation, $M = \langle D,\oplus,0 \rangle$. The binary operator,
    $\oplus$, must be commutative and associative; otherwise,
    the outcome of the operation is undefined.
    
    \item[{\sf u}]     ({\sf IN}) The GraphBLAS vector on which
    reduction will be performed.

    \item[{\sf desc}] ({\sf IN}) An optional operation descriptor. If
    a \emph{default} descriptor is desired, {\sf GrB\_NULL} should be
    specified. Non-default field/value pairs are listed as follows:  \\

    \begin{tabular}{lllp{2.5in}}
        Param & Field  & Value & Description \\
        \hline
    \end{tabular}

    \emph{Note:} This argument is defined for consistency with the other GraphBLAS operations.
    There are currently no non-default field/value pairs that can be set for this operation.
\end{itemize}

\paragraph{Return Values}

\begin{itemize}[leftmargin=2.1in]
    \item[{\sf GrB\_SUCCESS}]         In blocking or non-blocking mode, 
    the operation completed successfully, and the
    output scalar {\sf val} is ready to be used in the next method of 
    the sequence.

    \item[{\sf GrB\_PANIC}]           Unknown internal error.

    \item[{\sf GrB\_INVALID\_OBJECT}] This is returned in any execution mode 
    whenever one of the opaque GraphBLAS objects (input or output) is in an invalid 
    state caused by a previous execution error.  Call {\sf GrB\_error()} to access 
    any error messages generated by the implementation.

    \item[{\sf GrB\_OUT\_OF\_MEMORY}] Not enough memory available for the operation.

    \item[{\sf GrB\_UNINITIALIZED\_OBJECT}] One or more of the GraphBLAS objects 
    has not been initialized by a call to {\sf new} (or {\sf Vector\_dup} for vector
    parameters).
    
    \item[{\sf GrB\_DOMAIN\_MISMATCH}]    The domains of input and output arguments are
    incompatible with the corresponding domains of the accumulation operator, 
    or reduce operator.

    \item[{\sf GrB\_NULL\_POINTER}]  {\sf val} pointer is {\sf NULL}.
\end{itemize}

\paragraph{Description}

This variant of {\sf GrB\_reduce} computes the result of performing
a reduction across each of the elements of an input vector:
${\sf val} = \bigoplus {\sf u}(:)$; or, if an optional binary accumulation 
operator is provided, ${\sf val} = {\sf val} \odot (\bigoplus {\sf u}(:))$, 
where $\bigoplus = \bigodot({\sf op})$ and $\odot = \bigodot({\sf accum})$.  

Logically, this operation occurs in three steps:
\begin{enumerate}[leftmargin=0.85in]
\item[\bf Setup] The internal vector used in the computation is formed 
and its domain is tested for compatibility.
\item[\bf Compute] The indicated computations are carried out.
\item[\bf Output] The result is written into the output scalar.
\end{enumerate}

One vector argument is used in this {\sf GrB\_reduce} operation:
\begin{enumerate}
	\item ${\sf u} = \langle \bold{D}({\sf u}),\bold{size}({\sf u}),
    \bold{L}({\sf u}) = \{(i,u_i) \} \rangle$

\end{enumerate}

The output scalar, argument vector, reduction operator and accumulation 
operator (if provided) are tested for domain compatibility as follows:
\begin{enumerate}

	\item If {\sf accum} is {\sf GrB\_NULL}, then $\bold{D}({\sf val})$ must be
    compatible with $\bold{D}({\sf op})$ of the reduction binary operator.

	\item If {\sf accum} is not {\sf GrB\_NULL}, then $\bold{D}({\sf val})$ must be
    compatible with $\bDin1({\sf accum})$ and $\bDout({\sf accum})$ of the accumulation operator and 
    $\bold{D}({\sf op})$ of the reduction binary operator must be compatible with $\bDin2({\sf accum})$ of the accumulation operator.

	\item $\bold{D}({\sf u})$ must be compatible with $\bold{D}({\sf op})$ of the binary reduction operator.
\end{enumerate}
Two domains are compatible with each other if values from one domain can be cast 
to values in the other domain as per the rules of the C language.
In particular, domains from Table~\ref{Tab:PredefinedTypes} are all compatible 
with each other. A domain from a user-defined type is only compatible with itself.
If any compatibility rule above is violated, execution of {\sf GrB\_reduce} ends and 
the domain mismatch error listed above is returned.

From the argument vector, the internal vector used in 
the computation is formed ($\leftarrow$ denotes copy):
\begin{enumerate}
	\item Vector $\vector{\widetilde{u}} \leftarrow {\sf u}$.
\end{enumerate}

We are now ready to carry out the reduce and any additional 
associated operations.  
First, an intermediate scalar result $t$ is computed using the recurrence:
\[
	t \leftarrow \bold{0}({\sf op}),
\]
\[
	t \leftarrow t \oplus \vector{u}(i), \forall i \in \bold{ind}(\vector{u}). 
\]
Where $\oplus = \bigodot({\sf op})$, and $\bold{0}({\sf op})$ is the identity of the monoid.

The final reduction value {\sf val} is computed as follows:
\begin{itemize}
	\item If ${\sf accum} = {\sf GrB\_NULL}$, then ${\sf val} \leftarrow t$.

	\item If ${\sf accum}$ is a binary operator, then ${\sf val} \leftarrow {\sf val} \odot t$,
    where $\odot  = \bigodot({\sf accum})$. 
\end{itemize}

In both {\sf GrB\_BLOCKING} and {\sf GrB\_NONBLOCKING} modes, the method exits with return value 
{\sf GrB\_SUCCESS} and the new contents of {\sf val} is as defined above
and fully computed.


%-----------------------------------------------------------------------------

\subsubsection{{\sf reduce}: Matrix-scalar variant}
\label{Sec:Reduce_matrix_scalar}

Reduce all stored values into a single scalar.

\paragraph{\syntax}

\begin{verbatim}
        GrB_Info GrB_reduce(<type>               *val,
                            const GrB_BinaryOp    accum,
                            const GrB_Monoid      op,
                            const GrB_Matrix      A,
                            const GrB_Descriptor  desc);
\end{verbatim}

\paragraph{Parameters}

\begin{itemize}[leftmargin=1.1in]
    \item[{\sf val}]    ({\sf INOUT}) Scalar to store final reduced value into. On input,
    the scalar provides a value that may be accumulated with the result of the
    reduction operation.  On output, this scalar holds the results of the
    operation.

    \item[{\sf accum}] ({\sf IN}) An optional binary operator used for accumulating
    entries into existing {\sf val} value. If assignment rather than accumulation is
    desired, {\sf GrB\_NULL} should be specified.

    \item[{\sf op}]    ({\sf IN}) The monoid used in the element-wise reduction
    operation, $M = \langle D,\oplus,0 \rangle$. The binary operator,
    $\oplus$, must be commutative and associative; otherwise,
    the outcome of the operation is undefined.
    
    \item[{\sf A}]     ({\sf IN}) The GraphBLAS matrix on which
    reduction will be performed.

    \item[{\sf desc}] ({\sf IN}) An optional operation descriptor. If
    a \emph{default} descriptor is desired, {\sf GrB\_NULL} should be
    specified. Non-default field/value pairs are listed as follows:  \\

    \begin{tabular}{lllp{2.5in}}
        Param & Field  & Value & Description \\
        \hline
    \end{tabular}

    \emph{Note:} This argument is defined for consistency with the other GraphBLAS operations.
    There are currently no non-default field/value pairs that can be set for this operation.
\end{itemize}

\paragraph{Return Values}

\begin{itemize}[leftmargin=2.1in]
    \item[{\sf GrB\_SUCCESS}]         In blocking or non-blocking mode, 
    the operation completed successfully, and the
    output scalar {\sf val} is ready to be used in the next method of 
    the sequence.

    \item[{\sf GrB\_PANIC}]           Unknown internal error.

    \item[{\sf GrB\_INVALID\_OBJECT}] This is returned in any execution mode 
    whenever one of the opaque GraphBLAS objects (input or output) is in an invalid 
    state caused by a previous execution error.  Call {\sf GrB\_error()} to access 
    any error messages generated by the implementation.

    \item[{\sf GrB\_OUT\_OF\_MEMORY}] Not enough memory available for the operation.

    \item[{\sf GrB\_UNINITIALIZED\_OBJECT}] One or more of the GraphBLAS objects 
    has not been initialized by a call to {\sf new} (or {\sf Matrix\_dup} for matrix
    parameters).
    
    \item[{\sf GrB\_DOMAIN\_MISMATCH}]    The domains of input and output arguments are
    incompatible with the corresponding domains of the accumulation operator, 
    or reduce operator.

    \item[{\sf GrB\_NULL\_POINTER}]  {\sf val} pointer is {\sf NULL}.
\end{itemize}

\paragraph{Description}

This variant of {\sf GrB\_reduce} computes the result of performing
a reduction across each of the elements of an input matrix:
${\sf val} = \bigoplus {\sf A}(:,:)$; or, if an optional binary accumulation 
operator is provided, ${\sf val} = {\sf val} \odot (\bigoplus {\sf A}(:,:))$, 
where $\bigoplus = \bigodot({\sf op})$ and $\odot = \bigodot({\sf accum})$.  

Logically, this operation occurs in three steps:
\begin{enumerate}[leftmargin=0.85in]
\item[\bf Setup] The internal matrix used in the computation is formed 
and its domain is tested for compatibility.
\item[\bf Compute] The indicated computations are carried out.
\item[\bf Output] The result is written into the output scalar.
\end{enumerate}

One matrix argument is used in this {\sf GrB\_reduce} operation:
\begin{enumerate}
	\item ${\sf A} = \langle \bold{D}({\sf A}),\bold{size}({\sf A}),
    \bold{L}({\sf A}) = \{(i,j,A_{i,j}) \} \rangle$

\end{enumerate}

The output scalar, argument matrix, reduction operator and accumulation 
operator (if provided) are tested for domain compatibility as follows:
\begin{enumerate}

	\item If {\sf accum} is {\sf GrB\_NULL}, then $\bold{D}({\sf val})$ must be
    compatible with $\bold{D}({\sf op})$ of the reduction binary operator.

	\item If {\sf accum} is not {\sf GrB\_NULL}, then $\bold{D}({\sf val})$ must be
    compatible with $\bDin1({\sf accum})$ and $\bDout({\sf accum})$ of the accumulation operator and 
    $\bold{D}({\sf op})$ of the reduction binary operator must be compatible with $\bDin2({\sf accum})$ of the accumulation operator.

	\item $\bold{D}({\sf A})$ must be compatible with $\bold{D}({\sf op})$ of the binary reduction operator.
\end{enumerate}
Two domains are compatible with each other if values from one domain can be cast 
to values in the other domain as per the rules of the C language.
In particular, domains from Table~\ref{Tab:PredefinedTypes} are all compatible 
with each other. A domain from a user-defined type is only compatible with itself.
If any compatibility rule above is violated, execution of {\sf GrB\_reduce} ends and 
the domain mismatch error listed above is returned.

From the argument matrix, the internal matrix used in 
the computation is formed ($\leftarrow$ denotes copy):
\begin{enumerate}
	\item Matrix $\matrix{\widetilde{A}} \leftarrow {\sf A}$.
\end{enumerate}

We are now ready to carry out the reduce and any additional 
associated operations.  
First, an intermediate scalar result $t$ is computed using the recurrence:
\[
	t \leftarrow \bold{0}({\sf op}),
\]
\[
	t \leftarrow t \oplus \matrix{A}(i,j), \forall (i,j) \in \bold{ind}(\matrix{A}). 
\]
Where $\oplus = \bigodot({\sf op})$, and $\bold{0}({\sf op})$ is the identity of the monoid.

The final reduction value {\sf val} is computed as follows:
\begin{itemize}
	\item If ${\sf accum} = {\sf GrB\_NULL}$, then ${\sf val} \leftarrow t$.

	\item If ${\sf accum}$ is a binary operator, then ${\sf val} \leftarrow {\sf val} \odot t$,
    where $\odot  = \bigodot({\sf accum})$. 
\end{itemize}

In both {\sf GrB\_BLOCKING} and {\sf GrB\_NONBLOCKING} modes, the method exits with return value 
{\sf GrB\_SUCCESS} and the new contents of {\sf val} is as defined above
and fully computed.


%=========================================================================

\subsection{{\sf transpose}: Transpose rows and columns of a matrix}

This version computes a new matrix that is the transpose of the source matrix.

\paragraph{\syntax}

\begin{verbatim}
        GrB_Info GrB_transpose(GrB_Matrix             C,
                               const GrB_Matrix       Mask,
                               const GrB_BinaryOp     accum,
                               const GrB_Matrix       A,
                               const GrB_Descriptor   desc);
\end{verbatim}

\paragraph{Parameters}

\begin{itemize}[leftmargin=1.1in]
    \item[{\sf C}]    ({\sf INOUT}) An existing GraphBLAS matrix. On input,
    the matrix provides values that may be accumulated with the result of the
    transpose operation.  On output, the matrix holds the results of the
    operation.

    \item[{\sf Mask}] ({\sf IN}) An optional ``write'' mask that controls which
    results from this operation are stored into the output matrix {\sf C}. The 
    mask dimensions must match those of the matrix {\sf C} and the domain of the 
    {\sf Mask} matrix must be of type {\sf bool} or any of the predefined 
    ``built-in'' types in Table~\ref{Tab:PredefinedTypes}.  If the default
    matrix is desired (\ie, with correct dimensions filled with {\sf true}), 
    {\sf GrB\_NULL} should be specified.

    \item[{\sf accum}] ({\sf IN}) An optional binary operator used for accumulating
    entries into existing {\sf C} entries.
    If assignment rather than accumulation is
    desired, {\sf GrB\_NULL} should be specified.

    \item[{\sf A}]     ({\sf IN}) The GraphBLAS matrix on which
    transposition will be performed.

    \item[{\sf desc}] ({\sf IN}) An optional operation descriptor. If
    a \emph{default} descriptor is desired, {\sf GrB\_NULL} should be
    specified. Non-default field/value pairs are listed as follows:  \\

    \begin{tabular}{lllp{2.5in}}
        Param & Field  & Value & Description \\
        \hline
        {\sf C}    & {\sf GrB\_OUTP} & {\sf GrB\_REPLACE} & Output matrix {\sf C}
        is cleared (all elements removed) before the result is stored in it.\\

        {\sf Mask} & {\sf GrB\_MASK} & {\sf GrB\_SCMP}   & Use the structural
        complement of {\sf Mask}. \\

        {\sf A}    & {\sf GrB\_INP0} & {\sf GrB\_TRAN}   & Use transpose of {\sf A}
        for the operation. \\
    \end{tabular}
\end{itemize}

\paragraph{Return Values}

\begin{itemize}[leftmargin=2.1in]
    \item[{\sf GrB\_SUCCESS}]         In blocking mode, the operation completed
    successfully. In non-blocking mode, this indicates that the compatibility 
    tests on dimensions and domains for the input arguments passed successfully. 
    Either way, output matrix, {\sf C}, is ready to be used in the next method of
    the sequence.

    \item[{\sf GrB\_PANIC}]           Unknown internal error.

    \item[{\sf GrB\_INVALID\_OBJECT}] This is returned in any execution mode 
    whenever one of the opaque GraphBLAS objects (input or output) is in an invalid 
    state caused by a previous execution error.  Call {\sf GrB\_error()} to access 
    any error messages generated by the implementation.

    \item[{\sf GrB\_OUT\_OF\_MEMORY}] Not enough memory available for the operation.

    \item[{\sf GrB\_UNINITIALIZED\_OBJECT}] One or more of the GraphBLAS objects 
    has not been initialized by a call to {\sf new} (or {\sf Matrix\_dup} for matrix
    parameters).

    \item[{\sf GrB\_DIMENSION\_MISMATCH}]  {\sf mask}, {\sf C} and/or {\sf A} 
    dimensions are incompatible.

    \item[{\sf GrB\_DOMAIN\_MISMATCH}]    The domains of the various matrices are
    incompatible with the corresponding domains of the 
    accumulation operator, or the mask's domain is not compatible with {\sf bool}.
\end{itemize}

\paragraph{Description}

{\sf GrB\_transpose} computes the result of performing a transpose of the input matrix:
${\sf C} = {\sf A}^T$; or, if an optional binary accumulation 
operator ($\odot$) is provided, ${\sf C} = {\sf C} \odot {\sf A}^T$.  
We note that the input matrix {\sf A} can itself be optionally transposed before the operation,
which would cause either an assignment from {\sf A} to {\sf C} or an
accumulation of {\sf A} into {\sf C}.

Logically, this operation occurs in three steps:
\begin{enumerate}[leftmargin=0.75in]
\item[\bf Setup] The internal matrix and mask used in the computation are formed 
and their domains and dimensions are tested for compatibility.
\item[\bf Compute] The indicated computations are carried out.
\item[\bf Output] The result is written into the output matrix, possibly under 
control of a mask.
\end{enumerate}

Up to three matrix arguments are used in this {\sf GrB\_transpose} operation:
\begin{enumerate}
	\item ${\sf C} = \langle \bold{D}({\sf C}),\bold{nrows}({\sf C}),
    \bold{ncols}({\sf C}),\bold{L}({\sf C}) = \{(i,j,C_{ij}) \} \rangle$

	\item ${\sf Mask} = \langle \bold{D}({\sf Mask}),\bold{nrows}({\sf Mask}),
    \bold{ncols}({\sf Mask}),\bold{L}({\sf Mask}) = \{(i,j,M_{ij}) \} \rangle$ (optional)

	\item ${\sf A} = \langle \bold{D}({\sf A}),\bold{nrows}({\sf A}),
    \bold{ncols}({\sf A}),\bold{L}({\sf A}) = \{(i,j,A_{ij}) \} \rangle$
\end{enumerate}

The argument matrices and accumulation 
operator (if provided) are tested for domain compatibility as follows:
\begin{enumerate}
	\item The domain of {\sf Mask} (if not {\sf GrB\_NULL}) must be from one of
    the pre-defined types of Table~\ref{Tab:PredefinedTypes}.

	\item If {\sf accum} is {\sf GrB\_NULL}, then $\bold{D}({\sf C})$ must be
    compatible with $\bold{D}({\sf A})$ of the input matrix.

	\item If {\sf accum} is not {\sf GrB\_NULL}, then $\bold{D}({\sf C})$ must be
    compatible with $\bDin1({\sf accum})$ and $\bDout({\sf accum})$ of the accumulation operator and 
    $\bold{D}({\sf A})$ of the input matrix must be compatible with $\bDin2({\sf accum})$ of the accumulation operator.

\end{enumerate}
Two domains are compatible with each other if values from one domain can be cast 
to values in the other domain as per the rules of the C language.
In particular, domains from Table~\ref{Tab:PredefinedTypes} are all compatible 
with each other. A domain from a user-defined type is only compatible with itself.
If any compatibility rule above is violated, execution of {\sf GrB\_transpose} ends and 
the domain mismatch error listed above is returned.

From the argument matrices, the internal matrices and mask used in 
the computation are formed ($\leftarrow$ denotes copy):
\begin{enumerate}
	\item Matrix $\matrix{\widetilde{C}} \leftarrow {\sf C}$.

	\item Two-dimensional mask, $\matrix{\widetilde{M}}$, is computed from
    argument {\sf Mask} as follows:
	\begin{enumerate}
		\item	If ${\sf Mask} = {\sf GrB\_NULL}$, then $\matrix{\widetilde{M}} = 
        \langle \bold{nrows}({\sf C}), \bold{ncols}({\sf C}), \{(i,j), 
        \forall i,j : 0 \leq i <  \bold{nrows}({\sf C}), 0 \leq j < 
        \bold{ncols}({\sf C}) \} \rangle$.

		\item	Otherwise, $\matrix{\widetilde{M}} = \langle \bold{nrows}({\sf Mask}), 
        \bold{ncols}({\sf Mask}), \{(i,j) : (i,j) \in \bold{ind}({\sf Mask}) \wedge 
        ({\sf bool}){\sf Mask}(i,j) = \true\} \rangle$.

		\item	If ${\sf desc[GrB\_MASK].GrB\_SCMP}$ is set, then 
        $\matrix{\widetilde{M}} \leftarrow \neg \matrix{\widetilde{M}}$.
	\end{enumerate}

	\item Matrix $\matrix{\widetilde{A}} \leftarrow
    {\sf desc[GrB\_INP0].GrB\_TRAN} \ ? \ {\sf A}^T : {\sf A}$.
\end{enumerate}

The internal matrices and masks are checked for dimension compatibility. The following
conditions must hold:
\begin{enumerate}
	\item $\bold{nrows}(\matrix{\widetilde{C}}) = \bold{nrows}(\matrix{\widetilde{M}})$.

	\item $\bold{ncols}(\matrix{\widetilde{C}}) = \bold{ncols}(\matrix{\widetilde{M}})$.

	\item $\bold{nrows}(\matrix{\widetilde{C}}) = \bold{ncols}(\matrix{\widetilde{A}})$.

	\item $\bold{ncols}(\matrix{\widetilde{C}}) = \bold{nrows}(\matrix{\widetilde{A}})$.
\end{enumerate}
If any compatibility rule above is violated, execution of {\sf GrB\_transpose} ends and
the dimension mismatch error listed above is returned.

From this point forward, in {\sf GrB\_NONBLOCKING} mode, the method can 
optionally exit with {\sf GrB\_SUCCESS} return code and defer any computation 
and/or execution error codes.

We are now ready to carry out the matrix transposition and any additional 
associated operations.  We describe this in terms of two intermediate matrices:
\begin{itemize}
    \item $\matrix{\widetilde{T}}$: The matrix holding the  transpose of
    $\matrix{\widetilde{A}}$.
    \item $\matrix{\widetilde{Z}}$: The matrix holding the result after 
    application of the (optional) accumulation operator.
\end{itemize}

The intermediate matrix
\[
\matrix{\widetilde{T}} = \langle
\bold{D}(\matrix{\sf A}), \bold{ncols}(\matrix{\widetilde{A}}),
\bold{nrows}(\matrix{\widetilde{A}}), \bold{L}(\matrix{\widetilde{T}}) =
\{(j,i,A_{ij}) \forall (i,j) \in \bold{ind}(\matrix{\widetilde{A}}) 
\rangle
\]
is created.  

%\emph{Standard Matrix Accumulate Option}

The intermediate matrix $\matrix{\widetilde{Z}}$ is created as follows:
\begin{itemize}
    \item If ${\sf accum} = {\sf GrB\_NULL}$, then $\matrix{\widetilde{Z}} = \matrix{\widetilde{T}}$.

    \item If ${\sf accum}$ is a binary operator, then $\matrix{\widetilde{Z}}$ is defined as
        \[ \langle \bDout({\sf accum}), \bold{nrows}(\matrix{\widetilde{C}}), \bold{ncols}(\matrix{\widetilde{C}}),
        %\bold{L}(\matrix{\widetilde{Z}}) =
        \{(i,j,Z_{ij})  \forall (i,j) \in \bold{ind}(\matrix{\widetilde{C}}) \cup 
        \bold{ind}(\matrix{\widetilde{T}}) \} \rangle.\]

    The values of the elements of $\matrix{\widetilde{Z}}$ are computed based on the
    relationships between the sets of indices in $\matrix{\widetilde{C}}$ and 
    $\matrix{\widetilde{T}}$.
\[
    Z_{ij} = \matrix{\widetilde{C}}(i,j) \odot \matrix{\widetilde{T}}(i,j), \ \mbox{if}\  
    (i,j) \in  (\bold{ind}(\matrix{\widetilde{T}}) \cap \bold{ind}(\matrix{\widetilde{C}})),
\]
\[
    Z_{ij} = \matrix{\widetilde{C}}(i,j), \ \mbox{if}\  
    (i,j) \in (\bold{ind}(\matrix{\widetilde{C}}) - (\bold{ind}(\matrix{\widetilde{T}})
    \cap \bold{ind}(\matrix{\widetilde{C}}))),
\]
\[
    Z_{ij} = \matrix{\widetilde{T}}(i,j), \ \mbox{if}\  (i,j) \in  
    (\bold{ind}(\matrix{\widetilde{T}}) - (\bold{ind}(\matrix{\widetilde{T}})
    \cap \bold{ind}(\matrix{\widetilde{C}}))),
\]
where $\odot  = \bigodot({\sf accum})$, and the difference operator refers to set difference.
\end{itemize}



Finally, the set of output values that make up matrix $\matrix{\widetilde{Z}}$ 
are written into the final result matrix {\sf C}, using what is called
a \emph{standard matrix mask and replace}. 
This is carried out under control of the mask which acts as a ``write mask''.
\begin{itemize}
\item If {\sf desc[GrB\_OUTP].GrB\_REPLACE} is set, then any values in {\sf C} on
input to this operation are deleted and the contents of the new output matrix,
{\sf C}, is defined as,
\[
\bold{L}({\sf C}) = \{(i,j,Z_{ij}) : (i,j) \in (\bold{ind}(\matrix{\widetilde{Z}}) 
\cap \bold{ind}(\matrix{\widetilde{M}})) \}. 
\]

\item If {\sf desc[GrB\_OUTP].GrB\_REPLACE} is not set, the elements of 
$\matrix{\widetilde{Z}}$ indicated by the mask are copied into the result 
matrix, {\sf C}, and elements of {\sf C} that fall outside the set 
indicated by the mask are unchanged:
\[
\bold{L}({\sf C}) = \{(i,j,C_{ij}) : (i,j) \in (\bold{ind}({\sf C}) 
\cap \bold{ind}(\neg \matrix{\widetilde{M}})) \} \cup \{(i,j,Z_{ij}) : (i,j) \in 
(\bold{ind}(\matrix{\widetilde{Z}}) \cap \bold{ind}(\matrix{\widetilde{M}})) \}.
\]
\end{itemize}

In {\sf GrB\_BLOCKING} mode, the method exits with return value 
{\sf GrB\_SUCCESS} and the new content of matrix {\sf C} is as defined above
and fully computed.
In {\sf GrB\_NONBLOCKING} mode, the method exits with return value 
{\sf GrB\_SUCCESS} and the new content of matrix {\sf C} is as defined above
but may not be fully computed; however, it can be used in the next GraphBLAS 
method call in a sequence.

