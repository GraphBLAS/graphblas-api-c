\section{Generalized User-Defined Types}
\label{Sec:UsrTypes}

Currently, GraphBLAS only supports a limited form of user-defined
types. In particular, objects of the data type must have a flat memory
representation, so that two objects can be copied with a simple {\tt
memcpy} operation.  It is desirable to lift this restriction. One
possibility would be to add a version of {\sf GrB\_Type\_new} that
supports arbitrary user-defined types as follows.

When the user creates a new type, he or she must pass three functions
that perform the most basic operations in that type:
\begin{enumerate}
	\item A \emph{create} function that creates an object of the
	user-defined type. That includes allocating storage for the
	object and initializing that object to a default state.

	\item A \emph{copy} function that copies the state of a source
	object of the user-defined type to a target object of the same
	user-defined type.

	\item A \emph{destroy} function that destroys an object of the
	user-defined type, releasing any resources the object uses.
\end{enumerate}
Optionally, we could allow create, copy and destroy methods for arrays
of user-defined objects, in order to avoid the overhead of function
calls and memory management at the level of each individual object.
The description of this new form of {\sf GrB\_Type\_new} is shown
in Figure~\ref{Fig:GrB_Type_new}.

\begin{figure}
\hrule
\footnotesize
\paragraph{Syntax}

\begin{verbatim}
GrB_Info GrB_Type_new(GrB_Type   *utype,
                      void       *create,
                      void       *destroy,
                      void       *copy);
\end{verbatim}

\paragraph{Parameters}

\begin{itemize}[leftmargin=0.6in]
\item[{\sf utype}] ({\sf INOUT}) On successful return, contains a handle to the newly created user-defined GraphBLAS type object.
\item[{\sf create}] ({\sf IN})    A pointer to a function that creates and initializes (to a default state) an object of the user-defined type. Such function must return a {\tt void*} pointer to the new object. Its signature is \verb|void* create()|.
\item[{\sf destroy}] ({\sf IN}) A pointer to a function that destroys an object of the user-defined type, releasing any resources the object uses. Its signature is \verb|void destroy(void* obj)|.
\item[{\sf copy}] ({\sf IN}) A pointer to a function that copies the contents from a source object of the user-defined type to a destination object of the same user-defined type. Its signature is \verb|void copy(void* tgt, const void* src)|.
\end{itemize}

\paragraph{Return Values}

\begin{itemize}[leftmargin=1.6in]
\item[{\sf GrB\_SUCCESS}]           operation completed successfully.
\item[{\sf GrB\_PANIC}]             unknown internal error.
\item[{\sf GrB\_OUT\_OF\_MEMORY}]          not enough memory available for operation.
\item[{\sf GrB\_NULL\_POINTER}]    at least one of {\sf utype}, {\sf create}, {\sf destroy}, {\sf copy} pointers is {\sf NULL}.
\end{itemize}

	\caption{Definition of a {\sf GrB\_Type\_new} GraphBLAS method that can support arbitrary user-defined types.}
	\label{Fig:GrB_Type_new}
\hrule
\end{figure}
