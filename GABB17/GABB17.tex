\documentclass[10pt, conference, compsocconf]{IEEEtran}   	
%\usepackage{geometry}                		% See geometry.pdf to learn the layout options. There are lots.
%\geometry{letterpaper}                   		% ... or a4paper or a5paper or ... 
%\geometry{landscape}                		% Activate for for rotated page geometry
%\usepackage[parfill]{parskip}    		% Activate to begin paragraphs with an empty line rather than an indent
\usepackage{graphicx}				% Use pdf, png, jpg, or eps§ with pdflatex; use eps in DVI mode
								% TeX will automatically convert eps --> pdf in pdflatex

\usepackage{color}
\newcommand{\fix}[1]{\textcolor{red}{#1}}
\newcommand{\highlight}[1]{\textcolor{red}{#1}}
\usepackage{amssymb, amsmath}
\usepackage{mathtools}
\usepackage{booktabs}
%\usepackage[noend]{algpseudocode}
%\usepackage[ruled,vlined]{algorithm2e}
\usepackage{verbatim}
\newcommand*\rot{\rotatebox{90}}
%\usepackage[compatible]{algpseudocode}
\usepackage{algorithm}
\usepackage[noend]{algpseudocode}
\usepackage{varwidth}
\usepackage{multirow}
\usepackage{rotating, xcolor}
\usepackage{subfig}
\usepackage{enumitem}
%\usepackage[caption=false]{subfig}
\usepackage[hyphens]{url}
\usepackage{float}
\newfloat{algorithm}{t}{lop}

\usepackage{tcolorbox}
\definecolor{mycolor}{rgb}{0.122, 0.435, 0.698}
%\begin{tcolorbox}[width=\linewidth, boxsep=0pt, left=-4pt, boxrule=0.5pt]
%\end{tcolorbox} 
%\newtcolorbox{mybox}{colback=red!5!white,colframe=mycolor}
\makeatletter
\newcommand{\mybox}[1]{%
  %\setbox0=\hbox{#1}%
  %\setlength{\@tempdima}{\dimexpr\wd0+13pt}%
  \begin{tcolorbox}[colframe=mycolor,boxrule=0.5pt,arc=4pt,
      left=-4pt,right=6pt,top=6pt,bottom=6pt,boxsep=0pt,width=\linewidth]
    #1
  \end{tcolorbox}
}
\makeatother

\algnewcommand{\LineComment}[1]{\State \(\triangleright\) #1}
%\newcommand{\procdecl}[1]   {\proc{#1}\vrule width0pt height0pt depth 7pt \relax}
\newcommand{\lilabel}[1]        {\label{li:#1}}

\newcommand{\erdosrenyi}{Erd\H os-R\'{e}nyi }
\newcommand{\qg}{\u{g}}
\newcommand{\qG}{\u{G}}
\newcommand{\qc}{\c{c}}
\newcommand{\qC}{\c{C}}
\newcommand{\qs}{\c{s}}
\newcommand{\qS}{\c{S}}
\newcommand{\qu}{\"{u}}
\newcommand{\qU}{\"{U}}
\newcommand{\qo}{\"{o}}
\newcommand{\qO}{\"{O}}
\newcommand{\qI}{\.{I}}
\newcommand{\wa}{\^{a}}
\newcommand{\wA}{\^{A}}
\usepackage{mathtools}
\DeclarePairedDelimiter\ceil{\lceil}{\rceil}
\DeclarePairedDelimiter\floor{\lfloor}{\rfloor}


\newcommand{\minusone}{\text{-}1}

%% ABAB: Shortcuts/macros below are not used but it can be referenced as a cheatsheet
\newcommand{\liref}[1]      {line~\ref{li:#1}}
\newcommand{\Liref}[1]      {Line~\ref{li:#1}}
\newcommand{\lirefs}[2]     {lines \ref{li:#1}--\ref{li:#2}}
\newcommand{\Lirefs}[2]     {Lines \ref{li:#1}--\ref{li:#2}}
\newcommand{\lireftwo}[2]   {lines \ref{li:#1} and~\ref{li:#2}}
\newcommand{\lirefthree}[3] {lines \ref{li:#1}, \ref{li:#2}, and~\ref{li:#3}}

\def\Cpp{C{}\texttt{++}}
\newcommand{\mA}{\mathbf{A}} 
\newcommand{\mL}{\mathbf{L}}
\newcommand{\mU}{\mathbf{U}}
\newcommand{\transpose}     {^{\mbox{\scriptsize \sf T}}}
\newcommand{\mB}{\mathbf{B}}
\newcommand{\mC}{\mathbf{C}}
\newcommand{\dimN}{n}
\newcommand{\dimM}{m}
\newcommand{\dimK}{k}
\newcommand{\dnzc}{\id{nzc}}
\newcommand{\dnzr}{\id{nzr}}
\newcommand{\dni}{\id{ni}}
\newcommand{\dnnz}{\id{nnz}}
\newcommand{\dsort}{\id{sort}}
\newcommand{\dscan}{\id{scan}}
\newcommand{\dsearch}{\id{search}}
\newcommand{\dmin}{\func{min}}
\newcommand{\dmax}{\func{max}}
\newcommand{\dth}{th}
\newcommand{\dlen}{\id{len}}
\newcommand{\matlab}{{\sc Matlab}}

%
%TGM: included and new commands from our graphBLAS C API document
%
\usepackage{listings}
\renewcommand{\vector}[1]{{\bf #1}}
\renewcommand{\matrix}[1]{{\bf #1}}
\renewcommand{\arg}[1]{{\sf #1}}
\newcommand{\zip}{{\mbox{zip}}}
\newcommand{\zap}{{\mbox{zap}}}
\newcommand{\ewiseadd}{{\mbox{\bf ewiseadd}}}
\newcommand{\ewisemult}{{\mbox{\bf ewisemult}}}
\newcommand{\mxm}{{\mbox{\bf mxm}}}
\newcommand{\vxm}{{\mbox{\bf vxm}}}
\newcommand{\mxv}{{\mbox{\bf mxv}}}
\newcommand{\gpit}[1]{{\sf #1}}
\newcommand{\ie}{\emph{i.e.}}
\newcommand{\eg}{\emph{e.g.}}
\newcommand{\nan}{{\sf NaN}}
\newcommand{\nil}{{\bf nil}}
\newcommand{\ifif}{{\bf if}}
\newcommand{\ifthen}{{\bf then}}
\newcommand{\ifelse}{{\bf else}}
\newcommand{\ifendif}{{\bf endif}}
\newcommand{\zero}{{\bf 0}}
\newcommand{\one}{{\bf 1}}
\newcommand{\true}{{\sf true}}
\newcommand{\false}{{\sf false}}
\newcommand{\syntax}{{C Syntax}}
%%%%%%

%\input{algobox}

\title{Design of the GraphBLAS API for C}

\author{
Ayd\i n Bulu\qc , Tim Mattson, Scott McMillan, Jos\'e Moreira, Carl Yang}

\date{}	

\begin{document}
\maketitle

\begin{abstract}
The GraphBLAS effort aims to standardize linear-algebraic building blocks for graph computations. 
A time consuming part of this standardization effort is to translate the mathematical specification to
an actual Application Programming Interface (API) that (i) is faithful to the mathematics  
and (ii) enables efficient implementations on modern hardware. This paper 
documents the efforts taken by the C language specification subcommittee and presents the main concepts, 
constructs, and objects within the GraphBLAS API.

\end{abstract}

%%%%%%%%%%%%%%%%%%%%%%% file intro.tex %%%%%%%%%%%%%%%%%%%%%%%%%
%
% This file contains the introduction section of the paper
%
%%%%%%%%%%%%%%%%%%%%%%%%%%%%%%%%%%%%%%%%%%%%%%%%%%%%%%%%%%%%%%%%%%%
\section{Introduction}
\label{sec:intro}
The GraphBLAS are great and you'll love them.  In this paper, we'll talk about the even greater stuff coming in the future
\section{GraphBLAS Math}
\label{sec:math}

Graphs permit a dual representation as an 
collection of vertices/edges or as a matrix.  The matrix representation
of a graph can take different forms.  A common one uses an 
\emph{ Adjacency matrix}; for which the rows and columns designate the vertices
of a graph and nonzero $(i,j)^{th}$ matrix elements signify an edge between 
vertices $i$ and $j$.   The degree of vertices in a graph are almost always a small
compared to the number of vertices; hence, the matrices used to represent
graphs are sparse.

Multiplying an Adjacency matrix times a second matrix is equivalent to a breath 
first search from multiple starting locations.   Whole classes of algorithms
based on this basic pattern can be constructed from this core operation.
We can extend the range of graph algorithms by keeping the basic
memory access pattern of a matrix-matrix multiplication, but varying
the operations used in the operation.  By carefully choosing operations 
that support the algebraic properties of commutativity and associativity, familiar
algebraic relations between matrix objects are retained thereby enabling
composable graph algorithms.

We do this using the concept of an algebraic semiring.   The most 
common semirings used in the Graph Algorithms community are 
shown in table~\ref{Tab:semirings}.

  
\begin{table}[h]
\hrule
\begin{center}
\caption{Common Semirings used with Graph Algorithyms.}
\label{Tab:semirings}
\begin{tabular}{llll}
{\sf Semiring} & \multicolumn{2}{c}{operators} & Domain \\
\hline
Standard Arithmetic         & $\oplus   \equiv + $ & $\otimes \equiv   \times $  & $ \mathbb{R} $\\
max-plus Algebras           & $\oplus   \equiv max $ & $\otimes \equiv  + $  & $  \{-\infty \cup  \mathbb{R} \}$\\
max-min Algebras           & $\oplus   \equiv max $ & $\otimes \equiv  \times $  & $  \infty \cup  \mathbb{R}_{\leq 0}\} $\\
Finite (Galois) Fields (e.g. GF2)      & $\oplus   \equiv xor $ & $\otimes \equiv  and $  & $  \{0, 1\}$\\
Power set Algebras         & $\oplus   \equiv \cup $ & $\otimes \equiv  \cap $  & $  \mathbb{Z} $\\
\end{tabular}
\end{center}
\hrule
\end{table}

In the C-binding to the GraphBLAS, this means we define a separate object for the semiring 
that is passed into functions.  Since in many cases the full
semiring is not required, we also support passing monoids or
even operators; which basically means the semiring is implied but not 
explicitly stated.

In addition to matrix multiplication, a range of operations are included
in the GraphBLAS.  These are summarized in table~\ref{Tab:GraphBLASOps}.

\begin{table}[h]
\hrule
\begin{center}
\caption{A Mathematical overview of the fundamental GraphBLAS operations supported
in this specification. $\matrix{A}$, $\matrix{B}$, and $\matrix{C}$ are GraphBLAS matrices, 
$\vector{u}$ and $\vector{v}$ are GraphBLAS vectors, $i$ and $j$ are indices, and $v$ is a scalar 
indicating the value of an element of a GraphBLAS object.  $f()$ is a function and $m$ and $n$ are 
integers indicating the size of GraphBLAS object dimensions.  In most cases, the input matrices $\matrix{A}$ and $\matrix{B}$ may be selected for transposition prior to the operation and masks can be used to control
which values are written to the output GraphBLAS object.}
\label{Tab:GraphBLASOps}
\begin{tabular}{l|rrl}
{\sf Operation Name} & \multicolumn{3}{c}{Mathematical Description}  \\
\hline
{\sf mxm}          & $\matrix{C}$ & $\oplus=$ & $\matrix{A} \oplus.\otimes \matrix{B}$  \\
{\sf mxv}          & $\vector{u}$ & $\oplus=$ & $\matrix{A} \oplus.\otimes \vector{v}$  \\
{\sf vxm}          & $\vector{u}$ & $\oplus=$ & $\vector{v} \oplus.\otimes \matrix{A}$  \\
{\sf eWiseMult}    & $\matrix{C}$ & $\oplus=$ & $\matrix{A} \otimes \matrix{B}$  \\
{\sf eWiseAdd}     & $\matrix{C}$ & $\oplus=$ & $\matrix{A} \oplus  \matrix{B}$  \\
{\sf reduce} (row) & $\vector{u}$ & $\oplus=$ & $\oplus_j\matrix{A}(:,j)$  \\
{\sf apply}        & $\matrix{C}$ & $\oplus=$ & $f(\matrix{A})$ \\
{\sf transpose}    & $\matrix{C}$ & $\oplus=$ & $\matrix{A}$ \\
{\sf extract}      & $\matrix{C}$ & $\oplus=$ & $\matrix{A}(\vector{i},\vector{j})$ \\
{\sf assign}       & $\matrix{C}(\vector{i},\vector{j})$ & $\oplus=$ & $\matrix{A}$ \\
{\sf buildMatrix}  & $\matrix{C}$ & $\oplus=$ & $\mathbb{S}^{m\times n}(\vector{i},\vector{j},\vector{v},\oplus_{dup})$ \\
{\sf buildVector}  & $\vector{u}$ & $\oplus=$ & $\mathbb{S}^{n}(\vector{i},\vector{v})$ \\
{\sf extractTuples}& $(\vector{i},\vector{j},\vector{v})$ & $=$ & $\matrix{A}$ \\
\end{tabular}
\end{center}
\hrule
\end{table}


\section{Basic Concepts}
\label{sec:concepts}

The GraphBLAS standard defines a set of matrix and vector operations 
over semiring algebraic structures.  These operations can be used
to express a wide range of graph algorithms.  

The GraphBLAS C API is built on a collection of objects exposed to 
the C programmer as opaque data types. Functions that manipulate these
objects are referred to as {\it methods}.  These methods fully define the 
interface to GraphBLAS objects to create or destroy them, modify their 
contents, and copy the contents of opaque objects into non-opaque objects the 
contents of which are under direct control of the programmer.


\section{The GraphBLAS C API}
\label{sec:Capi}

Definition of the full GraphBLAS C API is  beyond the scope of this paper.  Instead, we 
focus on those parts of the API that are needed to understand the betweenness centrality
example in Section~\ref{sec:example}. 

We begin with data types used for the objects in the example.
These are listed in Table~\ref{Tab:GrBdataTypes}.  Except for the types that map directly onto
C language basic types ({\sf GrB\_Info}, {\sf GrB\_Index}, and {\sf GrB\_Type}), these data types
define handles to opaque objects manipulated by GraphBLAS methods.
\begin{table}[h]
\hrule
\begin{center}
\caption{GraphBLAS data types.}
\label{Tab:GrBdataTypes}
\begin{tabular}{lp{5.25cm}}
Data type                     & Description  \\
\hline
	{\sf GrB\_Info}           & Return value from any GraphBLAS method \\
	{\sf GrB\_Index}          & Vector and matrix indices \\
	{\sf GrB\_Type}		      & Type identifier \vspace{.1cm} \\ 
	{\sf GrB\_Descriptor}     & Opaque GraphBLAS descriptor object \\
	{\sf GrB\_Monoid}         & Opaque GraphBLAS monoid object  \\
	{\sf GrB\_Semiring}       & Opaque GraphBLAS semiring object  \\
	{\sf GrB\_Matrix}         & Opaque GraphBLAS matrix object \\
	{\sf GrB\_Vector}         & Opaque GraphBLAS vector object \\
\end{tabular}
\end{center}
\hrule
\end{table}
Objects corresponding to algebraic structures ({\sf GrB\_Monoid} and {\sf GrB\_Semiring}) are constructed from lower-level
operators. The GraphBLAS C API provides a mechanism for creating user-defined operators, but for this 
paper we consider only the predefined operators used in the example (summarized
in Table~\ref{Tab:GrBops}). 
\begin{table}[h]
\hrule
\begin{center}
\caption{Some predefined GraphBLAS operators.}
\label{Tab:GrBops}
\begin{tabular}{lp{5.25cm}}
Operator                       & Description  \\
\hline
	{\sf GrB\_TIMES\_INT32}    & Binary operation, returns  product of two 32-bit integer values \\
	{\sf GrB\_PLUS\_INT32}     & Binary operation, returns  sum of two 32-bit integer values \\
	{\sf GrB\_PLUS\_FP32}      & Binary operation, returns  sum of two 32-bit floating-point values \\
	{\sf GrB\_TIMES\_FP32}     & Binary operation, returns  product of two 32-bit floating-point values \\
	{\sf GrB\_MINV\_FP32}      & Unary operation, returns  multiplicative inverse of the input 32-bit floating-point value \\
	{\sf GrB\_IDENTITY\_BOOL}  & Unary operation, returns input boolean value \\
\end{tabular}
\end{center}
\hrule
\end{table}
A number of constant literal values are used in the GraphBLAS methods to 
choose among options or to define return values.
The most commonly used GraphBLAS literals are listed in Table~\ref{Tab:GrBliterals}.
\begin{table}[h]
\hrule
\begin{center}
\caption{GraphBLAS literals used in Section~\ref{sec:example}.}
\label{Tab:GrBliterals}
\begin{tabular}{lp{5.9cm}}
Literal                 & Description  \\
\hline
	{\sf GrB\_OUTP}      & Descriptor field for the output argument. \\
	{\sf GrB\_MASK}      & Descriptor field for the mask. \\
	{\sf GrB\_INP0}      & Descriptor field for the first input argument. \\
	{\sf GrB\_INP1}      & Descriptor field for the second input argument. \\ 
	{\sf GrB\_SCMP}      & Descriptor value to indicate use of the structural complement of the mask. \\
	{\sf GrB\_TRAN}      & Descriptor value to indicate use of the transpose of the corresponding input matrix. \\
	{\sf GrB\_REPLACE}   & Descriptor value to indicate that the output object should be replaced by the result of the method. \\ 
	{\sf GrB\_ALL}       & All of an object's indices in order. \\
	{\sf GrB\_NULL}      & \emph{Null} value used to indicate when a parameter is not provided and a default behavior should be used. \\
	{\sf GrB\_SUCCESS}   & Return value indicating that a method has returned without encountering an error condition. \\
	{\sf GrB\_BOOL}		   & Identifier for boolean type. \\
	{\sf GrB\_INT32}	   & Identifier for 32-bit integer type. \\
	{\sf GrB\_FP32}		   & Identifier for 32-bit floating point type. \\
\end{tabular}
\end{center}
\hrule
\end{table}

We illustrate the principles behind the GraphBLAS C API with a single method: 
the {\sf GrB\_mxm()} operation, shown in Figure~\ref{Fig:mxm}.
{\sf GrB\_mxm()} takes three input matrices $\matrix{A}$, $\matrix{B}$, and $\matrix{C}$; 
computes a matrix product of $\matrix{A}$ and $\matrix{B}$; and either copies the result into the matrix $\matrix{C}$
or accumulates the product into the matrix $\matrix{C}$.  Based on the arguments 
and the descriptor, the semantics of {\sf GrB\_mxm()} can vary considerably.
The function of the descriptor is consistent across all methods, hence its use for
{\sf GrB\_mxm()} is applicable to all methods.

\begin{figure*}[ht!]
\hrule
	\caption{The {\sf GrB\_mxm()} function signature, parameters, and return values.}
\label{Fig:mxm}
\paragraph{Signature}
\footnotesize
\begin{verbatim}
        GrB_Info GrB_mxm(GrB_Matrix              *C,
                         const GrB_Matrix         Mask,
                         const GrB_BinaryOp       accum,
                         const GrB_Semiring       op,
                         const GrB_Matrix         A, 
                         const GrB_Matrix         B,
                         const GrB_Descriptor     desc);
\end{verbatim}

\paragraph{Parameters}

\begin{itemize}[leftmargin=1.1in]
    \item[{\sf C}]    ({\sf INOUT}) An existing GraphBLAS matrix. On
    input, the matrix provides values that may be accumulated with the
    result of the matrix product.   On output, the matrix holds the
    results of this operation.

    \item[{\sf Mask}] ({\sf IN}) A ``write'' mask that controls which
    results from this operation are stored into the output matrix
    {\sf C} (optional).  If no mask is desired,  {\sf GrB\_NULL}
    is specified. The mask dimensions must match those of the
    matrix {\sf C}, and the domain of the {\sf Mask} matrix must be
    of type {\sf bool} or any ``built-in'' GraphBLAS type.

    \item[{\sf accum}] ({\sf IN}) A binary operator for accumulating entries
    with an existing \matrix{\sf C} entries.  Use {\sf GrB\_NULL} for assignment rather than accumulation.

    \item[{\sf op}] ({\sf IN}) Semiring used in the matrix-matrix
    multiply: ${\sf op}=\langle D_1,D_2,D_3,\oplus,\otimes,0 \rangle$.

    \item[{\sf A}] ({\sf IN}) The GraphBLAS matrix holding the values
    for the left-hand matrix in the multiplication.

    \item[{\sf B}] ({\sf IN}) The GraphBLAS matrix holding the values
    for the right-hand matrix in the multiplication.

    \item[{\sf desc}] ({\sf IN}) Operation descriptor (optional). If
    a \emph{default} descriptor is desired, {\sf GrB\_NULL} should be
    used. Valid fields are as follows: 

    \begin{tabular}{lllp{2.75in}}
    Argument   & Field           & Value               & Description \\ \hline
    {\sf C}    & {\sf GrB\_OUTP} & {\sf GrB\_REPLACE}  & Output matrix {\sf C} is cleared before result is stored. \\
    {\sf Mask} & {\sf GrB\_MASK} & {\sf GrB\_SCMP}     & Use the structural complement of {\sf Mask}. \\
    {\sf A}    & {\sf GrB\_INP0} & {\sf GrB\_TRAN}     & Use transpose of {\sf A} for operation. \\
    {\sf B}    & {\sf GrB\_INP1} & {\sf GrB\_TRAN}     & Use transpose of {\sf B} for operation. \\
    \end{tabular}
\end{itemize}

\paragraph{Return Values}

\begin{itemize}[leftmargin=2.1in]

	\item[{\sf GrB\_SUCCESS}]	Blocking mode: operation
	completed successfully. Nonblocking mode: input argument consistency tests
	passed.  

	\item[{\sf GrB\_PANIC}]		      Unknown internal error.

	\item[{\sf GrB\_OUTOFMEM}]	      Not enough memory available.
	for operation

	\item[{\sf GrB\_DIMENSION\_MISMATCH}] Matrix dimensions are
	incompatible.

	\item[{\sf GrB\_DOMAIN\_MISMATCH}]    The domains of the matrices are incompatible with the 
	accumulator, semiring, or mask domains.

\end{itemize}

\hrule
\end{figure*}
The semantics of GraphBLAS methods associated with the operations from Table~\ref{Tab:GraphBLASOps} follow a similar pattern:
\begin{enumerate}
\item The internal matrices and mask used in the computation are formed from the input parameters.  Their domains and dimensions are tested for consistency.
\item The indicated computations are carried out using the internal matrices and producing an internal result.
\item The internal result is written into the output matrix, possibly under control of a mask.
\end{enumerate}
In the case of the {\sf GrB\_mxm} operation, internal matrices $\matrix{A}$, $\matrix{B}$, $\matrix{C}$, and mask $\matrix{Mask}$ are formed from
the corresponding arguments according to the descriptor.  Depending on
the values in the descriptor fields {\sf GrB\_INP0} and {\sf GrB\_INP1},  $\matrix{A}$ and/or $\matrix{B}$ may be the transpose of the corresponding argument.
The descriptor field {\sf GrB\_MASK} may also indicate that the structural complement of the mask 
should be used.  If the domains and sizes of the objects are mathematically consistent, the indicated operation is carried out.
This produces an internal matrix $\matrix{T}$ equal to the product of matrices $\matrix{A}$ and $\matrix{B}$.
(We emphasize that an implementation
of the GraphBLAS is not required to materialize the matrix $\matrix{T}$).

If an optional binary accumulator function {\sf accum} is provided, it is used to combine the elements of 
matrix $\matrix{C}$ and the internal matrix $\matrix{T}$.  This forms a new internal matrix $\matrix{Z}$.   

At this point the elements of $\matrix{Mask}$ are used as a \emph{write mask} to select which elements of $\matrix{Z}$ are 
used to form the final output result.  Basically, the elements of the boolean write mask that exist and are true 
correspond to the elements of the output matrix that might be replaced by the corresponding elements of $\matrix{Z}$.
Two options are supported: 
\begin{itemize}
	\item \emph{Replace mode}: If the descriptor field {\sf GrB\_OUTP} is set to {\sf GrB\_REPLACE}, the 
		values in the $\matrix{C}$ matrix are deleted before masked elements of $\matrix{Z}$ are stored 
		in $\matrix{C}$.  In essence the computed matrix $\matrix{Z}$ \emph{replaces} the original matrix $\matrix{C}$.
	\item \emph{Merge mode}: Otherwise, the elements from the computation selected by the 
		write mask are written into the output $\matrix{C}$ matrix without changing elements
		that do not overlap with the mask.
\end{itemize}
%This discussion has been largely qualitative, and there are many low level details an implementor
%of {\sf GrB\_mxm()} would need to address.  The high level description of this operation, however, should
%give a basic idea of matrix multiplication in GraphBLAS.

The basic pattern used for {\sf GrB\_mxm()} is used for most of the GraphBLAS operations.
For example, the mathematical operations in 
Table~\ref{Tab:GraphBLASOps}  from {\sf GrB\_mxm} to {\sf GrB\_reduce}
use descriptors to modify input matrices or vectors, and write masks.  
We can't list all methods in the GraphBLAS API, but for the methods used in the example from Section~\ref{sec:example}
we list method names and descriptions in Table~\ref{Tab:GrBmethods}.    
%Low level details may vary, but the basic pattern used in {\sf GrB\_mxm} holds.
 
\begin{table}[h]
\hrule
\begin{center}
\caption{Methods used in the example in section~\ref{sec:example}.}
\label{Tab:GrBmethods}
\begin{tabular}{lp{5.25cm}}
Method Name                     & Description    \\
\hline
	{\sf GrB\_Monoid\_new}      & Creates a new monoid with specified domain, operator, and identity element.    \\
	{\sf GrB\_Semiring\_new}    & Creates a new semiring with specified domain, monoid, and operators.               \\
	{\sf GrB\_Vector\_new}      & Creates a new vector with specified domain and size.                                \\
	{\sf GrB\_Matrix\_new}      & Creates a new matrix with specified domain and dimensions.                          \\
	{\sf GrB\_Matrix\_nrows}    & Retrieves the number of rows in a matrix.                                            \\
	{\sf GrB\_Matrix\_nvals}    & Retrieves the number of stored elements (tuples) in a matrix.                        \\
	{\sf GrB\_Descriptor\_new}  & Creates a new (empty) descriptor.                                                  \\
	{\sf GrB\_Descriptor\_set}  & Sets the content (details of an operation) for a field of an existing descriptor.   \\
	{\sf GrB\_Matrix\_build}    & Copies elements from tuples into a matrix.                                         \\
	{\sf GrB\_mxm}              & Performs matrix multiplication over a semiring.                              \\
	{\sf GrB\_eWiseMult}        & Performs an element-wise multiplication on the elements of two matrices.            \\
	{\sf GrB\_eWiseAdd}         & Performs an element-wise addition on the elements of two matrices.                  \\
	{\sf GrB\_extract}          & Extracts a subgraph from an input matrix and copies them into an output matrix.        \\
	{\sf GrB\_assign}           & Assigns an input scalar value to each element of a specified subgraph.              \\
	{\sf GrB\_apply}            & Applies a unary operator to matrix elements.                                 \\
	{\sf GrB\_reduce}           & Reduces across matrix rows into a vector.                    \\
\end{tabular}
\end{center}
\hrule
\end{table}



\section{Example: Betweenness Centrality}
\label{sec:example}


Betweenness centrality (BC) is a popular metric to assess the centrality of 
vertices in a graph. It is based on shortest paths where the BC score of a
vertex $v$ is the normalized ratio of the number of shortest paths between 
any pair of vertices that go through $v$ to the total number of shortest paths 
in the graph.  Equation~\ref{eqn:bc} formally defines BC where $\sigma_{st}$ 
denotes the number of shortest paths from vertex $s$ to vertex $t$, and 
$\sigma_{st}(v)$ is the number of such paths passing through vertex $v$:
\begin{equation}
	BC(v) = \sum_{s \neq v \neq t \in V} \frac{\sigma_{st}(v)}{\sigma_{st}}
\label{eqn:bc}
\end{equation}
The BC score is efficiently computed using Brandes' 
algorithm~\cite{brandes2001faster}, 
which runs in $O(mn)$ time on unweighted graphs and avoids the expensive 
explicit all-pairs shortest paths computation.  For each starting vertex, $s$, 
Brandes' algorithm computes the BC contributions from the shortest paths starting
at $s$ that pass through every other vertex.

A batched version of Brandes' algorithm using linear-algebraic primitives
exists in the literature~\cite{combblas,bader2006designing,robinson2011complex}. 
We have implemented this batched version, where BC contributions from multiple 
source vertices are computed simultaneously, using the C GraphBLAS API. 
Figure~\ref{Fig:BClisting} shows the subroutine, {\tt BC\_update}, that computes
BC contributions from a subset of source vertices. Compared to previous work, the 
flexibility offered by the GraphBLAS API, especially masks, accumulators, 
and descriptors, enabled fewer functions calls within the main loops. The 
result is fewer intermediate objects created and less data moved.

At a high level, the {\tt BC\_update} function performs two sweeps over the 
graph. The forward sweep performs multiple simultaneous
breadth-first searches (one for each source vertex) where it also keeps track 
of the number of independent shortest paths that reach every vertex from 
the source. This is performed by the do-while loop starting at 
line~\ref{line:dowhile}. The backward sweep rolls back and tallies the BC 
contributions to every vertex. This is performed by the for loop starting 
at line~\ref{line:forloop}.  

In the forward sweep, variable {\tt numsp} keeps track of the number of
independent shortest paths that reach every other vertex from the source 
vertices, variable {\tt frontier} contains the current frontiers for each 
source vertex, and {\tt sigma} stores the final breadth-first search trees. 
GrB\_mxm call in line~\ref{line:mxm1} forms the next frontier in one step 
by both expanding the current frontier (i.e. discovering the 1-hop neighbors 
of the set of vertices in the current frontier) and pruning the vertices 
that have already been discovered before. This is achieved by setting the 
descriptor object {\tt desc} to use the structural complement of the mask 
and by passing the variable {\tt frontier} as the mask parameter. The 
implicit cast of {\tt frontier} to Boolean allows GrB\_mxm to interpret 
{\tt frontier} as the set of previously discovered vertices.  The 
GrB\_Matrix\_nvals call calculates the number of newly discovered vertices 
in that iteration and stores it in variable {\tt nvals}. The forward sweep 
is over when {\tt nvals} is zero.

The actual BC contributions are calculated during the tallying phase that 
performs a backwards sweep using the previously stored BFS trees. We first 
initialize the {\tt bcu} variable that holds the per-source BC contributions 
to all ones, in order to avoid issues with the treatment of zeros. In 
line~\ref{line:tallyewm1}, the contributions of each ``end'' vertex to its 
predecessors are divided by the number of shortest paths that reach them. The 
GrB\_mxm call in line~\ref{line:mxm2} discovers predecessors (as opposed to 
successors in the forward sweep) by its use of the descriptor {\tt desc\_r} 
(defined in line~\ref{line:desc}) that does not use the transpose of the 
adjacency matrix. The algorithm makes sure that the BC contributions are 
transferred only to direct parents on the BFS tree by passing the previous 
level of BFS tree ({\tt sigma[i-1]}) as a mask to GrB\_mxm. 

The remainder of this section describes the 
GraphBLAS implementation in detail.  
Note that for the sake of brevity and clarity, the examination of 
{\tt GrB\_info} return codes and handling of any errors is omitted.


\subsection{Preliminaries}

In line~\ref{line:include}, a single header file, {\tt GraphBLAS.h} is provided
that will define all collections, algebraic objects and signatures provided by the
API. In line~\ref{line:sig}, the signature of {\tt BC\_update} contains an uninitialized
output vector, {\tt delta}, that will be initialized and filled with the BC
contributions that are computed; the adjacency matrix of the graph, {\tt A}, 
defined on the {\tt GrB\_INT32} domain where edges are represented by a stored 1; the
array of source vertices in {\tt s}; and the number source vertices in the array,
{\tt nsver}. 

After finding the number of vertices in the graph (i.e. the number of rows in 
{\tt A}), the output vector is initialized to the appropriate size in 
line~\ref{line:init_output}.  Note that since BC contributions are not integer
values, a floating point domain is specified (single-precision {\tt GrB\_FP32} 
in this case).


\begin{figure*}[h]
\caption{C function using GraphBLAS primitives that computes the BC-metric
updates ${\it delta}$, given Boolean $n \times n$ adjacency matrix $A$, a
set of source vertices $s$, and the number of source vertices (i.e. the 
length of s) ${\it nsver}$.}
\label{Fig:BClisting}
{\scriptsize
\lstinputlisting[language=C,escapechar=|,numbers=left]{GabbBC4M.c}
}
\end{figure*}

\subsection{BFS Initialization}
 
Starting at line~\ref{line:int_arithmetic}, the algebriac objects used by the
BFS foward sweep are declared.  A 32-bit integer addition monoid {\tt Int32Add}
and the corresponding 32-bit integer arithmetic semiring {\tt Int32AddMul} are
initialized.  Both of these operate on inputs and produce an output that are
{\tt GrB\_INT32} types.  Note, that these objects have been proposed as
extensions to the API Specification, and if adopted, these declarations will not
be necessary.

In lines~\ref{line:numsp_begin}--\ref{line:numsp_end}, {\tt numsp} is initialized.
This is an {\tt n}$\times${\tt nsver} matrix where 
a single element in each column, corresponding to its source vertex, 
is set to one.  Mathematically,
\begin{equation}
	{\tt numsp}({{\tt s}_i},i) = 1, \text{ for } i \in [0,{\tt nsver}).
\label{eqn:bc}
\end{equation}
This is accomplished using the {\tt buildMatrix} operation that
requires two index arrays and one value array.  The row index array comes from 
the {\tt s} parameter while the column indices are created in the {\tt i\_nsver}
array.  An array of {\tt nsver} values set to one in {\tt ones}.  The call to
{\tt buildMatrix}, specifies no accumulation, mask, or descriptor.  It specifies
the integer addition operator, {\tt GrB\_PLUS\_INT32}, in case there are any
duplicate entries, but this does not occur.

In lines~\ref{line:frontier_begin}--\ref{line:frontier_end}, the 
{\tt n}$\times${\tt nsver} frontier matrix is initialized.  Each column of this 
matrix is initialized to the out vertices of of the corresponding source
vertex.  This could be accomplished by performing a single BFS step using the 
{\tt numsp} matrix as the input frontier; however, a alternate (more efficient?)
approach is shown here using the GraphBLAS {\tt extract}
operation.  This operation can be used because each
column of the {\tt numsp} matrix contains a single element equal to 1. 
The Descriptor for this operation transposes ({\tt GrB\_TRANS}) the {\tt A} 
matrix meaning that the {\tt GrB\_ALL} specified for all {\tt n} row indices 
selects all columns of {\tt A}, and the {\tt s} array specified for {\tt nsver} 
column indices selects
each row of {\tt A} corresponding to the source indices. The {\tt numsp} matrix
is specified as the mask. Because the Descriptor complements {\tt numsp}'s 
structure ({\tt GrB\_SCMP}), it masks out the source vertices themselves from 
each column of the result
of extract operation.  Since the {\tt frontier} matrix is already empty, the 
Descriptor's {\tt GrB\_REPLACE} parameter has no effect.
Note that it would be equivalent and valid to not specify a mask, ignore the
corresponding Descriptor parameters and allow any src vertices to remain in the
first set of frontiers.  

The final data structures needed for the BFS phase are a set of Matrices that
capture the current frontier at each step of the BFS phase.  This is stored in
an array of {\tt Sigma} matrices.  A set of {\tt n} of these are dynamically
allocated at line~\ref{line:sigma_init}.  Note that the actual number of matrices 
needed is bounded by the diameter of the graph which is bounded by the number of 
vertices in the graph.

\subsection{BFS Phase (Forward Sweep)}

The BFS phase of the computation begins with the do-loop on line~\ref{line:dowhile}. 
The first step is to initialize an {\tt n}$\times${\tt nsver} Sigma matrix for 
this step.  On line~\ref{line:sigma_set}, the current frontier is stored in 
{\tt sigmad[d]} using the {\tt apply} operation is used with the identity UnaryOp on
the boolean domain, {\tt GrB\_IDENTITY\_BOOL}.  Because the {\tt frontier} is uses
an integer domain to capture the number of paths at each step, the apply using the
boolean identity operator has the effect of casting the integer frontier to the boolean
sigma matrix.

On line~\ref{line:add_paths}, The path counts for the current frontier are 
accumulated.  The {\tt eWiseAdd} operation with the 
{\tt Int32Add} monoid is used to add the contents of the current {\tt frontier} matrix is
added to the {\tt numsp} matrix.

On line~\ref{line:mxm1}, the {\tt mxm} operation is used to advance the frontiers
of the BFS traversals by multiplying the transpose of the graph ({\tt A}) and the 
current {\tt frontier}. The result of this matrix multiplication is masked by the
structural complement of {\tt numsp} to remove any vertices that have already
been visited by the traversal.

The loop ends by computing the number of values in the new frontier using the
matrix method, {\tt GrB\_Matrix\_nvals}.  If the result is zero, there are no
vertices in the frontier and the BFS traversal is complete.
is complete

\subsection{Tally Initialization}

To prepare for the Tally phase (backward sweep), an number of floating point 
algebraic operations are initialized on the {\tt GrB\_FP32} domain starting on 
line~\ref{line:fp_arithmetic}.  Arithmetic monoids for addition and multiplication
({\tt F32Add} and {\tt F32Mul}) and the arithmetic semiring ({\tt FP32AddMul}) 
are initialized.

Starting on line~\ref{nspinv}, the element-wise inverse of {\tt numsp} is 
computed using the {\tt apply} operation along with the multiplicative inverse
unary function defined for 32-bit floating point, {\tt GrB\_MINV\_FP32}.

Following this, the {\tt n}$\times${\tt nsver} BC update matrix, {\tt bcu} is 
initialized starting on line~\ref{line:bcu_init}.  A variant of the {\tt assign} 
operation that allows the same value to be assigned to a subgraph.  However, 
since {\tt GrB\_ALL} is specified
for both row and column indices, this has the effect of filling the entire
matrix with ones to deal with sparsity issues during element-wise operations in
the next phase.

Finally, a Descriptor object, {\tt desc\_r}, needed by the Tally Phase is 
initialized starting on line~\ref{line:desc}.  The only parameter needed in this
phase is ``replace'' semantics when using the mask. 

\subsection{Tally Phase (Backward Sweep)}

After the initialization of a temporary workspace matrix, {\tt w}, the Tally
Phase begins on line~\ref{line:forloop}.  On line~\ref{line:tallyewm1}, the
current BC updates in {\tt bcu} are (normalized) scaled by the inverse of the
number of shortest paths by using the {\tt eWiseMult} operation.  This is masked
by the sigma matrix of the current BFS step (starting with the last and working
backwards through the traversal) to only operate on contributions from current
nodes in the traversal.  [I NEED HELP WITH THIS EXPLANATION]

On line~\ref{line:mxm2}, a single backward BFS step is performed using the {\tt w}
matrix as the set of frontiers.  The result of this {\tt mxm} operation is masked
by the previous sigma matrix to filter out paths that were not part of the original
forward traversal.  The resulting values in the computation are assigned to {\tt w}
by replacing all existing values (per the Descriptor).

Finally, the elements of the {\tt w} matrix are scaled by the number of shortest
paths with the {\tt eWiseMult} operation on line~\ref{line:accum_bcu}.  This result
is accumulated into the BC update matrix, {\tt bcu}.

This loop terminates when the original source vertices are reached.

\subsection{Wrapping Up}

To compute the BC updates for all vertices for this set of source vertices, 
the elements in each row of {\tt bcu} are accumulated using the {\tt reduce} 
operation on line~\ref{line:bcu_reduce}.  This result, is biased because 
{\tt bcu} began the loop filled with 1's.  As a result all elements of the reduction 
need to be adjusted by the number source vertices.  This is accomplished, by
accumulating the reduction result (in line~\ref{line:bcu_reduce}) with the output
vector, {\tt delta}, 
that has been filled with {\tt -nsver} (initialized in line~\ref{line:compensate}).

The remainder of the subroutine, involves freeing all of the resources allocated
to perform this computation.  Note that {\tt GrB\_free\_all} is a macro that 
expands to {\tt GrB\_free} for each of its parameters.
\section{Results}
\label{sec:results}

The subroutine in Figure~\ref{Fig:BClisting} was implemented using the
the GraphBLAS Template Library (GBTL)\cite{gbtl-cuda16}. This is a C++ library whose
goal is to implement functionally equivalent operations as the GraphBLAS C
Specification with similar signatures.  As of the writing of this paper, it is 
under active development as the specification approaches 
completion.  The latest development snapshot, including the working BC 
implementation and its unit test code, can be found on github \cite{gbtl-github}.  


%%%%%%%%%%%%%%%%%%%%%%% file conclusion.tex %%%%%%%%%%%%%%%%%%%%%%%%%
%
% This file contains the summary and conclusions of the paper.
%
%%%%%%%%%%%%%%%%%%%%%%%%%%%%%%%%%%%%%%%%%%%%%%%%%%%%%%%%%%%%%%%%%%%
\section{Conclusion}
\label{sec:conclusion}

The GraphBLAS C API 1.0 provisional specification was released in May of 2017.  The qualifier ``provisional'' will be dropped once
two conformant implementations of the specification have been completed.  To manage the scope of the project,
we had to defer many planned features for a future release of the GraphBLAS C API.  

A successful API evolves over time to meet the needs of its user community.   This means a dialog
between the users of the API and the team working on the API is critical.  This paper is the start of the next
phase in that dialog; to launch the ongoing discussion of the future GraphBLAS C API version 2.0.


\section*{Acknowledgments}
Ayd\i n Bulu\qc's work was supported by the Applied Mathematics Program of the DOE Office of Advanced Scientific
Computing Research under contract number DE-AC02-05\-CH\-11231.  Scott McMillan's work was supported by the 
Department of Defense under Contract No. FA8721-05-C-0003 with Carnegie Mellon University for the operation of 
the Software Engineering Institute, a federally funded research and development center. 


\bibliographystyle{IEEEtran}
\bibliography{GABB17}


\end{document}  