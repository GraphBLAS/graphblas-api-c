\documentclass[10pt, conference, compsocconf]{IEEEtran}   	
%\usepackage{geometry}                		% See geometry.pdf to learn the layout options. There are lots.
%\geometry{letterpaper}                   		% ... or a4paper or a5paper or ... 
%\geometry{landscape}                		% Activate for for rotated page geometry
%\usepackage[parfill]{parskip}    		% Activate to begin paragraphs with an empty line rather than an indent
\usepackage{graphicx}				% Use pdf, png, jpg, or eps§ with pdflatex; use eps in DVI mode
								% TeX will automatically convert eps --> pdf in pdflatex

\usepackage{color}
\newcommand{\fix}[1]{\textcolor{red}{#1}}
\newcommand{\highlight}[1]{\textcolor{red}{#1}}
\usepackage{amssymb, amsmath}
\usepackage{mathtools}
\usepackage{booktabs}
%\usepackage[noend]{algpseudocode}
%\usepackage[ruled,vlined]{algorithm2e}
\usepackage{verbatim}
\newcommand*\rot{\rotatebox{90}}
%\usepackage[compatible]{algpseudocode}
\usepackage{algorithm}
\usepackage[noend]{algpseudocode}
\usepackage{varwidth}
\usepackage{multirow}
\usepackage{rotating, xcolor}
\usepackage{subfig}
\usepackage{enumitem}
%\usepackage[caption=false]{subfig}
\usepackage[hyphens]{url}
\usepackage{float}
\newfloat{algorithm}{t}{lop}

\usepackage{tcolorbox}
\definecolor{mycolor}{rgb}{0.122, 0.435, 0.698}
%\begin{tcolorbox}[width=\linewidth, boxsep=0pt, left=-4pt, boxrule=0.5pt]
%\end{tcolorbox} 
%\newtcolorbox{mybox}{colback=red!5!white,colframe=mycolor}
\makeatletter
\newcommand{\mybox}[1]{%
  %\setbox0=\hbox{#1}%
  %\setlength{\@tempdima}{\dimexpr\wd0+13pt}%
  \begin{tcolorbox}[colframe=mycolor,boxrule=0.5pt,arc=4pt,
      left=-4pt,right=6pt,top=6pt,bottom=6pt,boxsep=0pt,width=\linewidth]
    #1
  \end{tcolorbox}
}
\makeatother

\algnewcommand{\LineComment}[1]{\State \(\triangleright\) #1}
%\newcommand{\procdecl}[1]   {\proc{#1}\vrule width0pt height0pt depth 7pt \relax}
\newcommand{\lilabel}[1]        {\label{li:#1}}

\newcommand{\erdosrenyi}{Erd\H os-R\'{e}nyi }
\newcommand{\qg}{\u{g}}
\newcommand{\qG}{\u{G}}
\newcommand{\qc}{\c{c}}
\newcommand{\qC}{\c{C}}
\newcommand{\qs}{\c{s}}
\newcommand{\qS}{\c{S}}
\newcommand{\qu}{\"{u}}
\newcommand{\qU}{\"{U}}
\newcommand{\qo}{\"{o}}
\newcommand{\qO}{\"{O}}
\newcommand{\qI}{\.{I}}
\newcommand{\wa}{\^{a}}
\newcommand{\wA}{\^{A}}
\usepackage{mathtools}
\DeclarePairedDelimiter\ceil{\lceil}{\rceil}
\DeclarePairedDelimiter\floor{\lfloor}{\rfloor}


\newcommand{\minusone}{\text{-}1}

%% ABAB: Shortcuts/macros below are not used but it can be referenced as a cheatsheet
\newcommand{\liref}[1]      {line~\ref{li:#1}}
\newcommand{\Liref}[1]      {Line~\ref{li:#1}}
\newcommand{\lirefs}[2]     {lines \ref{li:#1}--\ref{li:#2}}
\newcommand{\Lirefs}[2]     {Lines \ref{li:#1}--\ref{li:#2}}
\newcommand{\lireftwo}[2]   {lines \ref{li:#1} and~\ref{li:#2}}
\newcommand{\lirefthree}[3] {lines \ref{li:#1}, \ref{li:#2}, and~\ref{li:#3}}

\def\Cpp{C{}\texttt{++}}
\newcommand{\mA}{\mathbf{A}} 
\newcommand{\mL}{\mathbf{L}}
\newcommand{\mU}{\mathbf{U}}
\newcommand{\transpose}     {^{\mbox{\scriptsize \sf T}}}
\newcommand{\mB}{\mathbf{B}}
\newcommand{\mC}{\mathbf{C}}
\newcommand{\dimN}{n}
\newcommand{\dimM}{m}
\newcommand{\dimK}{k}
\newcommand{\dnzc}{\id{nzc}}
\newcommand{\dnzr}{\id{nzr}}
\newcommand{\dni}{\id{ni}}
\newcommand{\dnnz}{\id{nnz}}
\newcommand{\dsort}{\id{sort}}
\newcommand{\dscan}{\id{scan}}
\newcommand{\dsearch}{\id{search}}
\newcommand{\dmin}{\func{min}}
\newcommand{\dmax}{\func{max}}
\newcommand{\dth}{th}
\newcommand{\dlen}{\id{len}}
\newcommand{\matlab}{{\sc Matlab}}


%\input{algobox}


\title{Design of the GraphBLAS API for C}


\author{
Ayd\i n Bulu\qc , Tim Mattson, Scott McMillan, Jos\'e Moreira, Carl Yang}

\date{}	

\begin{document}
\maketitle


\begin{abstract}
The GraphBLAS effort aims to standardize linear-algebraic building blocks for graph computations. 
A time consuming part of this standardization effort is to translate the mathematical specification to
an actual Application Programming Interface (API) that (i) is faithful to the mathematics as much as possible
and (ii) enables efficient implementations on modern hardware. This paper 
documents the efforts taken by the C language specification subcommittee and presents the main concepts, 
constructs, and objects within the GraphBLAS API.

\end{abstract}



%%%%%%%%%%%%%%%%%%%%%%% file intro.tex %%%%%%%%%%%%%%%%%%%%%%%%%
%
% This file contains the introduction section of the paper
%
%%%%%%%%%%%%%%%%%%%%%%%%%%%%%%%%%%%%%%%%%%%%%%%%%%%%%%%%%%%%%%%%%%%
\section{Introduction}
\label{sec:intro}
The GraphBLAS are great and you'll love them.  In this paper, we'll talk about the even greater stuff coming in the future
\section{Basic Concepts}
\label{sec:concepts}

The GraphBLAS standard defines a set of matrix and vector operations 
over semiring algebraic structures.  These operations can be used
to express a wide range of graph algorithms.  

The GraphBLAS C API is built on a collection of objects exposed to 
the C programmer as opaque data types. Functions that manipulate these
objects are referred to as {\it methods}.  These methods fully define the 
interface to GraphBLAS objects to create or destroy them, modify their 
contents, and copy the contents of opaque objects into non-opaque objects the 
contents of which are under direct control of the programmer.



\section*{Acknowledgments}
This work is supported by the Applied Mathematics Program of the DOE Office of Advanced Scientific
Computing Research under contract number DE-AC02-05\-CH\-11231.


\bibliographystyle{IEEEtran}
\bibliography{GABB17}


\end{document}  