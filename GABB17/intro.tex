\section{Introduction}
\label{sec:intro}

Graphs are a fundamental abstraction in computer science.  They represent
relationships between a finite collection of objects.   The objects, or
\emph{vertices}, in a graph are connected by \emph{edges}.  This leads
to the common view of a graph as two sets;  a set of vertices and a set
of edges.

Graphs can also be represented as matrices.   For example, the
\emph{adjacency matrix} for a graph is constructed by labeling rows and
columns of the matrix by the vertices of the graph.  The elements of
the matrix denote the edges in the graph with matrix element $A_{ij}$
defining the edge from vertex $i$ to vertex $j$.  Most
vertices in a large graphi, such as those arising in social networks,
are not connected to each other so the matrices used for graphs tend to
be very sparse.

Many graph algorithms have been defined in the ``language of linear
algebra''~\cite{kepner2011graph}.  Mapping linear algebra algorithms over
sparse matrices onto modern computer architectures is well understood,
hence a number of groups have taken advantage of this fact to build high
performance graph libraries based on sparse linear algebra~\cite{combblas,
gadepally2015graphulo, gpi2016, sundaram2015graphmat}.  A group
of graph algorithm researchers, from academia, industry, and
federally-funded research laboratories, have formed the GraphBLAS
forum~\cite{graphblas_web} to standardize the low level building
blocks used in these graph algorithms.  The Forum has completed the
mathematical formalizations of GraphBLAS, which are summarized in a
recent paper~\cite{mathgraphblas16}.  The next task was to define the
binding of the C-programming language to the mathematical definition of
the GraphBLAS -- the so-called GraphBLAS C API specification.   This work
has been undertaken by a subcommittee of the GraphBLAS forum, comprised of
the authors of this paper.  We had to define the concepts, the objects,
and function signatures in the GraphBLAS C API specification as we
balanced often conflicting objectives: (i) simplicity and ease of use,
(ii) enabling high-performance implementations, and (iii) adherence to
the underlying mathematics.

This paper summarizes the GraphBLAS C API specification and the
motivation behind our key decisions.  We begin by summarizing the
essential mathematical ideas behind the GraphBLAS and how those ideas
influence the notation used in our work.  We then explain data structures,
algebraic objects, and objects that control the semantics of the functions
defined by the the GraphBLAS C API specification.   We then define the
core operations in the GraphBLAS C API specification and the signatures
for a subset of the functions  within the API.  We present a detailed
example of the GraphBLAS C API specification that computes a metric for
betweenness centrality of vertices in a graph.  We close with results
and concluding remarks.
