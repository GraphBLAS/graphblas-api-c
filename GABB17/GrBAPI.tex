\section{The GraphBLAS C API}
\label{sec:Capi}

Definition of the full GraphBLAS C API is  beyond the scope of this paper.  Instead, we 
focus on those parts of the API that are needed to understand the betweenness centrality
example in Section~\ref{sec:example}. 

We begin with data types used for the objects in the example.
These are listed in Table~\ref{Tab:GrBdataTypes}.  Except for the types that map directly onto
C language basic types ({\sf GrB\_Info}, {\sf GrB\_Index}, and {\sf GrB\_Type}), these data types
define handles to opaque objects manipulated by GraphBLAS methods.
\begin{table}[h]
\hrule
\begin{center}
\caption{GraphBLAS data types.}
\label{Tab:GrBdataTypes}
\begin{tabular}{lp{5.25cm}}
Data type                     & Description  \\
\hline
	{\sf GrB\_Info}           & Return value from any GraphBLAS method \\
	{\sf GrB\_Index}          & Vector and matrix indices \\
	{\sf GrB\_Type}		      & Type identifier \vspace{.1cm} \\ 
	{\sf GrB\_Descriptor}     & Opaque GraphBLAS descriptor object \\
	{\sf GrB\_Monoid}         & Opaque GraphBLAS monoid object  \\
	{\sf GrB\_Semiring}       & Opaque GraphBLAS semiring object  \\
	{\sf GrB\_Matrix}         & Opaque GraphBLAS matrix object \\
	{\sf GrB\_Vector}         & Opaque GraphBLAS vector object \\
\end{tabular}
\end{center}
\hrule
\end{table}
Objects corresponding to algebraic structures ({\sf GrB\_Monoid} and {\sf GrB\_Semiring}) are constructed from lower-level
operators. The GraphBLAS C API provides a mechanism for creating user-defined operators, but for this 
paper we consider only the predefined operators used in the example (summarized
in Table~\ref{Tab:GrBops}). 
\begin{table}[h]
\hrule
\begin{center}
\caption{Some predefined GraphBLAS operators.}
\label{Tab:GrBops}
\begin{tabular}{lp{5.25cm}}
Operator                       & Description  \\
\hline
	{\sf GrB\_TIMES\_INT32}    & Binary operation, returns  product of two 32-bit integer values \\
	{\sf GrB\_PLUS\_INT32}     & Binary operation, returns  sum of two 32-bit integer values \\
	{\sf GrB\_PLUS\_FP32}      & Binary operation, returns  sum of two 32-bit floating-point values \\
	{\sf GrB\_TIMES\_FP32}     & Binary operation, returns  product of two 32-bit floating-point values \\
	{\sf GrB\_MINV\_FP32}      & Unary operation, returns  multiplicative inverse of the input 32-bit floating-point value \\
	{\sf GrB\_IDENTITY\_BOOL}  & Unary operation, returns input boolean value \\
\end{tabular}
\end{center}
\hrule
\end{table}
A number of constant literal values are used in the GraphBLAS methods to 
choose among options or to define return values.
The most commonly used GraphBLAS literals are listed in Table~\ref{Tab:GrBliterals}.
\begin{table}[h]
\hrule
\begin{center}
\caption{GraphBLAS literals used in Section~\ref{sec:example}.}
\label{Tab:GrBliterals}
\begin{tabular}{lp{5.9cm}}
Literal                 & Description  \\
\hline
	{\sf GrB\_OUTP}      & Descriptor field for the output argument. \\
	{\sf GrB\_MASK}      & Descriptor field for the mask. \\
	{\sf GrB\_INP0}      & Descriptor field for the first input argument. \\
	{\sf GrB\_INP1}      & Descriptor field for the second input argument. \\ 
	{\sf GrB\_SCMP}      & Descriptor value to indicate use of the structural complement of the mask. \\
	{\sf GrB\_TRAN}      & Descriptor value to indicate use of the transpose of the corresponding input matrix. \\
	{\sf GrB\_REPLACE}   & Descriptor value to indicate that the output object should be replaced by the result of the method. \\ 
	{\sf GrB\_ALL}       & All of an object's indices in order. \\
	{\sf GrB\_NULL}      & \emph{Null} value used to indicate when a parameter is not provided and a default behavior should be used. \\
	{\sf GrB\_SUCCESS}   & Return value indicating that a method has returned without encountering an error condition. \\
	{\sf GrB\_BOOL}		   & Identifier for boolean type. \\
	{\sf GrB\_INT32}	   & Identifier for 32-bit integer type. \\
	{\sf GrB\_FP32}		   & Identifier for 32-bit floating point type. \\
\end{tabular}
\end{center}
\hrule
\end{table}

We illustrate the principles behind the GraphBLAS C API with a single method: 
the {\sf GrB\_mxm()} operation, shown in Figure~\ref{Fig:mxm}.
{\sf GrB\_mxm()} takes three input matrices $\matrix{A}$, $\matrix{B}$, and $\matrix{C}$; 
computes a matrix product of $\matrix{A}$ and $\matrix{B}$; and either copies the result into the matrix $\matrix{C}$
or accumulates the product into the matrix $\matrix{C}$.  Based on the arguments 
and the descriptor, the semantics of {\sf GrB\_mxm()} can vary considerably.
The function of the descriptor is consistent across all methods, hence its use for
{\sf GrB\_mxm()} is applicable to all methods.

\begin{figure*}[ht!]
\hrule
	\caption{The {\sf GrB\_mxm()} function signature, parameters, and return values.}
\label{Fig:mxm}
\paragraph{Signature}
\footnotesize
\begin{verbatim}
        GrB_Info GrB_mxm(GrB_Matrix              *C,
                         const GrB_Matrix         Mask,
                         const GrB_BinaryOp       accum,
                         const GrB_Semiring       op,
                         const GrB_Matrix         A, 
                         const GrB_Matrix         B,
                         const GrB_Descriptor     desc);
\end{verbatim}

\paragraph{Parameters}

\begin{itemize}[leftmargin=1.1in]
    \item[{\sf C}]    ({\sf INOUT}) An existing GraphBLAS matrix. On
    input, the matrix provides values that may be accumulated with the
    result of the matrix product.   On output, the matrix holds the
    results of this operation.

    \item[{\sf Mask}] ({\sf IN}) A ``write'' mask that controls which
    results from this operation are stored into the output matrix
    {\sf C} (optional).  If no mask is desired,  {\sf GrB\_NULL}
    is specified. The mask dimensions must match those of the
    matrix {\sf C}, and the domain of the {\sf Mask} matrix must be
    of type {\sf bool} or any ``built-in'' GraphBLAS type.

    \item[{\sf accum}] ({\sf IN}) A binary operator for accumulating entries
    with an existing \matrix{\sf C} entries.  Use {\sf GrB\_NULL} for assignment rather than accumulation.

    \item[{\sf op}] ({\sf IN}) Semiring used in the matrix-matrix
    multiply: ${\sf op}=\langle D_1,D_2,D_3,\oplus,\otimes,0 \rangle$.

    \item[{\sf A}] ({\sf IN}) The GraphBLAS matrix holding the values
    for the left-hand matrix in the multiplication.

    \item[{\sf B}] ({\sf IN}) The GraphBLAS matrix holding the values
    for the right-hand matrix in the multiplication.

    \item[{\sf desc}] ({\sf IN}) Operation descriptor (optional). If
    a \emph{default} descriptor is desired, {\sf GrB\_NULL} should be
    used. Valid fields are as follows: 

    \begin{tabular}{lllp{2.75in}}
    Argument   & Field           & Value               & Description \\ \hline
    {\sf C}    & {\sf GrB\_OUTP} & {\sf GrB\_REPLACE}  & Output matrix {\sf C} is cleared before result is stored. \\
    {\sf Mask} & {\sf GrB\_MASK} & {\sf GrB\_SCMP}     & Use the structural complement of {\sf Mask}. \\
    {\sf A}    & {\sf GrB\_INP0} & {\sf GrB\_TRAN}     & Use transpose of {\sf A} for operation. \\
    {\sf B}    & {\sf GrB\_INP1} & {\sf GrB\_TRAN}     & Use transpose of {\sf B} for operation. \\
    \end{tabular}
\end{itemize}

\paragraph{Return Values}

\begin{itemize}[leftmargin=2.1in]

	\item[{\sf GrB\_SUCCESS}]	Blocking mode: operation
	completed successfully. Nonblocking mode: input argument consistency tests
	passed.  

	\item[{\sf GrB\_PANIC}]		      Unknown internal error.

	\item[{\sf GrB\_OUTOFMEM}]	      Not enough memory available.
	for operation

	\item[{\sf GrB\_DIMENSION\_MISMATCH}] Matrix dimensions are
	incompatible.

	\item[{\sf GrB\_DOMAIN\_MISMATCH}]    The domains of the matrices are incompatible with the 
	accumulator, semiring, or mask domains.

\end{itemize}

\hrule
\end{figure*}
The semantics of GraphBLAS methods associated with the operations from Table~\ref{Tab:GraphBLASOps} follow a similar pattern:
\begin{enumerate}
\item The internal matrices and mask used in the computation are formed from the input parameters.  Their domains and dimensions are tested for consistency.
\item The indicated computations are carried out using the internal matrices and producing an internal result.
\item The internal result is written into the output matrix, possibly under control of a mask.
\end{enumerate}
In the case of the {\sf GrB\_mxm} operation, internal matrices $\matrix{A}$, $\matrix{B}$, $\matrix{C}$, and mask $\matrix{Mask}$ are formed from
the corresponding arguments according to the descriptor.  Depending on
the values in the descriptor fields {\sf GrB\_INP0} and {\sf GrB\_INP1},  $\matrix{A}$ and/or $\matrix{B}$ may be the transpose of the corresponding argument.
The descriptor field {\sf GrB\_MASK} may also indicate that the structural complement of the mask 
should be used.  If the domains and sizes of the objects are mathematically consistent, the indicated operation is carried out.
This produces an internal matrix $\matrix{T}$ equal to the product of matrices $\matrix{A}$ and $\matrix{B}$.
(We emphasize that an implementation
of the GraphBLAS is not required to materialize the matrix $\matrix{T}$).

If an optional binary accumulator function {\sf accum} is provided, it is used to combine the elements of 
matrix $\matrix{C}$ and the internal matrix $\matrix{T}$.  This forms a new internal matrix $\matrix{Z}$.   

At this point the elements of $\matrix{Mask}$ are used as a \emph{write mask} to select which elements of $\matrix{Z}$ are 
used to form the final output result.  Basically, the elements of the boolean write mask that exist and are true 
correspond to the elements of the output matrix that might be replaced by the corresponding elements of $\matrix{Z}$.
Two options are supported: 
\begin{itemize}
	\item \emph{Replace mode}: If the descriptor field {\sf GrB\_OUTP} is set to {\sf GrB\_REPLACE}, the 
		values in the $\matrix{C}$ matrix are deleted before masked elements of $\matrix{Z}$ are stored 
		in $\matrix{C}$.  In essence the computed matrix $\matrix{Z}$ \emph{replaces} the original matrix $\matrix{C}$.
	\item \emph{Merge mode}: Otherwise, the elements from the computation selected by the 
		write mask are written into the output $\matrix{C}$ matrix without changing elements
		that do not overlap with the mask.
\end{itemize}
%This discussion has been largely qualitative, and there are many low level details an implementor
%of {\sf GrB\_mxm()} would need to address.  The high level description of this operation, however, should
%give a basic idea of matrix multiplication in GraphBLAS.

The basic pattern used for {\sf GrB\_mxm()} is used for most of the GraphBLAS operations.
For example, the mathematical operations in 
Table~\ref{Tab:GraphBLASOps}  from {\sf GrB\_mxm} to {\sf GrB\_reduce}
use descriptors to modify input matrices or vectors, and write masks.  
We can't list all methods in the GraphBLAS API, but for the methods used in the example from Section~\ref{sec:example}
we list method names and descriptions in Table~\ref{Tab:GrBmethods}.    
%Low level details may vary, but the basic pattern used in {\sf GrB\_mxm} holds.
 
\begin{table}[h]
\hrule
\begin{center}
\caption{Methods used in the example in section~\ref{sec:example}.}
\label{Tab:GrBmethods}
\begin{tabular}{lp{5.25cm}}
Method Name                     & Description    \\
\hline
	{\sf GrB\_Monoid\_new}      & Creates a new monoid with specified domain, operator, and identity element.    \\
	{\sf GrB\_Semiring\_new}    & Creates a new semiring with specified domain, monoid, and operators.               \\
	{\sf GrB\_Vector\_new}      & Creates a new vector with specified domain and size.                                \\
	{\sf GrB\_Matrix\_new}      & Creates a new matrix with specified domain and dimensions.                          \\
	{\sf GrB\_Matrix\_nrows}    & Retrieves the number of rows in a matrix.                                            \\
	{\sf GrB\_Matrix\_nvals}    & Retrieves the number of stored elements (tuples) in a matrix.                        \\
	{\sf GrB\_Descriptor\_new}  & Creates a new (empty) descriptor.                                                  \\
	{\sf GrB\_Descriptor\_set}  & Sets the content (details of an operation) for a field of an existing descriptor.   \\
	{\sf GrB\_Matrix\_build}    & Copies elements from tuples into a matrix.                                         \\
	{\sf GrB\_mxm}              & Performs matrix multiplication over a semiring.                              \\
	{\sf GrB\_eWiseMult}        & Performs an element-wise multiplication on the elements of two matrices.            \\
	{\sf GrB\_eWiseAdd}         & Performs an element-wise addition on the elements of two matrices.                  \\
	{\sf GrB\_extract}          & Extracts a subgraph from an input matrix and copies them into an output matrix.        \\
	{\sf GrB\_assign}           & Assigns an input scalar value to each element of a specified subgraph.              \\
	{\sf GrB\_apply}            & Applies a unary operator to matrix elements.                                 \\
	{\sf GrB\_reduce}           & Reduces across matrix rows into a vector.                    \\
\end{tabular}
\end{center}
\hrule
\end{table}


